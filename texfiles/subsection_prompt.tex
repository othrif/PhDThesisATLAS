\subsection{Associate $t\bar t+W/Z$ production}
\label{subsec:promptbkg_ttv}

The production of a $t\bar t$ pair in association with a leptonically-decaying $W$ or $Z$ boson (including non-resonant contributions) 
constitutes the main source of background with prompt same-sign leptons for event selections including $b$-jets. 


\par{\bf Details on event generation and normalization\\}

Simulated events for these processes were generated at NLO with \AMCATNLO v2.2.2~\cite{Alwall:2014hca} interfaced to \textsc{Pythia} 8, 
further details can be found in Ref.~\cite{ATL-PHYS-PUB-2016-005}. 
The samples are normalised to the inclusive process NLO cross-section using appropriate $k$-factors 
computed by the PMG group~\cite{twiki-xsections-ttx,ATL-PHYS-PUB-2016-005} with a methodology similar to Ref.~\cite{Alwall:2014hca}. 


\par{\bf Theory uncertainties \\}

The theoretical uncertainties on the $ttW$ and $ttZ^{(*)}$ processes are evaluated following the recommendations of the Physics Modelling Group and using the LHE3 weights included within the nominal MC samples. The variation weights considered are the following:

\begin{itemize}
\item Normalization and factorization scales varied independently up and down by a factor of two from the central scale $\mu_0 = H_{\rm T}/2$ as detailed in Ref.~\cite{ATL-PHYS-PUB-2016-005}. The largest deviation with respect to the nominal is used as the symmetric uncertainty.
\item PDF variations 
%, following the recommendations at \\\href{https://twiki.cern.ch/twiki/bin/view/AtlasProtected/PdfRecommendations}{https://twiki.cern.ch/twiki/bin/view/AtlasProtected/PdfRecommendations}
\item Alternative \textsc{Sherpa} (v2.2) samples produced using LO with 1 extra parton in the matrix element for $ttW$ and 2 extra partons for $ttZ$~\cite{ATL-PHYS-PUB-2016-005}. The yield comparison for all SRs is shown in Table~\ref{tab:ttVGenComp}, with negligible differences in some SRs and up to 28\% in the worst case.
\end{itemize}

More details can be found in Appendix~\ref{app:mc_modeling_uncertainties}. 
Based on these studies and the cross-section uncertainties, the total theory uncertainty for these processes is at the level of 13-33\% in the signal and validation regions used in the analysis. 
In addition, the cross-sections used to normalise the MC samples are varied according to the uncertainty on the cross-section calculation, that is, 13\% for $\ttbar W$ and 12\% $\ttbar Z$ production~\cite{Alwall:2014hca}. 


\begin{table}[!htb]
\caption{Comparison of the event yields for the $\ttbar V$ backgrund processes between \AMCATNLO (default generator) and \textsc{Sherpa} in the SRs, as well as their relative difference.
}
\label{tab:ttVGenComp}
\def\arraystretch{1.1}
\centering
\begin{tabular}{|c|c|c|c|}
\hline\hline
   SR    & \textsc{Sherpa} & aMCATNLO & Relative diff.\\ \hline
Rpc2L0bH   &   0.25 $\pm$ 0.03   &   0.20 $\pm$ 0.05   &   25\% \\
Rpc2L0bS   &   0.60 $\pm$ 0.06   &   0.82 $\pm$ 0.10   &   -26\% \\
Rpc2L1bH   &   3.84 $\pm$ 0.14   &   3.86 $\pm$ 0.20   &   $<$1\% \\
Rpc2L1bS   &   3.55 $\pm$ 0.13   &   3.94 $\pm$ 0.20   &   -9\% \\
Rpc2L2bH   &   0.35 $\pm$ 0.04   &   0.41 $\pm$ 0.05   &   -14\% \\
Rpc2L2bS   &   1.57 $\pm$ 0.08   &   1.57 $\pm$ 0.12   &   $<$1\% \\
Rpc2Lsoft1b   &   1.01 $\pm$ 0.07   &   1.24 $\pm$ 0.11   &   -18\% \\
Rpc2Lsoft2b   &   1.13 $\pm$ 0.07   &   1.15 $\pm$ 0.10   &   -1\% \\
Rpc3L0bH   &   0.23 $\pm$ 0.02   &   0.18 $\pm$ 0.04   &   27\% \\
Rpc3L0bS   &   0.90 $\pm$ 0.05   &   0.99 $\pm$ 0.09   &   -9\% \\
Rpc3L1bH   &   1.54 $\pm$ 0.08   &   1.52 $\pm$ 0.11   &   1\% \\
Rpc3L1bS   &   6.95 $\pm$ 0.16   &   7.02 $\pm$ 0.23   &   $<$1\% \\
Rpc3LSS1b   &   0.00 $\pm$ 0.00   &   0.00 $\pm$ 0.00   &   - \\
Rpv2L0b   &   0.18 $\pm$ 0.03   &   0.14 $\pm$ 0.04   &   28\% \\
Rpv2L1bH   &   0.51 $\pm$ 0.04   &   0.54 $\pm$ 0.07   &   -5\% \\
Rpv2L1bM   &   1.78 $\pm$ 0.08   &   1.40 $\pm$ 0.12   &   27\% \\
Rpv2L1bS   &   9.85 $\pm$ 0.19   &   9.91 $\pm$ 0.30   &   $<$1\% \\
Rpv2L2bH   &   0.52 $\pm$ 0.04   &   0.53 $\pm$ 0.08   &   -1\% \\
Rpv2L2bS   &   5.09 $\pm$ 0.14   &   4.80 $\pm$ 0.21   &   6\%  \\
\hline\hline
\end{tabular}
\end{table}

\subsection{Diboson $WZ, ZZ, W^\pm W^\pm$ production}
\label{subsec:promptbkg_vv}

The production of multiple $W,Z$ bosons decaying leptonically 
constitutes the main source of background with prompt same-sign leptons for event selections vetoing $b$-jets. 

\par{\bf Details on event generation and normalization\\}

Diboson processes with four charged leptons, three charged leptons and one neutrino, or two charged leptons and two neutrinos 
were simulated at NLO using the \textsc{Sherpa} 2.2.1 generator~\cite{Gleisberg:2008ta}, as described in detail in Ref.~\cite{ATL-PHYS-PUB-2016-002}. 
The main samples simulate $qq \to VV\to\text{leptons}$ production including the doubly resonant WZ and ZZ processes, 
non-resonant contributions as well as Higgs-mediated contributions, and their interferences; 
up to three extra partons were included (at LO) in the matrix elements. 
Note that these samples include the contributions from tri-boson production with one of the vector bosons decaying hadronically, 
while tri-boson production with only leptonic decays was simulated separately (see section~\ref{subsec:promptbkg_rare}). 

Simulated events for the $W^\pm W^\pm jj$ process (including non-resonant contributions) were produced at LO with up to one extra parton, 
separately for QCD-induced $\left(\mathcal{O}(\alpha_\text{em}^4)\right)$ 
and VBS-induced $\left(\mathcal{O}(\alpha_\text{em}^6)\right)$ production -- the interferences being neglected. 
Additional samples for VBS-induced $qq\to 3\ell\nu jj$ and $qq\to 4\ell$ and loop-induced $gg\to WZ^{(*)}/ZZ^{(*)}$ processes
were also produced with the same configuration.

The samples generated at NLO are directly normalized to the cross-sections provided by the generator. 
The complete list of samples and cross-sections can be found in appendix~\ref{app:samples}, Table~\ref{tab:BGSamples1}. 

\par{\bf Theory uncertainties\\}

The theoretical uncertainties on the $WZ$ and $ZZ$ processes are evaluated following the recommendations of the Physics Modelling Group and using the LHE3 weights included within the nominal MC samples where available. The variation considered are the following:

\begin{itemize}
\item Normalization and factorization scales varied independently up and down by a factor of two from the central scale choice. The largest deviation with respect to the nominal is used as the symmetric uncertainty.
\item PDF variations
%, following the recommendations at\\ \href{https://twiki.cern.ch/twiki/bin/view/AtlasProtected/PdfRecommendations}{https://twiki.cern.ch/twiki/bin/view/AtlasProtected/PdfRecommendations}
\item Resummation scale varied up and down by a factor of two from the nominal value.
\item CKKW merging scale scale varied up and down to a value of 15 and 30 \GeV (with a value of 20 \GeV used in the nominal samples).
\end{itemize}

More details can be found in Appendix~\ref{app:mc_modeling_uncertainties}. 
Based on these studies and the cross-section uncertainties, the total theory uncertainty for these processes is at the level of 25-40\% in the signal and validation regions used in the analysis. In addition, the cross-sections used to normalise the MC samples are varied according to the uncertainty on the cross-section calculation of 6\%.

No theoretical uncertainties have been evaluated specifically for the $W^\pm W^\pm jj$ process, 
to which we assign the same uncertainties as for $WZ$, by lack of a better choice. 
But it should be noted that contributions from this process are minor in the SRs and typically  
smaller than those from $WZ$ and $ZZ$.

\subsection{Other rare processes}
\label{subsec:promptbkg_rare}

Production of a Higgs boson in association with a $\ttbar$ pair is simulated using \AMCATNLO~\cite{Alwall:2014hca} 
(in \MADGRAPH v2.2.2) interfaced to \HERWIG 2.7.1~\cite{Corcella:2000bw}.  
The UEEE5 underlying-event tune is used together with the CTEQ6L1~\cite{Pumplin:2002vw} (matrix element) and CT10~\cite{Lai:2010vv} (parton shower) PDF sets.
Simulated samples of SM Higgs boson production in association with a $W$ or $Z$ boson are produced with \PYTHIA 8.186, using the \textsc{A14} tune and the \textsc{NNPDF23LO} PDF set. Events are normalised with cross-sections calculated at NLO~\cite{Dittmaier:2012vm}. 

\MADGRAPH v2.2.2~\cite{Alwall:2011uj} is used to simulate the $t\bar{t}WW$, $tZ$, $\ttbar\ttbar$ and $\ttbar t$ processes, and the generator cross-section is used for $tZ$ and $\ttbar t$. \MADGRAPH interfaced to \textsc{Pythia} 8 is used to generate $\ttbar WZ$ processes, and appropriate $k$-factors are taken from~\cite{Alwall:2014hca}. \AMCATNLO interfaced to \PYTHIA 8 is used for the generation of the $tWZ$ process, with an alternative sample generated with \AMCATNLO interfaced to \HERWIG used to evaluate the parton shower uncertainty.  

Fully leptonic triboson processes ($WWW$, $WWZ$, $WZZ$ and $ZZZ$) with up to six charged leptons are simulated using \SHERPA~v2.1.1 
and described in Ref.~\cite{ATL-PHYS-PUB-2016-002}. 
The $4\ell$ and $2\ell+2\nu$ processes are calculated at next-to-leading order (NLO) for up to one additional parton; 
final states with two and three additional partons are calculated at leading order (LO). 
The $WWZ\to 4\ell+2\nu$ or $2\ell+4\nu$ processes are calculated at LO with up to two additional partons. 
The $WWW/WZZ\to 3\ell+3\nu$, $WZZ\to 5\ell+1\nu$, $ZZZ\to 6\ell+0\nu$, $4\ell+2\nu$ or $2\ell+4\nu$ processes 
are calculated is calculated at NLO with up to two extra partons at LO. 
The CT10~\cite{Lai:2010vv} parton distribution function (PDF) set is used for all \SHERPA samples in conjunction with 
a dedicated tuning of the parton shower parameters developed by the \SHERPA authors. 
The generator cross-sections (at NLO for most of the processes) are used when normalising these backgrounds.

A conservative 50\% uncertainty is assigned on the summed contributions 
of all these processes ($\ttbar H$, $tZ$, $tWZ$, $\ttbar\ttbar$, $\ttbar WW$, $\ttbar WZ$, $WH$, $ZH$, $VVV$), 
which is generally quite larger than the uncertainties on their inclusive production cross-sections, 
and assumes a similar level of mismodelling as for diboson or $t\bar t V$ processes. 


\subsection{$W^\pm W^\pm$ production via Double Parton Scattering}
\label{subsec:dps}

Double parton scattering (DPS) occurs when two partons interact simultaneously in a proton-proton collision leading to two hard scattering processes overlapping in a detector event. 
Accordingly, two single $W$ production processes can lead to a $W^\pm$ + $W^\pm$ final state via DPS. 
In the 2015 version of the present analysis, the background originated from $W^\pm W^\pm$ production via DPS was roughly estimated by overlaying at analysis level events from the \textsc{Sherpa} $W$+jets samples. The estimated contribution was at the level of 0.001-0.01 events for 4~fb$^{-1}$ in a SR similar to SR0b1-2 ($b$-jet veto), although slightly looser since it only required 5 jets.


DPS effects are implemented in Sherpa diboson samples used in the analysis in the case of $V+jj$, but not for $W^\pm+W^\pm$. 
In the current analysis, the sensitivity to this background and its contribution to the signal regions have been revisited using simulated DPS $WW\to\ell\nu\ell\nu$ events (with all possible sign combinations) generated at LO with \textsc{Pythia}8\textsc{EvtGen}. In addition, a sample with single $W$+jets events produced via SPS has been generated with the same Monte Carlo framework. This sample was used together with a \SHERPA $W$+jets sample to derive \SHERPA/\PYTHIA correction factors to address the insufficient prediction from \PYTHIA8 at high jet multiplicities. These factors were then applied to the actual DPS sample in order to provide a corrected Monte Carlo based estimation of the DPS $W^\pm W^\pm$ process in the signal regions.  
A detailed description of the samples used according validation plots can be found in appendix~\ref{app:dps}.

To estimate a conservative upper bound on cross-section for $WW$ events which might arise from DPS, a standard ansatz is adopted: 
in this, for a collision in which a hard process (X) occurs, the probability that 
an additional (distinguishable) process (Y) occurs is parametrized as:
\begin{equation}
\sigma^{DPS}_{XY} = \sigma^{}_{X}\sigma^{}_{Y}/\sigma^{}_{eff}
\end{equation} 
where $\sigma^{}_{X}$ is the production cross section of the hard 
process X and $\sigma^{}_{eff}$ (effective area parameter) 
parameterizes the double-parton interaction part of the production 
cross section for the composite system (X+Y). 
A value of $\sigma^{}_{eff}$ is 10-20~mb is assumed in this study (as obtained from 7~TeV measurements, and with no observed dependence on $\sqrt{s}$), and it is independent on the processes involved. For the case of $W^\pm+W^\pm$ production:
\begin{equation}
\sigma^{DPS}_{W^\pm W^\pm} = \frac{ \sigma^{}_{W^+}\sigma^{}_{W^+} + \sigma^{}_{W^-}\sigma^{}_{W^-} + 2\sigma^{}_{W^+}\sigma^{}_{W^-}}{\sigma^{}_{eff} } \simeq 0.19-0.38\text{ pb.}
\end{equation} 

After the application of the SR criteria, only 4 raw MC events in the DPS $WW\to\ell\nu\ell\nu$ remain. 
Table~\ref{tab:DPS_SR} shows the expected contribution in the SRs where some MC event survives all the cuts. The ranges quoted in the tables reflect the range in the predicted $\sigma^{DPS}_{W^\pm W^\pm}$ cross-section above, as well as the combinatorics for scaling the jet multiplicity\footnote{For instance, a DPS event with 6 jets can be due to the overlap of two events with 6+0 jets, or 5+1, 4+2 or 3+3 jets. All possible combinations are considered and the range quoted in the table shows the combinations leading to the smallest and largest correction factors.}.
Due to the large uncertainties involved in these estimates, some of them difficult to quantify (such as the modelling of DPS by {\sc Pythia} at LO), the contribution from this background is not included in the final SR background estimates. 
Note that the estimated DPS contribution is typically much smaller than the uncertainty on the total background for the SRs.

\begin{table}[!htb]
\caption{Number of raw MC events and its equivalent for 36.1 \ifb with and without the correction as a function of the jet multiplicity described in Appendix~\ref{app:dps}. Only the SRs where at least one MC event passes all the cuts are shown.}
\label{tab:DPS_SR}
\centering
\begin{tabular}{l|c|c|c}
\hline
SR       & Raw MC evts & Without $N_{\text{jet}}$ correction & With $N_{\text{jet}}$ correction \\\hline
Rpc2L0bS & 2 & 0.016-0.033 & 0.09-0.38 \\ 
Rpc2L0bH & 1 & 0.006-0.012 & 0.05-0.17 \\ 
Rpv2L0b  & 1 & 0.006-0.012 & 0.05-0.17 \\ 
\hline
\end{tabular}
\end{table}



\subsection{Experimental uncertainties}
\label{subsec:promptbkg_expsyst}

All the  experimental systematics provided by the SUSYTools {\tt getSystInfoList()} method have been considered. 
The list of sources of uncertainty and the corresponding names of the variations are:

\textbf{Jet energy scale ({\tt{Jet.JESNPSet:1}})}  \\  
One of the strongly reduced uncertainty sets provided by the JetEtMiss group for Run-2 searches is used in this note. 
These sets are intended for use by analyses which are not sensitive to jet-by-jet correlations arising from changes to the jet energy scale 
(as expected for many early SUSY searches), 
and we use the scenario {\tt JES2015\_SR\_Scenario1.config} (as included in the JetUncertainties package, 4 nuisance parameters). 
We checked that the uncertainties obtained from one of the 3 other scenarios did not lead to significant changes. 
The jet energy is scaled up and down (in a fully correlated way) by the $\pm 1\sigma$ uncertainty of each nuisance parameter.

\textbf{Jet energy resolution ({\tt{JET\_JER\_SINGLE\_NP\_\_1up}})} \\ % OK
An extra $\pt$ smearing is added to the jets based on their $\pt$ and $\eta$ to account for a possible underestimate of the jet energy resolution in the MC simulation. This is done by the {\tt JERSmearingTool} in the JetResolution package. We are using the {\tt Simple} systematic mode configuration (1 nuisance parameter). 

\textbf{Jet vertex tagger ({\tt{JvtEfficiency\{down,up\}}})}\\   % OK
The uncertainties are applied via the {\tt JetJvtEfficiency} tool, which account for the residual contamination from pile-up jets after pile-up suppression and the MC generator choice.

\textbf{Flavor tagging ({\tt{FT\_EFF\_\{B,C,Light\}\_systematics\_\_1\{up,down\}}}, {\tt{FT\_EFF\_extrapolation\_\_1\{up,down\}}}, {\tt{FT\_EFF\_extrapolation\_from\_charm\_\_1\{up,down\}}})} \\  % OK
Similarly to the case of the JES, a significant reduction in the number of nuisance parameters was provided by the Flavour Tagging CP group and used in this analysis.

\textbf{Egamma resolution ({\tt{EG\_RESOLUTION\_ALL\_\_1\{up,down\}}}) and scale ({\tt{EG\_SCALE\_ALL\_\_1\{up,down\}}})} \\ % OK
As the analysis is weakly sensitive to the energy scale of electrons, the special simplified correlation model ({\tt 1NPCOR\_PLUS\_UNCOR}) is used. It has only 2 systematic variations, one for the scale, one for the resolution. In this scheme all the effects are considered fully correlated in $\eta$ and they are summed in quadrature.

\textbf{Electron efficiency ({\tt{EL\_EFF\_\{RECO,ID,Iso,CFT\}\_TOTAL\_1NPCOR\_PLUS\_UNCOR\_1\{up,down\}}})}\\  % OK
These uncertainty sources are associated with the electron efficiency scale factors provided by the Egamma CP group. The default correlation model ({\tt TOTAL}) is used, which provides 1 uncertainty sources, for reconstruction, identification, isolation and charge flip tagger, separately. 

\textbf{Muon efficiency ({\tt{MUON\_EFF\_\{STAT,SYS\}\_\_1\{up,down\}}}, {\tt{MUON\_EFF\_\{STAT,SYS\}\_LOWPT\_\_1\{up,down\}}}, % OK
{\tt{MUON\_ISO\_\{STAT,SYS\}\_\_1\{up,down\}}})}\\ % OK
This uncertainty corresponds to the statistical and systematic uncertainties in the muon efficiency scale factors provided by the Muon CP group on the muon reconstruction and isolation. The *\_LOWPT\_* sources of uncertainties are associated to  muons with $\pt<15 \GeV$.

\textbf{Muon resolution uncertainty  ({\tt{MUONS\_ID\_\_1\{up,down\}}}, {\tt{MUONS\_MS\_\_1\{up,down\}}})} \\  % OK
This is evaluated as variations in the smearing of the inner detector and muon spectrometer tracks associated to the muon objects by $\pm 1\sigma$ their uncertainty

\textbf{Muon momentum scale ({\tt{MUONS\_SCALE\_\_1\{up,down\}}})} \\  % OK
This is evaluated as variations in the scale of the momentum of the muon objects.

\textbf{Muon scale corrections ({\tt{MUON\_SAGITTA\_RHO\_\_1\{up,down\}}}, {\tt{MUON\_SAGITTA\_RESBIAS\_\_1\{up,down\}}})} \\  % OK
All analyses are recommended to set the ``SagittaCorr" property to true. The flag corrects data for charge dependent local effects do to misalignments mainly in the ID and smaller local effects due to local misalignments in the MS. The {\tt{MUON\_SAGITTA\_RHO}} systematics describe a variation in the scale of the momentum (charge dependent), based on combination of correction on combined ($Z$ scale) and recombination of the corrections. The {\tt{MUON\_SAGITTA\_RESBIAS}} systematic describe a variation in the scale of the momentum (charge dependent), based on the residual charge-dependent bias after correction.

\textbf{Muon TTVA ({\tt{MUON\_TTVA\_\{STAT,SUSY\}\_\_1\{up,down\}}})}\\ % OK
Uncertainties associated to the TTVAs (track to vertex association) SFs for 2016 data (periods A-I) to account for worse ID $d_0$ resolution, resulting in lower efficiency and $\varphi$-dependent SFs, 1-2\% effects (new SFs are parametrized in $\pt-\eta-\varphi$, before only in $\pt-\eta$).

\textbf{Bad muon veto ({\tt{MUON\_BADMUON\_\{STAT,SYST\}\_\_1\{up,down\}}})}\\ % OK
Systematic associated to the bad-muon veto for high-pT muons (to reject tracks with poor resolution in the MS).

\textbf{\met\ soft term uncertainties  ({\tt{MET\_SoftTrk\_Reso\{Pare,Perp\}}}, {\tt{MET\_SoftTrk\_Scale\{up,down\}}})}\\  % OK
Note that the effect of the hard object uncertainties (most notably JES and JER) are also propagated to the $\met$.

\textbf{Pileup reweighting ({\tt{PRW\_DATASF\_\_1\{up,down\}}})}\\  % OK
This uncertainty is obtained by re-scaling the $\mu$ value in data by 1.00 and 1/1.18, 
covering the full difference between applying and not-applying the nominal $\mu$ correction of 1/1.09, 
as well as effects resulting from uncertainties on the luminosity measurements, which are expected to dominate.

\textbf{Luminosity}\\  % OK
See section~\ref{subsubsec:samples_data}.

\textbf{Trigger}\\
Trigger uncertainties provided by CP groups and propagated via \texttt{TrigGlobalEfficiencyCorrection} package.

The uncertainty on the beam energy is neglected. \\

All the experimental uncertainties are applied also on the signal samples when computing exclusion limits on SUSY scenarios. 



\subsection{Expected yields in the signal regions}
\label{subsec:promptbkg_yields}

The predicted event yields in the signal regions are presented in Table~\ref{tab:prompt_sr_yields}, while the contributions of particular rare processes to the signal regions, relative to the summed contributions of all these processes, are shown in Table~\ref{tab:prompt_rare_contributions}.

\begin{table}[!htb]
\caption{Expected yields for background processes with prompt leptons, 
in the SRs proposed in Section~\ref{sec:signalregions}, for 36.1 \ifb. 
Quoted uncertainties include statistical sources only. %and theory uncertainties. 
Rare category includes $\ttbar WW$, $\ttbar WZ$, 3$t$, $tZ$, $tWZ$, $WH$, $ZH$ and $VVV$, and detailed contributions of these processes can be found in Table~\ref{tab:prompt_rare_contributions}. 
}
\label{tab:prompt_sr_yields}
\def\arraystretch{1.1}
\centering
\resizebox{0.8\textwidth}{!}{
\begin{tabular}{|c|c|c|c|c|c|}
\hline\hline
& $t\bar t V$ & $VV$ & $t\bar tH$ & $t\bar tt \bar t $ & rare  \\\hline\hline
 Rpc2L0bH  &     0.20 $\pm$ 0.05   &     1.14 $\pm$ 0.23  &     0.08 $\pm$ 0.04  &     0.02 $\pm$ 0.01    &     0.17 $\pm$ 0.04  \\
 Rpc2L0bS  &     0.82 $\pm$ 0.10   &     3.13 $\pm$ 0.21  &     0.26 $\pm$ 0.05  &     0.01 $\pm$ 0.00    &     0.20 $\pm$ 0.04  \\
 Rpc2L1bH  &     3.86 $\pm$ 0.20   &     0.61 $\pm$ 0.06  &     1.01 $\pm$ 0.10  &     0.53 $\pm$ 0.03    &     0.97 $\pm$ 0.12  \\
 Rpc2L1bS  &     3.94 $\pm$ 0.20   &     0.48 $\pm$ 0.05  &     1.28 $\pm$ 0.10  &     0.33 $\pm$ 0.03    &     0.87 $\pm$ 0.12  \\
 Rpc2L2bH  &     0.41 $\pm$ 0.05   &     0.04 $\pm$ 0.01  &     0.10 $\pm$ 0.03  &     0.17 $\pm$ 0.02    &     0.14 $\pm$ 0.04  \\
 Rpc2L2bS  &     1.57 $\pm$ 0.12   &     0.10 $\pm$ 0.03  &     0.44 $\pm$ 0.06  &     0.25 $\pm$ 0.02    &     0.32 $\pm$ 0.05  \\
 Rpc2Lsoft1b  &     1.24 $\pm$ 0.11   &     0.14 $\pm$ 0.02  &     0.44 $\pm$ 0.06  &     0.09 $\pm$ 0.01    &     0.18 $\pm$ 0.04  \\
 Rpc2Lsoft2b  &     1.15 $\pm$ 0.10   &     0.05 $\pm$ 0.02  &     0.37 $\pm$ 0.06  &     0.20 $\pm$ 0.02    &     0.17 $\pm$ 0.03  \\
 Rpc3L0bH  &     0.18 $\pm$ 0.04   &     2.64 $\pm$ 0.12  &     0.03 $\pm$ 0.02  &     0.01 $\pm$ 0.00    &     0.29 $\pm$ 0.04  \\
 Rpc3L0bS  &     0.99 $\pm$ 0.09   &     8.95 $\pm$ 0.21  &     0.12 $\pm$ 0.04  &     0.02 $\pm$ 0.01    &     0.75 $\pm$ 0.07  \\
 Rpc3L1bH  &     1.52 $\pm$ 0.11   &     0.48 $\pm$ 0.05  &     0.25 $\pm$ 0.06  &     0.28 $\pm$ 0.03    &     0.87 $\pm$ 0.12  \\
 Rpc3L1bS  &     7.02 $\pm$ 0.23   &     1.44 $\pm$ 0.10  &     1.36 $\pm$ 0.10  &     0.69 $\pm$ 0.04    &     2.51 $\pm$ 0.22  \\
 Rpc3LSS1b  &     0.00 $\pm$ 0.00   &     0.00 $\pm$ 0.00  &     0.21 $\pm$ 0.04  &     0.00 $\pm$ 0.00    &     0.09 $\pm$ 0.01  \\
 Rpv2L0b  &     0.14 $\pm$ 0.04   &     0.52 $\pm$ 0.10  &     0.02 $\pm$ 0.02  &     0.01 $\pm$ 0.00    &     0.10 $\pm$ 0.04  \\
 Rpv2L1bH  &     0.54 $\pm$ 0.07   &     0.12 $\pm$ 0.02  &     0.07 $\pm$ 0.03  &     0.34 $\pm$ 0.02    &     0.29 $\pm$ 0.09  \\
 Rpv2L1bM  &     1.40 $\pm$ 0.12   &     0.41 $\pm$ 0.04  &     0.28 $\pm$ 0.06  &     0.53 $\pm$ 0.03    &     0.79 $\pm$ 0.12  \\
 Rpv2L1bS  &     9.91 $\pm$ 0.30   &     1.66 $\pm$ 0.08  &     1.93 $\pm$ 0.15  &     1.79 $\pm$ 0.06    &     2.40 $\pm$ 0.19  \\
 Rpv2L2bH  &     0.53 $\pm$ 0.08   &     0.04 $\pm$ 0.01  &     0.12 $\pm$ 0.03  &     0.48 $\pm$ 0.03    &     0.19 $\pm$ 0.05  \\
 Rpv2L2bS  &     4.80 $\pm$ 0.21   &     0.25 $\pm$ 0.03  &     0.87 $\pm$ 0.11  &     1.52 $\pm$ 0.05    &     1.11 $\pm$ 0.11  \\
 
\hline\hline
\end{tabular}}
\end{table}

\begin{table}[!htb]
\caption{Contributions of particular rare processes to the signal regions, relative to the summed contributions of all these processes.  
}
\label{tab:prompt_rare_contributions}
\def\arraystretch{1.1}
\centering
\resizebox{0.7\textwidth}{!}{
\begin{tabular}{|c|c|c|c|c|c|c|c|}
\hline\hline
         & $VVV$ & $VH$ & 3$t$ & $tZ$ & $t \bar t WW$ & $t WZ$ & $t \bar t WZ$\\\hline\hline
 Rpc2L0bH  & 23\%  & 0\% & 2\% & 3\% & 25\% & 43\%  & 1\% \\
 Rpc2L0bS  & 50\%  & 0\% & 3\% & 15\% & 14\% & 16\%  & 0\% \\
 Rpc2L1bH  & 2\%  & 0\% & 7\% & 4\% & 41\% & 41\%  & 2\% \\
 Rpc2L1bS  & 2\%  & 0\% & 6\% & 3\% & 34\% & 50\%  & 2\% \\
 Rpc2L2bH  & 3\%  & 0\% & 15\% & 4\% & 47\% & 27\%  & 1\% \\
 Rpc2L2bS  & 2\%  & 0\% & 13\% & 2\% & 42\% & 36\%  & 2\% \\
 Rpc2Lsoft1b  & 3\%  & 0\% & 9\% & 0\% & 76\% & 7\%  & 2\% \\
 Rpc2Lsoft2b  & 2\%  & 0\% & 17\% & 4\% & 54\% & 19\%  & 2\% \\
 Rpc3L0bH  & 52\%  & 0\% & 0\% & 3\% & 1\% & 40\%  & 1\% \\
 Rpc3L0bS  & 50\%  & 0\% & 0\% & 4\% & 2\% & 39\%  & 1\% \\
 Rpc3L1bH  & 3\%  & 0\% & 3\% & 3\% & 17\% & 70\%  & 1\% \\
 Rpc3L1bS  & 2\%  & 0\% & 3\% & 7\% & 18\% & 64\%  & 2\% \\
 Rpc3LSS1b  & 25\%  & 0\% & 0\% & 0\% & 0\% & 0\%  & 74\% \\
 Rpv2L0b  & 10\%  & 0\% & 3\% & 0\% & 66\% & 20\%  & 0\% \\
 Rpv2L1bH  & 2\%  & 0\% & 8\% & 0\% & 35\% & 52\%  & 0\% \\
 Rpv2L1bM  & 1\%  & 0\% & 6\% & 0\% & 19\% & 69\%  & 1\% \\
 Rpv2L1bS  & 1\%  & 0\% & 8\% & 4\% & 28\% & 54\%  & 2\% \\
 Rpv2L2bH  & 0\%  & 0\% & 17\% & 0\% & 35\% & 45\%  & 0\% \\
 Rpv2L2bS  & 0\%  & 0\% & 14\% & 4\% & 32\% & 45\%  & 3\% \\
\hline\hline
\end{tabular}}
\end{table}


