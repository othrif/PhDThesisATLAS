Uncertainties associated with the measurement and reconstruction of the 
physics objects used in the analysis (leptons, jets, etc.) must be accounted
 for when interpreting the results.
The systematic uncertainties from the data-driven method have already been 
discussed in Section~\ref{sec:bkg.red}. In fact, these data-driven backgrounds 
are affected by the same systematic uncertainty as in data to which they 
are being compared to. As a result, only systematic uncertainties on 
backgrounds estimated with MC simulation and detector simulation needs to 
be considered. The uncertainties considered for the analysis and 
recommended by the ATLAS SUSY group are:
% TO DO: check references.
\textbf{Jet energy scale (JES)}  \\  
In order to account for inefficiencies in the calorimeter cells
and the varying response to charged and neutral particles passing through 
them, the energies of the jets used in this analysis were corrected. 
The calibration procedure uses a combination of simulation and test beam 
and in situ data ~\cite{Aaboud:2017jcu} with an uncertainty correlated 
between all events.
As a result, all distributions used in the final result are produced 
with the nominal calibration as well as an up and down variation of the 
the jet energy scale (in a fully correlated way) by the 
$\pm 1\sigma$ uncertainty of each nuisance parameter.
A combined version of several independent sources contributing to the 
calibration was used in the analysis 
to reduce the number of nuisance variables in the fitting procedure.


\textbf{Jet energy resolution (JER)} \\ % OK
An extra $\pt$ smearing is added to the jets based on their $\pt$ and $\eta$ 
to account for a possible underestimate of the jet energy resolution 
in the MC simulation. A systematic
uncertainty is considered to account for this defect on the final result. 
The JER in data has previously been estimated by ATLAS in dijet events. %[?]


\textbf{Jet vertex tagger}
The uncertainties account for the residual contamination from pile-up jets 
after pile-up suppression and the MC generator choice
~\cite{ATLAS-CONF-2014-018}.

\textbf{Flavor tagging}
The MC simulation does not reproduce correctly the $b$-tagging, 
charm identification, and light jet reject efficiencies of the detector. 
A \ttbar MC simulation and di--jet measurements are used to derive 
correction factors to be applied to MC simulation
~\cite{ATL-PHYS-PUB-2015-022,ATL-PHYS-PUB-2016-012}.
These correction factors are then varied within
their uncertainties to produce up and down variations.


\textbf{Lepton energy scale, resolution, and Identification efficiencies}
Similar to the case of jets, electrons and muons also have corresponding 
energy scale and resolution systematic uncertainties. Corrections are 
also applied to take into account any variations in the identification 
efficiency in the detector and its simulation~\cite{ATLAS-CONF-2016-024,Aad:2016jkr,ATLAS-CONF-2016-024}.

\textbf{\met\ soft term uncertainties}
The main effect come from the hard object uncertainties (most notably JES and 
JER) that are propagated to the $\met$.

\textbf{Pileup re-weighting}
This uncertainty is obtained by re-scaling the $\mu$ value in data by 1.00 and 1/1.18, 
covering the full difference between applying and not-applying the nominal $\mu$ correction of 1/1.09, 
as well as effects resulting from uncertainties on the luminosity measurements, which are expected to dominate.

\textbf{Luminosity}
The integrated luminosities in data corresponds to 3.2 \ifb and 32.9 \ifb 
for 2015 and 2016 respectively. The combined luminosity error for 2015 and 2016 is 3.2\%, assuming partially correlated uncertainties in 2015 and 2016.


\textbf{Trigger}
To account for any differences between the trigger efficiency in simulation 
and data, corrections factors are derived to correct for them. 
Uncertainties on the correction factors as well as inefficiencies 
related to the plateau of the trigger are propagated to the final result.

The uncertainty on the beam energy is neglected. 
All the experimental uncertainties are applied also on the signal samples when computing exclusion limits on SUSY scenarios. 


All of these uncertainties are fed into the fitting and limit setting 
machinery by treating them as uncorrelated uncertainties, and thus 
treated independently. 
