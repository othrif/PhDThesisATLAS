While I have not found evidence for supersymmetry in the work presented in this dissertation, 
my search for it led me to meet some incredible and talented people, many of whom became close
friends and mentors to me during this exciting journey. 
I am deeply grateful for the support of so many that  
constantly encouraged me to think critically and independently, and to take initiatives
while entrusting me with important responsabilities early on in my training.
%I became a better scientist 
During this process, I learned to appreciate the elegance in the world of elementary particles 
and their interactions, and to realize that the closer we looked the more there was to see. 
%There is a very long list of people I would like to thank. 

First and foremost, thanks to my advisor, Brad Abbott, for his continuous support and encouragment over these past five years. 
Brad gave me enough freedom and space to pursue the projects I felt most pationate about and at the same time be always
there to guide me through all the decisions I had to make during my PhD. I could always rely on his perspective to set realistic 
expectations and to navigate the waters of a large collaboration such as ATLAS, and for that I will be always grateful.

Thanks to the faculty of the High Energy Experimental Physics (HEP) group at OU%.: Michael Strauss, Patrick Skubic, Phillip Gutierrez, and John Stupak. 
%I would like to thank you 
for financially supporting my research. I have enjoyed interacting with all of you and getting your feedback on the numerous 
presentations I gave to the group. 
I would like to thank my doctoral committee for evaluating my progress and reviewing my work through the years.
I also would like to thank the OU HEP students and Postdocs that I overlapped with;
The upperclassmen, Callie Bradley, David Bertsche, and Scarlet Norberg,  who never hesitated in sharing their experiences throughout their PhD and after they graduated.
Students in my year, David Shope for the many scuba diving adventures we had and Yu-Ting Shen for our work on supersymmetry.
Our postdocs, Muhammad Alhroob  made my stay at CERN more enjoyable and Muhammad Saleem who was one of the first to offer his advice.

Thanks to the whole Argonne National Laboratory HEP group. The one year I spent with the group was one of the  best periods of my PhD experience.
I would like to particularly thank Bob Blair, Jeremy Love, and Jinlong Zhang in working together to make the evolution of the ATLAS 
Region of Interest Builder (RoIB) a success. 
We replaced the two-decade-old RoIB that ``discovered the Higgs'', as Jeremy likes to say, with a modern solution.
The new system integration and commissioning in ATLAS went so smoothly that barely anyone noticed, a testament of a job well done,
keeping in mind that a failure of this system would have hindered the data-taking ability of ATLAS.
I would like to thank Sergei Chekanov with whom I made my first contribution to an ATLAS paper.
I am particularly grateful for having worked with Sasha Paramonov in a physics measurement and a search for supersymmetry. 
Sasha would always find simple ways to convey his ideas and take the time to answer my many questions. 
I am thankful to James Proudfoot who continued offereing his advice throughout my stay at Argonne and afterwards.

Thanks to William Vazquez and Wainer Vandelli in their invaluable help while developing the RoIB software.
I still remember my excitment one Friday afternoon at CERN when we finally figured out how to get the RoIB to operate at 
over 100 kHz rate. It was one of the highlights of the project.

Thanks to my collaborators in the ``SS/3L'' supersymmetry seach group. In particular, I am thankful to 
Otilia Ducu, Julien Maurer, and Ximo Poveda for the countless hours of debugging, checking and cross-checking results, and the 
long Skype conversations. I have not only learned from you the intricate details of searching for rare signals and estimating 
the difficult detector backgrounds but also crisis management skills with the three conference rushes we had to endure. 


thanks for the good times in the mountains, on the slopes, aboard boats, under water, and at the office.

%Mak Hussein, Heidar Albandar, Salah Chafik, and many others
%Mohamed Fnine, 

Ximo Poveda, Julien Maurer, Otilia Ducu

Peter Tornambe, Fabio Cardillo

physics was more fun with your collaboration

first ATLAS conference note for ICHEP2016, second ATLAS SUSY paper with the full 2015 and 2016 data.

thanks for the good times 

I treasure all the time I spent with many friends at CERN and the Geneva area,
either be it summitting peaks in the alps, climbing, skiing on the many world class resorts in the area, 
sailing and diving in lake Geneva (or lake Leman), and taking several diving trips to lakes in the area , 
the mediternean, and the red sea.

I shared my office with Jeremy Love and Bing with whom I had such a pleasant time. 

The list is very long of the 
Edoardo Farina
Ondrej Hladik
Artem Basalaev
Giulia Ripellion
Sameed Mohammad
Nicolas Hoffmann
Philipp Millet
Guy Crockford
Denis Regat


Family and friends in Morocco. 

Abdelfatah Mouatani

Finally, I want to thank the most important people in my life, my parents, Chouaib and Latifa, and my sister, Hafsa, 
your endless support and love during this adventure kept me always going. This dissertation is dedicated to you.

