% why study elementary particles
% the sucess of the standard model
% the need for bsm: supersymmetry
% the tool to do it: LHC and ATLAS 
% the challenge


Of what is the universe made? A question that has intrigued the human curiosity since the dawn of time. 
Today, we are confident that we do not know the answer to this question. 
Yet, attempts to answer it has gone through many iterations where it was deemed solved at times 
and unreachable at others. 
Over two thousands years ago, the ancient Greeks  postulated that all is made of Earth, Air, Fire and Water. 
Fastfoward to the end of the 19th century, Mendeleev and others made the astonishing remark that by organizing the 
relative atomic masses of chemical elements in an ascending order, elements with similar chemical properties followed a pattern
leading to the formulation of the periodic table of elements. The importance of this achievement stems from the
predictive power of the periodic table leading to the anticipation of new elements and their properties that were later discovered.
However, the table lacked compactness and necessitated a more fundamental underlying structure that could 
connect the different elements. At the turn of the century, several important discoveries established 
the existence of the atom and its constituents: electrons bound to the nucleus via the electromagnetic force.
The substructure behind the organization of the periodic table were understood
with the discovery of the proton and neutron that were ``glued'' togehter by 
the strong nuclear force to form the atomic nucleus. 
Quantum ideas were applied to the atom offering a quantitative description of the origin of structure in atoms and molecules, including the chemical elements 
and their properties.
The decades that followed refined our understanding of the composition of matter through a series of experimental results.
Several 

The radioactive $\beta$ decay of nuclei via the weak interactions. A pletor of new particles with properties similar to the proton 
and neutron but heavier start being observed.
A new pattern start emerging simular to the Mandelev. 
the discovery of hadrons. 
electron and neutrino to form fermions with an intrsici property like electric chage called spin 1/2.

 

Electrons are bound to the atomic nucleus via the electromagnetic force
and the nucleus itself is composed of protons and neutrons ``glued'' togehter by the strong nuclear force.



After decades of development, particle physics is the most ambitious and organized attempt to answering the question.
