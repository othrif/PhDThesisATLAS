% why study elementary particles
% the sucess of the standard model
% the need for bsm: supersymmetry
% the tool to do it: LHC and ATLAS 
% the challenge

% How did we get here?

Of what is the universe made? A question that has intrigued the human curiosity since the dawn of time. 
Today, we are confident that we do not know the answer to this question. 
Yet, attempts to answer it has gone through many iterations where it was deemed solved at times 
and unreachable at others. 
Over two thousands years ago, the ancient Greeks  postulated that all is made of Earth, Air, Fire and Water. 
The idea behind this picture is to give a description of the world we observe from a compact number of building blocks
that follow simple principles. Obvisoulsy, the ancients got it wrong.
Fastfoward to the end of the 19th century, Mendeleev and others made the astonishing remark that by organizing the 
relative atomic masses of chemical elements in an ascending order, elements with similar chemical properties followed a pattern.
The periodic table of elements was born. 
The predictive power of the periodic table led to the anticipation of new elements and their properties that were later discovered.
However, the table lacked compactness and necessitated a more fundamental underlying structure that could 
connect the different elements. At the turn of the century, several important discoveries established 
the existence of the atom and its constituents. Electrons are bound to the nucleus via the electromagnetic force and 
protons and neutrons are ``glued'' togehter by the strong nuclear force (or strong force) to form the atomic nucleus.
These elements formed the underlying substructure that explained qualitatively the systematic organization of the periodic table.
Moreover, quantum ideas were applied to the atom offering a quantitative description of the origin of structure in atoms and molecules, 
including the chemical elements and their properties.
The decades that followed refined our understanding of the composition of matter through a series of experimental results.
By studying the collisions of protons and neutrons in the 50's and 60's, we came to uncover a plethora of new particles from the same 
family as the proton and neutron which interacted via the strong force, called \textit{hadrons}.
These particles could not be all elementary, a classic replay of the argument that atoms were composite based on Mendeleev's table.
A new layer of structure was unfolded to reveal the existence of \textit{quarks}, \textit{elementary} 
particles that interact via the strong force and make up all hadrons. 
Another puzzle of the 20th century was the continuous energy spectra in the radioactive $\beta$ decay of nuclei which pointed to the 
existence of neutrinos to remedy the energy conservation law in the decay. This process proceeds via the weak nuclear force 
repsonsible for the decay of atoms and radiation.
It turns out that electrons and neutrinos had other relatives regarded as \textit{elementary}, referred to as \textit{leptons}.

% Today's success

The physics of elementary particles became the most ambitious and organized attempt to answer the question of what the universe is 
made out of. 
The picture we have today was achieved through a mixture of both theorertical insight and experimental input to arrive at 
quarks and leptons as the building blocks of matter interacting via the strong, weak, and electomactic forces.
A theoretical framework was constructed to translate all the developments of particle physics in a quantitative 
calculational tool that became known as the \textit{Standard Model of particle physics} (SM). 
The SM entails two classes of particles. 
The quarks and leptons are collectively referred as fermions and
share a spin of $s=\frac{1}{2}$, an intrinsic property of elementary particles.
The interaction forces are mediated by elementary particles with an integer spin, collectively called bosons.



% Limitation


% my work
