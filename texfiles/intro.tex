% why study elementary particles
% the sucess of the standard model
% the need for bsm: supersymmetry
% the tool to do it: LHC and ATLAS 
% the challenge

% How did we get here?

Of what is the universe made? A question that has intrigued the human curiosity since the dawn of time. 
Today, we are confident that we do not know the answer to this question but a lot of progress has been made
The overeaching aim was to reduce the diversity of the physicsal phenomena to a compact set of constituents
and a unified set of principles.
% Unification and symmetry
%Yet, attempts to answer it has gone through many iterations in order to find a compact set of principles
%that offers a syntheis of the phenomena we observe in the world around us. 
%would describe the world we observe.
%where it was deemed solved at times 
%and unreachable at others. 
Over two thousands years ago, the ancient Greeks  postulated that all is made of Earth, Air, Fire and Water. 
The idea behind this picture is to give a description of the world we observe based on a compact set of principles 
Obvisoulsy, the ancients got it wrong.
Fastfoward to the end of the 19th century, Mendeleev and others made the astonishing remark that by organizing the 
relative atomic masses of chemical elements in an ascending order, elements with similar chemical properties followed a pattern.
The periodic table of elements was born. 
The predictive power of the periodic table led to the anticipation of new elements and their properties that were later discovered.
However, the table lacked compactness and necessitated a more fundamental underlying structure that could 
connect the different elements. At the turn of the century, several important discoveries established 
the existence of the atom and its constituents. The mass of the atom lies in its nucleus with electrons bound to it via the 
electromagnetic force. The nucleus itself is formed from 
protons and neutrons that are ``glued'' togehter by the strong nuclear force (or strong force).
These elements formed the underlying substructure that explained qualitatively the systematic organization of the periodic table.
Moreover, quantum ideas were applied to the atom offering a quantitative description of the origin of structure in atoms and molecules, 
including the chemical elements and their properties.
The decades that followed refined our understanding of the composition of matter through a series of experimental results.
By studying the collisions of protons and neutrons in the 50's and 60's, we came to uncover a plethora of new particles from the same 
family as the proton and neutron which interacted via the strong force, called \textit{hadrons}.
These particles could not be all elementary
\footnote{Elementary particles refer to  particles that cannot be decomposed into further constituents.}
, a classic replay of the argument that atoms were composite based on Mendeleev's table.
A new layer of structure was unfolded to reveal the existence of \textit{quarks}, elementary 
particles that interact via the strong force and make up all hadrons. 
Another puzzle of the 20th century was the continuous energy spectra in the radioactive $\beta$ decay of nuclei which pointed to the 
existence of neutrinos to remedy the energy conservation law in the decay. This process proceeds via the weak nuclear force 
repsonsible for the radioactivity and nuclear fusion, the process that powers the stars.
It turns out that electrons and neutrinos had other relatives referred to as \textit{leptons}.
Similarly, several other quarks were discovered.%, with the latest top quark discovered in 1995. 
The quarks and leptons are collectively referred as fermions and
share a half integer spin, an intrinsic property of elementary particles.
The strong, weak, and electromagnetic interactions are mediated by 
gluons, $W$ and $Z$ bosons, and photons. These are 
elementary particles with an integer spin, 
collectively called bosons. Apart from gravity, all aspects of daily life can 
be described in terms of these interactions.
%The gravitational force is so weak that it only acts at the macroscopic scale.
The lastest discovery happened in 2012 of a new boson, the Higgs boson, 
that allows the quarks and leptons and the $W$ and $Z$ bosons to 
acquire mass.

%A success story
The physics of elementary particles became the most ambitious and organized attempt to answer the question of what the universe is 
made out of. 
Through a mixture of both theorertical insight and experimental input, we now 
know that everything we see in our daily life is formed from quarks and leptons
that interact via the strong, weak, and electromagnetic forces. 
The forms of these forces are determined from basic princples of 
symmetry and invariance.
As a result, a theoretical framework was constructed to synthesize all these 
developments in a quantitative 
calculational tool that became known as the \textit{Standard Model of particle physics} (SM). 
The only inputs needed by the SM are the interaction strengths of the forces and quark and lepton masses to make 
very accurate predictions about the behavior of elementary particles.
For example, the magnetic moment of the electron calculated in the SM is found to agree with the experimental measurement to nine(?) decimal places.
Over the past 30 years, the SM has been vigorously tested across the strong, weak, and electromagnetic interactions. 
It came out triumphant every time. Yet, we know it is not the complete story. 

% the dark side
In 1933, an observation of the Coma Cluster by Zwicky suggested that the galaxies in the cluster were moving too fast to be explained 
by the luminous matter present. The same observation was repeated when looking at the rotation speeds of individual galaxies which 
suggested a dark component of mass, Dark Matter. 
Several independent measurements established that 
not only dark matter is not made out of baryons but it is more 
abundant. 
For example, anistropies in the cosmic microwave background, a radiation 
left over from the Big Bang, were consistent with 
quantum fluctuations from an inflationary epoch \cite{Hu:2001bc,2009AIPC}. 
These fluctuations encoded details about the density of matter 
in the form of 
cosmological parameters as they travelled through space and time to reach 
our experiments.
The astonishing conclusion was that the universe has nearly five times 
as much dark matter as ordinary matter \cite{Bertone:2004pz}.
The supernovae surveys gave direct evidence for an accelerating universe
 \cite{Perlmutter:1998np},
% From the rate of expansion, we can play the scenario back in time and 
%deduce that the universe was compacted on itself some 14 billion years ago.
% the explosive erruption from that state is referred to as the Big Bang
a view that was cemented by the measurement of cosmological parameters
\cite{Adam:2015rua,Ade:2015xua}
which led to the startling discovery that most of the energy density of 
the Universe is in the 
form of an unkown negative-pressure, called dark energy \cite{Scranton:2003in}.
There is an extensive program of experiments 
which will probe the dark energy.% and is beyond the scope of this work. 
%Instead, we turn to dark matter. 
Astrophysics and cosmology told us about 
the existence of dark matter and measured its density to a remarkable 
precision. Particle physics holds the hope to uncover what dark matter is.
In short, all experimental evidence are consistent with a universe 
constructed of 
\begin{itemize}
\item baryons (everyday matter): $\sim 5\%$ 
\item dark matter: $\sim 20\%$ 
\item dark energy: $\sim 75\%$ 
\item neutrinos, photons: a tiny fraction
\end{itemize}

Today, we are in front of many puzzles related to our view of the universe.
Everything we know of, including all particles present in the standard model, 
constitutes only 5\% of the energy budget of the universe. 
The universe is also predominatly composed of matter as opposed to 
anti-matter even though at the start of the universe, they were in equal 
amounts%, baryon assymetry
. The standard model describes the content of 
everyday matter 
and how it interacts but without telling us why it is that way.
Moreover, the standard model only describes these phenomena 
up to an energy scale of $\mathcal{O}\left(100\right)$ \GeV, weak scale.
Beyond this scale lies the realm of phenomena not described by the standard 
model that extend all the way to the Planck scale of 
$\mathcal{O}\left(10^{19}\right)$ \GeV, the limit where the known laws of 
physics apply. 
There is no mechanism to generate mass for 
neutrinos in the standard model. Last but not least, the standard model 
does not incorporate gravity, the fourth fundamental force.
The SM has many limitations since it is unable to account for observed 
features in the universe. 
Thus, there is a need for a theory beyond the standard model.

One of the most prominent extensions of the standard model, 
that addresses many of the shortcomings mentioned above, is a theory based on 
a new symmetry, called supersymmetry.
This symmetry is between the matter particles, fermions, and particles whose
exchange mediates the forces, bosons. Our current description of the world
treats fermions and bosons differently. Supersymmetry puts forward the idea
that fermions and bosons can be treated in a fully symmetric way. 
In other words,
if we exchange fermions and bosons in the equations of the theory, the 
equations will still look the same. An immediate consequence of the theory
is that every standard model particle will have a ``superpartner''.
As a result, we can design experiments to search for these 
supersymmetric particles. The work presented in this dissertation is about the search for supersymmetric particles with a specific signature.
The many benefits of supersymmetry will be discussed later but here it is worth 
mentioning two important features of the theory: it addresses the question 
of why there is a huge gap between the Planck scale and the weak scale, 
and it provides a dark matter candidate particle. 

Now that we understand what we are trying to do, it is time to address 
the question of how to do it.
There are two challenges related to studying elementary particles: 
distance and lifetime.


In order to see an object, the wave must scatter off 
the object.
light acts as a wave.  This becomes harder as the object gets smaller. 
In fact, the wavelenght of light must be smaller than the object being 
probed. For this reason it is hard for us to see atoms.
In order to see the atom, we use not just light but also particles, 
since they can also be represented by waves.
Particulary charged particles have the advantage of being able to be 
manipulated, accelerated and steared to reach high energies.

This allows us to probe ever shorter distances.
particles have wavelike character, and conversely, waves can act as 
a bundle of particles, known as quanta. 
An electromagnetic wave is a bust of quanta, photons.
energy and wavelenght can be realted by 
\[
E = \frac{hc}{\lambda}
\]
where $hc \sim 10^{-6} eV m$, 1 \ev corresponds to $10^{-6}$ m and 1 \TeV
corresponds to $10^{-18}$ m, our current limit.

To probe deep within atoms, the technique is to use particles, such as protons,
and speed them in electric fields. The higher their speed, the greater their 
energy and momentum and the shorter their associated wavelenght.
To be able to resolve very small distances, we have to speed up the particles
by giving them more and more energy to get to wavelenght smaller than the 
distance we desire to probe as far as our technology allows.
Today, we are in the realm of \TeV beam energy which allows us to probe 
matter at distances smaller than $10^{-18}$ m.
It is possible that electrons and quarks have some structure which are 
beyond our ability to resolve in experiment. For this reason, we consider 
them as not having any deeper structure, pointlike objects. All the 
developement that we made 
describes phenomena happening at distances larger than about $10^{-18}$ m.

There is a paradox between needing to probe very short distances, and thus 
smaller wavelenghts which require higher energy particles which can only 
be reached by bigger accelerators to probe the smallest distances.

First, the human eye can resolve pieces of dust up to $10^{-4}$ m to 
$10^{-5}$ m. 
Light is an electromagnetic wave with the visible wavelength, as the 
rainbow, between $10^{-6}$ m to $10^{-7}$ m. 
Atoms are in the order of $10^{-10}$ m which makes them beyond the reach of 
our normal vision.
Its central nucleus is about $10^{-14}$ m to $10^{-15}$ m
(relative size of nucleus to atom is 1/10000).
Individual protons and neutrons are about $10^{-15}$ m
composed of yet small parts, quarks.
The limit of our current resolution is $10^{-18}$ m
If quarks and electrons had any intrinsic size, it is no 
larger than $10^{-18}$
(relative size of quark to proton is 1/10000).

It is necessary to expand our senses using instruments.
Light microscopes can reveal what things are like down to 
$10^{-6}$ m, the scale of 
bacteria and molecules. 
Electron microscopes can go down to $10^{-9}$ m, separation between two 
atoms in a solid. 
A special type of microscope is needed to probe smaller distances, which 
is known as a particle accelerator.


Over the last century, beams of particles have been used to uncover the inner
structure of matter that was presented so far. 
Initially, beams orginated from phenomena that were 
already occuring, such as alpha and beta particles, coming from radioactive 
decays, and cosmic rays.
Cosmic rays were very energetic ??, however, they came at random and 
with a low intensity. To remedy these problems, 
high energy particle accelerators were used to deliver high intensity beams of 
electrons, protons, and other particles under controlled conditions.
It is by combining information from the interaction of matter with these 
different particles in the laboratory that we were able to build the model of 
elementary particles and their interactions.
These acclerators rely on the basic concept that 

The basic concept is to collide two sufficiently energetic particles 
which in turn creates new particles according to Einstein's 

By accelerating charged particles using electric fields to a fraction 
of the speed of light then smashing them to a target or to one another, 
we can reveal the deep structre of matter. 
they not only have shown quarks but also other exotic form of matter 
that is not prevalent today.
In fact, these exotic forms were abudant the first moment of the Big Bang. 
In other words, by colliding high energy particles, it is possible to 
recreate momentarly conditions similar to those of the universe when it 
was newly born?.
Thus study 
So the goal of high energy physics, a term often times uses 
synonumous to particle physics, is to reveal the deep inner structure 
of matter and to understand the exotic forms that were predominant 
in the early universe, and also search for new froms of matter 
predicted by our theories. One of the forms of matter we are searching for 
is evidence for supersymmetric particles, the subject of this dissertation.

Einstein's famous equation $E = mc^2$ tells us that energy can be exchanged 
for mass, and vice versa, the exchange rate being $c^2$, the square of the 
velocity of light. For example, an electron has a mass of 0.5 \MeV 
and can only be created in pair of particle and its antiparticle counterpart, 
1 \MeV of energy is needed for an electron-positron pair production.
Modern accelerators can generate energy in the \TeV scale, an energy that 
was present about a billionth of a second of the Big Bang.
At such energies, particles and antiparticles were created, including 
exotic forms no longer common today.
Most of them died out very shortely thereafter, including exotic forms of 
matter no longer common today, producing radiation and more of the basic 
particles such as electrons and quarks that make up most of the matter today.


%For this endeavour, we require two key instruments: accelerators and detectors.
The Large Hadron Collider (LHC) at CERN is the world's most energetic 
particle accelerator producing the highest intensity beam of protons 
travelling at 99.99993\% the speed of light. 
colliding beams of proton beams at a center of mass 
energy of 13 \TeV. %7 times more energetic than the tevatron. 
The ring is 27 km in circumference special magnets steer two counter-rotating 
beams of protons to meet head on.
the benefit of collding head-on is that the energy is spent on the interaction
between them. the debries of this collision fly off in all directions and 
the energy is redistributed with it.
This brings us to the second component, how to detect the debris of these 
collisions that happen very quickly.
When two protons collide, many new paerticles are 
created out of the vacuum and the signals of interest are extremely rare, 
like looking for a needle in a haystack.
For example, the higgs boson is produced once in 20 million million collisions.
In more practical terms, a higgs boson might appear once a day in each 
of the LHC experiments.
The challenge is to recognize one higgs event and record it to tape 
out of hundred thousand billion other collions.
A big computational challenge is to be able to recognize this event 
and record it.
If they exist, supersymmetric particles this discertaition is looking for are 
even more rare. 

needle-like bunches of protons would pass through each other at the heart of 
the detector 40 million times per second, or spaced by 25 nanoseconds.
each time they cross there are on average XX collisions, leading to 
about a billion collision per second that needs to be processed.
Data storage and processing capabilities aloow for about 1000 carefully 
selected events per second to be recorded for analysis.
Moreover, the total amount of data generated amounts to tens of 
petabytes per year that needs to be accessed and processed by scientits 
around the world. The largest computing grid infrastructure for data 
distribution, processing, and storage was developed which include data 
storage and 
computing resources at centers all around the world working in unison 
with the CERN on-site computing facilities.

even though there are millions of particles in each bunch, the particles 
were thinly dispersed so interactions were very rare. 
An interesting collision, ``event'', only occured once every..., 
The challenge was to identify and collect the interesting events, and not 
to miss them while recording something more mandane.
A multilevel real time system would decide on the events in real time
weither something interesting has occured in 25 ns. If it had, the process
of reading out and combining information from all the detector would begin.

cylindrical detector surrounding the interaction point exploring the new 
energy region and looking for both expected and unexpected effects.

The ATLAS detector is five storeys high (20 m) and able to measure particle 
tracks down to 0.01m.
the entire detector is designed to record debris from collision that occur
a billion time each second.
by contrast, the early days of particle detector such as cloud chambers could 
record one event a minute, or even bubble chambers at one event per second.

Among the debris produced in collisiions are phenomena , at energies 
far exceeding any particle accelerator, some unexpected phenomea or excepceted.

The particles generated in high energy collisions are extremely short lived.
Every day matter is formed of electrons and protons which have lower limit 
on their lifetime of $10^{22}$ years and $10^{31}$ years, respectively, 
much longer than the lifeteime of the universe of about $10^{10}$ years.
A free neutron decays in $10^{3}$ seconds, but stable in nuclei.
However, all other particles have lifetimes in the range of 
$10^{-24} - 10^{-6}$ seconds where most of them live less than $10^{-20}$ 
seconds. In collision of two energetic particles, new particles 
are created. It is important to understand that these are particles 
are not components of the decaying particle but rather created at the moment 
of the decay.


Many years of experience from Tevatron collider at Fermilab\cite{tevatron}, 
HERA at DESY\cite{hera}, and LEP\cite{lep} at CERN improved the understanding of the complex strong interaction dynamics.

It is one of the these particle detectors, the ATLAS detector, 
that supplied the data analyzed in this dissertation. 
The ATLAS detector is the largest experiment ever built.
It is an international endavour of more than 3000 physicists coming from 
178 laboratories and universities from around the world.


This dissertation will expand upon the concepts presented in the 
introduction and describe in detail the search for supersymmetric particles
completed by the author.
First, the motivation behind the work will begin with an overview of the 
Standard Model of particle physics and Supersymmetry in 
Chapter~\ref{chap:theory} followed by the design of the ATLAS detector 
at the LHC in Chapter~\ref{chap:exp}.
The Region of Interest Builder that processes every event recorded by ATLAS 
is covered in Chapter~\ref{chap:roib}.
The detailed description of the search starts in Chapter~\ref{chap:strategy}
covering the basic analysis strategy and the supersymmetric models considered.
The most important part of the analysis is the estimation of detector 
backgrounds with novel techniques developed by the author and covered in 
Chapters~\ref{chap:fake} and ~\ref{chap:bkg}. The statistical 
methodology and interpretation of the results is presented in 
Chapters~\ref{chap:stat} and ~\ref{chap:res}.
This analysis represents an important search for supersymmetric particles 
with the early dataset collected by ATLAS at a new energy.
The strength of the search lies in exploring regions of the parameter space 
with a small mass difference between the supersymmetric particles, regions
that are difficult to probe with other searches for new phsyics.
