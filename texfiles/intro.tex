% why study elementary particles
% the sucess of the standard model
% the need for bsm: supersymmetry
% the tool to do it: LHC and ATLAS 
% the challenge

% How did we get here?

Of what is the universe made? A question that has intrigued the human curiosity since the dawn of time. 
Today, we are confident that we do not know the answer to this question but a lot of progress has been made
The overeaching aim was to reduce the diversity of the physicsal phenomena to a compact set of constituents
and a unified set of principles.
% Unification and symmetry
%Yet, attempts to answer it has gone through many iterations in order to find a compact set of principles
%that offers a syntheis of the phenomena we observe in the world around us. 
%would describe the world we observe.
%where it was deemed solved at times 
%and unreachable at others. 
Over two thousands years ago, the ancient Greeks  postulated that all is made of Earth, Air, Fire and Water. 
The idea behind this picture is to give a description of the world we observe based on a compact set of principles 
Obvisoulsy, the ancients got it wrong.
Fastfoward to the end of the 19th century, Mendeleev and others made the astonishing remark that by organizing the 
relative atomic masses of chemical elements in an ascending order, elements with similar chemical properties followed a pattern.
The periodic table of elements was born. 
The predictive power of the periodic table led to the anticipation of new elements and their properties that were later discovered.
However, the table lacked compactness and necessitated a more fundamental underlying structure that could 
connect the different elements. At the turn of the century, several important discoveries established 
the existence of the atom and its constituents. The mass of the atom lies in its nucleus with electrons bound to it via the 
electromagnetic force. The nucleus itself is formed from 
protons and neutrons that are ``glued'' togehter by the strong nuclear force (or strong force).
These elements formed the underlying substructure that explained qualitatively the systematic organization of the periodic table.
Moreover, quantum ideas were applied to the atom offering a quantitative description of the origin of structure in atoms and molecules, 
including the chemical elements and their properties.
The decades that followed refined our understanding of the composition of matter through a series of experimental results.
By studying the collisions of protons and neutrons in the 50's and 60's, we came to uncover a plethora of new particles from the same 
family as the proton and neutron which interacted via the strong force, called \textit{hadrons}.
These particles could not be all elementary
\footnote{Elementary particles refer to  particles that cannot be decomposed into further constituents.}
, a classic replay of the argument that atoms were composite based on Mendeleev's table.
A new layer of structure was unfolded to reveal the existence of \textit{quarks}, elementary 
particles that interact via the strong force and make up all hadrons. 
Another puzzle of the 20th century was the continuous energy spectra in the radioactive $\beta$ decay of nuclei which pointed to the 
existence of neutrinos to remedy the energy conservation law in the decay. This process proceeds via the weak nuclear force 
repsonsible for the radioactivity and nuclear fusion, the process that powers the stars.
It turns out that electrons and neutrinos had other relatives referred to as \textit{leptons}.
Similarly, several other quarks were discovered.%, with the latest top quark discovered in 1995. 
The quarks and leptons are collectively referred as fermions and
share a half integer spin, an intrinsic property of elementary particles.
The strong, weak, and electromagnetic interactions are mediated by elementary particles with an integer spin, 
collectively called bosons. Apart from gravity, all aspects of daily life can be described in terms of these interactions.
The lastest discovery happened in 2012 of a new boson, the Higgs boson, as the quanta of a field that generates masses for the weak 
bosons.

%A success story
The physics of elementary particles became the most ambitious and organized attempt to answer the question of what the universe is 
made out of. 
The picture we have today was achieved through a mixture of both theorertical insight and experimental input to arrive at 
fermions  as the building blocks of matter and bosons as the carriers of the strong, weak, and electomactic forces.
A theoretical framework was constructed to synthesize all the developments of particle physics in a quantitative 
calculational tool that became known as the \textit{Standard Model of particle physics} (SM). 
The only inputs needed by the SM are the interaction strengths of the forces and quark and lepton masses to make 
very accurate predictions about the behavior of elementary particles.
For example, the magnetic moment of the electron calculated in the SM is found to agree 
with the experimental measurement to nine(?) decimal places. 
Over the past 30 years, the SM has been vigorously tested across the strong, weak, and electromagnetic interactions. 
It came out triumphant every time. Yet, we know it is not the complete story. 

In 1933, an observation of the Coma Cluster by Zwicky suggested that the galaxies in the cluster were moving too fast to be explained 
by the luminous matter present. The same observation was repeated when looking at the rotation speeds of individual galaxies which 
suggested a dark component of mass, Dark Matter. There is now strong evidence from several sources that dark matter is not made 
from baryons. 
For instance, the most precise measurement of the Dark Matter abundance comes from the Cosmic Microwave Background Radiation, radiation left over 
from the Big Bang which provides a wealth of information about cosmological prameters.
From a fit to data the dark matter density is $\Omega DM h^2 = 0.106\pm0.008$ and the combined mass parameter of 
baryons give $\Omega baryons h^2 = 0.0223 ^{0.0007}{0.0009}$. In other words, the universe has nearly six times as much Dark 
Matter as ordinary matter.  



% My work
The SM has some limitation since it is unable to account for observed features in the universe sush as the existence of dark matter, 
the baryon asymmetry, and neutrino masses. It is in this context that I conducted my research.

Many years of experience from Tevatron collider at Fermilab, HERA at DESY, and LEP at CERN improved the understanding of the 
complex strong interaction dynamics 


% complex detectors
