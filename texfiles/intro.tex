% why study elementary particles
% the sucess of the standard model
% the need for bsm: supersymmetry
% the tool to do it: LHC and ATLAS 
% the challenge

% How did we get here?

Of what is the universe made? A question that has intrigued the human curiosity since the dawn of time. 
Today, we are confident that we do not know the answer to this question but a lot of progress has been made
The overeaching aim was to reduce the diversity of the physicsal phenomena to a compact set of constituents
and a unified set of principles.
% Unification and symmetry
%Yet, attempts to answer it has gone through many iterations in order to find a compact set of principles
%that offers a syntheis of the phenomena we observe in the world around us. 
%would describe the world we observe.
%where it was deemed solved at times 
%and unreachable at others. 
Over two thousands years ago, the ancient Greeks  postulated that all is made of Earth, Air, Fire and Water. 
The idea behind this picture is to give a description of the world we observe based on a compact set of principles 
Obvisoulsy, the ancients got it wrong.
Fastfoward to the end of the 19th century, Mendeleev and others made the astonishing remark that by organizing the 
relative atomic masses of chemical elements in an ascending order, elements with similar chemical properties followed a pattern.
The periodic table of elements was born. 
The predictive power of the periodic table led to the anticipation of new elements and their properties that were later discovered.
However, the table lacked compactness and necessitated a more fundamental underlying structure that could 
connect the different elements. At the turn of the century, several important discoveries established 
the existence of the atom and its constituents. The mass of the atom lies in its nucleus with electrons bound to it via the 
electromagnetic force. The nucleus itself is formed from 
protons and neutrons that are ``glued'' togehter by the strong nuclear force (or strong force).
These elements formed the underlying substructure that explained qualitatively the systematic organization of the periodic table.
Moreover, quantum ideas were applied to the atom offering a quantitative description of the origin of structure in atoms and molecules, 
including the chemical elements and their properties.
The decades that followed refined our understanding of the composition of matter through a series of experimental results.
By studying the collisions of protons and neutrons in the 50's and 60's, we came to uncover a plethora of new particles from the same 
family as the proton and neutron which interacted via the strong force, called \textit{hadrons}.
These particles could not be all elementary
\footnote{Elementary particles refer to  particles that cannot be decomposed into further constituents.}
, a classic replay of the argument that atoms were composite based on Mendeleev's table.
A new layer of structure was unfolded to reveal the existence of \textit{quarks}, elementary 
particles that interact via the strong force and make up all hadrons. 
Another puzzle of the 20th century was the continuous energy spectra in the radioactive $\beta$ decay of nuclei which pointed to the 
existence of neutrinos to remedy the energy conservation law in the decay. This process proceeds via the weak nuclear force 
repsonsible for the radioactivity and nuclear fusion, the process that powers the stars.
It turns out that electrons and neutrinos had other relatives referred to as \textit{leptons}.
Similarly, several other quarks were discovered.%, with the latest top quark discovered in 1995. 
The quarks and leptons are collectively referred as fermions and
share a half integer spin, an intrinsic property of elementary particles.
The strong, weak, and electromagnetic interactions are mediated by 
gluons, $W$ and $Z$ bosons, and photons. These are 
elementary particles with an integer spin, 
collectively called bosons. Apart from gravity, all aspects of daily life can 
be described in terms of these interactions.
%The gravitational force is so weak that it only acts at the macroscopic scale.
The lastest discovery happened in 2012 of a new boson, the Higgs boson, 
that allows the quarks and leptons and the $W$ and $Z$ bosons to 
acquire mass.

%A success story
The physics of elementary particles became the most ambitious and organized attempt to answer the question of what the universe is 
made out of. 
Through a mixture of both theorertical insight and experimental input, we now 
know that everything we see in our daily life is formed from quarks and leptons
that interact via the strong, weak, and electromagnetic forces. 
The forms of these forces are determined from basic princples of 
symmetry and invariance.
As a result, a theoretical framework was constructed to synthesize all these 
developments in a quantitative 
calculational tool that became known as the \textit{Standard Model of particle physics} (SM). 
The only inputs needed by the SM are the interaction strengths of the forces and quark and lepton masses to make 
very accurate predictions about the behavior of elementary particles.
For example, the magnetic moment of the electron calculated in the SM is found to agree with the experimental measurement to nine(?) decimal places.
Over the past 30 years, the SM has been vigorously tested across the strong, weak, and electromagnetic interactions. 
It came out triumphant every time. Yet, we know it is not the complete story. 

% the dark side
In 1933, an observation of the Coma Cluster by Zwicky suggested that the galaxies in the cluster were moving too fast to be explained 
by the luminous matter present. The same observation was repeated when looking at the rotation speeds of individual galaxies which 
suggested a dark component of mass, Dark Matter. 
Several independent measurements established that 
not only dark matter is not made out of baryons but it is more 
abundant. 
For example, anistropies in the cosmic microwave background, a radiation 
left over from the Big Bang, were consistent with 
quantum fluctuations from an inflationary epoch \cite{Hu:2001bc,2009AIPC}. 
These fluctuations encoded details about the density of matter 
in the form of 
cosmological parameters as they travelled through space and time to reach 
our experiments.
The astonishing conclusion was that the universe has nearly five times 
as much dark matter as ordinary matter \cite{Bertone:2004pz}.
The supernovae surveys gave direct evidence for an accelerating universe
 \cite{Perlmutter:1998np},
a view that was cemented by the measurement of cosmological parameters
\cite{Adam:2015rua,Ade:2015xua}
which led to the startling discovery that most of the energy density of 
the Universe is in the 
form of an unkown negative-pressure, called dark energy \cite{Scranton:2003in}.
There is an extensive program of experiments 
which will probe the dark energy.% and is beyond the scope of this work. 
%Instead, we turn to dark matter. 
Astrophysics and cosmology told us about 
the existence of dark matter and measured its density to a remarkable 
precision. Particle physics holds the hope to uncover what dark matter is.
In short, all experimental evidence are consistent with a universe 
constructed of 
\begin{itemize}
\item baryons (everyday matter): $\sim 5\%$ 
\item dark matter: $\sim 20\%$ 
\item dark energy: $\sim 75\%$ 
\item neutrinos, photons: a tiny fraction
\end{itemize}

Today, we are in front of many puzzles related to our view of the universe.
Everything we know of, including all particles present in the standard model, 
constitutes only 5\% of the energy budget of the universe. 
The universe is also predominatly composed of matter as opposed to 
anti-matter even though at the start of the universe, they were in equal 
amounts%, baryon assymetry
. The standard model describes the content of 
everyday matter 
and how it interacts but without telling us why it is that way.
Moreover, the standard model only describes these phenomena 
up to an energy scale of $\mathcal{O}\left(100\right)$ \GeV, weak scale.
Beyond this scale lies the realm of phenomena not described by the standard 
model that extend all the way to the Planck scale of 
$\mathcal{O}\left(10^{19}\right)$ \GeV, the limit where the known laws of 
physics apply. 
There is no mechanism to generate mass for 
neutrinos in the standard model. Last but not least, the standard model 
does not incorporate gravity, the fourth fundamental force.
The SM has many limitations since it is unable to account for observed 
features in the universe. 
Thus, there is a need for a theory beyond the standard model.

One of the most prominent extensions of the standard model, 
that addresses many of the shortcomings mentioned above, is a theory based on 
a new symmetry, called supersymmetry.
This symmetry is between the matter particles, fermions, and particles whose
exchange mediates the forces, bosons. Our current description of the world
treats fermions and bosons differently. Supersymmetry puts forward the idea
that fermions and bosons can be treated in a fully symmetric way. 
In other words,
if we exchange fermions and bosons in the equations of the theory, the 
equations will still look the same. An immediate consequence of the theory
is that every standard model particle will have a ``superpartner''.
As a result, we can design experiments to search for these 
supersymmetric particles. The work presented in this thesis is about the search for supersymmetric particles with a specific signature.
The many benefits of supersymmetry will be discussed but here it is worth 
mentioning two important features of the theory: it addresses the question 
of why there is a huge gap between the Planck scale and the weak scale, 
and it provides a dark matter candidate particle. 

Now that we understand what we are trying to do, it is time to address 
the question of how to do it.
For this endeavour, we require two key elements: accelerators and detectors.
The Large Hadron Collider (LHC) at CERN is the world's most energetic 
particle accelerator producing the highest intensity beam of protons 
travelling at 99.99993\% the speed of light. 

colliding beams of proton beams at a center of mass 
energy of 13 \TeV. When two protons collide, many new paerticles are 
created out of 


Many years of experience from Tevatron collider at Fermilab\cite{tevatron}, 
HERA at DESY\cite{hera}, and LEP\cite{lep} at CERN improved the understanding of the 
complex strong interaction dynamics 


% complex detectors
