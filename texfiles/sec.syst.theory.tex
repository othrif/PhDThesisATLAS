The cross-sections used to normalize the MC samples are varied according to the uncertainty in the 
cross-section calculation, which is 13\% for $\ttbar W$, 12\% for $\ttbar Z$ production~\cite{YR4}, 6\% for diboson
production~\cite{ATL-PHYS-PUB-2016-002}, 8\% for $\ttbar H$~\cite{YR4} and 30\% for 4$t$~\cite{Alwall:2014hca}. 
Additional uncertainties are assigned to some of these backgrounds to account for the theoretical modelling of the kinematic 
distributions in the MC simulation.

\subsection*{Associate $t\bar t+W/Z$ production}

The theoretical uncertainties on the $ttW$ and $ttZ^{(*)}$ processes are evaluated by several variations added in quadrature:

\begin{itemize}
\item Normalization and factorization scales varied independently up and down by a factor of two from the central scale $\mu_0 = H_{\rm T}/2$ as detailed in Ref.~\cite{ATL-PHYS-PUB-2016-005}. 
The largest deviation with respect to the nominal is used as the symmetric uncertainty.
\item Variation of the PDF used.
The standard deviation of the yields obtained using different PDF sets was used as the absolute uncertainty due to PDF. 
The relative uncertainty is then computed by dividing the standard deviation by the mean yield.
\item Comparison of the nominal \AMCATNLO MC samples to alternative \textsc{Sherpa} (v2.2) samples produced at leading-order with one extra parton in the matrix element for $ttW$ and 2 extra partons 
for $ttZ$~\cite{ATL-PHYS-PUB-2016-005}. 
The yield comparison for all SRs is shown in Table~\ref{tab:ttVGenComp}, with negligible differences in some SRs and up to 28\% in the worst case.
\end{itemize}
As a result of these studies, the total theory uncertainty for these processes 
is at the level of 15-35\% in the signal and validation regions used in the 
analysis. 
\begin{table}[!htb]
\caption{Comparison of the event yields for the $\ttbar V$ backgrund processes between \AMCATNLO (default generator) and \textsc{Sherpa} in the SRs, as well as their relative difference.
}
\label{tab:ttVGenComp}
\def\arraystretch{1.1}
\centering
\begin{tabular}{|c|c|c|c|}
\hline\hline
   SR    & \textsc{Sherpa} & aMCATNLO & Relative diff.\\ \hline
Rpc2L0bH   &   0.25 $\pm$ 0.03   &   0.20 $\pm$ 0.05   &   25\% \\
Rpc2L0bS   &   0.60 $\pm$ 0.06   &   0.82 $\pm$ 0.10   &   -26\% \\
Rpc2L1bH   &   3.84 $\pm$ 0.14   &   3.86 $\pm$ 0.20   &   $<$1\% \\
Rpc2L1bS   &   3.55 $\pm$ 0.13   &   3.94 $\pm$ 0.20   &   -9\% \\
Rpc2L2bH   &   0.35 $\pm$ 0.04   &   0.41 $\pm$ 0.05   &   -14\% \\
Rpc2L2bS   &   1.57 $\pm$ 0.08   &   1.57 $\pm$ 0.12   &   $<$1\% \\
Rpc2Lsoft1b   &   1.01 $\pm$ 0.07   &   1.24 $\pm$ 0.11   &   -18\% \\
Rpc2Lsoft2b   &   1.13 $\pm$ 0.07   &   1.15 $\pm$ 0.10   &   -1\% \\
Rpc3L0bH   &   0.23 $\pm$ 0.02   &   0.18 $\pm$ 0.04   &   27\% \\
Rpc3L0bS   &   0.90 $\pm$ 0.05   &   0.99 $\pm$ 0.09   &   -9\% \\
Rpc3L1bH   &   1.54 $\pm$ 0.08   &   1.52 $\pm$ 0.11   &   1\% \\
Rpc3L1bS   &   6.95 $\pm$ 0.16   &   7.02 $\pm$ 0.23   &   $<$1\% \\
Rpc3LSS1b   &   0.00 $\pm$ 0.00   &   0.00 $\pm$ 0.00   &   - \\
Rpv2L0b   &   0.18 $\pm$ 0.03   &   0.14 $\pm$ 0.04   &   28\% \\
Rpv2L1bH   &   0.51 $\pm$ 0.04   &   0.54 $\pm$ 0.07   &   -5\% \\
Rpv2L1bM   &   1.78 $\pm$ 0.08   &   1.40 $\pm$ 0.12   &   27\% \\
Rpv2L1bS   &   9.85 $\pm$ 0.19   &   9.91 $\pm$ 0.30   &   $<$1\% \\
Rpv2L2bH   &   0.52 $\pm$ 0.04   &   0.53 $\pm$ 0.08   &   -1\% \\
Rpv2L2bS   &   5.09 $\pm$ 0.14   &   4.80 $\pm$ 0.21   &   6\%  \\
\hline\hline
\end{tabular}
\end{table}

\subsection*{Diboson $WZ, ZZ, W^\pm W^\pm$ production}

The theoretical uncertainties on the  $WZ$ and $ZZ$ processes are evaluated by several variations added in quadrature:
\begin{itemize}
\item Normalization and factorization scales varied independently up and down by a factor of two from the central scale choice. The largest deviation with respect to the nominal is used as the 
symmetric uncertainty.
\item The standard deviation of the yields obtained using different PDF sets was used as the absolute uncertainty due to PDF. 
The relative uncertainty is then computed by dividing the standard deviation by the mean yield.
\item Resummation scale varied up and down by a factor of two from the nominal value.
\item The scale for calculating the overlap between jets from the matrix element and the parton shower is varied from the nominal value of 20 GeV down to 15 GeV 
 and up to 30 GeV. The largest deviation with respect to the nominal is used as the symmetric uncertainty due to matrix element matching.
\item An alternative recoil scheme is considered to estimate the uncertainty associated with mismodeling of jet multiplicities larger than three.
\end{itemize}

Based on these studies and the cross-section uncertainties, the total theory uncertainty for these processes is at the level of 25-40\% in the signal and validation regions used in the analysis. 

No theoretical uncertainties have been evaluated specifically for the $W^\pm W^\pm jj$ process, 
to which we assign the same uncertainties as for $WZ$, by lack of a better choice. 
But it should be noted that contributions from this process are minor in the SRs and typically  
smaller than those from $WZ$ and $ZZ$.

\subsection*{Other rare processes}
A conservative 50\% uncertainty is assigned on the summed contributions 
of all these processes ($\ttbar H$, $tZ$, $tWZ$, $\ttbar\ttbar$, $\ttbar WW$, $\ttbar WZ$, $WH$, $ZH$, $VVV$), 
which is generally quite larger than the uncertainties on their inclusive production cross-sections, 
and assumes a similar level of mismodelling as for diboson or $t\bar t V$ processes. 
