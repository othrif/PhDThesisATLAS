The likelihood for a set of parameters $\left(\mu,  \vect{\theta}\right)$
given all the data that might have been observed  $\vect{X}$ is the probability 
of observing the data given the parameters

\begin{equation}
\mathcal{L} \left( \mu, \vect{\theta} \rvert \vect{X} \right) = 
\Pr\left(\vect{X} \rvert \mu, \vect{\theta} \right).
\end{equation}

The data $\vect{X}$ includes observation in the signal regions as well as 
other auxilary experiments such as control regions used to constain 
backgrounds. 
In this analysis, hypothesis tests are performed on one signal region 
at the time, single-binned, and without control regions. 
As a result, the observed data $\vect{X}$ has one-dimensional component 
with value $X$ representing the count of events in the signal region.
The first parameter of interest represents the `strength' of 
the signal process $\mu > 0$ that will increase the number of 
expected events in the signal region given that the signal of the new physics 
model tested is present. In practice, the signal strength $\mu$ is used to 
scale the nominal expected cross section for the signal process, or the
number of expected signal events $s$. 
Thus, the predicted background will be of the form $\mu s + \sum_i b_i$ where 
$b_i$ represents the standard model background processes expected to 
contribute to the signal region. 
The parameter $\vect{\theta}$ refers to 
the nuisance parameters used to parametrize the systematic uncertainties
(luminosity, JES, JER, etc.)
\footnote{The parameters represented by $\vect{\theta}$ are called nuisance 
parameters since the aim is not to set a limit on them.}.
Thus, the likelihood is built as the product of a Poisson probability density 
function describing the observed number of events in the signal 
region and Gaussian distributions for each of the sources of systematic 
uncertainties.
The likelihood takes the simple form
\begin{equation}
\Pr\left(X \rvert \mu, \vect{\theta} \right) = 
\mathcal{P} \left(X~\rvert~\mu s\left(\vect{\theta}\right)
+ \sum_i b_i\left(\vect{\theta}\right)
+ \sum_j \theta_{j}^{\left(0\right)}
\right)
\times  
\prod_{j}  \mathcal{G} \left(\theta_{j}\right)
\end{equation}
where $\mathcal{P} \left(X~\rvert \nu\right) = e^{-\nu}\nu^X/X!$ 
and $\mathcal{G} \left(\theta\right)  = \frac{1}{\sqrt{2\pi\sigma_{\theta}^2}}
e^{-\frac{\left(\theta-\theta^0\right)^2}{2\sigma_\theta^2}}$
 are the Poisson and Gaussian probablity density functions. 
Both signal and backgrounds depend on the nuisance parameter
$\vect{\theta}$ which controls all independent sources of uncertainty 
and will be \textit{profiled} (or constrained) in the $CL_s$ procedure
described next.
To give an example, the luminosity uncertainty will have a mean 
as the luminosity central value $ \theta_{\text{lumi}}^{\left(0\right)}$ and the 
width as an experimentally determined 
uncertainty $\sigma_{\theta_{\text{lumi}}}$. When evaluating the effect of the 
luminosity uncertainty on the likelihood function, all terms involving the 
nuisance parameter $\theta_{\text{lumi}}$ will be scaled in the same way 
since luminosity is correlated across all terms.




\begin{equation}
q_{\mu} = 
\begin{cases}
  -2\ln\left(
  \frac
      {\mathcal{L} \left( \mu, \hat{\hat{\vect{\theta}}} \rvert \vect{X} \right)}
      {\mathcal{L} \left( \hat{\mu}, \hat{\vect{\theta}} \rvert \vect{X} \right)}
 \right), & \text{if}\ \mu > \hat{\mu} \\
      0, & \text{otherwise}
    \end{cases}
\end{equation}


\begin{equation}
p_{\mu} = \int_{q_{\mu}^{\left(\text{obs}\right)}}^{\infty} f\left(q_{\mu} \rvert \mu\right) \diff q_{\mu}
\end{equation}


\begin{equation}
  CL_\mathrm{s+b}  = \Pr\left( q_{\mu} \geq q_{\mu}^{\left(\text{obs}\right)}
  \rvert \mu
  \right) < \alpha
\end{equation}


\begin{equation}
  CL_\mathrm{s}  = \frac
  {\Pr\left( q_{\mu} \geq q_{\mu}^{\left(\text{obs}\right)}\rvert \mu \right)}
  {\Pr\left( q_{\mu} \geq q_{\mu}^{\left(\text{obs}\right)}\rvert \mu = 0 \right)}
< \alpha
\end{equation}


\begin{equation}
  q_{\mu} = 
  \begin{cases}
    \frac{\left(\mu - \hat{\mu}\right)^2}{\sigma^2} + \mathcal{O}\left(\frac{1}{\sqrt{N}}\right)
    , & \text{if}\ \mu > \hat{\mu} \\
    0, & \text{otherwise}
  \end{cases}
\end{equation}
