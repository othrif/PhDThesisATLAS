The reconstruction of `fake` leptons can be an instrumental effect related to the inability to identify the object based on 
its measured properties by the detector. In this case, the reconstructed lepton is not a real lepton and the production process 
will be different for electrons and muons.

The reconstruction of electrons relies on the observation of well aligned particle hits in the layers of the ID that are consistent 
with an energy deposition in the EM calorimeter. Photons can mimic this signature since they deposit energy in the EM 
calorimeter that happens to be aligned with a charged track. A jet for example containing charged and neutral pions can 
lead to such scenarios. It is possible for the jet to have one charged pion leaving a track similar to that of an electron.
The decay of $\pi^0$ mesons to photons in this jet can deposit energy in the EM calorimeter leading to the required signature.
Another mechanism that can lead to fake electrons is the emission of photons via Brehmstrahlung from high energy muons. 
The muon track can be mistaken for that of an electron and the photons interact with the EM calorimeter leading to a
signature similar to that of electrons. An additional process is that of photon conversions into a $e^+e^-$.

The reconstruction of muons relies on the observation of tracks from the ID matched to tracks from the muon spectrometer.
It is possible for charged hadrons with long lifetime to traverse the calorimeter layers and leave hits in the muon spectrometer.
These hits may coincide with other hits from the ID due to the random activity in the event. As a result, a muon can get
reconstructed. Another instance may occur when pions or kaons decay in-flight to muons in the muon spectrometer
and happen to align with the primary vertex.

The leptons that are used in the physics analyses must be coming from the hard scatter, generally referred to as prompt leptons.
There is another case where the reconstructed lepton is a real lepton but is not a lepton coming from the hard interaction,
referred to as non-prompt leptons. Non-prompt leptons can be produced from heavy flavor meson decays with a low energy activity 
around the lepton which allows it to pass isolation requirements. A good example of this type of process is the 
semi-leptonic decay of top quark pair which contribute to final states with two leptons. 

For the rest of the thesis, the fake leptons will be referred to as fake/non-prompt (FNP) leptons.
There are several methods used to perform the estimation of FNP lepton backgrounds. 
A method that the author developed will be described next along with a standard method for estimating this type of backgrounds.
The benefit of having two methods for estimating the FNP lepton background is to have enough confidence in the final estimate. 
The two methods use different assumptions which naturally leads to a more robust estimation of this difficult background.
Moreover, the final estimate of the FNP lepton background is taken as a statistical combination of the estimates from the 
two methods leading to a reduction of the systematic uncertainties on the estimate.
