As discussed in Chapter ??, supersymmetry (SUSY) 
is a theoretically favoured extension of the Standard Model (SM),
which for each degree of freedom of the SM predicts another degree of
freedom with a different spin. These degrees of freedom combine into
physical superpartners of the SM particles: scalar partners of quarks
and leptons (squarks (\sq) and sleptons), fermionic
partners of gauge and Higgs bosons (gluinos (\gl), charginos
($\tilde{\chi}^{\pm}_{i}$, with $i$ = 1,2) and neutralinos
($\tilde{\chi}^{0}_{i}$ with $i$ = 1,2,3,4)), all with identical 
quantum numbers to their SM partners, except spin.
If $R$-parity is conserved~
%\cite{Farrar:1978xj} 
the lightest supersymmetric particle (LSP) is stable 
and is typically the lightest neutralino \neut which is a viable 
dark matter candidate. 

As some of these particles are expected to be in the TeV-range, 
the discovery (or exclusion) of weak-scale SUSY is one of the highest
physics priorities for the LHC. Both the ATLAS and CMS collaborations 
have carried out a vigorous search program for SUSY in various final states.

In this chapter, a detailed description will be given for 
the search for supersymmetric particles with an experimental signature 
involving multiple leptons in the final state. The search strategy will 
be covered along with the event selection procedure and the mechanisms by which 
the various backgrounds are estimated. The results of the analysis 
will be interpretted in the context of simplified 
and popular supersymmetric models and also recast in terms of model independent 
limits. 


In order to address the SM hierarchy problem with SUSY models
%~\cite{Sakai:1981gr,Dimopoulos:1981yj,Ibanez:1981yh,Dimopoulos:1981zb}
, the masses of the partners of the gluons (gluinos~\gl) 
and of the top quark chiral degrees of freedom 
(top squarks~$\tilde t_L$ and~$\tilde t_R$), and the partners of the bottom 
quark (sbottom) are expected to be in the \TeV range.
The production cross sections 
for gluino pairs (\glgl), top--antitop squark pairs (\stst) 
and bottom--antibottom squark pairs (\sbsb) 
are relatively large which makes them a primary target for the early 
search program with the LHC data~cite{Borschensky:2014cia}.
The cascade decay of these SUSY particles may proceed
via intermediate neutralinos~$\tilde\chi^0_{2,3,4}$ or 
charginos~$\tilde\chi^\pm_{1,2}$ that in turn lead to $W$, $Z$ or Higgs bosons, 
or to lepton superpartners (sleptons~\slep) which will 
lead to isolated leptons and neutralinos~\neut.


This searches focuses  on  final states with two leptons (electrons or muons) 
of the same electric charge (referred to as same-sign (SS) leptons) or 
three leptons (3L) in any charge combination or of same electric charge, 
jets and missing transverse momentum 
(\Ptmiss, whose magnitude is referred to as~\met). 
 Despite the penalty in signal yields due to the low branching ratio to SS/3L, 
the extreme background reduction achieved in 
this signature is a very good opportunity to discover new physics, 
particularly in scenarios with small mass differences between SUSY particles 
(``compressed scenarios''). 
This search is thus sensitive to a wide variety of models based on very different assumptions.

The results covered in the next sections use the full data collected in 
2015 and 2016 by the ATLAS experiment in proton--proton ($pp$) collisions 
at a center-of-mass energy of $\sqrt s=13\TeV$, 
corresponding to a total integrated luminosity of 36.1 \ifb.

It uses as signal benchmarks a few SUSY scenarios described in section~\ref{sec:signals}. 
Observed data and predicted Standard Model yields are compared 
in a set of unbinned signal regions defined in section~\ref{sec:signalregions} (``cut and count''), 
and chosen to provide good sentivity to the signal benchmarks. 
The background prediction relies on Monte Carlo simulation of the relevant processes with prompt same-sign leptons 
and theoretical computations of their cross-sections (section~\ref{sec:promptbkg}), 
and data-driven estimates of the contributions from processes with charge-flipped electrons (section~\ref{sec:chargeflip}) 
or non-prompt or fake leptons (section~\ref{sec:fakes}). 
The predictions are compared to observed data in a set of suitably chosen validation regions (section~\ref{sec:validation}) 
enriched in the different processes. 
Statistical interpretation of the observations in the signal regions is performed through the framework described in section~\ref{sec:histfitter}, 
which is also used by convenience to determine the total background yields in the signal regions, 
accounting for correlations between systematic uncertainties associated to the different background processes. 
These yields, compared to observed data, are reported in section~\ref{sec:results}. 
In the absence of significant excess in observed data over the Standard Model prediction, 
exclusion limits are set on the masses of SUSY partners involved in the benchmark SUSY scenarios, and are presented in section~\ref{sec:limits}. 
Final conclusions are stated in section~\ref{sec:conclusion}.
