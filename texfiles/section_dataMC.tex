The data used in this analysis were collected during 2015 and 2016 with a peak 
instantaneous luminosity of $L=1.4\times~10^{34}$~cm$^{-2}$s$^{-1}$. 
The mean number of $pp$ interactions per bunch crossing 
(pile-up) in the data set is 24. After the application of beam, detector and data-quality requirements, 
the integrated luminosity considered corresponds to 36.1 \ifb.
The uncertainty in the combined 2015+2016 integrated luminosity is 3.2\%. 
It is derived, following a methodology similar to that detailed in Ref.~\cite{Aaboud:2016hhf}, 
from a preliminary calibration of the luminosity scale using $x$--$y$ beam-separation scans performed in August 2015 and May 2016.

Monte Carlo (MC) simulated event samples are used to model the SUSY signals and to estimate the irreducible SM background with two 
same-sign and/or three ``prompt'' leptons. Prompt leptons are produced directly in the hard-scattering process, or in the subsequent decays 
of $W,Z,H$ bosons or prompt $\tau$ leptons. The reducible background, mainly 
arising from $\ttbar$ production, is estimated from the data as described in Section~\ref{sec:DD_bkg}. 
The MC samples were processed through a detailed ATLAS detector simulation~\cite{Aad:2010ah} based on 
\textsc{Geant4}~\cite{Agostinelli:2002hh} or a fast simulation using a parameterization of the calorimeter response 
and \textsc{Geant4} for the ID and MS~\cite{ATL-PHYS-PUB-2010-013}. To simulate the effects of additional $pp$ collisions 
in the same and nearby bunch crossings, inelastic interactions were generated using the soft strong-interaction processes 
of \PYTHIA 8.186~\cite{Sjostrand:2007gs} with a set of tuned parameters referred to as the A2 tune~\cite{ATL-PHYS-PUB-2012-003} and the MSTW2008LO parton 
distribution function (PDF)~\cite{Martin:2009iq}. These MC events were overlaid onto the simulated hard-scatter event 
and reweighted to match the pile-up conditions observed in the data. 
Table~\ref{tab:MC} presents, for all samples, the event generator, parton shower, cross-section normalization, PDF
set and the set of tuned parameters for the modelling of the parton shower, hadronization and underlying event. 
In all MC samples, except those produced by the \SHERPA generator, 
the \textsc{EvtGen}~v1.2.0 program~\cite{EvtGen} was used 
to model the properties of bottom and charm hadron decays. 

\begin{table*}[ht]
\begin{center}
\scriptsize
\resizebox{\textwidth}{!}
{
\begin{tabular}{|l|l|c|c|c|c|}
\hline
Physics process    & Event generator & Parton shower & Cross-section & PDF set & Set of tuned \\
                   &	      & 	      & normalization & 	& parameters  \\
\hline
\hline
Signal            &                      			&                      			&                               	&               &      \\
RPC   	   	& \AMCATNLO 2.2.3~\cite{Alwall:2014hca} 	& \PYTHIA 8.186~\cite{Sjostrand:2007gs}	& NLO+NLL  & NNPDF2.3LO~\cite{Ball:2012cx} & A14~\cite{ATL-PHYS-PUB-2014-021} \\
RPV except Fig.~\ref{fig:feynm_rpv_gl112} & \AMCATNLO 2.2.3     & \PYTHIA 8.210       			& or			 		& NNPDF2.3LO	& A14 \\
RPV Fig.~\ref{fig:feynm_rpv_gl112}   & \HERWIG 2.7.1~\cite{Corcella:2000bw}       & \HERWIG 2.7.1  	& NLO-Prospino2 ~\cite{Beenakker:1996ch,Kulesza:2008jb,Kulesza:2009kq,Beenakker:2009ha,Beenakker:2011fu,Kramer:2012bx}     				&CTEQ6L1~\cite{Pumplin:2002vw} & UEEE5~\cite{Gieseke:2012ft}  \\
\hline
\hline
$\ttbar +X$            &                      			&                      			&                               	&               &      \\
$\ttbar W$, $\ttbar Z/\gamma^{*}$ & \AMCATNLO 2.2.2 		& \PYTHIA 8.186	  			& NLO~\cite{YR4}     			& NNPDF2.3LO	& A14    \\
$\ttbar H$	   & \AMCATNLO 2.3.2        			& \PYTHIA 8.186  			& NLO~\cite{YR4}   			& NNPDF2.3LO	& A14  \\
4$t$    	& \AMCATNLO 2.2.2       			& \PYTHIA 8.186        			& NLO~\cite{Alwall:2014hca}	  	& NNPDF2.3LO	& A14  \\
\hline
Diboson            &                      			&                      			&                               	&               &      \\
$ZZ$, $WZ$    	   & \SHERPA 2.2.1~\cite{gleisberg:2008ta}      & \SHERPA 2.2.1				& NLO~\cite{ATL-PHYS-PUB-2016-002}	&NNPDF2.3LO & \SHERPA default \\
Other (inc. $W^{\pm}W^{\pm}$)   & \SHERPA 2.1.1 		& \SHERPA 2.1.1				& NLO~\cite{ATL-PHYS-PUB-2016-002}	&CT10~\cite{Lai:2010vv} & \SHERPA default \\
\hline
Rare               &                      			&                      			&                               	&               &      \\
$\ttbar WW$, $\ttbar WZ$     & \AMCATNLO 2.2.2       		& \PYTHIA 8.186      			& NLO~\cite{Alwall:2014hca}	  	& NNPDF2.3LO	 & A14  \\
$tZ$, $tWZ$, $t\ttbar$    & \AMCATNLO 2.2.2        		& \PYTHIA 8.186       			& LO		                   	& NNPDF2.3LO     & A14  \\
$WH$, $ZH$	   & \AMCATNLO 2.2.2        			& \PYTHIA 8.186      			& NLO~\cite{Dittmaier:2012vm}   	& NNPDF2.3LO     & A14  \\
Triboson	   & \SHERPA 2.1.1         			& \SHERPA 2.1.1        			& NLO~\cite{ATL-PHYS-PUB-2016-002}       & CT10	     	& \SHERPA default \\
\hline
\end{tabular}
}
\caption{Simulated signal and background event samples: the corresponding event generator, parton shower, cross-section normalization, PDF set and 
set of tuned parameters are shown for each sample. Because of their very small contribution to the signal-region background estimate, 
$\ttbar WW$, $\ttbar WZ$, $tZ$, $tWZ$, $t\ttbar$, $WH$, $ZH$ and triboson are summed and labelled ``rare'' in the following. 
NLO-Prospino2 refers to RPV down squark models of Figures~\ref{fig:feynm_rpv_sd313} and \ref{fig:feynm_rpv_sd321}, as well as the NUHM2 model.}
\label{tab:MC}
\end{center}
\end{table*}

The SUSY signals from Figure~\ref{fig:feynman} are defined by an effective Lagrangian describing the interactions of a small number of new 
particles~\cite{Alwall:2008ve,Alwall:2008ag,Alves:2011wf}. All SUSY particles not included 
in the decay of the pair-produced squarks and gluinos are effectively decoupled. These simplified models assume one 
production process and one decay channel with a 100\% branching fraction. Apart from Figure~\ref{fig:feynm_rpv_gl112}, where 
events were generated with \HERWIG++~\cite{Corcella:2000bw}, all simplified models were generated from leading-order (LO) matrix elements 
with up to two extra partons in the matrix element (only up to one for the $\gluino\to q\bar q(\ell\ell/\nu\nu)\ninoone$ model) 
using \AMCATNLO 2.2.3~\cite{Alwall:2014hca} interfaced to \PYTHIA 8 with the A14 tune~\cite{ATL-PHYS-PUB-2014-021} for the 
modelling of the parton shower, hadronization and underlying event.
Jet--parton matching was realized following the CKKW-L prescription~\cite{Lonnblad:2011xx}, with a matching scale set to one quarter of 
the pair-produced superpartner mass. All signal models were generated 
with prompt decays of the SUSY particles. Signal cross-sections were calculated at next-to-leading order (NLO) in the strong coupling constant, 
adding the resummation of soft-gluon emission at next-to-leading-logarithmic 
accuracy (NLO+NLL)~\cite{Beenakker:1996ch,Kulesza:2008jb,Kulesza:2009kq,Beenakker:2009ha,Beenakker:2011fu}, except 
for the RPV models of Figures~\ref{fig:feynm_rpv_sd313} and~\ref{fig:feynm_rpv_sd321} and the NUHM2 model where NLO 
cross-sections were used~\cite{Beenakker:1996ed,Beenakker:1996ch}. The nominal cross-sections and the uncertainties were taken from 
envelopes of cross-section predictions using different PDF sets 
and factorization and renormalization scales, as described in Refs.~\cite{Kramer:2012bx,Borschensky:2014cia}. 
Typical pair-production cross-sections are: $4.7 \pm 1.2$~fb for gluinos with a mass of 1.7 \TeV, $28 \pm 4$~fb
for bottom squarks with a mass of 800 \GeV, and $15.0\pm 2.0$~fb for down 
squark-rights with a mass of 800 \GeV and a gluino mass of 2.0 \TeV.

The two dominant irreducible background processes are $\ttbar V$ (with $V$ being a $W$ or $Z/\gamma^*$ boson) 
and diboson production with final states of four charged leptons $\ell$,\footnote{All lepton flavours are included here and $\tau$
leptons subsequently decay leptonically or hadronically.} three charged leptons and one neutrino, or 
two same-sign charged leptons and two neutrinos. The MC simulation samples for these are described in 
Refs.~\cite{ATL-PHYS-PUB-2016-005} and~\cite{ATL-PHYS-PUB-2016-002}, 
respectively. For diboson production, the matrix elements contain the doubly resonant diboson processes and all other diagrams with four 
or six electroweak vertices, such as $W^\pm W^\pm jj$, with one ($W^\pm W^\pm jj$) or two ($WZ$, $ZZ$) extra partons.
NLO cross-sections for $\ttbar W$, $\ttbar Z/\gamma^*(\rightarrow \ell \ell)$\footnote{This cross-section is computed 
using the configuration described in Refs.~\cite{Alwall:2014hca,Frixione:2015zaa}.} 
and leptonic diboson processes are respectively 600.8~fb~\cite{YR4}, 124~fb and 
6.0~pb~\cite{ATL-PHYS-PUB-2016-002}. The processes $\ttbar H$ and 4$t$, with NLO cross-sections of 507.1~fb~\cite{YR4} and 
9.2~fb~\cite{Alwall:2014hca} respectively, are also considered.

Other background processes, with small cross-sections and no significant contribution to any of the signal regions, 
are grouped into a category labelled ``rare''. This category contains 
$\ttbar WW$ and $\ttbar WZ$ events generated with no extra parton in the matrix element, and $tZ$, $tWZ$, $t\ttbar$, $WH$ and $ZH$ as well as 
triboson ($WWW$, $WWZ$, $WZZ$ and $ZZZ$) production with fully leptonic decays, leading to up to six charged leptons. 
The processes $WWW$, $WZZ$ and $ZZZ$ were generated at NLO 
with additional LO matrix elements for up to two extra partons, 
while $WWZ$ was generated at LO with up to two extra partons. 
