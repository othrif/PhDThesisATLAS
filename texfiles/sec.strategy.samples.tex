\subsection{Collision Data}
The analysis uses $pp$--collisions data at $\sqrt s=13$ \TeV~ 
collected by the ATLAS detector during 2015 and 2016
 with a peak instantaneous luminosity of 
$L=1.4\times~10^{34}$~cm$^{-2}$s$^{-1}$.
The total integrated luminosity considered corresponds to 36.1 \ifb~ 
(3.2 \ifb~in 2015 and 32.9 \ifb~in 2016) recorded 
after applying beam, detector, and data-quality requirements.
The combined luminosity uncertainty for 2015 and 2016 is 3.2\%, 
assuming partially correlated uncertainties in 2015 and 2016.
The integrated luminosity was established following the same methodology as 
that detailed in Ref.~\cite{Aaboud:2016hhf},
from a preliminary calibration of the luminosity scale using a pair of $x$-$y$ 
beam separation scans.% performed in August 2015 and May 2016. 

\subsection{Simulated Event Samples}

Monte Carlo (MC) simulated event samples are used to model the SUSY signal 
and SM backgrounds. 
The irreducible SM backgrounds refer to processes that lead to two 
same-sign and/or three ``prompt'' leptons where the prompt leptons
are produced directly in the hard-scattering process, 
or in the subsequent decays 
of $W,Z,H$ bosons or prompt $\tau$ leptons. 
The reducible backgrounds, mainly 
arising from $\ttbar$ and $V$+jets production, are estimated either from data 
or from MC simulation as described in Section~\ref{chap:fake}. 

Table~\ref{tab:MC} presents the event generator, parton shower, cross-section 
normalization, PDF
set and the set of tuned parameters for the modelling of the parton shower, 
hadronization and underlying event. 
Apart from the MC samples produced by the \SHERPA generator, all MC samples
used the \textsc{EvtGen}~v1.2.0 program~\cite{EvtGen} 
to model the properties of bottom and charm hadron decays. 

\begin{table*}[!ht]
\begin{center}
%\scriptsize
\Large
\resizebox{\textwidth}{!}
{
\begin{tabular}{|l|l|c|c|c|c|c|}
\hline
Physics process    & Event generator & Parton shower & Cross-section & Cross-section & PDF set & Set of tuned \\
                   &          &               & order & value (fb)&         & parameters  \\
\hline
\hline
Signal                  & \AMCATNLO 2.2.3~\cite{Alwall:2014hca}         & \PYTHIA 8.186~\cite{Sjostrand:2007gs} & NLO+NLL
& See Table~\ref{tab:signal_xsections} & NNPDF2.3LO~\cite{Ball:2012cx} & A14~\cite{ATL-PHYS-PUB-2014-021} \\
\hline
$\ttbar +X$            &                                        &                                       & 
             &               &    &   \\
$\ttbar W$,$\ttbar Z/\gamma^{*}$ & \AMCATNLO 2.2.2            & \PYTHIA 8.186                         & NLO~\cite{YR4} & $600.8$, $123.7$                     & NNPDF2.3LO    & A14    \\
$\ttbar H$	   & \AMCATNLO 2.3.2        			& \PYTHIA 8.186  			& NLO~\cite{YR4}  & 	$507.1$		& NNPDF2.3LO	& A14  \\
4$t$    	& \AMCATNLO 2.2.2       			& \PYTHIA 8.186        			& NLO~\cite{Alwall:2014hca}	&  
$9.2$	& NNPDF2.3LO	& A14  \\
\hline
Diboson            &                   &   			&                      			&                               	&               &      \\
$ZZ$, $WZ$       & \SHERPA 2.2.1~\cite{gleisberg:2008ta}      & \SHERPA 2.2.1& NLO~\cite{ATL-PHYS-PUB-2016-002}& 
$1.3\cdot 10^3$,$4.5\cdot 10^3$
&NNPDF2.3LO & \SHERPA default \\
inc. $W^{\pm}W^{\pm}$   & \SHERPA 2.1.1 		& \SHERPA 2.1.1				& NLO~\cite{ATL-PHYS-PUB-2016-002}  &	
$86$
&CT10~\cite{Lai:2010vv} & \SHERPA default \\
\hline
Rare               &                  &    			&                      			&                               	&               &      \\
$\ttbar WW$, $\ttbar WZ$     & \AMCATNLO 2.2.2       & \PYTHIA 8.186      & NLO~\cite{Alwall:2014hca}  & 
$9.9$, $0.36$
& NNPDF2.3LO & A14  \\
$tZ$, $tWZ$, $t\ttbar$    & \AMCATNLO 2.2.2        & \PYTHIA 8.186       & LO                   & 
$240$, $16$, $1.6$
& NNPDF2.3LO     & A14  \\
$WH$, $ZH$   & \AMCATNLO 2.2.2        & \PYTHIA 8.186      & NLO~\cite{Dittmaier:2012vm}   & 
$1.4\cdot 10^3$, $868$  
& NNPDF2.3LO     & A14  \\
Triboson	   & \SHERPA 2.1.1         			& \SHERPA 2.1.1        			& NLO~\cite{ATL-PHYS-PUB-2016-002}
& $14.9$
& CT10	     	& \SHERPA default \\
\hline
Irreducible (Incl.)            &                      			&                      		&	&                               	&               &      \\
$W$+Jets      & \POWHEGBOX       		& \PYTHIA 8.186      			& NNLO	 & 
2.0 $\cdot 10^7$	& CT10      	 & AZNLO\cite{AZNLO:2014}\\
$Z$+Jets      & \POWHEGBOX       		& \PYTHIA 8.186      			& NNLO	 & 1.9 $\cdot 10^7$	& CT10      	 & AZNLO\cite{AZNLO:2014}\\
$\ttbar$    	   & \POWHEGBOX       		& \PYTHIA 6.428      			& NNLO+NNLL~\cite{Czakon:2011xx}	&  
8.3 $\cdot 10^5$ 	& CT10      	 & PERUGIA2012 (P2012) \cite{tt:perugia}\\
\hline
\end{tabular}
}
\caption{Simulated signal and background event samples: the corresponding event generator, parton shower, cross-section normalization, PDF set and 
set of tuned parameters are shown for each sample. Because of their very small contribution to the signal-region background estimate, 
$\ttbar WW$, $\ttbar WZ$, $tZ$, $tWZ$, $t\ttbar$, $WH$, $ZH$ and triboson are summed and labelled ``rare''.}
\label{tab:MC}
\end{center}
\end{table*}

The MC samples were processed through either a full ATLAS detector 
simulation~\cite{Aad:2010ah} based on 
\textsc{Geant4}~\cite{Agostinelli:2002hh} or a fast simulation using a 
parameterization of the calorimeter response 
and \textsc{Geant4} for the inner detector and muon spectrometer
~\cite{ATL-PHYS-PUB-2010-013},
and are reconstructed in the same manner as the data. 
All simulated samples are generated with a range of minimum-bias interactions 
using {\sc Pythia 8}~\cite{Sjostrand:2007gs} 
with the MSTW2008LO PDF set~\cite{Sherstnev:2007nd} and the A2 tune overlaid on the hard-scattering event 
to account for the multiple $pp$ interactions in the same bunch crossing 
(in-time pileup) 
and neighbouring bunch crossing (out-of-time pileup). 
The distribution of the average number of interactions per bunch crossing 
$\langle\mu\rangle$ ranges from 0.5 to 39.5, 
with a profile set as an estimate of the combined 2015+2016 data 
$\langle\mu\rangle$ profile. 
With larger luminosity collected during this year and the $\mu$ distribution in data being closer to that in the MC profile,
the simulated samples are re-weighted to reproduce the observed distribution 
of the average number of collisions per bunch crossing ($\mu$).


\subsection*{Background process simulation}

The two dominant irreducible background processes are $\ttbar V$ (with $V$ being a $W$ or $Z/\gamma^*$ boson) 
and diboson production with final states of four charged leptons $\ell$,\footnote{All lepton flavours are included here and $\tau$
leptons subsequently decay leptonically or hadronically.} three charged leptons and one neutrino, or 
two same-sign charged leptons and two neutrinos. 
%The MC simulation samples for these are described in 
%Refs.~\cite{ATL-PHYS-PUB-2016-005} and~\cite{ATL-PHYS-PUB-2016-002}, respectively.

The production of a $\ttbar V$ 
constitutes the main source of background with prompt same-sign leptons for event selections including $b$-jets. 
Simulated events for these processes were generated at NLO with \AMCATNLO v2.2.2~\cite{Alwall:2014hca} interfaced to \textsc{Pythia} 8,
with up to two ($ttW$) or one ($ttZ^{(*)}$) extra parton included in the matrix elements~\cite{ATL-PHYS-PUB-2016-005}. 
The samples are normalised to the inclusive process NLO cross-section using appropriate $k$-factors~\cite{Alwall:2014hca}.

The production of multiple $W,Z$ bosons decaying leptonically 
constitutes the main source of background with prompt same-sign leptons for event selections vetoing $b$-jets. 
Diboson processes with four charged leptons, three charged leptons and one neutrino, or two charged leptons and two neutrinos 
were simulated at NLO using the \textsc{Sherpa} 2.2.1 generator~\cite{gleisberg:2008ta}, as described in detail in Ref.~\cite{ATL-PHYS-PUB-2016-002}. 
The main samples simulate $qq \to VV\to\text{leptons}$ production including the doubly resonant $WZ$ and $ZZ$ processes, 
non-resonant contributions as well as Higgs-mediated contributions, and their interferences; 
up to three extra partons were included (at LO) in the matrix elements. 
Simulated events for the $W^\pm W^\pm jj$ process (including non-resonant contributions) were produced at LO with up to one extra parton, 
separately for QCD-induced $\left(\mathcal{O}(\alpha_\text{em}^4)\right)$ 
and VBS-induced $\left(\mathcal{O}(\alpha_\text{em}^6)\right)$ production -- the interferences being neglected. 
Additional samples for VBS-induced $qq\to 3\ell\nu jj$ and $qq\to 4\ell$ and loop-induced $gg\to WZ^{(*)}/ZZ^{(*)}$ processes
were also produced with the same configuration.
The samples generated at NLO are directly normalized to the cross-sections provided by the generator. 

Production of a Higgs boson in association with a $\ttbar$ pair is simulated using \AMCATNLO~\cite{Alwall:2014hca} 
(in \MADGRAPH v2.2.2) interfaced to \HERWIG 2.7.1~\cite{Corcella:2000bw}.  
The UEEE5 underlying-event tune is used together with the CTEQ6L1~\cite{Pumplin:2002vw} (matrix element) and CT10~\cite{Lai:2010vv} (parton shower) PDF sets.
Simulated samples of SM Higgs boson production in association with a $W$ or $Z$ boson are produced with \PYTHIA 8.186, using the \textsc{A14} tune and the \textsc{NNPDF23LO} PDF set. Events are normalised with cross-sections calculated at NLO~\cite{Dittmaier:2012vm}.

\MADGRAPH v2.2.2~\cite{Alwall:2011uj} is used to simulate the $t\bar{t}WW$, $tZ$, $\ttbar\ttbar$ and $\ttbar t$ processes, and the generator cross-section is used for $tZ$ and $\ttbar t$. \MADGRAPH interfaced to \textsc{Pythia} 8 is used to generate $t\bar{t}WZ$ processes, and appropriate $k$-factors are taken from~\cite{Alwall:2014hca}. \AMCATNLO interfaced to \PYTHIA 8 is used for the generation of the $tWZ$ process, with an alternative sample generated with \AMCATNLO interfaced to \HERWIG used to evaluate the parton shower uncertainty.  
Fully leptonic triboson processes ($WWW$, $WWZ$, $WZZ$ and $ZZZ$) with up to six charged leptons are simulated using \SHERPA~v2.1.1 
and described in Ref.~\cite{ATL-PHYS-PUB-2016-002}. 
The $4\ell$ and $2\ell+2\nu$ processes are calculated at next-to-leading order (NLO) for up to one additional parton; 
final states with two and three additional partons are calculated at leading order (LO). 
The $WWZ\to 4\ell+2\nu$ or $2\ell+4\nu$ processes are calculated at LO with up to two additional partons. 
The $WWW/WZZ\to 3\ell+3\nu$, $WZZ\to 5\ell+1\nu$, $ZZZ\to 6\ell+0\nu$, $4\ell+2\nu$ or $2\ell+4\nu$ processes 
are calculated at NLO with up to two extra partons at LO. 
The CT10~\cite{Lai:2010vv} parton distribution function (PDF) set is used for all \SHERPA samples in conjunction with 
a dedicated tuning of the parton shower parameters developed by the \SHERPA authors. 
The generator cross-sections (at NLO for most of the processes) are used when normalising these backgrounds.

Double parton scattering (DPS) occurs when two partons interact simultaneously in a proton-proton collision leading to two hard scattering 
processes overlapping in a detector event. 
Accordingly, two single $W$ production processes can lead to a $W^\pm$ + $W^\pm$ final state via DPS. 
This background is expected to have a negligible contribution to signal regions with high jet multiplicities.
To estimate a conservative upper bound on cross-section for $WW$ events which might arise from DPS, a standard ansatz is adopted: 
in this, for a collision in which a hard process (X) occurs, the probability that 
an additional (distinguishable) process (Y) occurs is parametrized as:
\begin{equation}
\sigma^{DPS}_{XY} = \sigma^{}_{X}\sigma^{}_{Y}/\sigma^{}_\text{eff}
\end{equation} 
where $\sigma^{}_{X}$ is the production cross section of the hard 
process X and $\sigma^{}_\text{eff}$ (effective area parameter) 
parametrizes the double-parton interaction part of the production 
cross section for the composite system (X+Y). 
A value of $\sigma^{}_\text{eff} = 10-20$ mb is assumed in this study (as obtained from 7~TeV~measurements, and with no observed dependence on $\sqrt{s}$), and it is independent on the processes involved. For the case of $W^\pm+W^\pm$ production:
\begin{equation}
\sigma^{DPS}_{W^\pm W^\pm} = \frac{ \sigma^{}_{W^+}\sigma^{}_{W^+} + \sigma^{}_{W^-}\sigma^{}_{W^-} + 2\sigma^{}_{W^+}\sigma^{}_{W^-}}{\sigma^{}_\text{eff} } \simeq 0.19-0.38\text{ pb.}
\end{equation} 

After the application of the SR criteria, only 4 raw MC events in the DPS $WW\to\ell\nu\ell\nu$ remain. 
Table~\ref{tab:DPS_SR} shows the expected contribution in the SRs where some MC event survives all the cuts. The ranges quoted in the tables reflect the range in the predicted $\sigma^{DPS}_{W^\pm W^\pm}$ cross-section above, as well as the combinatorics for scaling the jet multiplicity\footnote{For instance, a DPS event with 6 jets can be due to the overlap of two events with 6+0 jets, or 5+1, 4+2 or 3+3 jets. All possible combinations are considered and the range quoted in the table shows the combinations leading to the smallest and largest correction factors.}.
Due to the large uncertainties involved in these estimates, some of them difficult to quantify (such as the modelling of DPS by {\sc Pythia} at LO), the contribution from this background is not included in the final SR background estimates. 
Note that the estimated DPS contribution is typically much smaller than the uncertainty on the total background for the SRs.

\begin{table}[!htb]
\caption{Number of raw MC events and its equivalent for 36.1 \ifb with and without the correction as a function of the jet multiplicity. 
Only the SRs where at least one MC event passes all the cuts are shown.}
\label{tab:DPS_SR}
\centering
\begin{tabular}{l|c|c|c}
\hline
SR       & Raw MC events & Without $N_{\text{jet}}$ correction & With $N_{\text{jet}}$ correction \\\hline
Rpc2L0bS & 2 & 0.016-0.033 & 0.09-0.38 \\ 
Rpc2L0bH & 1 & 0.006-0.012 & 0.05-0.17 \\ 
\hline
\end{tabular}
\end{table}




\subsection*{Signal cross-sections and simulations}

\begin{table}[t!]
\centering
\caption{Signal cross-sections [pb] and related uncertainties [\%] for scenarios featuring \glgl\ (top table) or \sbsb\  (bottom table) production, 
as a function of the pair-produced superpartner mass, reproduced from Ref.~\cite{twiki-SusyCrossSections}.}
\label{tab:signal_xsections}
\resizebox{\textwidth}{!}{
\begin{tabular}{|c|c|c|c|c|c|c|c|c|c|c|}
\hline\hline
Gluino mass (GeV) & 500 & 550 & 600 & 650 & 700 \\\hline
Cross section (pb) & $27.4 \pm 14\%$ & $15.6 \pm 14\%$ & $9.20 \pm 14\%$ & $5.60 \pm 14\%$ & $3.53 \pm 14\%$\\\hline\hline
750 & 800 & 850 & 900 & 950 & 1000\\\hline
$2.27 \pm 14\%$ & $1.49 \pm 15\%$ & $0.996 \pm 15\%$ & $0.677 \pm 16\%$ & $0.466 \pm 16\%$ & $0.325 \pm 17\%$\\\hline\hline
1050 & 1100 & 1150 & 1200 & 1250 & 1300\\\hline
$0.229 \pm 17\%$ & $0.163 \pm 18\%$ & $0.118 \pm 18\%$ & $0.0856 \pm 18\%$ & $0.0627 \pm 19\%$ & $0.0461 \pm 20\%$\\\hline\hline
1350 & 1400 & 1450 & 1500 & 1550 & 1600\\\hline
$0.0340 \pm 20\%$ & $0.0253 \pm 21\%$ & $0.0189 \pm 22\%$ & $0.0142 \pm 23\%$ & $0.0107 \pm 23\%$ & $0.00810 \pm 24\%$\\\hline\hline
\end{tabular}}

\resizebox{0.8\textwidth}{!}{
\begin{tabular}{|c|c|c|c|c|}
\hline\hline
Sbottom mass (GeV) & 400 & 450 & 500 & 550 \\\hline
Cross section (pb) & $1.84 \pm 14\%$ & $0.948 \pm 13\%$ & $0.518 \pm 13\%$ & $0.296 \pm 13\%$\\\hline\hline
600 & 650 & 700 & 750 & 800\\\hline
$0.175 \pm 13\%$ & $0.107 \pm 13\%$ & $0.0670 \pm 13\%$ & $0.0431 \pm 14\%$ & $0.0283 \pm 14\%$\\\hline\hline
\end{tabular}}
\end{table}

The signal processes are generated from leading order (LO) matrix elements with up to two extra partons (only one for the grid featuring slepton-mediated gluino decays), 
using the \textsc{Madgraph v5.2.2.3} generator~\cite{Alwall:2014hca} interfaced to \textsc{Pythia} 8.186~\cite{Sjostrand:2007gs} 
with the \textit{ATLAS 14} tune~\cite{ATL-PHYS-PUB-2014-021} for the modelling of the SUSY decay chain, parton showering, 
hadronization and the description of the underlying event. 
Parton luminosities are provided by the \textsc{NNPDF23LO}~\cite{Carrazza:2013axa} set of parton distribution functions. 
Jet-parton matching is realized following the CKKW-L prescription~\cite{Lonnblad:2011xx}, 
with a matching scale set to one quarter of the pair-produced superpartner mass. 

The signal samples are normalised to the next-to-next-to-leading order cross-section from Ref.~\cite{twiki-SusyCrossSections} 
including the re-summation of soft gluon emission at next-to-next-to-leading-logarithmic accuracy (NLO+NLL), 
as detailed in Ref.~\cite{Borschensky:2014cia}; 
some of these cross-sections are shown for illustration in Table~\ref{tab:signal_xsections}. 

Cross-section uncertainties are also taken from Ref.~\cite{twiki-SusyCrossSections} as well, 
and include contributions from varied normalization and factorization scales, as well as PDF uncertainties. 
They typically vary between 15 and 25\%. 
Uncertainties on the signal acceptance are not considered since 
 these are generally smaller than the uncertainties on the inclusive production cross-section. 

