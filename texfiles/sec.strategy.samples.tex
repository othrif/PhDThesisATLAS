\subsection{Collision Data}
The analysis uses $pp$--collisions data at $\sqrt s=13$ \TeV~ 
collected by the ATLAS detector during 2015 and 2016
 with a peak instantaneous luminosity of 
$L=1.4\times~10^{34}$~cm$^{-2}$s$^{-1}$.
The total integrated luminosity considered corresponds to 36.1 \ifb~ 
(3.2 \ifb~in 2015 and 32.9 \ifb~in 2016) recorded 
after applying beam, detector, and data-quality requirements.
The combined luminosity uncertainty for 2015 and 2016 is 3.2\%, 
assuming partially correlated uncertainties in 2015 and 2016.
The integrated luminosity was established following the same methodology as 
that detailed in Ref.~\cite{Aaboud:2016hhf},
from a preliminary calibration of the luminosity scale using a pair of $x$-$y$ 
beam separation scans.% performed in August 2015 and May 2016. 

\subsection{Simulated Event Samples}

Monte Carlo (MC) simulated event samples are used to model the SUSY signal 
and SM backgrounds. 
The irreducible SM backgrounds refer to processes that lead to two 
same-sign and/or three ``prompt'' leptons where the prompt leptons
are produced directly in the hard-scattering process, 
or in the subsequent decays 
of $W,Z,H$ bosons or prompt $\tau$ leptons. 
The reducible backgrounds, mainly 
arising from $\ttbar$ and $V$+jets production , are estimated either from data 
or from MC simulation as described in Section~\ref{chap:fake}. 

Table~\ref{tab:MC} presents the event generator, parton shower, cross-section 
normalization, PDF
set and the set of tuned parameters for the modelling of the parton shower, 
hadronization and underlying event. 
Apart from the MC samples produced by the \SHERPA generator, all MC samples
used the \textsc{EvtGen}~v1.2.0 program~\cite{EvtGen} 
to model the properties of bottom and charm hadron decays. 

\begin{table*}[!ht]
\begin{center}
\scriptsize
\resizebox{\textwidth}{!}
{
\begin{tabular}{|l|l|c|c|c|c|}
\hline
Physics process    & Event generator & Parton shower & Cross-section & PDF set & Set of tuned \\
                   &	      & 	      & normalization & 	& parameters  \\
\hline
\hline
Signal   	   	& \AMCATNLO 2.2.3~\cite{Alwall:2014hca} 	& \PYTHIA 8.186~\cite{Sjostrand:2007gs}	& NLO+NLL  & NNPDF2.3LO~\cite{Ball:2012cx} & A14~\cite{ATL-PHYS-PUB-2014-021} \\
\hline
\hline
$\ttbar +X$            &                      			&                      			&                               	&               &      \\
$\ttbar W$, $\ttbar Z/\gamma^{*}$ & \AMCATNLO 2.2.2 		& \PYTHIA 8.186	  			& NLO~\cite{YR4}     			& NNPDF2.3LO	& A14    \\
$\ttbar H$	   & \AMCATNLO 2.3.2        			& \PYTHIA 8.186  			& NLO~\cite{YR4}   			& NNPDF2.3LO	& A14  \\
4$t$    	& \AMCATNLO 2.2.2       			& \PYTHIA 8.186        			& NLO~\cite{Alwall:2014hca}	  	& NNPDF2.3LO	& A14  \\
\hline
Diboson            &                      			&                      			&                               	&               &      \\
$ZZ$, $WZ$    	   & \SHERPA 2.2.1~\cite{gleisberg:2008ta}      & \SHERPA 2.2.1				& NLO~\cite{ATL-PHYS-PUB-2016-002}	&NNPDF2.3LO & \SHERPA default \\
Other (inc. $W^{\pm}W^{\pm}$)   & \SHERPA 2.1.1 		& \SHERPA 2.1.1				& NLO~\cite{ATL-PHYS-PUB-2016-002}	&CT10~\cite{Lai:2010vv} & \SHERPA default \\
\hline
Rare               &                      			&                      			&                               	&               &      \\
$\ttbar WW$, $\ttbar WZ$     & \AMCATNLO 2.2.2       		& \PYTHIA 8.186      			& NLO~\cite{Alwall:2014hca}	  	& NNPDF2.3LO	 & A14  \\
$tZ$, $tWZ$, $t\ttbar$    & \AMCATNLO 2.2.2        		& \PYTHIA 8.186       			& LO		                   	& NNPDF2.3LO     & A14  \\
$WH$, $ZH$	   & \AMCATNLO 2.2.2        			& \PYTHIA 8.186      			& NLO~\cite{Dittmaier:2012vm}   	& NNPDF2.3LO     & A14  \\
Triboson	   & \SHERPA 2.1.1         			& \SHERPA 2.1.1        			& NLO~\cite{ATL-PHYS-PUB-2016-002}       & CT10	     	& \SHERPA default \\
\hline
Irreducible            &                      			&                      			&                               	&               &      \\
$\ttbar$    	   & \POWHEGBOX       		& \PYTHIA 6.428      			& NNLO+NNLL~\cite{Czakon:2011xx}	  	& CT10      	 & PERUGIA2012 (P2012) \cite{tt:perugia}\\
$W$+Jets, $Z$+Jets      & \POWHEGBOX       		& \PYTHIA 8.186      			& NNLO~\cite{Czakon:2011xx}	  	& CT10      	 & AZNLO\cite{AZNLO:2014}\\
\hline
\end{tabular}
}
\caption{Simulated signal and background event samples: the corresponding event generator, parton shower, cross-section normalization, PDF set and 
set of tuned parameters are shown for each sample. Because of their very small contribution to the signal-region background estimate, 
$\ttbar WW$, $\ttbar WZ$, $tZ$, $tWZ$, $t\ttbar$, $WH$, $ZH$ and triboson are summed and labelled ``rare'' in the following.}
\label{tab:MC}
\end{center}
\end{table*}

The MC samples were processed through either a full ATLAS detector 
simulation~\cite{Aad:2010ah} based on 
\textsc{Geant4}~\cite{Agostinelli:2002hh} or a fast simulation using a 
parameterization of the calorimeter response 
and \textsc{Geant4} for the ID and MS~\cite{ATL-PHYS-PUB-2010-013},
and are reconstructed in the same manner as the data. 
All simulated samples are generated with a range of minimum-bias interactions 
using {\sc Pythia 8}~\cite{Sjostrand:2007gs} 
with the MSTW2008LO PDF set~\cite{Sherstnev:2007nd} and the A2 tune overlaid on the hard-scattering event 
to account for the multiple $pp$ interactions in the same bunch crossing 
(in-time pileup) 
and neighbouring bunch crossing (out-of-time pileup). 
The distribution of the average number of interactions per bunch crossing 
$\langle\mu\rangle$ ranges from 0.5 to 39.5, 
with a profile set as an estimate of the combined 2015+2016 data 
$\langle\mu\rangle$ profile. 
With larger luminosity collected during this year and the $\mu$ distribution in data being closer to that in the MC profile,
the simulated samples are reweighted to reproduce the observed distribution 
of the average number of collisions per bunch crossing ($\mu$).


\subsection*{Background process simulation}


\subsection*{Signal cross-sections and simulations}

\begin{table}[t!]
\centering
\caption{Signal cross-sections [pb] and related uncertainties [\%] for scenarios featuring \glgl\ (top table) or \sbsb\  (bottom table) production, 
as a function of the pair-produced superpartner mass, reproduced from Ref.~\cite{twiki-SusyCrossSections}.}
\label{tab:signal_xsections}
\resizebox{\textwidth}{!}{
\begin{tabular}{|c|c|c|c|c|c|c|c|c|c|c|}
\hline\hline
Gluino mass (GeV) & 500 & 550 & 600 & 650 & 700 \\\hline
Cross section (pb) & $27.4 \pm 14\%$ & $15.6 \pm 14\%$ & $9.20 \pm 14\%$ & $5.60 \pm 14\%$ & $3.53 \pm 14\%$\\\hline\hline
750 & 800 & 850 & 900 & 950 & 1000\\\hline
$2.27 \pm 14\%$ & $1.49 \pm 15\%$ & $0.996 \pm 15\%$ & $0.677 \pm 16\%$ & $0.466 \pm 16\%$ & $0.325 \pm 17\%$\\\hline\hline
1050 & 1100 & 1150 & 1200 & 1250 & 1300\\\hline
$0.229 \pm 17\%$ & $0.163 \pm 18\%$ & $0.118 \pm 18\%$ & $0.0856 \pm 18\%$ & $0.0627 \pm 19\%$ & $0.0461 \pm 20\%$\\\hline\hline
1350 & 1400 & 1450 & 1500 & 1550 & 1600\\\hline
$0.0340 \pm 20\%$ & $0.0253 \pm 21\%$ & $0.0189 \pm 22\%$ & $0.0142 \pm 23\%$ & $0.0107 \pm 23\%$ & $0.00810 \pm 24\%$\\\hline\hline
\end{tabular}}

\resizebox{0.8\textwidth}{!}{
\begin{tabular}{|c|c|c|c|c|}
\hline\hline
Sbottom mass (GeV) & 400 & 450 & 500 & 550 \\\hline
Cross section (pb) & $1.84 \pm 14\%$ & $0.948 \pm 13\%$ & $0.518 \pm 13\%$ & $0.296 \pm 13\%$\\\hline\hline
600 & 650 & 700 & 750 & 800\\\hline
$0.175 \pm 13\%$ & $0.107 \pm 13\%$ & $0.0670 \pm 13\%$ & $0.0431 \pm 14\%$ & $0.0283 \pm 14\%$\\\hline\hline
\end{tabular}}
\end{table}

The signal processes are generated from leading order (LO) matrix elements with up to two extra partons (only one for the grid featuring slepton-mediated gluino decays), 
using the \textsc{Madgraph v5.2.2.3} generator~\cite{Alwall:2014hca} interfaced to \textsc{Pythia} 8.186~\cite{Sjostrand:2007gs} 
with the \textit{ATLAS 14} tune~\cite{ATL-PHYS-PUB-2014-021} for the modelling of the SUSY decay chain, parton showering, 
hadronisation and the description of the underlying event. 
For the RPV models, \textsc{Madgraph v5.2.3.3} and \textsc{Pythia} 8.210~\cite{Sjostrand:2007gs} were used instead. 
Parton luminosities are provided by the \textsc{NNPDF23LO}~\cite{Carrazza:2013axa} set of parton distribution functions. 
Jet-parton matching is realized following the CKKW-L prescription~\cite{Lonnblad:2011xx}, 
with a matching scale set to one quarter of the pair-produced superpartner mass. 

The signal samples are normalised to the next-to-next-to-leading order cross-section from Ref.~\cite{twiki-SusyCrossSections} 
including the resummation of soft gluon emission at next-to-next-to-leading-logarithmic accuracy (NLO+NLL), 
as detailed in Ref.~\cite{Borschensky:2014cia}; 
some of these cross-sections are shown for illustration in Table~\ref{tab:signal_xsections}. 
For the production of like-sign d-squark (RPV scenario), 
Prospino~\cite{Beenakker:1996ed} is used to scale the samples to their NLO cross-section.

Cross-section uncertainties are also taken from Ref.~\cite{twiki-SusyCrossSections} as well, 
and include contributions from varied normalization and factorization scales, as well as PDF uncertainties. 
They typically vary between 15 and 25\%. 
We do not consider any source of uncertainties on signal acceptance, 
experience having shown that these are generally smaller than the uncertainties on the inclusive production cross-section. 

