The hierarchy problem stated in the last section can be solved if the 
diagrams of Figure~\ref{fig:theory.sm.oneloopH} cancelled out. 
It is possible since the radiative contributions to the Higgs mass 
coming from the fermion loop has a minus sign while 
the boson loop contributes with a positive sign. 
By introducing a new symmetry between 
fermions and bosons, directly linking matter and gauge fields, 
the diagrams can mutually compensate one another.
This new symmetry is called \textit{supersymmetry} and commonly referred to as SUSY.
In this section, we describe briefly the principles of supersymmetry by focusing 
on the concepts rather than the technical implementation of the theory.
We also cover the motivations for examining supersymmetry and how it mitigates
many of the problems of the SM. Last but not least,
we will cover the phenomenology of the theory and its implications in a hadron 
collider like the LHC.

\subsection{Principles of Supersymmetry}

%% \subsection*{The $S$-matrix}
%% The scattering matrix, $S$-matrix, of quantum field theory is essential to the discussion that follows, 
%% so we start by defining it. The $S$-matrix is the unitary operator $S$ that determines the evolution 
%% of a state $ \Psi\left(t\right)$ at time $t$, from an initial state $ \left|\Phi_i \right>$  at $t=-\infty$ to a final state  $ \left|\Phi_f \right>$
%% at  $t=+\infty$. The $S$-matrix element, $S_{fi}$, is given by
%% \begin{equation}
%% \begin{split}
%% S_{fi} & = \lim_{t\to + \infty} \left<\Phi_f | \Psi\left(t\right)\right> \\
%% & \equiv \left< \Phi_f| \mathcal{H}_\text{int} | \Phi_i \right>,
%% \end{split}
%% \end{equation}
%% where $\mathcal{H}_\text{int}$ is the interaction Hamiltoinan for the system. 
%% In this sense, the $S$-matrix encodes the probabilities of various scattering processes of a given theory.
%% The perturbative treatment allows the calculation of the $S$-matrix which is commonly represented graphically
%% by Feynman diagrams.

%\subsection*{Symmetries}

The symmetries encountered in the SM are a direct product of the Poincare group that encodes the 
symmetries of space-time (translations, rotations, and boosts) and the internal symmetries 
($SU\left(3\right) \times SU\left(2\right) \times U\left(1\right)$). These symmetries
 do not affect the space-time geometric properties of the transformed states.
For instance, an isotropic rotation can transform a neutron into a proton, preserving the same spin, but cannot
transform a neutron into a pion, a particle with a different spin. 
Transformations of supersymmetry are very different in the sense that they directly associate 
fields of integer and half-integer spins, allowing fermions and bosons to be transformed into one another.

The development of the formalism of supersymmetry started with the famous no-go theorem from Coleman and 
Mandula \cite{Coleman:1967ad} who showed that there is no non-trivial way to mix the space-time symmetry 
group with the internal symmetry group in four dimensions and maintain non-zero scattering amplitudes. 
In this theorem, 
only commuting symmetry generators were considered which describe bosonic generators with integer spin.
Haag, Lopuszanski, and Sohnius generalized the theorem by extending the symmetry group to 
anticommutating generators that describe fermions \cite{Haag:1974qh}.
The super-Poincare group, which includes supersymmetry transformations linking bosons and 
fermions in addition to the other space-time symmetries, was constructed. 
The conclusion is that the most general framework for the symmetries of physics is a direct product of the 
super-Poincare group with the internal symmetry group.
 This group is represented by four supersymmetry generators $Q_\alpha$ and $\overline{Q}_{\dot{\alpha}}$, 
 where  $\alpha$ and  $\dot{\alpha}$ represent a left-handed and right-handed Weyl spinor index, respectively.
 They can act on a scalar state $\phi$ to obtain a spinor particle
 \begin{equation}
 Q_\alpha \left|\phi\right>  = \left|\psi_\alpha\right>,
 \end{equation}
 where the state $\psi$ represents a fermion. 
%%Inversely, if one acts on the fermion state
%% %\begin{equation}
%% %\epsilon^{\beta\alpha}Q_\beta \left|\psi_\alpha\right>  = \left|\phi\right>.
%% %\end{equation}
%% \begin{equation}
%% \overline{Q}_{\dot{\alpha}} \left|\psi_\alpha\right>  = \left|A_{\dot{\alpha}\alpha}\right> \neq 0,
%% \end{equation}
%% where the object $A_{\dot{\alpha}\alpha}$ 
The momentum operator and gauge transformation generators of internal symmetries commute with the operators 
$Q$ and $\bar{Q}$. As a result, the supersymmetric states contain bosonic and fermionic fields, 
commonly referred to as \textit{supermultiplet}. The supermultiplets can be either chiral, containing a boson
and a left-handed fermion, or anti-chiral, with a right-handed fermion.
As a consequence of the commutation properties, the particles within a supermultiplet have identical charges, 
such as the electric charge and colour charge, under all gauge symmetries.
The implication of this statement is that in a supersymmetric extension of the SM, there will be two 
superpartners, a boson and a fermion, with the same quantum numbers except spin. 

\subsection*{SUSY breaking} 

The momentum operator $P^{\mu}$ also commutes with $Q$, $\left[Q,P^\mu\right]=0$, which implies that if 
supersymmetry is exact, then that every bosonic state must have a corresponding fermionic partner with 
an identical mass. In other words, the supersymmetric partners must come in mass-degenerate pairs.
However, this possibility has been ruled out experimentally since we know that there is no superpartners 
with similar masses as the SM particles. Supersymmetry must then be a broken symmetry. 
%We return to this after 
%introducing the most common implementation of a supersymmetric version of the SM.
Supersymmetry breaking is not well understood, however, there is an appealing scheme that preserves 
most of the attractive features of supersymmetry which is known as \textit{soft supersymmetry breaking}.
In this scheme, the superpartner masses can be increased to an acceptable range with the current 
experimental bounds. Also, the scale of the mass splitting should be in the range of 
~$\mathrm{O}\left(100\right)$ \GeV to $\mathrm{O}\left(1\right)$ \TeV, since it can be linked to electroweak 
symmetry breaking\cite{Chung:2003fi}. 

\subsection{Supersymmetric Phenomenology}

\subsection*{Minimal Supersymmetric Standard Model} 

The simplest implementation of a supersymmetric SM is known as the Minimal Supersymmetric Standard Model, or 
MSSM. It is minimal since it contains the smallest number of new particle states and new interactions 
necessary such that the SM particles still exist in their current forms and within a supersymmetric framework.

In this model, each SM fermion is placed within a supermultiplet containing an additional boson. 
These new particles are called the same as their fermionic counterpart with a prepended `\textit{s-}'.
For instance, an electron ($e$) is partnered with a selectron ($\tilde{e}$), a quark ($q$) with a generic 
squark ($\tilde{q}$), etc. On the other hand, the SM bosons with spin 1, that is $B^0$, $W^{\pm}$, $W^0$ before 
electroweak symmetry breaking, are paired with fermionic superpartners with spin $\frac{1}{2}$ into gauge 
supermultiplets. These new particles are called the same as their bosonic counterpart but with the postfix 
`\textit{-ino}'. So we obtain gluinos (\gluino), winos ($\tilde{W}$) and binos ($\tilde{B}$).

The Higgs sector is chosen to consist of two left-chiral scalar superfields, $H_u$ and $H_d$, with different 
charges under $U\left(1\right)_Y$, $Y=1$ and $Y=-1$, respectively. The $H_u$ and $H_d$ supermultiplets are required 
since each gives mass to only the up or the down quarks. They are also introduced to ensure the cancellation 
of triangle anomalies in the SM, which would otherwise make the theory non-renormalizable.

\begin{table}
\scriptsize
%\small
\begin{center}
\vspace*{-0.035\textwidth}
\resizebox{1.\textwidth}{!}{
\begin{tabular}{ccccc}
\hline
\textbf{Particle group} & \textbf{Spin} & \textbf{$P_R$} & \textbf{Gauge eigenstates} & \textbf{Mass eigenstates} \\
\hline\hline
Higgs bosons & $0$ & $+1$ & \psm{H}{u}{0}, \psm{H}{d}{0}, \psm{H}{u}{+}, \psm{H}{d}{-} & \psm{h}{}{0}, \psm{H}{}{0}, \psm{A}{}{0}, \psm{H}{}{\pm}  \\
\hline
 & & & \psusy{u}{L}{}, \psusy{u}{R}{}, \psusy{d}{L}{}, \psusy{d}{R}{}  & (same) \\
squarks & $0$ & $-1$ &  \psusy{s}{L}{}, \psusy{s}{R}{}, \psusy{c}{L}{}, \psusy{c}{R}{}  & (same) \\
 & & & \stopL, \stopR, \sbottomL, \sbottomR & \stopone, \stoptwo, \sbottomone, \sbottomtwo\\
\hline
 & & & \psusy{e}{L}{}, \psusy{e}{R}{}, \psusy{\nu}{e}{} & (same) \\
sleptons & $0$ & $-1$ & \psusy{\mu}{L}{}, \psusy{\mu}{R}{}, \psusy{\nu}{\mu}{} &  (same) \\
 & & & \stauL, \stauR, \snustau & \stauone, \stautwo, \psusy{\nu}{\tau}{} \\
\hline
Neutralinos & $1/2$ & $-1$ & \psusy{B}{}{0}, \psusy{W}{}{0}, \psusy{H}{u}{0}, \psusy{H}{d}{0} & \ninoone, \ninotwo, \ninothree, \ninofour \\
\hline
Charginos & & & \psusy{W}{}{\pm}, \psusy{H}{u}{+}, \psusy{H}{d}{-} & \chinoonepm, \chinotwopm \\
\hline
gluino &  $1/2$ & $-1$ & \gluino & (same) \\
\hline
\end{tabular}}
\vspace*{-0.01\textheight}
\caption{Superpartners of the SM particles in the Minimal Supersymmetric Standard Model showing the mass eigenstates and which gauge eigenstates are mixed.
The two first generations of the squarks and sleptons are assumed to have negligible mixing.}
\label{tab:theory.susy.mssm}
\end{center}
\end{table}

The superpartners of the SM particles in the MSSM are shown in Table~\ref{tab:theory.susy.mssm}.
Since SUSY is a broken symmetry, the gaugino eigenstates mix with the Higgs multiplets to form a set of 
neutralinos ($\tilde{\chi}_{i}^{0}$, $i=1,2,3,4$), charginos ($\tilde{\chi}_{i}^{\pm}$, $i=1,2$), and Higgs bosons 
as a result of the SUSY breaking terms that are added. 
The neutralino and chargino states are ordered in terms of mass as
$m_{\ninoone} \leq m_{\ninotwo} \leq m_{\ninothree} \leq m_{\ninofour}$ and $m_{\chinoonepm} \leq m_{\chinotwopm}$.
It is worth noting that the MSSM expects to have five physical Higgs bosons: two CP-even Higgs bosons
$h^0$ and $H^0$, one CP-odd state $A^0$, and a pair of charged Higgs $H^\pm$. The observed Higgs boson of
125 \GeV~can be one of the two CP-even Higgs bosons. Unlike in the SM where the Higgs mass is one free parameter, 
%In the MSSM, 
the masses of the Higgs bosons at tree level and the mixing angle are expressed in terms of 
two parameters chosen to be the mass of $A^0$ ($m_A$) and the ratio of the two vacuum expectation values
($\tan \beta = v_u/v_d$). These vacuum expectation values, $v_u$ and $v_d$, correspond to the local minima 
of the scalar potential in which electroweak symmetry is spontaneously broken.

\subsection*{Simplified Models}

The details given so far have described the particle content of the MSSM. 
As far as the free parameters are concerned, in contrast to the SM which has 
nineteen free parameters, the MSSM has 124 free parameters. While a large portion of the 
parameter space is excluded, there are many degrees of freedom still remaining.
In principle, it is possible to reduce the number of parameters by making well-motivated assumptions
on the physics at higher energy scales. In fact, model builders attempt to formulate reasonable and 
economical models that are phenomenologically viable and falsifiable based on the current experimental results.

Another strategy is to completely decouple many of the particles in the SUSY spectrum, and assume a 100\% 
branching ratio for one specific decay mode, in what is known as \textit{simplified models}.
 In practice, the decoupling is achieved by arbitrarily 
tuning the SUSY breaking parameters in the Lagrangian to include the desired mass terms and couplings.
While such models are known to be not viable and may even 
break the renormalizability of the theory, they are considered as indicative  
of the reach of the analysis in probing the SUSY parameter space and can also be recast by theorists in terms 
of their own models. This is the strategy followed in most of the results shown in the analysis presented in this 
dissertation.


\subsection*{$R$-parity}

It is desirable to write down supersymmetric interactions that preserve baryon or lepton numbers since 
they are putatively good symmetries in the SM.
This can be achieved by requiring the conservation of a new quantity called \textit{$R$-parity}. 
For baryon number $B$, lepton number $L$, and particle spin $s$, the $R$-parity is defined as
\begin{equation}
P_R = \left(-1\right)^{3\left(B-L\right)+2s}.
\end{equation}
The MSSM is formulated as an $R$-parity conserving (RPC) theory. 
However, it can be extended to include a superpotential for the $R$-parity violating (RPV) interactions 
that can be written as
\begin{equation}
%W_{\slashed{P}_R} = \underbrace{\frac{1}{2}\lambda^{ijk}L_iL_jE_k + \lambda'^{ijk}L_iQ_jD_k - \kappa^i L_i H_d }_{\Delta L = 1} + \underbrace{\frac{1}{2}\lambda''^{ijk}U_iD_jD_k}_{\Delta B = 1},
W_{\slashed{P}_R} = \frac{1}{2}\lambda^{ijk}L_iL_jE_k + \lambda'^{ijk}L_iQ_jD_k - \kappa^i L_i H_d  + \frac{1}{2}\lambda''^{ijk}U_iD_jD_k,
\end{equation}
in which chiral quark and lepton superfields are denoted by $Q$, $U$, $D$ and $L$, $E$, respectively,
where $i$, $j$, and $k$ are flavour indices. 
The terms show the only interactions that violate baryon or lepton number conservation where 
$\lambda$ and $\lambda'$ couplings break lepton number conservation, while  $\lambda''$ coupling breaks 
baryon number conservation. The work presented in this dissertation will not address RPV scenarios
\cite{Barbier:2004ez}.


\subsection{The Hierarchy Problem}

As described previously, there are scalar and fermion loops that contribute to the radiative corrections 
to the Higgs mass that diverges as $\Lambda^2$. By introducing supersymmetric partners, the 
large fermionic  contribution  to the Higgs mass will be compensated by the scalar particle loop of the 
same mass but with an opposite sign. In the case of unbroken supersymmetry, this cancellation is exact and 
will thus eliminate the fine-tuning problem. However, we know that supersymmetry must be broken.
Naturalness is introduced to place limits on the masses of certain superpartners in order not to replace the 
fine-tuning problem of the SM with another in a supersymmetric model. As a result, there is strong 
motivation for having supersymmetry in the weak scale which will inevitably stabilize the electroweak 
symmetry breaking of the SM which suffered from the fine-tuning problem.

There are other benefits of supersymmetry that are beyond the scope of this 
work. We refer the reader to the literature 
\cite{howiebook,Martin:1997ns,Ellis:2015cva}.
