The Standard Model (SM) of particle physics is a description of the physical world around us in terms of fundamental particles 
and their interactions. 
The developement of the SM has been guided by both theoretical predictions and experimental discoveries.
The SM includes three of the four fundamental forces: electromagnetism, the strong interaction, and the weak interaction.
The mathematical formalism used relies on quantum field theory. %cite??

The fundamental particles are represented by the states of quantized fields.
Quarks and leptons constitute matter and are associated with fields of half integer spin, called fermion fields.
The dynamics of the system is defined by the Lagrangian, $\mathcal{L}$, a quantity that describes the motion and excitations 
in the fields.
The Lagrangian of the SM is invariant under spacetime dependent continuous internal transformations of the group 
$ SU\left(3\right) \times SU\left(2\right) \times U\left(1\right) $.
This invariance is called gauge invariance and is neccessary to ensure that the theory is renormalizable.
The renormalizability condition guarantees the predictive power of the theory.
To preserve gauge invariance, additional quantum fields with spin one are required, called gauge bosons.
As a result, twelve gauge fields are required to write a gauge invariant Lagrangian, 
 eight for the generators of $ SU\left(3\right) $, three for the generators of $SU\left(2\right)$, 
and one for the $U\left(1\right)$ generator.
The elements described are enough to write down the langrangian of the SM.

\subsection{Quantum Chromodynamics}

The $ SU\left(3\right) $ gauge symmetry coupled to the quarks describes Quantum chromodynamics (QCD), the theory of strong interactions.
The eight $ SU\left(3\right) $ gauge fields are associated to the different colored states of the gluon.
The QCD Lagrangian is given by

\begin{equation}
\mathcal{L} QCD = \frac{1}{4} G
\end{equation}

where $G$ are the gauge fields of QCD given by 

\begin{equation}
,
\end{equation}

and the covariant derivative, $D\mu$,  defined as

\begin{equation}
,
\end{equation}

where $gS$ is the strong coupling constant, and $\lambda A$ are the eight GellMass matrices.
The indices of the quarks, $i=1,2,3$, run over the colors: red, blue, green, and their anticolors.
While the indices of the gluons, $A,B,C = 1, \cdots, 8$, correspond to the combinations of colors and anticolors.
Color must be conserved in all QCD interactions, similar to the  electric charge.
Gluons have been observed experimentally cite?? and interact with quarks as predicted by the SM.

\subsection{The Electroweak Theory}

The $ SU\left(2\right) \times U\left(1\right) $ gauge symmetry describes the
electroweak theory that unifies the electromagnetic and weak interactions.
There are two problems with this part of the SM.
The four gauge fields of $ SU\left(2\right) $ and $ U\left(1\right) $ 
must be added without mass to preserve gauge invariance.
However, the gauge bosons of the weak force have a large mass according to observation,
and thus in direct contradiction with the prediction.
This problem was not relevant in the case of QCD since the gluons are massless. 
In addition, the weak interaction violates parity where it couples differently 
to the left and right-handed quark and lepton helicity states.
The solution is to treat the two helicity states of the leptons as different fields
with different couplings. A fermion mass term in the Lagrangian would couple to 
these different fields but will break gauge invariance.
Again to maintain gauge invariance, the fermion fields should be massless in 
direct contradiction with observation.

Both of the problems described can be resolved by introducing 
sponteneous symmetry breaking. The idea is to introduce new scalar fields with 
zero spin that couple to the electroweak  $ SU\left(2\right) \times U\left(1\right) $
gauge fields while preserving the gauge invariance of the Lagrangian.


 







