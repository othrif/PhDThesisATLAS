The chapters thus far described supersymmetry as an extension of the standard 
model, motivated the need to search for it, and described the experimental 
apparatus used for this search, the LHC and ATLAS. 
This chapter begins the discussion of the search for supersymmetry that the 
author has performed.

The main task in designing a search for new physics signatures is to ensure 
that the search regions, expected to have an enhancement of the signal
 (also referred to as the signal regions), are sensitivity to 
 a wide 
range of new physics models that are well motivated.
In addition, it is important to design the search in a way that minimizes the
contamination from the known physics processes of the Standard Model.
In other words, a typical signal region should have a maximum expected 
signal with a minimum expected background. 

The search for supersymmetry with two leptons of the same-electric charge
or more than three leptons meets both criteria. 
The search targets the strong production of supersymmetric particles 
which mainly involves gluino pair production. Since gluinos are Majorana 
fermions, they can decay to either a positive or negative lepton in each branch
of the pair production. As a result, each branch is as likely to have two leptons 
of the same electric charge as it is to have two leptons of opposite electric
charge. While opposite-sign lepton production is a common signature in the 
Standard Model, the same-sign lepton signature is very rare. 
Also, processes that involve more than two leptons are rare in the 
Standard Model. On one hand the electroweak processes leading to 
$W$ and $Z$ bosons have a low cross sections. On the other hand the low 
branching ratio to leptons leads to an extreme background reduction.
The requirement of three or more leptons allows the analysis 
to target supersymmetric models with longer decay chains. The presence of 
a third softer lepton also increases the sensitivity to scenarios with 
small mass difference between the supersymmetric particles.

The next important step in a search for new physics is to estimate the 
backgrounds present in the signal regions.
The Standard Model backgrounds with a same-sign lepton pair or three or more 
leptons predominantly comes
from the associated production of a top quark pair and a vector boson (\ttbar + $W$, \ttbar + $Z$), and multi-boson production (di-boson and tri-boson). 
These backgrounds that lead to a signature with same-sign leptons or three 
or more leptons are referred to as \textit{irreducible backgrounds}.
However, the high cross section processes from the Standard Model such as \ttbar, might contribute to the signal regions via a mis-reconstruction of this 
process by the detector. As a result, there are two very important backgrounds
 that affect the analysis that are referred to as 
\textit{reducible backgrounds}.
The top quark pair production (\ttbar) process may decay fully leptonically 
($\ttbar \to \left(b\ell^+\bar{\nu}\right)\left(\bar{b}\ell^{-}\nu\right)$) 
and contribute to the signal regions if the lepton charge is mis-measured.
In the case of a semi-leptonic decay of \ttbar 
($\ttbar \to \left(b\ell^+\bar{\nu}\right)\left(\bar{b}q \bar{q}'\right)$)
where the hadronic decay is mis-identified as a leptonic decay, 
the process will contribute with a ``fake'' lepton in the signal regions.
The background estimation methodology will aim at estimating both 
reducible and irreducible backgrounds with Monte Carlo simulation and 
data-driven methods.

Finally, we assess the compatibility between the observed data and the 
predicted background in one counting experiment for each signal region 
by doing a hypothesis test
of the background-only or the background as well as the sought 
after signal hypotheses.
If an excess is found in data, we proceed to evaluate if this excess can lead
to a rejection of the background-only hypothesis and we check the plausibility that 
the new signal can describe the data.
Otherwise, we set exclusion limits for a certain region of the parameter 
space of a defined model or we set model-independent upper limits on the number 
of events from a beyond the Standard Model process. 
