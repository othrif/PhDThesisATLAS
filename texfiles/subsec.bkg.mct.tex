As discussed in Section~\ref{sec:fake.mct}, the MC template method is
 a simulation-based method that provides an alternative estimate of the reducible backgrounds affecting the analysis.
It relies on the correct modelling of FNP leptons and charge-flipped electron kinematics in $\ttbar$ 
and $V$+jets.
FNP leptons are classified in four categories, namely electrons and muons coming 
from $b$ and light-quark jets. Normalisation factors for each of the five sources (including phton conversions) are computed to match the observed data 
in dedicated control regions. The fifth category is events with prompt electrons that have a charge mis-measurement 
(charge flip, EL HF, MU HF, EL LF, MU LF).
Six non-overlapping control regions are defined by the presence of $b$-jets and by the flavors of the same sign lepton pair in the event:
\begin{itemize}
\item CR0b: events without $b$-jets in $ee$, $e\mu$, and $\mu\mu$ channels.
\item CR1b: events with at least one $b$-jet in $ee$, $e\mu$, and $\mu\mu$ channels.
\end{itemize}
All the selected events contain two or more same-sign signal leptons and \\$\met >40~\GeV$ and 2 or more jets. 
Events satisfying the signal regions requirements are excluded from the control regions. 
The purpose of the \met requirement is to remove multi-jet events that have two or more FNP leptons and tend to have low \met. 
The six distributions are chosen for variables that provide the best separation between processes with prompt leptons and processes with 
fake leptons and charge flip and are shown 
before and after the fit in Figures \ref{f:prefit_CR0b}-\ref{f:prefit_CR1b} and Figures \ref{f:postfit_CR0b}-\ref{f:postfit_CR1b}, 
respectively. 
The multipliers obtained after the minimization of the negative log likelihood were given 
in Tables \ref{t:fake_factors_powheg} and \ref{t:fake_factors_sherpa}.
The tables represent the correction factors obtained from the fit upon using two different parton showers, Pythia and Sherpa
for the processes that lead to non-prompt leptons and charge flips.
The goal of varying the parton shower is to access the dependence of the fake and charge flip estimates on the choice of the 
parton shower. 

The MC template method is validated by looking at the agreement 
between observed data and prediction as shown in Figure~\ref{fig:bkg.val.met}
and Figure~\ref{fig:bkg.val.met}.
In the MC template method, the systematic uncertainty is obtained by
changing the generator from \POWHEGBOX to \SHERPA and propagating uncertainties from the control region fit to the global
normalization scale factors applied to the MC samples. 
The uncertainties in these scale factors are in the range 75--80\%,
depending on the SRs.
In practice, only \ttbar contributes to the SRs and the final yields with systamtic uncertainties from 
fit uncertainty, theory uncertainties on ttbar, comparison of different showers (Pythia 6 and Sherpa) is shown in Table \ref{tab:fakes_mcglobal}.
This table also shows a global correction factor derived by taking the ratio of the weighted \ttbar to raw MC \ttbar with
a global uncertainty that includes all systematic uncertainties used to obtain the final estimate. 

\linespread{1.0}
\begin{table}[!htb]
\caption{Expected yields for background processes with fake leptons,
in the signal regions with a global correction factor that represents the ratio of weighted \ttbar to raw MC \ttbar with
a global uncertainty that includes: fit uncertainty, theory uncertainties on ttbar, comparison of different showers. 
The fraction of the systematic uncertainty from the comparison between two showers (Pythia and Sherpa) is also shown.
}
\label{tab:fakes_mcglobal}
\centering
\resizebox{\textwidth}{!}{
\begin{tabular}{|c||c|c|c||}\hline
Region &              MC Template method  & Global correction & Shower systematic \\
\hline
Rpc2L0bH & $1.00 \pm 0.96 \pm 0.81$ & $2.80 \pm 2.10$ & 74\% \\
Rpc2L0bS & $1.68 \pm 1.02 \pm 1.26$ & $2.89 \pm 1.97$ & 65\% \\
Rpc2L1bH & $2.07 \pm 0.63 \pm 1.56$ & $1.22 \pm 1.14$ & 34\% \\
Rpc2L1bS & $2.33 \pm 1.17 \pm 2.10$ & $1.83 \pm 1.42$ & 81\% \\
Rpc2L2bH & $<0.5$ & $0 \pm 0$ & 0\% \\
Rpc2L2bS & $0.41 \pm 0.33 \pm 0.45$ & $1.47 \pm 1.12$ & 73\% \\
Rpc2Lsoft1b & $2.48 \pm 1.32 \pm 1.86$ & $1.59 \pm 1.31$ & 68\% \\
Rpc2Lsoft2b & $1.66 \pm 0.66 \pm 1.28$ & $1.72 \pm 1.29$ & 54\% \\
Rpc3L0bH & $<0.5$ & $0 \pm 0$ & 0\% \\
Rpc3L0bS & $0.21 \pm 0.15 \pm 0.16$ & $2.90 \pm 2.20$ & 71\% \\
Rpc3L1bH & $0.42 \pm 0.29 \pm 0.32$ & $1.59 \pm 1.25$ & 59\% \\
Rpc3L1bS & $3.55 \pm 1.80 \pm 2.76$ & $1.76 \pm 1.32$ & 67\% \\
Rpc3LSS1b & $0.90 \pm 0.14 \pm 0.69$ & $2.34 \pm 1.44$ & 56\% \\
Rpv2L0b & $1.02 \pm 0.96 \pm 0.76$  & $2.80 \pm 2.10$ & 66\% \\
Rpv2L1bH & $0.60 \pm 0.35 \pm 0.48$ & $1.32 \pm 0.92$ & 45\% \\
Rpv2L1bM & $1.20 \pm 0.69 \pm 0.95$ & $1.59 \pm 1.25$ & 58\% \\
Rpv2L1bS & $4.46 \pm 1.67 \pm 3.45$ & $1.33 \pm 0.80$ & 67\% \\
Rpv2L2bH & $<0.5$ & $0 \pm 0$ & 0\% \\
Rpv2L2bS & $7.24 \pm 2.36 \pm 5.43$ & $2.45 \pm 1.83$ & 73\% \\
\hline
\hline
\end{tabular}
}
\end{table}
\linespread{1.6}
