\subsection*{Approximate sampling distributions}

The remaining task is to determine the sampling distributions
$f\left(q_{0} \rvert 0\right)$ and $f\left(q_{\mu} \rvert \mu\right)$
needed to compute the $p$-values
used in the case of discovery and setting upper limits, respectively.
These distributions do not have an analytic form but can be obtained from 
pseudo-experiments or asymptotic approximations.
The pseudo-experiments are more accurate than the asymptotic 
approximations since a large number of 
datasets are generated which are drawn from a distribution that is consistent 
with those observed.
However, for a complex likelihood function where the procedure of generating 
pseudo-datasets needs to be repeated many times for each parameter point
of each model being considered, the pseudo-data set method is computing 
intensive and not practical. For this reason, the current analysis 
uses asymptotic formulae to approximate $q_{\mu}$ and shown to be valid in the 
large sample size limit \cite{Cowan:2010js}. The approximation is based on an important result 
by Wilks \cite{Wilks:1938dza} and Wald \cite{Wald:ams1943} who showed that for a single parameter of interest,
\begin{equation}
  q_{\mu} =
  \begin{cases}
    \frac{\left(\mu - \hat{\mu}\right)^2}{\sigma^2} + \mathcal{O}\left(\frac{1}{\sqrt{N}}\right)
    , & \text{if}\ \mu > \hat{\mu} \\
    0, & \text{otherwise}.
  \end{cases}
\label{eq:stat.model.asymp}
\end{equation}
where $\mu$ is a Gaussian distribution with a mean $\bar{\mu}$ and 
standard deviation $\sigma$, and a sample size $N$. In the case where 
$\mu = \hat{\mu}$, the test statistic $q_{\mu}$ follows a $Chi^2$ distribution
for one degree of freedom. 
The variance $\sigma^2$ is obtained from an artificial data set called the 
``Asimov data set'' that verifies $X_A = \bar{\mu} s + b$. 
From Eq.\ref{eq:stat.model.asymp}, the variance is then 
$\sigma^2 = \left( \mu - \bar{\mu} \right)^2/q_{\mu,A}$, where $q_{\mu,A}$ 
is evaluated from the exact expression of $q_{\mu}$ using the Asimov data set.

The results obtained from the asymptotic approximations have been compared to 
exclusion limits obtained with a limited number of pseudo-experiments.
A reasonable agreement has been observed which validated the use of 
the asymptotic formalism to obtain the exclusion limits on the different models.
