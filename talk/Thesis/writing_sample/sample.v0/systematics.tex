
\subsection{Experimental systematics}
\label{sec:syst_exp}

All the  experimental systematics provided by the SUSYTools {\tt getSystInfoList()} method have been considered. 
The list of sources of uncertainty and the corresponding names of the variations are:

\textbf{Jet energy scale ({\tt{JET\_GroupedNP\_\{1-3\}\_\_1\{up,down\}}})}  \\
One of the strongly reduced uncertainty sets provided by the JetEtMiss group for early Run-2 searches is used in this note. 
These sets are intended for use by analyses which are not sensitive to jet-by-jet correlations arising from changes to the jet energy scale 
(as expected for many early SUSY searches), 
and we use the scenario {\tt InsituJES2012\_3NP\_Scenario1.config} (as included in the JetUncertainties package). 
We checked that the uncertainties obtained from one of the 3 other scenarios did not lead to significant changes. 
The jet energy is scaled up and down (in a fully correlated way) by the $\pm 1\sigma$ uncertainty of each nuisance parameter.

\textbf{Jet energy resolution ({\tt{JET\_JER\_SINGLE\_NP\_\_1up}})} \\
An extra $\pt$ smearing is added to the jets based on their $\pt$ and $\eta$ to account for a possible underestimate of the jet energy resolution in the MC simulation. This is done by the {\tt JERSmearingTool} in the JetResolution package.

\textbf{Egamma resolution ({\tt{EG\_RESOLUTION\_ALL\_\_1\{up,down\}}})} \\
A nuisance filtering scheme to reduce the $\sim$8 NPs used to electron and photon resolution to only one as implemented in SUSYTools is used.

\textbf{Egamma scale ({\tt{EG\_SCALE\_ALL\_\_1\{up,down\}}})}\\
A nuisance filtering scheme to reduce the $\sim$16 NPs used to electron and photon resolution to only one as implemented in SUSYTools is used.

\textbf{Electron efficiency ({\tt{EL\_EFF\_\{ID,RECO,TRIGGER,Iso\}\_TotalCorrUncertainty\_\_1\{up,down\}}})} \\
These uncertainty sources are associated with the electron efficiency scale factors provided by the Egamma CP group.

\textbf{Muon efficiency ({\tt{MUON\_EFF\_\{STAT,SYS\}\_\_1\{up,down\}}}, \\
{\tt{MUON\_EFF\_Trig\{Stat,Syst\}Uncertainty\_\_1\{up,down\}}}, \\
{\tt{MUON\_ISO\_\{STAT,SYS\}\_\_1\{up,down\}}}) }\\
This uncertainty corresponds to the statiscal and systematic uncertainties in the muon efficiency scale factors provided by the Muon CP group on the muon reconstruction, trigger and isolation.

\textbf{Muon resolution uncertainty  ({\tt{MUONS\_ID\_\_1\{up,down\}}}, {\tt{MUONS\_MS\_\_1\{up,down\}}})} \\
This is evaluated as variations in the smearing of the inner detector and muon spectrometer tracks associated to the muon objects by $\pm 1\sigma$ their uncertainty

\textbf{Muon momentum scale ({\tt{MUONS\_SCALE\_\_1\{up,down\}}})} \\
This is evaluated as variations in the scale of the momentum of the muon objects

\textbf{\met\ soft term uncertainties  ({\tt{MET\_SoftTrk\_Reso\{Pare,Perp\}}}, {\tt{MET\_SoftCalo\_Scale\{Up,Down\}}})}\\
Note that the effect of the hard object uncertainties (most notably JES and JER) are also propagated to the $\met$.

\textbf{Flavor tagging ({\tt{FT\_EFF\_\{B,C,Light\}\_systematics\_\_1\{up,down\}}}, \\
{\tt{FT\_EFF\_\{B,C,Light\}\_extrapolation\_\_1\{up,down\}}}, \\
{\tt{FT\_EFF\_\{B,C,Light\}\_extrapolation from charm\_\_1\{up,down\}}})} \\
Similarly to the case of the JES, a significant reduction in the number of nuisance parameters was provided by the Flavour Tagging CP group at the beginning of Run-2.\\

\textbf{Pileup reweighting ({\tt{PRW\_DATASF\_\_1\{up,down\}}})}\\
This uncertainty is obtained by re-scaling the $\mu$ value in data by 1.00 and 1/1.23, covering the full difference between applying and not-applying the nominal $\mu$ correction of 1/1.16, as well as uncertainty on the luminosity measurement which is expected to dominate.\\


\subsection{Theoretical systematics}
\label{sec:syst_theo}

% In Run-1~\cite{paperSS3L,noteSS3L}, the systematic uncertainties associated with the $t \bar{t} + V$ and
% diboson MC were assessed by using samples where the factorization and
% renormalization scales have been varied. We plan to follow a similar approach, 
% following the recommendations by the Physics Modeling Group.
%once scale variation samples are available in mc15 productions.
%Other theoretical uncertainties were evaluated by comparing different generators and samples produced with different
%number of partons. A similar approach will be used in 2015, following the recommendations by the Physics Modeling Group.
The theoretical uncertainties on the $t \bar{t} + V$ production cross-sections are $22\%$ for $t\bar{t}W$ \cite{Campbell:2012dh} 
and $t\bar{t}Z$~\cite{Garzelli:2012bn} 
In addition, uncertainties on the signal region fiducial acceptance for these processes were assessed 
by using MC samples with varied factorization and renormalization scales. 
This led to overall uncertainties of $30\%$ on the $t \bar{t} + V$ contributions to the signal regions.  

For inclusive diboson production, cross-section uncertainties amount to $7\%$ (computed with MCFM \cite{Campbell:2011bn}. 
Delays in production of samples with scale variations prevented us from checking scale impact on the fiducial acceptance; 
however, after comparisons between {\sc Sherpa} and {\sc Powheg} predictions, 
we established an overall 30\% uncertainty as well for these processes. 

Normalisation uncertainties between 35\% and 100\% were applied to processes with smaller contributions (triboson 
production, $t\bar{t}h$, $t+Z$, etc.).

% For dibosons, the theory uncertainties will be evaluated by comparing the results from the nominal {\sc Sherpa} samples 
% with alternative samples generated with {\sc Powheg}. Note that currently {\sc Powheg} samples exist for all diboson 
% processes except same-sign $W^\pm W^\pm jj$ production, which has important contribution to the signal regions without $b$-jets.
% Also note that with an increase in production cross section of a factor $\sim$4 between $\sqrt{s}=8$~TeV and 13~TeV, $\ttbar H$ 
% production can be a relevant background in the signal regions containing $b$-jets. The theory uncertainties for this process 
% will be evaluated in a similar way as in other analyses in the Higgs working group.


