%-------------------------------------------------------------------------------
% This file provides a skeleton ATLAS document.
%-------------------------------------------------------------------------------
% \pdfoutput=1
% The \pdfoutput command is needed by arXiv/JHEP/JINST to ensure use of pdflatex.
% It should be included in the first 5 lines of the file.
%-------------------------------------------------------------------------------
% This file assumes that atlaslatex is installed in a central place like ~/texmf.
%-------------------------------------------------------------------------------
\documentclass[texmf,UKenglish,texlive=2013]{atlasdoc}
% The language of the document must be set: usually UKenglish or USenglish.
% british and american also work!
% Commonly used options:
%  texlive=YYYY          Specify TeX Live version (2013 is default).
%  atlasstyle=true|false Use ATLAS style for document (default).
%  coverpage             Create ATLAS draft cover page for collaboration circulation.
%                        See atlas-draft-cover.tex for a list of variables that should be defined.
%  cernpreprint          Create front page for a CERN preprint.
%                        See atlas-preprint-cover.tex for a list of variables that should be defined.
%  PAPER                 The document is an ATLAS paper (draft).
%  CONF                  The document is a CONF note (draft).
%  PUB                   The document is a PUB note (draft).
%  BOOK                  The document is of book form, like an LOI or TDR (draft)
%  txfonts=true|false    Use txfonts rather than the default newtx - needed for arXiv submission.
%  paper=a4|letter       Set paper size to A4 (default) or letter.

%-------------------------------------------------------------------------------
% Extra packages:
\usepackage{atlaspackage}
% Commonly used options:
%  biblatex=true|false   Use biblatex (default) or bibtex for the bibliography.
%  backend=biber         Use the biber backend rather than bibtex.
%  subfigure|subfig|subcaption  to use one of these packages for figures in figures.
%  minimal               Minimal set of packages.
%  default               Standard set of packages.
%  full                  Full set of packages.
%-------------------------------------------------------------------------------
% Style file with biblatex options for ATLAS documents.
\usepackage{atlasbiblatex}

% Package for creating list of authors and contributors to the analysis.
\usepackage{atlascontribute}

% Useful macros
\usepackage{atlasphysics}
% See doc/atlas_physics.pdf for a list of the defined symbols.
% Default options are:
%   true:  journal, misc, particle, unit, xref
%   false: BSM, heppparticle, hepprocess, hion, jetetmiss, math, process, other, texmf
% See the package for details on the options.

% Files with references for use with biblatex.
% Note that biber gives an error if it finds empty bib files.
\addbibresource{atlas-document.bib}
\addbibresource{ATLAS.bib}

% Paths for figures - do not forget the / at the end of the directory name.
\graphicspath{{logos/}{figures/}}

% Add you own definitions here (file atlas-document-defs.sty).
\usepackage{atlas-document-defs}

%-------------------------------------------------------------------------------
% Generic document information
%-------------------------------------------------------------------------------

% Title, abstract and document 
%-------------------------------------------------------------------------------
% This file contains the title, author and abstract.
% It also contains all relevant document numbers used by the different cover pages.
%-------------------------------------------------------------------------------

% Title
\AtlasTitle{AtlasTitle: Bare bones ATLAS document}

% Author - this does not work with revtex (add it after \begin{document})
\author{The ATLAS Collaboration}

% Authors and list of contributors to the analysis
% \AtlasAuthorContributor also adds the name to the author list
% Include package latex/atlascontribute to use this
% Use authblk package if there are multiple authors, which is included by latex/atlascontribute
% \usepackage{authblk}
% Use the following 3 lines to have all institutes on one line
% \makeatletter
% \renewcommand\AB@affilsepx{, \protect\Affilfont}
% \makeatother
% \renewcommand\Authands{, } % avoid ``. and'' for last author
% \renewcommand\Affilfont{\itshape\small} % affiliation formatting
% \AtlasAuthorContributor{First AtlasAuthorContributor}{a}{Author's contribution.}
% \AtlasAuthorContributor{Second AtlasAuthorContributor}{b}{Author's contribution.}
% \AtlasAuthorContributor{Third AtlasAuthorContributor}{a}{Author's contribution.}
% \AtlasContributor{Fourth AtlasContributor}{Contribution to the analysis.}
% \author[a]{First Author}
% \author[a]{Second Author}
% \author[b]{Third Author}
% \affil[a]{One Institution}
% \affil[b]{Another Institution}

% If a special author list should be indicated via a link use the following code:
% Include the two lines below if you do not use atlasstyle:
% \usepackage[marginal,hang]{footmisc}
% \setlength{\footnotemargin}{0.5em}
% Use the following lines in all cases:
% \usepackage{authblk}
% \author{The ATLAS Collaboration%
% \thanks{The full author list can be found at:\newline
%   \url{https://atlas.web.cern.ch/Atlas/PUBNOTES/ATL-PHYS-PUB-2016-007/authorlist.pdf}}
% }

% Draft version:
% Should be 1.0 for the first circulation, and 2.0 for the second circulation.
% If given, adds draft version on front page, a 'DRAFT' box on top of each other page, 
% and line numbers.
% Comment or remove in final version.
\AtlasVersion{0.1}

% ATLAS reference code, to help ATLAS members to locate the paper
\AtlasRefCode{GROUP-2016-XX}

% ATLAS note number. Can be an COM, INT, PUB or CONF note
% \AtlasNote{ATLAS-CONF-2016-XXX}
% \AtlasNote{ATL-PHYS-PUB-2016-XXX}
% \AtlasNote{ATL-COM-PHYS-2016-XXX}

% CERN preprint number
% \PreprintIdNumber{CERN-PH-2016-XX}

% ATLAS date - arXiv submission; usually filled in by the Physics Office
% \AtlasDate{\today}

% ATLAS heading - heading at top of title page. Set for TDR etc.
% \AtlasHeading{ATLAS ABC TDR}

% arXiv identifier
% \arXivId{14XX.YYYY}

% HepData record
% \HepDataRecord{ZZZZZZZZ}

% Submission journal and final reference
% \AtlasJournal{Phys.\ Lett.\ B.}
% \AtlasJournalRef{\PLB 789 (2014) 123}
% \AtlasDOI{}

% Abstract - % directly after { is important for correct indentation
\AtlasAbstract{%
  This is a bare bones ATLAS document. Put the abstract for the document here.
}

%-------------------------------------------------------------------------------
% The following information is needed for the cover page. The commands are only defined
% if you use the coverpage option in atlasdoc or use the atlascover package
%-------------------------------------------------------------------------------

% List of supporting notes  (leave as null \AtlasCoverSupportingNote{} if you want to skip this option)
% \AtlasCoverSupportingNote{Short title note 1}{https://cds.cern.ch/record/XXXXXXX}
% \AtlasCoverSupportingNote{Short title note 2}{https://cds.cern.ch/record/YYYYYYY}
%
% OR (the 2nd option is deprecated, especially for CONF and PUB notes)
%
% Supporting material TWiki page  (leave as null \AtlasCoverTwikiURL{} if you want to skip this option)
% \AtlasCoverTwikiURL{https://twiki.cern.ch/twiki/bin/view/Atlas/WebHome}

% Comment deadline
% \AtlasCoverCommentsDeadline{DD Month 2016}

% Analysis team members - contact editors should no longer be specified
% as there is a generic email list name for the editors
% \AtlasCoverAnalysisTeam{Peter Analyser, Susan Editor1, Jenny Editor2, Alphonse Physicien}

% Editorial Board Members - indicate the Chair by a (chair) after his/her name
% Give either all members at once (then they appear on one line), or separately
% \AtlasCoverEdBoardMember{EdBoard~Chair~(chair), EB~Member~1, EB~Member~2, EB~Member~3}
% \AtlasCoverEdBoardMember{EdBoard~Chair~(chair)}
% \AtlasCoverEdBoardMember{EB~Member~1}
% \AtlasCoverEdBoardMember{EB~Member~2}
% \AtlasCoverEdBoardMember{EB~Member~3}

% A PUB note has readers and not an EdBoard -- give their names here (one line or several entries)
% \AtlasCoverReaderMember{Reader~1, Reader~2}
% \AtlasCoverReaderMember{Reader~1}
% \AtlasCoverEdBoardMember{Reader~2}

% Editors egroup
% \AtlasCoverEgroupEditors{atlas-GROUP-2016-XX-editors@cern.ch}

% EdBoard egroup
% \AtlasCoverEgroupEdBoard{atlas-GROUP-2016-XX-editorial-board@cern.ch}


% Author and title for the PDF file
\hypersetup{pdftitle={ATLAS document},pdfauthor={The ATLAS Collaboration}}

%-------------------------------------------------------------------------------
% Content
%-------------------------------------------------------------------------------
\begin{document}

\maketitle

\tableofcontents

% List of contributors - print here or after the Bibliography.
%\PrintAtlasContribute{0.30}
%\clearpage

%-------------------------------------------------------------------------------
\section{Introduction}
\label{sec:intro}
%-------------------------------------------------------------------------------

Place your introduction here.


%-------------------------------------------------------------------------------
\section{ATLAS detector}
\label{sec:detector}
%-------------------------------------------------------------------------------

The ATLAS detector~\cite{PERF-2007-01} ...
% % Footnote with ATLAS coordinate system
\newcommand{\AtlasCoordFootnote}{%
ATLAS uses a right-handed coordinate system with its origin at the nominal interaction point (IP)
in the centre of the detector and the $z$-axis along the beam pipe.
The $x$-axis points from the IP to the centre of the LHC ring,
and the $y$-axis points upwards.
Cylindrical coordinates $(r,\phi)$ are used in the transverse plane, 
$\phi$ being the azimuthal angle around the $z$-axis.
The pseudorapidity is defined in terms of the polar angle $\theta$ as $\eta = -\ln \tan(\theta/2)$.
Angular distance is measured in units of $\Delta R \equiv \sqrt{(\Delta\eta)^{2} + (\Delta\phi)^{2}}$.}

%-------------------------------------------------------------------------------
\subsection{ATLAS detector example from a Letter}
\label{sec:atlas1}
%-------------------------------------------------------------------------------

The ATLAS experiment~\cite{PERF-2007-01} at the LHC is a multi-purpose particle detector
with a forward-backward symmetric cylindrical geometry and a near $4\pi$ coverage in 
solid angle.\footnote{\AtlasCoordFootnote}
It consists of an inner tracking detector surrounded by a thin superconducting solenoid
providing a \SI{2}{\tesla} axial magnetic field, electromagnetic and hadron calorimeters, and a muon spectrometer.
The inner tracking detector covers the pseudorapidity range $|\eta| < 2.5$.
It consists of silicon pixel, silicon micro-strip, and transition radiation tracking detectors.
Lead/liquid-argon (LAr) sampling calorimeters provide electromagnetic (EM) energy measurements
with high granularity.
A hadron (iron/scintillator-tile) calorimeter covers the central pseudorapidity range ($|\eta| < 1.7$).
The end-cap and forward regions are instrumented with LAr calorimeters
for both EM and hadronic energy measurements up to $|\eta| = 4.9$.
The muon spectrometer surrounds the calorimeters and is based on
three large air-core toroid superconducting magnets with eight coils each.
Its bending power is in the range from \num{2.0} to \SI{7.5}{\tesla\metre}.
It includes a system of precision tracking chambers and fast detectors for triggering.
A three-level trigger system is used to select events.
The first-level trigger is implemented in hardware and uses a subset of the detector information
to reduce the accepted rate to at most \SI{75}{\kilo\hertz}.
This is followed by two software-based trigger levels that
together reduce the accepted event rate to \SI{400}{\hertz} on average
depending on the data-taking conditions during 2012.


%-------------------------------------------------------------------------------
\subsection{ATLAS detector example from a paper making use of the whole detector}
\label{sec:atlas2}
%-------------------------------------------------------------------------------

The ATLAS detector~\cite{PERF-2007-01} at the LHC covers nearly the entire solid angle around the collision point.
It consists of an inner tracking detector surrounded by a thin superconducting solenoid, electromagnetic and hadronic calorimeters,
and a muon spectrometer incorporating three large superconducting toroid magnets.
The inner-detector system (ID) is immersed in a \SI{2}{\tesla} axial magnetic field 
and provides charged particle tracking in the range $|\eta| < 2.5$.

The high-granularity silicon pixel detector covers the vertex region and typically provides three measurements per track, 
the first hit being normally in the innermost layer.
It is followed by the silicon microstrip tracker which usually provides four two-dimensional measurement points per track.
These silicon detectors are complemented by the transition radiation tracker,
which enables radially extended track reconstruction up to $|\eta| = 2.0$. 
The transition radiation tracker also provides electron identification information 
based on the fraction of hits (typically 30 in total) above a higher energy deposit threshold corresponding to transition radiation.

The calorimeter system covers the pseudorapidity range $|\eta| < 4.9$.
Within the region $|\eta|< 3.2$, electromagnetic calorimetry is provided by barrel and 
endcap high-granularity lead/liquid-argon (LAr) electromagnetic calorimeters,
with an additional thin LAr presampler covering $|\eta| < 1.8$,
to correct for energy loss in material upstream of the calorimeters.
Hadronic calorimetry is provided by the steel/scintillating-tile calorimeter,
segmented into three barrel structures within $|\eta| < 1.7$, and two copper/LAr hadronic endcap calorimeters.
The solid angle coverage is completed with forward copper/LAr and tungsten/LAr calorimeter modules
optimised for electromagnetic and hadronic measurements respectively.

The muon spectrometer (MS) comprises separate trigger and
high-precision tracking chambers measuring the deflection of muons in a magnetic field generated by superconducting air-core toroids.
The precision chamber system covers the region $|\eta| < 2.7$ with three layers of monitored drift tubes,
complemented by cathode strip chambers in the forward region, where the background is highest.
The muon trigger system covers the range $|\eta| < 2.4$ with resistive plate chambers in the barrel, and thin gap chambers in the endcap regions.
A three-level trigger system is used to select interesting events~\cite{PERF-2011-02}.
The Level-1 trigger is implemented in hardware and uses a subset of detector information
to reduce the event rate to a design value of at most \SI{75}{\kHz}.
This is followed by two software-based trigger levels which together reduce the event rate to about \SI{200}{\Hz}.




%-------------------------------------------------------------------------------
\section{Analysis}
\label{sec:analysis}
%-------------------------------------------------------------------------------

You can find some text snippets that can be used in papers in \texttt{template/atlas-snippets.tex}.
Some of the snippets need the \texttt{jetetmiss} option passed to \texttt{atlasphysics}.
%\subsection{\Antikt}

The \antikt algorithm with a radius parameter of $R=0.4$ is used to reconstruct jets with a four-momentum recombination scheme, using \topos as inputs. Jet energy is calibrated to the hadronic scale with the effect of \pileup removed

\subsection{\Topos}

Hadronic jets are reconstructed from calibrated three-dimensional \topos.
Clusters are constructed from calorimeter cells that are grouped together using a topological clustering algorithm.
These objects provide a three-dimensional representation of energy depositions in the calorimeter and implement a nearest-neighbour noise suppression algorithm.
The resulting \topos are classified as either electromagnetic or hadronic based on their shape, depth and energy density.
Energy corrections are then applied to the clusters in order to calibrate them to the appropriate energy scale for their classification.
These corrections are collectively referred to as \textit{local cluster weighting}, or LCW, and jets that are calibrated using this procedure are referred to as LCW jets~\cite{PERF-2012-01}.

\subsection{Grooming}

Trimming removes subjets with $\ptsubji/\ptjet < \fcut$, where \ptsubji is the transverse momentum of the $i^{\text{th}}$ subjet, and $\fcut=0.05$.
Filtering proceeds similarly, but utilises the relative masses of the subjets defined and the original jet. For at least one of the configurations tested, trimming and filtering are both able to approximately eliminate the \pileup dependence of the jet mass.


%-------------------------------------------------------------------------------
\section{Results}
\label{sec:result}
%-------------------------------------------------------------------------------

Place your results here.

% All figures and tables should appear before the summary and conclusion.
% The package placeins provides the macro \FloatBarrier to achieve this.
% \FloatBarrier


%-------------------------------------------------------------------------------
\section{Conclusion}
\label{sec:conclusion}
%-------------------------------------------------------------------------------

Place your conclusion here.


%-------------------------------------------------------------------------------
\section*{Acknowledgements}
%-------------------------------------------------------------------------------

%% Acknowledgements for papers with collision data
% Version 7-Feb-2016

% Standard acknowledgements start here
%----------------------------------------------
We thank CERN for the very successful operation of the LHC, as well as the
support staff from our institutions without whom ATLAS could not be
operated efficiently.

We acknowledge the support of ANPCyT, Argentina; YerPhI, Armenia; ARC, Australia; BMWFW and FWF, Austria; ANAS, Azerbaijan; SSTC, Belarus; CNPq and FAPESP, Brazil; NSERC, NRC and CFI, Canada; CERN; CONICYT, Chile; CAS, MOST and NSFC, China; COLCIENCIAS, Colombia; MSMT CR, MPO CR and VSC CR, Czech Republic; DNRF and DNSRC, Denmark; IN2P3-CNRS, CEA-DSM/IRFU, France; GNSF, Georgia; BMBF, HGF, and MPG, Germany; GSRT, Greece; RGC, Hong Kong SAR, China; ISF, I-CORE and Benoziyo Center, Israel; INFN, Italy; MEXT and JSPS, Japan; CNRST, Morocco; FOM and NWO, Netherlands; RCN, Norway; MNiSW and NCN, Poland; FCT, Portugal; MNE/IFA, Romania; MES of Russia and NRC KI, Russian Federation; JINR; MESTD, Serbia; MSSR, Slovakia; ARRS and MIZ\v{S}, Slovenia; DST/NRF, South Africa; MINECO, Spain; SRC and Wallenberg Foundation, Sweden; SERI, SNSF and Cantons of Bern and Geneva, Switzerland; MOST, Taiwan; TAEK, Turkey; STFC, United Kingdom; DOE and NSF, United States of America. In addition, individual groups and members have received support from BCKDF, the Canada Council, CANARIE, CRC, Compute Canada, FQRNT, and the Ontario Innovation Trust, Canada; EPLANET, ERC, FP7, Horizon 2020 and Marie Sk{\l}odowska-Curie Actions, European Union; Investissements d'Avenir Labex and Idex, ANR, R{\'e}gion Auvergne and Fondation Partager le Savoir, France; DFG and AvH Foundation, Germany; Herakleitos, Thales and Aristeia programmes co-financed by EU-ESF and the Greek NSRF; BSF, GIF and Minerva, Israel; BRF, Norway; the Royal Society and Leverhulme Trust, United Kingdom.

The crucial computing support from all WLCG partners is acknowledged
gratefully, in particular from CERN and the ATLAS Tier-1 facilities at
TRIUMF (Canada), NDGF (Denmark, Norway, Sweden), CC-IN2P3 (France),
KIT/GridKA (Germany), INFN-CNAF (Italy), NL-T1 (Netherlands), PIC (Spain),
ASGC (Taiwan), RAL (UK) and BNL (USA) and in the Tier-2 facilities
worldwide.
%----------------------------------------------



The \texttt{atlaslatex} package contains the acknowledgements that were valid 
at the time of the release you are using.
These can be found in the \texttt{acknowledgements} subdirectory.
When your ATLAS paper or PUB/CONF note is ready to be published,
download the latest set of acknowledgements from:\\
\url{https://twiki.cern.ch/twiki/bin/view/AtlasProtected/PubComAcknowledgements}

The supporting notes for the analysis should also contain a list of contributors.
This information should usually be included in \texttt{mydocument-metadata.tex}.
The list should be printed either here or before the table of contents.


%-------------------------------------------------------------------------------
\clearpage
\appendix
\part*{Appendix}
\addcontentsline{toc}{part}{Appendix}
%-------------------------------------------------------------------------------

In a paper, an appendix is used for technical details that would otherwise disturb the flow of the paper.
Such an appendix should be printed before the Bibliography.


%-------------------------------------------------------------------------------
% If you use biblatex and either biber or bibtex to process the bibliography
% just say \printbibliography here
\printbibliography
% If you want to use the traditional BibTeX you need to use the syntax below.
%\bibliographystyle{atlasBibStyleWoTitle}
%\bibliography{atlas-document,bibtex/bib/ATLAS}
%-------------------------------------------------------------------------------

%-------------------------------------------------------------------------------
% Print the list of contributors to the analysis
% The argument gives the fraction of the text width used for the names
%-------------------------------------------------------------------------------
\clearpage
\PrintAtlasContribute{0.30}

%-------------------------------------------------------------------------------
\clearpage
\appendix
\part*{Auxiliary material}
\addcontentsline{toc}{part}{Auxiliary material}
%-------------------------------------------------------------------------------

In an ATLAS paper, auxiliary plots and tables that are supposed to be made public 
should be collected in an appendix that has the title \enquote{Auxiliary material}.
This appendix should be printed after the Bibliography.
At the end of the paper approval procedure, this information can be split into a separate document
-- see \texttt{atlas-auxmat.tex}.

In an ATLAS note, use the appendices to include all the technical details of your work
that are relevant for the ATLAS Collaboration only (e.g.\ dataset details, software release used).
This information should be printed after the Bibliography.

\end{document}
