The lepton fake rate used in the matrix method (section~\ref{sec:bkg_matrix_method}) depends to some extent on the origin of the fake lepton. 
All signal regions are dominated by fake leptons originating from $t\bar t$ processes; 
however the region SR3b requires 3 tagged $b$-jets, 
which might reduce significantly the fraction of non-prompt leptons produced in $B$ meson decay. 

We provide here as a cross-check an independent data-driven estimate of the fake lepton background in SR3b. 
For that we select events with $\ge 1$ lepton and 2 $b$-jets, 
and measure the probability of the production and reconstruction of a fake lepton, 
taken as the ratio between same-sign and one-lepton events. 
This ratio is then assumed to be invariant by the adjunction of additional cuts, 
and we use it to rescale the number of events with $\ge 1$ lepton satisfying the SR3b cuts (except lepton requirements) 
in order to estimate the fake lepton background yield in SR3b -- so a typical ``ABCD'' estimate : 

\begin{align}
N_\text{fakes}^\text{SR3b}=\frac{N_\text{1 lepton}^{SR3b}\times\left(N_\text{SS leptons}^\text{SR2b} - \text{prompt}\right)}{N_\text{1 lepton}^{SR2b}}
\end{align}

where \textrm{prompt} represents the subtracted yields for processes with prompt same-sign leptons or charge flip electrons in the sample with same-sign leptons and $\ge 2b$-jets. 
To avoid signal contamination or spoiling the SR3b blinding policy, the \met requirement used to select events in the $\ge 2b$-jets sample 
was loosened to $40\met<125$ GeV. 

Events were selected in the one-lepton sample from SUSY7 derivations, using the same framework as for the rest of the analysis, 
and relying on the lowest unprescaled isolated single lepton triggers 
(\textrm{e24\_lh\-me\-dium\_iloose\_L1EM20VH} and \textrm{mu24\_iloose\_L1MU15}). 
Strictly speaking, this is not fully consistent with our analysis phase space, 
as the trigger thresholds are higher than the offline $p_T>20$ GeV required on the leading lepton 
-- and to a lesser extent, the isolation criteria differ. 
But this affects only a small fraction of events, 
and is furthermore not a problem at first order thanks to the nature of the ABCD method (same SS/1L ratio used for $2b$ or $3b$ selections). 

\par{\bf Assessment of possible biases\\}
As defined above, the different regions do not select the same number of objects , 
which might impact e.g. on the effective mass, used in the selection. 
More important is the correlation between non-prompt leptons and tagged $b$-jets : 
the presence of a non-prompt lepton arising from a $B$ meson decay, and satisfying signal requirements, 
generally coincides with a failed tagging of the underlying jet (notably because of the lepton-jet overlap removal, isolation and impact parameter requirements). 
Therefore events with same-sign leptons (out of which one non-prompt lepton) and $N$ $b$-jets might be considered as events with $N+1$ $b$-jets. 

In order to assess the impact of these possible sources of bias on the results, 
we varied the definitions of the background-enriched regions (one lepton or $==2$ $b$-jets) 
by adding additional ($b$-) jets requirements in some of them, as can be seen in Table~\ref{tab:abcd_rates}. 

A closure test was also performed on a $t\bar t$ MC sample, checking the predicted rate of events 
with same-sign leptons and $\ge 3 b$-jets (without further cuts). 
A good agreement was found ($\sim 15\%$), well within statistical fluctuations (only $\sim 20$ raw MC events with SS+3$b$). 

\par{\bf Results\\}
\begin{table}[t!]
\centering
\begin{tabular}{c|c|c|c|c|}\hline
& SS+2b    & 1L+2b   &    1L+3b & Fake lepton estimate in SR3b\\\hline\hline
1 & \multicolumn{2}{c|}{$40<\met<125$, $==2b$} & $-$ & \textbf{$0.60 \pm 0.35$} $[=0.30\;\text{(stat SS)}\pm 0.02\;\text{(stat 1L)}\pm 0.17\;\text{(bkg)}]$\\\hline
2 & \multicolumn{2}{c|}{$\met>40$, $==2b$} & $-$ & \textbf{$0.37 \pm 0.29$} $[=0.22\;\text{(stat SS)}\pm 0.01\;\text{(stat 1L)}\pm 0.19\;\text{(bkg)}]$\\\hline
3 & \multicolumn{2}{c|}{(1) + $\ge 3j$} & $-$ & \textbf{$0.54 \pm 0.34$} $[=0.29\;\text{(stat SS)}\pm 0.02\;\text{(stat 1L)}\pm 0.17\;\text{(bkg)}]$\\\hline
4 & (1) & (1) + $\ge 3j$ & $\ge 4j$ & \textbf{$0.59 \pm 0.34$} $[=0.30\;\text{(stat SS)}\pm 0.02\;\text{(stat 1L)}\pm 0.17\;\text{(bkg)}]$\\\hline
5 & (1) & (1) + $==3b$ & $\ge 4 b$ & \textbf{$0.48 \pm 0.28$} $[=0.24\;\text{(stat SS)}\pm 0.05\;\text{(stat 1L)}\pm 0.14\;\text{(bkg)}]$\\\hline
\end{tabular}
\caption{Estimates of the fake lepton background in the SR3b signal region obtained with the ABCD method for $\mathcal L=3.2$ fb$^{-1}$, 
with different definitions of the background-enriched regions. 
The cuts mentioned are added to (or replace) the cuts defining SR3b (\met, \meff, jets). 
The uncertainties include statistical and systematic (prompt bkg subtraction) sources. }
\label{tab:abcd_rates}
\end{table}

The predicted yields of the fake lepton background in SR3b are detailed in Table~\ref{tab:abcd_rates} for $3.2$ fb$^{-1}$, 
and for the different variations mentioned above. 
For reference, the number of events observed in data in the different categories for the nominal case are respectively 15 (SS+2$b$), 10578 (1L+2$b$) and 827 (1L+3$b$), 
with a prompt+charge flip background of 7.3 events to be subtracted from the first number. 
An overall $30\%$ uncertainty is assigned on the subtraction of the prompt same-sign and charge flip backgrounds. 
One can see that the main limitation to the precision of the method is the very low number of events in the same-sign leptons + 2 $b$-jets selection. 
The variations of the selection have some impact on the results, but the large uncertainties prevent from identifying a systeamtic effect. 
Taking the largest difference (0.23) as an additional systematic uncertainty does not substantially increase the total uncertainty (from 0.35 to 0.42). 

In conclusion, the expected rate of fake leptons in the SR3b signal region for $\mathcal L=3.2$ fb$^{-1}$ is $0.6\pm 0.4$ 
(all sources of uncertainties included). 
It is larger than the one obtained with the matrix method prediction (cf Table~\ref{tab:sr_yields}), that is $0.13\pm 0.17$, 
but the two estimates are consistent within uncertainties. 
The statistical uncertainties of the two methods are comparable in size, as by coincidence the lepton fake rate ($10-20\%$) 
is similar to the probability of $b$-tagging a third jet in $t\bar t$-like events (ratio 1L+3$b$/1L+2$b$ $\sim 10\%$). 
As a final remark, one could also use the same method to cross-check the background prediction for a selection with three leptons and three $b$-jets -- 
which raised some interest during run-1~\cite{3l3b} -- using this time opposite-sign dilepton events instead of one-lepton events. 