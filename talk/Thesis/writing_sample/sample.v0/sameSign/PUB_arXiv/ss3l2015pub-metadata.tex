%-------------------------------------------------------------------------------
% This file contains the title, author and abstract.
% It also contains all relevant document numbers used by the different cover pages.
%-------------------------------------------------------------------------------

% Title
\AtlasTitle{Search for supersymmetry at $\sqrt{s}=13$~TeV in final states with jets and two same-sign leptons or three leptons with the ATLAS detector}

% Author - this does not work with revtex (add it after \begin{document})
\author{The ATLAS Collaboration}

% CERN preprint number
\PreprintIdNumber{CERN-EP-2016-027}

% ATLAS date - arXiv submission; to be filled in by the Physics Office
% \AtlasDate{\today}

% arXiv identifier
% \arXivId{14XX.YYYY}

% HepData record
% \HepDataRecord{ZZZZZZZZ}

% Submission journal and final reference
\AtlasJournal{Eur.\ Phys.\ J.\ C}
% \AtlasJournalRef{Eur.\ Phys.\ J.\ C XXX}
% \AtlasDOI{}

% Abstract - 
\AtlasAbstract{A search for strongly produced supersymmetric particles is conducted using signatures involving multiple energetic jets and either two isolated leptons ($e$ or $\mu$) with the same electric charge or at least three isolated leptons. The search also utilises $b$-tagged jets, missing transverse momentum and other observables to extend its sensitivity. The analysis uses a data sample of proton--proton collisions at $\sqrt{s}= 13$~TeV recorded with the ATLAS detector at the Large Hadron Collider in 2015 corresponding to a total integrated luminosity of 3.2~fb$^{-1}$. No significant excess over the Standard Model expectation is observed.  
The results are interpreted in several simplified supersymmetric models and extend the exclusion limits from previous searches.
In the context of exclusive production and simplified decay modes, gluino masses are excluded at $95\%$ confidence level 
up to 1.1--1.3~TeV for light neutralinos (depending on the decay channel), 
and bottom squark masses are also excluded up to 540 GeV. 
In the former scenarios, neutralino masses are also excluded up to 550-850 GeV for gluino masses around 1 TeV. 
}
