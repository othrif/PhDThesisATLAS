%\documentclass[usetikz]{atlasnote} % the 'usetikz' option loads tikz.sty in the proper place, 
                                   % avoiding conflicts with graphicx.sty.
                                   % Don't know what tikz.st is? Just ignore this line! :-)

%\documentclass[coverpage]{atlasnote} % the 'coverpage' option loads the ATLAS Cover Page package 
                                      % ans makes sure that the cover page is generated before the
                                      % note title page. Make sure that the latest version of
                                      % of 'atlascover.sty. is installed on your system!


\documentclass[11pt,a4paper]{atlasnote}
\graphicspath{{FIGURES/}}
\usepackage{atlasphysics}
\usepackage{subfigure}
\usepackage{mathrsfs}
\usepackage{multirow}
\usepackage{longtable}
\usepackage{slashed}
\usepackage{xspace}
\usepackage{rotating}
\usepackage{graphicx}
\usepackage{epstopdf}
\usepackage{subfigure}
\usepackage{tabularx}
\usepackage{rotating}
\usepackage{placeins}
\usepackage{hyperref}
\usepackage{listings}
\usepackage{pbox}
\usepackage{changepage}
\hypersetup{
colorlinks=true,
linkcolor=red,
citecolor=red,
urlcolor=blue
}
% Running line numbers: 
%Skip this for final version 
\linenumbers 
\usepackage{url}

\renewcommand{\topfraction}{0.99}
\renewcommand{\bottomfraction}{0.99}
\newcommand\quitesmall{\fontsize{8.2}{8.2}\selectfont}
%\usepackage{slashed}
%\setcounter{tocdepth}{4}

% SUSY specific object & variable definitions
\input{susydefs}
\def\btagging{$b$-tagging\xspace}


%%%%%%%%%%%%%%%%%%%%%%%%%%%%%%%%%%%%
%           Title page             % 
%%%%%%%%%%%%%%%%%%%%%%%%%%%%%%%%%%%%

\skipbeforetitle{100pt}

% Title
\title{Search for strongly produced superpartners in final states with same-sign leptons or three leptons and jets: preparing for 2015 analyses}

% Author
%if not given, typesets ``The ATLAS collaboration''
%\author{The ATLAS Collaboration}

% if multiple authors/affiliations are needed, use the authblk package
\usepackage{authblk}
\renewcommand\Authands{, } % avoid ``. and'' for last author
\renewcommand\Affilfont{\itshape\small} % affiliation formatting

%\author[1]{The SS/3L analysis team}
%\affil[1]{the team members institutes}
\input{authors.tex}

% Date: if not given, uses current date
\date{\today}

% Draft version: if given, adds draft version on front page, a
% 'DRAFT' box on top of each other page, and line numbers to easy
% commenting. Comment or remove in final version.
\draftversion{1.0.2}

% Journal: adds a 
%\journal{Phys. Lett. B} 

% Abstract
\abstracttext{
This note contains the documentation of the studies performed with DC14 Monte Carlo samples in preparation for Run-2 for the 
search for strongly produced supersymmetric particles using signatures involving multiple energetic jets and either two
isolated leptons ($e$ or $\mu$) with the same electric charge, or at least three isolated leptons.
The analysis also utilises jets originating from $b$-quarks, missing transverse momentum and other observables to extend its sensitivity.

}

%%%%%%%%%%%%%%%%%%%%%%%%%%%%%%%%%%%%
%            Content               % 
%%%%%%%%%%%%%%%%%%%%%%%%%%%%%%%%%%%%

\begin{document}

\tableofcontents
\clearpage

%\section{Notes to the reader}
%\subsection {Changes with respect to version X}
%For the moment, the document is empty. This will be fixed in version 1.0. 

\section{Introduction}
\label{sec:intro}
\subsection*{Historical Background}

Of what is the universe made? This question has intrigued human curiosity since the dawn of time. 
Today, we are confident that we do not know the complete answer to this question.
However, a lot of progress has been made with the aim of reducing the diversity of the physical phenomena 
observed around us to a limited number of constituents following fundamental principles.
Over two thousand years ago, the ancient Greeks  postulated that all is made of Earth, Air, Fire and Water.
%This view represents a compact yet incorrect description of the world.
Fast-forward to the end of the 19$^\text{th}$ century, Mendeleev and others made the astonishing remark that by organizing the 
relative atomic masses of chemical elements, elements with similar chemical properties followed a pattern.
The periodic table of elements was born. 
The predictive power of the periodic table led to the anticipation of new elements that were later discovered.
However, the table lacked compactness and needed a more fundamental underlying structure that could 
connect the different elements together. 

At the turn of the 20$^\text{th}$ century, several important discoveries established 
the existence of the atom and its constituents. 
The atom is formed by electrons bound via the electromagnetic force to a nucleus, where nearly all the mass resides.
The nucleus itself is formed from 
protons and neutrons that are ``glued'' together by the strong nuclear force (or strong force).
These elements formed the underlying substructure that explained qualitatively the systematic organization of the periodic table.
After 1913, quantum ideas were applied to the atom offering a quantitative description of the origin of structure in atoms and molecules, 
including the chemical elements and their properties.
The decades that followed refined our understanding of the composition of matter through a series of experimental results.
By studying the collisions of protons, neutrons, and electrons in the 1950's and 1960's, a plethora of new particles were discovered 
which belonged to the same family as the proton and neutron, called \textit{hadrons}. % which interacted via the strong force
These particles could not all be elementary
\footnote{Elementary particles refer to  particles that cannot be decomposed into further constituents.}.
By invoking a similar argument that atoms were composite based on Mendeleev's table,
a new layer of structure was unfolded to reveal the existence of \textit{quarks} as basic constituents of all hadrons.
Six types of quarks were discovered over the years with the top quark discovered at Fermilab in Chicago, Illinois 
in 1995, being the most massive elementary particle
\cite{PhysRevLett.74.2626,Abachi:1995iq}.

The observation of the continuous energy spectra in the radioactive $\beta$ decay of nuclei 
led to the discover of neutrinos to remedy  the energy conservation law in the decay. 
Neutrinos are very light neutral particles that interact via the weak nuclear force 
responsible for radioactivity and nuclear fusion, the process that powers the stars.
Electrons and neutrinos had other relatives collectively called \textit{leptons}.
The quarks and leptons are referred to as \textit{fermions} and
have a half integer spin, an intrinsic property of elementary particles.
The strong, weak, and electromagnetic interactions are mediated by 
gluons, $W$ and $Z$ particles, and photons, respectively. These particles 
have an integer spin and are called \textit{bosons}. 
%Apart from gravity, all aspects of daily life can 
%be described in terms of these interactions.
%
The latest addition to the known elementary particles happened in 2012 
 with the discovery of 
a new boson, the Higgs boson, 
that allows the quarks and leptons and the $W$ and $Z$ bosons to 
acquire mass\cite{Aad:2012tfa,Chatrchyan:2012xdj}.

%A success story
\subsection*{The Standard Model}
The physics of elementary particles became the most ambitious and organized attempt to answer the question of what the universe is 
made out of. 
Through a mixture of both theoretical insight and experimental input, we now 
know that everything we see in our daily life is formed from quarks and leptons
that interact via the strong, weak, and electromagnetic forces.\footnote{ 
The fourth fundamental force of gravity is extremely weak and
only acts at the macroscopic scale.}
The forms of these forces are determined from basic principles of 
symmetry and invariance.
As a result, a theoretical framework was constructed to synthesize all these 
developments in a quantitative 
calculational tool that became known as the \textit{Standard Model of particle physics} (SM). 
The only inputs needed by the SM are the interaction strengths of the forces and quark and lepton masses to make 
very accurate predictions about the behavior of elementary particles.
%For example, the magnetic moment of the electron calculated in the SM is found to agree with the experimental measurement to nine(?) decimal places.
Over the past 30 years, the SM has been vigorously tested by many 
experiments and has been shown to accurately describe particle 
interactions at the highest energies produced in the laboratory.
 Yet, we know it is not the complete story. 

% the dark side
\subsection*{Limitations}
In 1933, an observation of the Coma Cluster by Fritz Zwicky suggested that the galaxies in the cluster were moving too fast to be explained 
by the luminous matter present\cite{1933AcHPh}. 
The same observation was repeated when looking at the rotation speeds of individual galaxies which 
suggested an invisble component of mass, dark matter. 
The experimental evidence established that dark matter is not made out of baryons
and is more abundant than ordinary matter.
For example, anisotropies in the cosmic microwave background, a radiation 
left over from the Big Bang, that were consistent with 
quantum fluctuations from an inflationary epoch \cite{Hu:2001bc,2009AIPC}, 
encoded details about the density of matter 
in the form of 
cosmological parameters as they traveled through space and time to reach 
our experiments.
The astonishing conclusion was that the universe has nearly five times 
as much dark matter as ordinary matter \cite{Bertone:2004pz}.

Supernovae surveys gave direct evidence for an accelerating universe
 \cite{Perlmutter:1998np},
% From the rate of expansion, we can play the scenario back in time and 
%deduce that the universe was compacted on itself some 14 billion years ago.
% the explosive erruption from that state is referred to as the Big Bang
a view that was cemented by the measurement of cosmological parameters
\cite{Adam:2015rua,Ade:2015xua}
which led to the startling discovery that most of the energy density of 
the universe is in the 
form of an unknown negative-pressure, called dark energy \cite{Scranton:2003in}.
There is an extensive program of experiments 
which will probe the dark energy. % and is beyond the scope of this work. 
%Instead, we turn to dark matter. 

Astrophysics and cosmology told us about 
the existence of dark matter and measured its density to a remarkable 
precision. Particle physics holds the hope to uncover what dark matter is.
In short, all experimental evidence is consistent with a universe 
constructed of 
\begin{itemize}
\item baryons (everyday matter): $\sim 5\%$ 
\item dark matter: $\sim 20\%$ 
\item dark energy: $\sim 75\%$ 
\item neutrinos, photons: a tiny fraction
\end{itemize}

Today, we are confronted by many puzzles related to our view of the universe.
Everything we know of, namely all the particles of the Standard Model, 
constitute only 5\% of the energy budget of the universe. 
The universe is also predominately composed of matter as opposed to 
anti-matter even though at the start of the universe, they were in equal 
amounts%, baryon assymetry
. The Standard Model describes the content of 
everyday matter 
and how it interacts but without telling us why it is that way.
Moreover, the Standard Model only describes these phenomena 
up to an energy scale of $\mathcal{O}\left(100\right)$ \GeV, called the weak scale.
Beyond this scale lies the realm of phenomena not described by the standard 
model that extend all the way to the Planck scale of 
$\mathcal{O}\left(10^{19}\right)$ \GeV. There is no mechanism to generate mass for 
neutrinos in the Standard Model. Last but not least, the Standard Model 
does not incorporate gravity, the fourth fundamental force.
The SM is unable to account for these observed 
features in the universe. 
Thus, there is a need for a theory beyond the Standard Model.

\subsection*{Supersymmetry}

One of the most prominent extensions of the Standard Model, 
that addresses many of the shortcomings mentioned above, is a theory based on 
a new symmetry, called supersymmetry.
This symmetry is between the matter particles, fermions, and particles whose
exchange mediates the forces, bosons. Our current description of the world
treats fermions and bosons differently. Supersymmetry puts forward the idea
that fermions and bosons can be treated in a fully symmetric way. 
In other words,
if we exchange fermions and bosons in the equations of the theory, the 
equations will still look the same. An immediate consequence of the theory
is that every Standard Model particle will have a ``superpartner,'' 
none of which have yet been discovered.
As a result, we can design experiments to search for these 
supersymmetric particles. The work presented in this dissertation is about the search for supersymmetric particles with a specific signature.
The many benefits of supersymmetry will be discussed later but here it is worth 
mentioning two important features of the theory: 
it unifies the three interactions, electromagnetic, strong, and weak forces,
at very high energies 
%it addresses the question 
%of why there is a huge gap between the Planck scale and the weak scale, 
and it provides a dark matter candidate particle. Now that we understand 
what we are trying to do, it is time to address the question of how to do it.

\subsection*{Experimental techniques}

The human eye can resolve pieces of dust up to $10^{-5}$ m.
The subatomic distances we are interested in probing range
from  $10^{-15}$ m, the size of a proton, 
down to $10^{-18}$ m, the experimental limit to the maximum size of a quark.
Instruments are needed to extend our senses to 
probe these very small scales.
For instance, light microscopes can reveal the structure of things down to 
$10^{-6}$ m, the scale of bacteria and molecules. 
A special type of microscope is needed to probe smaller distances, 
 a particle accelerator.
The basic idea is that in order to see an object, a wave must scatter off 
this object and must have a wavelength smaller than the object being 
probed.
Since particles have a wavelike character, they can be used to 
probe ever shorter distances according to
\[
E = \frac{hc}{\lambda}
\]
where $E$ is the energy of the particle, $\lambda$ is its wavelength, 
and $hc \sim 10^{-6}$ eVm. As a result, 
the higher the speed of the particles, the greater their 
energy and momentum and the shorter their associated wavelength.
Modern accelerators can generate energy in the \TeV~scale and thus probe 
a distance of $10^{-18}$ m.
All the development that we have made 
describes phenomena happening at distances larger than about $10^{-18}$ m.
Thus, it is possible that electrons and quarks have some structure which is 
smaller than what we can resolve in experiment. For this reason, we 
currently consider them as not having any deeper structure, i.e. they are 
called pointlike objects. 

Over the last century, beams of particles were used to study the 
composition of matter.
Initially, beams originated from phenomena that were 
already naturally occurring, such as alpha and beta particles coming from radioactive 
decays and cosmic rays.
Some cosmic rays are much more energetic than what we can produce 
in the laboratory today, however, they occur at random and 
with a low intensity. Instead, 
high energy particle accelerators were used to deliver high intensity beams of 
electrons, protons, and other particles under controlled conditions.
For this reason, particle physics is also known as high energy physics.
By colliding two sufficiently energetic particles,  new particles will be created 
according to Einstein's equation $E = mc^2$ (or more generally $E = \sqrt{\left(mc^2\right)^2+\left(pc\right)^2}$),
where energy can be exchanged for mass, and vice versa, the exchange rate being $c^2$, the square of the 
speed of light. For example, an electron has a mass of 0.5 \MeV~ 
and can only be created in an electron-positron pair, thus 
1 \MeV~of energy is needed for an electron--positron pair to be produced at rest.
Energies in the \TeV~ range were present about a billionth of a second after the Big Bang.
In other words, by colliding high energy particles, it is possible to 
recreate momentarily conditions similar to those of the universe when it 
was newly born.
At such energies, particles and antiparticles were created, including 
exotic forms no longer common today.
Most of the particles generated in these collisions are extremely short lived 
with lifetimes less than $10^{-20}$ seconds, 
producing radiation and decaying to stable particles, such as electrons and quarks, that make up most of what we see today.
One of the exotic forms of matter that may exist is supersymmetry.
The search for evidence for supersymmetric particles using data collected at a
high energy particle accelerator is the subject of this dissertation.

The Large Hadron Collider (LHC) 
is the world's most energetic 
particle accelerator and the pinnacle of colliding beam technology.
Is it located at CERN, 
the European Laboratory for Particle Physics
\footnote{
The acronym comes from French ``Conseil Europ�en pour la Recherche Nucl�aire''
which was established to do fundamental physics research.
In 1952, this research concentrated on understanding the atom and its nucleus, hence the word ``nuclear''.
Today, our knowledge goes deeper than the nucleus which motivates the modified name.
}, near Geneva, Switzerland.
The LHC accelerates counter rotating beams of protons 
to 99.9999991\% the speed of light in a 27 km ring reaching an energy of 
6.5 \TeV~per beam. 
Magnets, cooled 
by the largest cryogenic system in the world to 1.9 K (-271.3 $^{\circ}{\textrm C}$), that keep the 
protons on track and bring the counter-rotating needle-like
beams 
%of few micrometers in diameter 
to meet head on 40 million times per second. 
The debris of each collision fly off in all directions, 
briefly producing less common exotic forms of matter
captured by large particle detectors in the form of ``snapshots'' of these collisions, called events.
The teams of scientists analyze these events to identify the different particles that were produced 
and reconstruct the full collision process.
With this information, it is possible to make precision measurements of rare Standard Model processes, like 
the production of the Higgs boson, or search for physics beyond the Standard Model, like evidence 
for supersymmetry.
ATLAS is one of the general-purpose particle detectors at the LHC that supplied the events 
analyzed in this dissertation to search for supersymmetry. 
The ATLAS detector is the largest-volume particle detector ever built --
the size of a seven-story building 46 meters high and 26 meters in diameter, 
weight 7000 tonnes, %has about 3000 km of cable, 
and able to measure particle trajectories down to 0.01 meters.
% The collaboration has nearly 3000 scientific authors from 182 institutions
% in 38 countries (June 2017).
Bunches of protons pass through each other at the heart of 
the ATLAS detector 40 million times per second.
Each time they cross there are on average 25 proton-proton collisions, leading to 
about a billion proton collisions per second. 
The data generated in these collisions amounts to about 60 terabytes per second, 
an amount far beyond what is technologically possible to store.
In fact, the processes of interest are extremely rare.
For example, the Higgs boson is produced once in 20 million million collisions.
In more practical terms, a Higgs boson might appear once a day during the LHC operations.
ATLAS has a big computational challenge to recognize this one Higgs event and record it to tape 
out of 35 million million other collisions each day.
The topic of this dissertation is to search for supersymmetric particles that 
are even rarer and thus more challenging to look for. 

This dissertation will give a detailed explanation on how we searched for supersymmetric 
particles using the ATLAS detector.
First, the motivation behind the work will begin with an overview of the 
Standard Model of particle physics and supersymmetry in 
Chapter~\ref{chap:theory} followed by the design of the ATLAS detector 
at the LHC in Chapter~\ref{chap:exp}.
The Region of Interest Builder that processes every event recorded by ATLAS 
is covered in Chapter~\ref{chap:roib}.
The detailed description of the search starts in Chapter~\ref{chap:strategy}
covering the basic analysis strategy and the supersymmetric models considered.
The most challenging part of the analysis is the estimation of Standard Model and 
detector backgrounds with novel techniques developed by the author and covered in 
Chapters~\ref{chap:fake} and ~\ref{chap:bkg}. The statistical 
methodology and interpretation of the results is presented in 
Chapters~\ref{chap:stat} and ~\ref{chap:res}.
This analysis represents an important search for supersymmetric particles 
with the early data-set collected by ATLAS at a new center of mass energy of 13 \TeV.
The strength of the search lies in exploring regions of the parameter space 
with a small mass difference between the supersymmetric particles, regions
that are difficult to probe with other searches for unknown physics.


\section{Signal models}
\label{sec:signal}
\subsection{Models already considered in Run-1}
Final states with two same-sign leptons and multiple jets are sensitive to a variety of new physics scenarios. 
In supersymmetric models in particular, such final states can be produced in the decays of heavy superpartners 
involving massive gauge bosons, sleptons or top quarks. 
We list in this section the different simplified models which we use as benchmarks for the choice of signal regions. 


\begin{figure}[h!]
\centering
\subfigure{\includegraphics[width=0.24\textwidth]{MODELS/gogo-ttttN1N1}}
\subfigure{\includegraphics[width=0.24\textwidth]{MODELS/sbsb-ttWWN1N1}}
\caption{Gluino decay via offshell stop (left), and direct sbottom pair production (right).}
\label{fig:feynman_3rdgen}
\end{figure}

\begin{figure}[t]
\centering
\subfigure{\includegraphics[width=0.49\textwidth]{MODELS/ATLAS_SUSY_Gtt}}
\subfigure{\includegraphics[width=0.49\textwidth]{MODELS/exclusion_sbottom_topC1_both_grids}}
\caption{Exclusion limits on the gluino-stop offshell (left) and direct sbottom (right) scenarios 
set by ATLAS with the 2012 dataset~\cite{DraftSquarkGluinoSummaryPaper}.}
\label{fig:run1excl_3rdgen}
\end{figure}

\par{\bf Gluino-stop offshell $\gluino\to t\bar t\neut$\\}
In this model inspired by naturalness arguments, gluinos are coupling preferentially to stops which are lighter than the other squarks. 
Gluinos are however considered lighter than stops, and decay directly into a $t\bar t\neut$ triplet via a virtual stop (Fig.~\ref{fig:feynman_3rdgen}). 
The pair production of gluinos leads to a final state containing four top quarks and two neutralinos. 
This characteristic final state is accessible through various experimental signatures, which is why this model 
is commonly used as a benchmark to estimate analyses sensitivities. 
The searches performed with run-1 data~\cite{DraftSquarkGluinoSummaryPaper}, 
summarized in Fig.~\ref{fig:run1excl_3rdgen}, showed that the same-sign leptons final state is competitive mainly at large neutralino mass. 
This region of the phase space is consequently given a particular attention in the choice of signal regions described further on. 
In the signal samples referenced in this document, the lightest stop mass is fixed to 10~\TeV and is mostly a $\widetilde{t}_R$ state. 
Only gluino pair production is considered, followed by an exclusive decay in the aforementioned channel. 
\\
\par{\bf Direct sbottom $\sbot\to t\chargino$\\}
In this model, bottom squarks are rather lights and assumed to decay in a top quark and a chargino $\chargino$ (Fig.~\ref{fig:feynman_3rdgen}), 
providing complementarity to the mainstream search which focuses on the channel $\sbot\to b\neut$. 
The final state resulting from the production of a sbottom pair contains pairs of top quarks, of $W$ bosons and of neutralinos. 
While this final state may lead to various experimental signatures, 
the model was considered in run-1~\cite{DraftSquarkGluinoSummaryPaper} 
only by the same-sign leptons and jets search, leading to the exclusion limits presented in Fig.~\ref{fig:run1excl_3rdgen}. 
In the signal samples used by the analysis, the neutralino mass is fixed to 60~\GeV, and the chargino mass to 150~\GeV, while the sbottom mass is varied. 
Only pair production of the lightest sbottom is considered, followed by an exclusive decay in the aforementioned channel. \\


\begin{figure}[h!]
\subfigure{\includegraphics[width=0.24\textwidth]{MODELS/gogo-qqqqWWZZN1N1-C1N2}}
\subfigure{\includegraphics[width=0.24\textwidth]{MODELS/sqsq-qqlllvN1N1-C1N2}}
\subfigure{\includegraphics[width=0.24\textwidth]{MODELS/gogo-qqqqlllvN1N1-C1N2}}
\subfigure{\includegraphics[width=0.24\textwidth]{MODELS/sqsq-qqWWZZN1N1-C1N2}}
\caption{Two-step decays of gluinos and squarks, mediated by gauginos (left) or sleptons (right).}
\label{fig:feynman_1stgen}
\end{figure}

\begin{figure}[t]
\centering
\subfigure{\includegraphics[width=0.49\textwidth]{MODELS/run1excluded_gluino2stepWZ}}
\subfigure{\includegraphics[width=0.49\textwidth]{MODELS/run1excluded_gluino2stepSleptons}}
\subfigure{\includegraphics[width=0.49\textwidth]{MODELS/run1excluded_squark2stepWZ}}
\subfigure{\includegraphics[width=0.49\textwidth]{MODELS/run1excluded_squark2stepSleptons}}
\caption{Exclusion limits on scenarios featuring gluino (top) and squarks (bottom) two-steps decays via gauginos (left) or sleptons (right) 
set by ATLAS with the 2012 dataset~\cite{DraftSquarkGluinoSummaryPaper}.}
\label{fig:run1excluded_1stgen}
\end{figure}

\par{\bf Gluinos and squarks 2-step decays via gauginos\\}
These scenarios feature a less oriented search for gluinos or squarks (save third generation) where gluinos couple preferentially to the latter, 
and squarks decay to charginos (Fig.~\ref{fig:feynman_1stgen} left) with the subsequent cascade $\chargino\to W\tilde{\chi}_{2}^{0} \to WZ\neut$. 
This leads to final states of two light quarks, two $W$ and $Z$ bosons, and two neutralinos 
(with two additional light quarks in the case of gluino pair production). 
Fig.~\ref{fig:run1excluded_1stgen} left shows the exclusion limits obtained with the 2012 dataset~\cite{DraftSquarkGluinoSummaryPaper}; 
in the gluino scenario, the same-sign leptons + jets search provided complementarity at large neutralino mass, 
while its sensitivity in the squark scenario dominated completely. 
In the signal samples used here, the chargino mass is set halfway between the neutralino and squark (or gluino) masses, 
while the neutralino $\tilde{\chi}_{2}^{0}$ mass is set halfway between the chargino and neutralino masses. 
Gluino and squark-antisquark pair production are considered separately in distinct scenarios.  
\\
\par{\bf Gluinos and squarks 2-step decays via sleptons\\}
In these scenarios, gluinos couple preferentially to the squarks of the first two generations, and
the latter decay either to a chargino $\chargino$ or a neutralino $\tilde{\chi}_{2}^{0}$, 
which are assumed to be mass-degenerate, and decay in turn to sleptons (Fig.~\ref{fig:feynman_1stgen} right) 
with $\mathcal{BR}(\chargino\to\nu\tilde\ell) = \mathcal{BR}(\chargino\to\ell\tilde\nu) 
= \mathcal{BR}(\tilde{\chi}_{2}^{0}\to\nu\tilde\nu) = \mathcal{BR}(\tilde{\chi}_{2}^{0}\to\ell\tilde\ell)=50\%$. 
The corresponding final state may contain zero to four charged leptons, neutrinos, two light quarks and two neutralinos 
(with two additional light quarks in the case of gluino pair production). 
Because of the sleptons replacing the gauge bosons featured in the scenarios presented in the previous paragraph, 
these scenarios have comparatively a lower jet multiplicity but a significantly enhanced acceptance in multi-lepton experimental signatures. 
As can be seen on Fig.~\ref{fig:run1excluded_1stgen} right, which presents the exclusion limits obtained with the 2012 dataset~\cite{DraftSquarkGluinoSummaryPaper}, 
the same-sign leptons and jets signature is again very competitive. 
In the signal samples used here, the chargino $\chargino$ and neutralino $\tilde{\chi}_{2}^{0}$ masses are set equal, 
halfway between the neutralino and squark (or gluino) masses, 
while the degenerate sleptons masses are set halfway between these gauginos and the lightest neutralino masses. 
Furthermore, gauginos decay to any slepton flavor with equal probability. 
Gluino and squark-antisquark pair production are considered separately in distinct scenarios. 
\\
\par{\bf Models not considered for the moment\\}
In the publications~\cite{paperSS3L,DraftSquarkGluinoSummaryPaper} of the analysis results obtained with run-1 data, 
exclusion limits were also provided for other signal models, often 
These scenarios included the $\gluino\to tbW\neut$ and $\gluino\to tcW\neut$ simplified models, as well as minimal models featuring 
$R$-parity violation through bilinear terms, gauge-mediated SUSY breaking, or universal extra dimensions. 
These models are not considered here, although interpretations might be proposed for them again in the future. 

\subsection{New models}

%\subsubsection{pMSSM inspired [Sebastien]}

\subsubsection{RPV inspired}
\label{subsec:RPVmodel}

 In supersymmetry, the following superpotential is present :
 \begin{align}
   W = \mu H L + \frac{1}{2} \lambda_{ijk} L_i L_j E_k + \lambda'_{ijk} L_i Q_j D_k + \frac{1}{2} \lambda''_{ijk} U_i D_j D_k
   \label{rpvpotential}
 \end{align}
 where $H$, $L$, $Q$, $E$, $U$ and $D$ are respectively the superpotential associated to the Higgs doublet, the lepton-neutrino doublet, the quark up-down doublet, the right-handed electron, the right-handed up quark and the right-handed down quark.
 The indices $i$, $j$ and $k$ are the flavor indices and $\mu$ , $\lambda_{ijk}$ , $\lambda'_{ijk}$ , $\lambda''_{ijk}$ are the coupling constants.
\\

 The leptonic number violation and the baryonic number violation implied by this potential have an important impact in the physic at low energy
 and do not respect some low energy constraints like the proton decay time limit.
 In $R$-parity conserving (RPC) SUSY models, the $R$-parity is added in order to remove these terms and keep the proton stable.
 However, one can play with the couplings $\mu$ , $\lambda_{ijk}$ , $\lambda'_{ijk}$ and $\lambda''_{ijk}$ in order to violate $R$-parity while respecting the low energy constraints.
 This is called the $R$-parity violation (RPV) SUSY models.
\\

 In this note, we will only consider the case where the coupling constants $\mu$, $\lambda_{ijk}$, $\lambda'_{ijk}$ and $\lambda''_{(i \neq 3) jk}$ are suppressed.
 Therefore, the only non-negligible terms are $\lambda''_{321}TSD$ , $\lambda''_{331}TBD$ and $\lambda''_{323}TSB$ where $T$, $B$, $D$ and $S$ are the superfields associated to the top, bottom, down and strange quark.
 This scenario is predicted by some RPV models like the Minimal Flavor Violation (MFV) scenarios~\cite{Nikolidakis:2007fc,Csaki:2011ge} and leads to the production of same-sign top quarks~\cite{Durieux:2013uqa} (see Fig.~\ref{fig:rpv_diagram}).

\begin{figure}[h!]
\centering
\subfigure{\includegraphics[width=0.24\textwidth]{MODELS/ddfusion.png}}
\subfigure{\includegraphics[width=0.27\textwidth]{MODELS/ggfusion.png}}
\caption{Example of diagrams for RPV model for the \textit{d-quark fusion} topology (left), and the \textit{gluon fusion} topology (right) involving the $\lambda''_{321}$ coupling. In both cases, the baryonic number is violated by two units. }
\label{fig:rpv_diagram}
\end{figure}

 In these scenarios, the $R$-parity is not conserved and the LSP is not stable (and therefore cannot be a good dark matter candidate).
 However, the interesting part of this model is that it allows B-violating processes (see Fig.~\ref{fig:rpv_diagram}).
 Actually, the baryonic asymmetry in the universe and the fact that the standard model does not respect this symmetry provides a good motivation for the search for $B$-violation processes at high energy.
 In addition, it was also proved that the violation of the baryonic number by two units involving top quarks could respect the low denergy constraints like the proton decay time limit~\cite{Durieux:2012gj}.
% Therefore, RPV SUSY is good generic model for the search for $B$-violation.
\\

In this analysis, two kind of topologies will be considered:
\begin{itemize}
\item \textbf{d-quark fusion}: Production of two on-shell $d$-squarks with a gluino in $t$-channel and the $d$-squarks decay to an anti-top quark and to one extra jet.
  The squarks will lighter than the gluino and the cross section will mostly depend on the mass of the squark (see Fig.~\ref{fig:rpv_diagram}).
  \begin{align}
    dd \to \tilde{d} \tilde{d}\ ,\ \tilde{d} \to \bar{t}q    
  \end{align}
\item \textbf{gluon fusion}: Production of two on-shell gluinos which decay to a top or an anti-top quark and to two extra jets.
  The gluino will be lighter than the squarks and the cross section will only depend on the mass of the gluino (see Fig.~\ref{fig:rpv_diagram} ).
  \begin{align}
    gg \to \gluino\gluino\ ,\ \gluino \to tqq\ /\ \gluino \to \bar{t}qq'
  \end{align}
\end{itemize}
The extra jets could be either $b$-jet or light jets, depending on the coupling being considered. 
In order to have access to the charge of the top quark, we will only consider the case where the top quark decays leptonically.
At the end, the final states will be composed of two same-sign leptons, at least two $b$-jets, low missing transverse energy (coming from the neutrinos) and extra jets.
\\
% Cross section : soon

 This model was already constrained by ATLAS~\cite{Aad:2014pda} and CMS~\cite{Chatrchyan:2013fea} in Run-1, but those analyses 
 only considered the coupling $\lambda''_{323}$ and the topology of \textit{gluon fusion}.
 A limit of around 900 GeV was found on the mass of the gluino. The full hadronic final state was also exploited in ATLAS~\cite{Aad:2013wta}.



\section{MC samples}
\label{sec:samples}
This section summarizes the signal and background samples used for the
studies presented in this note, as well some details about the Monte Carlo production and analysis framework.
All samples used in this note are part of the DC14 Monte-Carlo campaign which exhibits a number 
of deficiencies~\cite{dc14twiki}, most notably the inaccessibility of detailed trigger information. 

\subsection{Derivation versions and analysis model}
\label{sec:ntuples}

In all cases, the SUSY1 DxAOD derivations are used. 
Derivation tag p1872 (19.1.4.9 cache) is used for most of the samples used in this note, although some signal 
samples were not initially available in that tag and p1863 (19.1.4.8 cache) was used instead. 
Both derivation tags included the fix of lepton isolation 
variables (see Section~\ref{sec:isolation} for more details).

Most of the studies shown in this note were performed using flat ROOT ntuples produced on the grid using code based on the 
{\tt{SUSYAnalysiExample}} EventLoop package and SUSYTools-00-05-00-26 tag. 
They contained basic object information, although no overlap removal 
or isolation cuts were applied to allow flexibility for optimization studies. 
No event skimming or systematic uncertainties were included. 
The total size of these ntuples is 70~GB (both background and signal samples).
These files were shared among the groups participating in the analysis, although some 
of the groups developed their own framework for xAOD analysis.

In addition, dedicated flat ROOT ntuples with all the systematic variations included in the SUSYTools {\tt getSystInfoList()} method 
were also produced for the samples containing prompt SS/3L ($\ttbar+V$, $\ttbar+h$, diboson, signals). To reduce the size of the output files, only events that contained in any of the systematic variations
at least 2 leptons with $\pt>10$~GeV, and either $\et>100$~GeV or at least 3 b-tagged jets ($\pT>20$ GeV) or $H_{\rm T}>300$~GeV were kept.  
In order to benefit from the recently implemented grouping of 
the Egamma systematics, the SUSYTools-00-05-00-29 tag was used for this production, which contains only other minor 
updates with respect to SUSYTools-00-05-00-26. The total size of these ntuples is below 50~GB.


\subsection{Signal Samples}

This section describes the signal samples used for in this note. 
The baseline signal models used are the ones described
in Section~\ref{sec:signal}, and all the signal samples used are listed in Table~\ref{tab:SignalSamples}.

The gluino-stop offshell ($\gluino\to t\bar t\neut$) and direct sbottom ($\sbot\to t\chargino$ with $\tilde{b}_1\to t\tilde{\chi}^{\pm}_1$) samples are produced with {\sc Herwig++}~\cite{Corcella:2000bw}. The UE-EE-4 {\sc Herwig++} tune~\cite{Gieseke:2012ft} is used to describe the underlying event.
%and the generator settings are equivalent to those of the 8 TeV 
%analysis~\cite{noteSS3L} except that the stop mass has been raised to 5
%TeV and the center of mass energy has been adjusted to 13 TeV.
The other signal samples used in this note (gluinos and squarks 2-step decays via gauginos or sleptons) are produced using the {\sc Madgraph v5} generator~\cite{madgraph} using leading-order (LO) matrix elements with up to two additional partons.
{\sc Pythia} 6.425~\cite{Sjostrand:2006za} with the {\sc AUET2B} tune~\cite{pub-2011-009,ATLAS:2011krm} is used for hadronisation and to describe the underlying event. 
In all the signal samples, the {\sc CTEQ6L1}~\cite{Pumplin:2002vw} set of parton distribution functions is used in all the signal samples. The signal samples are normalised to the next-to-next-to-leading order cross-section including the resummation of
soft gluon emission at next-to-next-to-leading-logarithmic accuracy
(NLO+NLL), as detailed in Ref.~\cite{Borschensky:2014cia}. 


\begin{sidewaystable}[htb]
\begin{center}
\resizebox{0.95\textwidth}{!}{
\tiny
\begin{tabular}{lllllll}
\hline
\hline
datasetID & Sample name & $\sigma$ [pb] & $k$-Factor & $\epsilon_{filter}$ & $N_{gen}$ & $L_{equiv}$\\
\hline
\multicolumn{7}{c}{\textbf{Gluino-stop offshell} ($\gluino\to t\bar t\neut$)} \\ \hline
204533  &  mc14\_13TeV.204533.Herwigpp\_UEEE4\_CTEQ6L1\_Gtt\_G1000\_T5000\_L100.merge.DAOD\_SUSY1.e3094\_s1982\_s2008\_r5787\_r5853\_p1872/  &  0.3254  &  1.00  &  1.000  &  20000  &  61.5 \\
204534  &  mc14\_13TeV.204534.Herwigpp\_UEEE4\_CTEQ6L1\_Gtt\_G1300\_T5000\_L100.merge.DAOD\_SUSY1.e3094\_s1982\_s2008\_r5787\_r5853\_p1872/  &  0.0461  &  1.00  &  1.000  &  19500  &  423.0 \\
204535  &  mc14\_13TeV.204535.Herwigpp\_UEEE4\_CTEQ6L1\_Gtt\_G1600\_T5000\_L100.merge.DAOD\_SUSY1.e3094\_s1982\_s2008\_r5787\_r5853\_p1872/  &  0.0081  &  1.00  &  1.000  &  20000  &  2469.1 \\
204536  &  mc14\_13TeV.204536.Herwigpp\_UEEE4\_CTEQ6L1\_Gtt\_G1900\_T5000\_L100.merge.DAOD\_SUSY1.e3094\_s1982\_s2008\_r5787\_r5853\_p1872/  &  0.0016  &  1.00  &  1.000  &  20000  &  12500.0 \\
204537  &  mc14\_13TeV.204537.Herwigpp\_UEEE4\_CTEQ6L1\_Gtt\_G2200\_T5000\_L100.merge.DAOD\_SUSY1.e3094\_s1982\_s2008\_r5787\_r5853\_p1872/  &  0.0004  &  1.00  &  1.000  &  20000  &  50000.0 \\
204539  &  mc14\_13TeV.204539.Herwigpp\_UEEE4\_CTEQ6L1\_Gtt\_G1300\_T5000\_L400.merge.DAOD\_SUSY1.e3094\_s1982\_s2008\_r5787\_r5853\_p1872/  &  0.0461  &  1.00  &  1.000  &  15000  &  325.4 \\
204540  &  mc14\_13TeV.204540.Herwigpp\_UEEE4\_CTEQ6L1\_Gtt\_G1300\_T5000\_L700.merge.DAOD\_SUSY1.e3094\_s1982\_s2008\_r5787\_r5853\_p1872/  &  0.0461  &  1.00  &  1.000  &  20000  &  433.8 \\
204541  &  mc14\_13TeV.204541.Herwigpp\_UEEE4\_CTEQ6L1\_Gtt\_G1300\_T5000\_L900.merge.DAOD\_SUSY1.e3094\_s1982\_s2008\_r5787\_r5853\_p1872/  &  0.0461  &  1.00  &  1.000  &  20000  &  433.8 \\
204912  &  mc14\_13TeV.204912.Herwigpp\_UEEE4\_CTEQ6L1\_Gtt\_G1500\_T5000\_L200.merge.DAOD\_SUSY1.e3328\_s1982\_s2008\_r5787\_r5853\_p1872/  &  0.0142  &  1.00  &  1.000  &  20000  &  1408.5 \\
204913  &  mc14\_13TeV.204913.Herwigpp\_UEEE4\_CTEQ6L1\_Gtt\_G1500\_T5000\_L400.merge.DAOD\_SUSY1.e3328\_s1982\_s2008\_r5787\_r5853\_p1872/  &  0.0142  &  1.00  &  1.000  &  20000  &  1408.5 \\
204914  &  mc14\_13TeV.204914.Herwigpp\_UEEE4\_CTEQ6L1\_Gtt\_G1500\_T5000\_L600.merge.DAOD\_SUSY1.e3328\_s1982\_s2008\_r5787\_r5853\_p1872/  &  0.0142  &  1.00  &  1.000  &  20000  &  1408.5 \\
204915  &  mc14\_13TeV.204915.Herwigpp\_UEEE4\_CTEQ6L1\_Gtt\_G1500\_T5000\_L800.merge.DAOD\_SUSY1.e3328\_s1982\_s2008\_r5787\_r5853\_p1872/  &  0.0142  &  1.00  &  1.000  &  20000  &  1408.5 \\
204916  &  mc14\_13TeV.204916.Herwigpp\_UEEE4\_CTEQ6L1\_Gtt\_G1500\_T5000\_L1000.merge.DAOD\_SUSY1.e3328\_s1982\_s2008\_r5787\_r5853\_p1872/  &  0.0142  &  1.00  &  1.000  &  20000  &  1408.5 \\ \hline
\multicolumn{7}{c}{\textbf{Direct sbottom} ($\sbot\to t\chargino$ with $\tilde{b}_1\to t\tilde{\chi}^{\pm}_1$)} \\ \hline
204973  &  mc14\_13TeV.204973.Herwigpp\_UEEE4\_CTEQ6L1\_sbottom\_tchr\_2lep\_B450\_C150\_N60.merge.DAOD\_SUSY1.e3356\_s1982\_s2008\_r5787\_r5853\_p1863/  &  0.9483  &  0.41  &  0.409  &  20000  &  125.8 \\
204974  &  mc14\_13TeV.204974.Herwigpp\_UEEE4\_CTEQ6L1\_sbottom\_tchr\_2lep\_B550\_C150\_N60.merge.DAOD\_SUSY1.e3356\_s1982\_s2008\_r5787\_r5853\_p1863/  &  0.2961  &  0.43  &  0.432  &  11000  &  200.0 \\
204975  &  mc14\_13TeV.204975.Herwigpp\_UEEE4\_CTEQ6L1\_sbottom\_tchr\_2lep\_B650\_C150\_N60.merge.DAOD\_SUSY1.e3356\_s1982\_s2008\_r5787\_r5853\_p1863/  &  0.1070  &  0.45  &  0.451  &  20000  &  921.0 \\
204976  &  mc14\_13TeV.204976.Herwigpp\_UEEE4\_CTEQ6L1\_sbottom\_tchr\_2lep\_B750\_C150\_N60.merge.DAOD\_SUSY1.e3356\_s1982\_s2008\_r5787\_r5853\_p1863/  &  0.0431  &  0.47  &  0.472  &  20000  &  2091.8 \\
204977  &  mc14\_13TeV.204977.Herwigpp\_UEEE4\_CTEQ6L1\_sbottom\_tchr\_2lep\_B850\_C150\_N60.merge.DAOD\_SUSY1.e3356\_s1982\_s2008\_r5787\_r5853\_p1863/  &  0.0190  &  0.48  &  0.483  &  20000  &  4540.3 \\
\hline
\multicolumn{7}{c}{\textbf{gluino 2-step decays via gauginos}} \\ \hline
204952  &  mc14\_13TeV.204952.MadGraphPythia\_AUET2BCTEQ6L1\_SMGG2WWZZ\_1100\_600\_100.merge.DAOD\_SUSY1.e3345\_s1982\_s2008\_r5787\_r5853\_p1872/  &  0.1635  &  1.00  &  1.000  &  19500  &  119.3 \\
204953  &  mc14\_13TeV.204953.MadGraphPythia\_AUET2BCTEQ6L1\_SMGG2WWZZ\_1300\_700\_100.merge.DAOD\_SUSY1.e3345\_s1982\_s2008\_r5787\_r5853\_p1872/  &  0.0461  &  1.00  &  1.000  &  5000  &  108.5 \\
204954  &  mc14\_13TeV.204954.MadGraphPythia\_AUET2BCTEQ6L1\_SMGG2WWZZ\_1500\_800\_100.merge.DAOD\_SUSY1.e3345\_s1982\_s2008\_r5787\_r5853\_p1872/  &  0.0142  &  1.00  &  1.000  &  20000  &  1408.5 \\
204955  &  mc14\_13TeV.204955.MadGraphPythia\_AUET2BCTEQ6L1\_SMGG2WWZZ\_1700\_900\_100.merge.DAOD\_SUSY1.e3345\_s1982\_s2008\_r5787\_r5853\_p1872/  &  0.0047  &  1.00  &  1.000  &  15000  &  3191.5 \\
204956  &  mc14\_13TeV.204956.MadGraphPythia\_AUET2BCTEQ6L1\_SMGG2WWZZ\_1900\_1000\_100.merge.DAOD\_SUSY1.e3345\_s1982\_s2008\_r5787\_r5853\_p1872/  &  0.0016  &  1.00  &  1.000  &  15500  &  9687.5 \\
204957  &  mc14\_13TeV.204957.MadGraphPythia\_AUET2BCTEQ6L1\_SMGG2WWZZ\_2100\_1100\_100.merge.DAOD\_SUSY1.e3345\_s1982\_s2008\_r5787\_r5853\_p1872/  &  0.0006  &  1.00  &  1.000  &  5000  &  8333.3 \\
204958  &  mc14\_13TeV.204958.MadGraphPythia\_AUET2BCTEQ6L1\_SMGG2WWZZ\_2300\_1200\_100.merge.DAOD\_SUSY1.e3345\_s1982\_s2008\_r5787\_r5853\_p1872/  &  0.0002  &  1.00  &  1.000  &  13500  &  67500.0 \\
204959  &  mc14\_13TeV.204959.MadGraphPythia\_AUET2BCTEQ6L1\_SMGG2WWZZ\_1000\_700\_400.merge.DAOD\_SUSY1.e3345\_s1982\_s2008\_r5787\_r5853\_p1872/  &  0.3254  &  1.00  &  1.000  &  5000  &  15.4 \\
204960  &  mc14\_13TeV.204960.MadGraphPythia\_AUET2BCTEQ6L1\_SMGG2WWZZ\_1000\_750\_500.merge.DAOD\_SUSY1.e3345\_s1982\_s2008\_r5787\_r5853\_p1872/  &  0.3254  &  1.00  &  1.000  &  7500  &  23.0 \\
204961  &  mc14\_13TeV.204961.MadGraphPythia\_AUET2BCTEQ6L1\_SMGG2WWZZ\_1000\_800\_600.merge.DAOD\_SUSY1.e3345\_s1982\_s2008\_r5787\_r5853\_p1872/  &  0.3254  &  1.00  &  1.000  &  20000  &  61.5 \\
204962  &  mc14\_13TeV.204962.MadGraphPythia\_AUET2BCTEQ6L1\_SMGG2WWZZ\_1000\_850\_700.merge.DAOD\_SUSY1.e3345\_s1982\_s2008\_r5787\_r5853\_p1872/  &  0.3254  &  1.00  &  1.000  &  20000  &  61.5 \\
204963  &  mc14\_13TeV.204963.MadGraphPythia\_AUET2BCTEQ6L1\_SMGG2WWZZ\_1000\_900\_800.merge.DAOD\_SUSY1.e3345\_s1982\_s2008\_r5787\_r5853\_p1872/  &  0.3254  &  1.00  &  1.000  &  20000  &  61.5 \\
204964  &  mc14\_13TeV.204964.MadGraphPythia\_AUET2BCTEQ6L1\_SMGG2WWZZ\_1000\_950\_900.merge.DAOD\_SUSY1.e3345\_s1982\_s2008\_r5787\_r5853\_p1872/  &  0.3254  &  1.00  &  1.000  &  10000  &  30.7 \\
\hline
\multicolumn{7}{c}{\textbf{squark 2-step decays via gauginos}} \\ \hline
204965  &  mc14\_13TeV.204965.MadGraphPythia\_AUET2BCTEQ6L1\_SMSS2WWZZ\_650\_375\_100.merge.DAOD\_SUSY1.e3345\_s1982\_s2008\_r5787\_r5853\_p1872/  &  0.0695  &  1.00  &  1.000  &  5000  &  71.9 \\
204966  &  mc14\_13TeV.204966.MadGraphPythia\_AUET2BCTEQ6L1\_SMSS2WWZZ\_750\_425\_100.merge.DAOD\_SUSY1.e3345\_s1982\_s2008\_r5787\_r5853\_p1872/  &  0.0219  &  1.00  &  1.000  &  20000  &  913.2 \\
204967  &  mc14\_13TeV.204967.MadGraphPythia\_AUET2BCTEQ6L1\_SMSS2WWZZ\_850\_475\_100.merge.DAOD\_SUSY1.e3345\_s1982\_s2008\_r5787\_r5853\_p1872/  &  0.0075  &  1.00  &  1.000  &  17500  &  2333.3 \\
204968  &  mc14\_13TeV.204968.MadGraphPythia\_AUET2BCTEQ6L1\_SMSS2WWZZ\_950\_525\_100.merge.DAOD\_SUSY1.e3345\_s1982\_s2008\_r5787\_r5853\_p1872/  &  0.0027  &  1.00  &  1.000  &  19000  &  7037.0 \\
204969  &  mc14\_13TeV.204969.MadGraphPythia\_AUET2BCTEQ6L1\_SMSS2WWZZ\_650\_475\_300.merge.DAOD\_SUSY1.e3345\_s1982\_s2008\_r5787\_r5853\_p1872/  &  0.0695  &  1.00  &  1.000  &  20000  &  287.8 \\
204970  &  mc14\_13TeV.204970.MadGraphPythia\_AUET2BCTEQ6L1\_SMSS2WWZZ\_650\_525\_400.merge.DAOD\_SUSY1.e3345\_s1982\_s2008\_r5787\_r5853\_p1872/  &  0.0695  &  1.00  &  1.000  &  20000  &  287.8 \\
204971  &  mc14\_13TeV.204971.MadGraphPythia\_AUET2BCTEQ6L1\_SMSS2WWZZ\_650\_575\_500.merge.DAOD\_SUSY1.e3345\_s1982\_s2008\_r5787\_r5853\_p1872/  &  0.0695  &  1.00  &  1.000  &  20000  &  287.8 \\
204972  &  mc14\_13TeV.204972.MadGraphPythia\_AUET2BCTEQ6L1\_SMSS2WWZZ\_650\_625\_600.merge.DAOD\_SUSY1.e3345\_s1982\_s2008\_r5787\_r5853\_p1872/  &  0.0695  &  1.00  &  1.000  &  20000  &  287.8 \\
\hline
\hline
\end{tabular}}
\end{center}
\caption{List of simulated samples for the signal models used in this note. The dataset ID, the 
  cross-section $\sigma$, the $k$-Factor, the generator filter
  efficiency $\epsilon_{filter}$, the total number of
  generated events $N_{gen}$ and the equivalent luminosity ($L_{equiv}$) are shown.}
\label{tab:SignalSamples}
\end{sidewaystable}



\subsection{Background Samples}
\label{sec:BGSamples}

All the background samples used are listed in Tables~\ref{tab:BGSamples1}-\ref{tab:BGSamples3}.

Simulated $t\bar{t}$ events are generated using the {\sc Powheg} generator
version r2330.2 ~\cite{Nason:2004rx,Frixione:2007vw,Alioli:2010xd}, which implements
the next-to-leading order matrix element for inclusive $t\bar{t}$
production and uses the
CT10 PDF set~\cite{Lai:2010vv}. {\sc Powheg} is interfaced to {\sc Pythia} 6.427~\cite{Sjostrand:2006za}
with the CTEQ6L1 PDF set using the
Perugia2012 tune~\cite{Skands:2010ak}. The $t\bar{t}$ samples are normalised to their
next-to-next-to-leading order cross-section including the resummation of
soft gluon emission at next-to-next-to-leading-logarithmic accuracy
using Top++2.0 ~\cite{Czakon:2011xx}.

Simulated $W$/$Z$+jets samples are produced using {\sc Sherpa} 1.4.5
~\cite{gleisberg:2008ta} with massive $b$-, $c$-quarks with up to four
additional partons in the matrix element and parton shower and are
normalised to their next-to-next-to-leading order QCD theoretical
cross sections~\cite{Catani:2009sm}.

Samples of single top quark backgrounds corresponding to the $t$-, $s$-
and $Wt$ production mechanisms are generated with {\sc Powheg} version r2330.2
using the CT10 PDF set. All single-top samples are interfaced to {\sc Pythia} 6.427
with the CTEQ6L1 set of parton distribution functions using the
Perugia2012 tune. The leading-order cross-sections obtained from the
generator is used for these samples.

The processes of $t\bar{t}+V$ production are generated with {\sc MadGraph} 5
2.1.1~\cite{madgraph} + {\sc Pythia} 6.427 using the CTEQ6L1 set of parton
distribution functions and the AUET2B underlying event
tune~\cite{ATLAS:2011krm}. The samples are normalised to their
next-to-leading order cross-sections using the $k$-factors computed for $\sqrt{s}=14$~TeV~\cite{Campbell:2012dh,Kardos:2011na}. 

% 
Simulated samples for diboson processes are simulated with {\sc Sherpa} 1.4.5 ($WZ/ZZ\to\ell\ell\nu\nu$ and $W^{\pm}W^{\pm}jj\to\ell^{\pm}\ell^{\pm}\nu\nu jj$ electroweak production), {\sc Powheg} r2330.3 + {\sc Pythia 8} ($WW$, $WZ$ and $ZZ$) and {\tt gg2VV}~\cite{Kauer:2013qba,Kauer:2012hd} in {\sc Powheg} r2330.3 + {\sc Pythia 8} ($gg\to h\to WW$).

QCD multijets are not included in this study since these
backgrounds are expected to be very small.
Note that some background processes that had a non-negligible (despite small) contribution to the analysis in Run-1,
such as tri-boson or $t+Z$ production, were not included in the DC14 productions.

% Processes involving Higgs production are generated with PYTHIA8.
\begin{sidewaystable}[h]
\begin{center}
\resizebox{0.9\textwidth}{!}{
\tiny
\begin{tabular}{lllllll}
\hline
\hline
datasetID & Sample name & $\sigma$ [pb] & $k$-Factor & $\epsilon_{filter}$ & $N_{gen}$ & $L_{equiv}$\\
\hline
110401  &  mc14\_13TeV.110401.PowhegPythia\_P2012\_ttbar\_nonallhad.merge.DAOD\_SUSY1.e2928\_s1982\_s2008\_r5787\_r5853\_p1872/  &  831.7600  &  1.00  &  0.543  &  9970500  &  22.1 \\
119353  &  mc14\_13TeV.119353.MadGraphPythia\_AUET2BCTEQ6L1\_ttbarW.merge.DAOD\_SUSY1.e3214\_s1982\_s2008\_r5787\_r5853\_p1872/  &  0.2014  &  1.22  &  1.000  &  399500  &  1625.9 \\
119355  &  mc14\_13TeV.119355.MadGraphPythia\_AUET2BCTEQ6L1\_ttbarZ.merge.DAOD\_SUSY1.e3214\_s1982\_s2008\_r5787\_r5853\_p1872/  &  0.1857  &  1.56  &  1.000  &  400000  &  1380.8 \\
119583  &  mc14\_13TeV.119583.MadgraphPythia\_AUET2B\_CTEQ6L1\_ttbarWW.merge.DAOD\_SUSY1.e3214\_s1982\_s2008\_r5787\_r5853\_p1872/  &  0.0030  &  1.00  &  1.000  &  10000  &  3333.3 \\
174830  &  mc14\_13TeV.174830.MadGraphPythia\_AUET2BCTEQ6L1\_ttbarWjExcl.merge.DAOD\_SUSY1.e3214\_s1982\_s2008\_r5787\_r5853\_p1872/  &  0.1313  &  1.22  &  1.000  &  399500  &  2494.0 \\
174831  &  mc14\_13TeV.174831.MadGraphPythia\_AUET2BCTEQ6L1\_ttbarWjjIncl.merge.DAOD\_SUSY1.e3214\_s1982\_s2008\_r5787\_r5853\_p1872/  &  0.1627  &  1.22  &  1.000  &  395000  &  1990.0 \\
174832  &  mc14\_13TeV.174832.MadGraphPythia\_AUET2BCTEQ6L1\_ttbarZjExcl.merge.DAOD\_SUSY1.e3214\_s1982\_s2008\_r5787\_r5853\_p1872/  &  0.1663  &  1.56  &  1.000  &  399500  &  1539.9 \\
174833  &  mc14\_13TeV.174833.MadGraphPythia\_AUET2BCTEQ6L1\_ttbarZjjIncl.merge.DAOD\_SUSY1.e3214\_s1982\_s2008\_r5787\_r5853\_p1872/  &  0.2076  &  1.56  &  1.000  &  399500  &  1233.6 \\
110070  &  mc14\_13TeV.110070.PowhegPythia\_P2012\_singletop\_tchan\_lept\_top.merge.DAOD\_SUSY1.e3049\_s1982\_s2008\_r5787\_r5853\_p1872/  &  43.7550  &  1.00  &  1.000  &  999000  &  22.8 \\

110071  &  mc14\_13TeV.110071.PowhegPythia\_P2012\_singletop\_tchan\_lept\_antitop.merge.DAOD\_SUSY1.e3049\_s1982\_s2008\_r5787\_r5853\_p1872/  &  25.7780  &  1.00  &  1.000  &  1000000  &  38.8 \\

110302  &  mc14\_13TeV.110302.PowhegPythia\_P2012\_st\_schan\_lep.merge.DAOD\_SUSY1.e3049\_s1982\_s2008\_r5787\_r5853\_p1872/  &  3.3514  &  1.00  &  1.000  &  999500  &  298.2 \\

110305  &  mc14\_13TeV.110305.PowhegPythia\_P2012\_st\_Wtchan\_incl\_DR.merge.DAOD\_SUSY1.e3049\_s1982\_s2008\_r5787\_r5853\_p1872/  &  68.4590  &  1.00  &  1.000  &  958500  &  14.0 \\
%\hline
%161105 & mc14\_13TeV.161105.Pythia8\_AU2CTEQ6L1\_WH125\_WW2lep.merge.DAOD\_SUSY1.e2743\_s1982\_s2008\_r5787\_r5853\_p1872/ & 0.24 & 1.00 & 0.104 & 0.025 & 100000  \\ 
%161155 & mc14\_13TeV.161155.Pythia8\_AU2CTEQ6L1\_ZH125\_WW2lep.merge.DAOD\_SUSY1.e2743\_s1982\_s2008\_r5787\_r5853\_p1872/ & 0.01 & 1.00 & 1.000 & 0.014 & 100000  \\ 
%161305 & mc14\_13TeV.161305.Pythia8\_AU2CTEQ6L1\_ttH125\_WWinclusive.merge.DAOD\_SUSY1.e2743\_s1982\_s2008\_r5787\_r5853\_p1872/ & 0.07 & 1.00 & 1.000 & 0.071 & 199500  \\ 
\hline
167740  &  mc14\_13TeV.167740.Sherpa\_CT10\_WenuMassiveCBPt0\_BFilter.merge.DAOD\_SUSY1.e2822\_s1982\_s2008\_r5787\_r5853\_p1872/  &  18795.0000  &  1.07  &  0.018  &  497000  &  1.4 \\
167741  &  mc14\_13TeV.167741.Sherpa\_CT10\_WenuMassiveCBPt0\_CJetFilterBVeto.merge.DAOD\_SUSY1.e2822\_s1982\_s2008\_r5787\_r5853\_p1872/  &  18804.0000  &  1.07  &  0.063  &  494500  &  0.4 \\
167742  &  mc14\_13TeV.167742.Sherpa\_CT10\_WenuMassiveCBPt0\_CJetVetoBVeto.merge.DAOD\_SUSY1.e2822\_s1982\_s2008\_r5787\_r5853\_p1872/  &  18843.0000  &  1.07  &  0.919  &  1000000  &  0.1 \\
167743  &  mc14\_13TeV.167743.Sherpa\_CT10\_WmunuMassiveCBPt0\_BFilter.merge.DAOD\_SUSY1.e2822\_s1982\_s2008\_r5787\_r5853\_p1872/  &  18793.0000  &  1.07  &  0.018  &  499500  &  1.4 \\
167744  &  mc14\_13TeV.167744.Sherpa\_CT10\_WmunuMassiveCBPt0\_CJetFilterBVeto.merge.DAOD\_SUSY1.e2822\_s1982\_s2008\_r5787\_r5853\_p1872/  &  18788.0000  &  1.07  &  0.056  &  498000  &  0.4 \\
167745  &  mc14\_13TeV.167745.Sherpa\_CT10\_WmunuMassiveCBPt0\_CJetVetoBVeto.merge.DAOD\_SUSY1.e2822\_s1982\_s2008\_r5787\_r5853\_p1872/  &  18797.0000  &  1.07  &  0.926  &  989500  &  0.1 \\
167746  &  mc14\_13TeV.167746.Sherpa\_CT10\_WtaunuMassiveCBPt0\_BFilter.merge.DAOD\_SUSY1.e2822\_s1982\_s2008\_r5787\_r5853\_p1872/  &  18795.0000  &  1.07  &  0.018  &  500000  &  1.4 \\
167747  &  mc14\_13TeV.167747.Sherpa\_CT10\_WtaunuMassiveCBPt0\_CJetFilterBVeto.merge.DAOD\_SUSY1.e2822\_s1982\_s2008\_r5787\_r5853\_p1872/  &  18808.0000  &  1.07  &  0.060  &  500000  &  0.4 \\
167748  &  mc14\_13TeV.167748.Sherpa\_CT10\_WtaunuMassiveCBPt0\_CJetVetoBVeto.merge.DAOD\_SUSY1.e2822\_s1982\_s2008\_r5787\_r5853\_p1872/  &  18800.0000  &  1.07  &  0.923  &  499500  &  0.0 \\
167761  &  mc14\_13TeV.167761.Sherpa\_CT10\_WenuMassiveCBPt70\_140\_BFilter.merge.DAOD\_SUSY1.e2822\_s1982\_s2008\_r5787\_r5853\_p1872/  &  557.8700  &  1.07  &  0.055  &  349500  &  10.6 \\
167762  &  mc14\_13TeV.167762.Sherpa\_CT10\_WenuMassiveCBPt70\_140\_CJetFilterBVeto.merge.DAOD\_SUSY1.e2822\_s1982\_s2008\_r5787\_r5853\_p1872/  &  557.9200  &  1.07  &  0.227  &  350000  &  2.6 \\
167763  &  mc14\_13TeV.167763.Sherpa\_CT10\_WenuMassiveCBPt70\_140\_CJetVetoBVeto.merge.DAOD\_SUSY1.e2822\_s1982\_s2008\_r5787\_r5853\_p1872/  &  558.3700  &  1.07  &  0.718  &  985500  &  2.3 \\
167764  &  mc14\_13TeV.167764.Sherpa\_CT10\_WmunuMassiveCBPt70\_140\_BFilter.merge.DAOD\_SUSY1.e2822\_s1982\_s2008\_r5787\_r5853\_p1872/  &  557.8700  &  1.07  &  0.055  &  345500  &  10.5 \\
167765  &  mc14\_13TeV.167765.Sherpa\_CT10\_WmunuMassiveCBPt70\_140\_CJetFilterBVeto.merge.DAOD\_SUSY1.e2822\_s1982\_s2008\_r5787\_r5853\_p1872/  &  558.2500  &  1.07  &  0.220  &  350000  &  2.7 \\
167766  &  mc14\_13TeV.167766.Sherpa\_CT10\_WmunuMassiveCBPt70\_140\_CJetVetoBVeto.merge.DAOD\_SUSY1.e2822\_s1982\_s2008\_r5787\_r5853\_p1872/  &  557.4800  &  1.07  &  0.724  &  500000  &  1.2 \\
167767  &  mc14\_13TeV.167767.Sherpa\_CT10\_WtaunuMassiveCBPt70\_140\_BFilter.merge.DAOD\_SUSY1.e2822\_s1982\_s2008\_r5787\_r5853\_p1872/  &  557.9200  &  1.07  &  0.055  &  349500  &  10.6 \\
167768  &  mc14\_13TeV.167768.Sherpa\_CT10\_WtaunuMassiveCBPt70\_140\_CJetFilterBVeto.merge.DAOD\_SUSY1.e2822\_s1982\_s2008\_r5787\_r5853\_p1872/  &  557.6800  &  1.07  &  0.224  &  349500  &  2.6 \\
167769  &  mc14\_13TeV.167769.Sherpa\_CT10\_WtaunuMassiveCBPt70\_140\_CJetVetoBVeto.merge.DAOD\_SUSY1.e2822\_s1982\_s2008\_r5787\_r5853\_p1872/  &  558.5900  &  1.07  &  0.720  &  999500  &  2.3 \\
167770  &  mc14\_13TeV.167770.Sherpa\_CT10\_WenuMassiveCBPt140\_280\_BFilter.merge.DAOD\_SUSY1.e2822\_s1982\_s2008\_r5787\_r5853\_p1872/  &  81.8640  &  1.07  &  0.075  &  198000  &  30.1 \\
167771  &  mc14\_13TeV.167771.Sherpa\_CT10\_WenuMassiveCBPt140\_280\_CJetFilterBVeto.merge.DAOD\_SUSY1.e2822\_s1982\_s2008\_r5787\_r5853\_p1872/  &  81.9180  &  1.07  &  0.255  &  200000  &  8.9 \\
167772  &  mc14\_13TeV.167772.Sherpa\_CT10\_WenuMassiveCBPt140\_280\_CJetVetoBVeto.merge.DAOD\_SUSY1.e2822\_s1982\_s2008\_r5787\_r5853\_p1872/  &  81.7640  &  1.07  &  0.671  &  398000  &  6.8 \\
167773  &  mc14\_13TeV.167773.Sherpa\_CT10\_WmunuMassiveCBPt140\_280\_BFilter.merge.DAOD\_SUSY1.e2822\_s1982\_s2008\_r5787\_r5853\_p1872/  &  81.7750  &  1.07  &  0.074  &  349500  &  54.0 \\
167774  &  mc14\_13TeV.167774.Sherpa\_CT10\_WmunuMassiveCBPt140\_280\_CJetFilterBVeto.merge.DAOD\_SUSY1.e2822\_s1982\_s2008\_r5787\_r5853\_p1872/  &  81.8130  &  1.07  &  0.250  &  349000  &  15.9 \\
167775  &  mc14\_13TeV.167775.Sherpa\_CT10\_WmunuMassiveCBPt140\_280\_CJetVetoBVeto.merge.DAOD\_SUSY1.e2822\_s1982\_s2008\_r5787\_r5853\_p1872/  &  81.9250  &  1.07  &  0.676  &  999500  &  16.9 \\
167776  &  mc14\_13TeV.167776.Sherpa\_CT10\_WtaunuMassiveCBPt140\_280\_BFilter.merge.DAOD\_SUSY1.e2822\_s1982\_s2008\_r5787\_r5853\_p1872/  &  81.8670  &  1.07  &  0.075  &  200000  &  30.4 \\
167777  &  mc14\_13TeV.167777.Sherpa\_CT10\_WtaunuMassiveCBPt140\_280\_CJetFilterBVeto.merge.DAOD\_SUSY1.e2822\_s1982\_s2008\_r5787\_r5853\_p1872/  &  81.7000  &  1.07  &  0.253  &  199927  &  9.0 \\
167778  &  mc14\_13TeV.167778.Sherpa\_CT10\_WtaunuMassiveCBPt140\_280\_CJetVetoBVeto.merge.DAOD\_SUSY1.e2822\_s1982\_s2008\_r5787\_r5853\_p1872/  &  81.7680  &  1.07  &  0.673  &  400000  &  6.8 \\
167779  &  mc14\_13TeV.167779.Sherpa\_CT10\_WenuMassiveCBPt280\_500\_BFilter.merge.DAOD\_SUSY1.e2822\_s1982\_s2008\_r5787\_r5853\_p1872/  &  6.2271  &  1.00  &  0.098  &  399500  &  654.6 \\
167780  &  mc14\_13TeV.167780.Sherpa\_CT10\_WenuMassiveCBPt280\_500\_CJetFilterBVeto.merge.DAOD\_SUSY1.e2822\_s1982\_s2008\_r5787\_r5853\_p1872/  &  6.2128  &  1.07  &  0.273  &  99500  &  54.8 \\
167781  &  mc14\_13TeV.167781.Sherpa\_CT10\_WenuMassiveCBPt280\_500\_CJetVetoBVeto.merge.DAOD\_SUSY1.e2822\_s1982\_s2008\_r5787\_r5853\_p1872/  &  6.2357  &  1.07  &  0.630  &  200000  &  47.6 \\
167782  &  mc14\_13TeV.167782.Sherpa\_CT10\_WmunuMassiveCBPt280\_500\_BFilter.merge.DAOD\_SUSY1.e2822\_s1982\_s2008\_r5787\_r5853\_p1872/  &  6.2188  &  1.07  &  0.097  &  199500  &  309.1 \\
167783  &  mc14\_13TeV.167783.Sherpa\_CT10\_WmunuMassiveCBPt280\_500\_CJetFilterBVeto.merge.DAOD\_SUSY1.e2822\_s1982\_s2008\_r5787\_r5853\_p1872/  &  6.2277  &  1.07  &  0.266  &  199000  &  112.3 \\
167784  &  mc14\_13TeV.167784.Sherpa\_CT10\_WmunuMassiveCBPt280\_500\_CJetVetoBVeto.merge.DAOD\_SUSY1.e2822\_s1982\_s2008\_r5787\_r5853\_p1872/  &  6.2271  &  1.07  &  0.635  &  399500  &  94.4 \\
167785  &  mc14\_13TeV.167785.Sherpa\_CT10\_WtaunuMassiveCBPt280\_500\_BFilter.merge.DAOD\_SUSY1.e2822\_s1982\_s2008\_r5787\_r5853\_p1872/  &  6.2235  &  1.07  &  0.097  &  399500  &  618.5 \\
167786  &  mc14\_13TeV.167786.Sherpa\_CT10\_WtaunuMassiveCBPt280\_500\_CJetFilterBVeto.merge.DAOD\_SUSY1.e2822\_s1982\_s2008\_r5787\_r5853\_p1872/  &  6.2311  &  1.07  &  0.271  &  100000  &  55.3 \\
167787  &  mc14\_13TeV.167787.Sherpa\_CT10\_WtaunuMassiveCBPt280\_500\_CJetVetoBVeto.merge.DAOD\_SUSY1.e2822\_s1982\_s2008\_r5787\_r5853\_p1872/  &  6.2060  &  1.07  &  0.631  &  200000  &  47.7 \\
167788  &  mc14\_13TeV.167788.Sherpa\_CT10\_WenuMassiveCBPt500\_BFilter.merge.DAOD\_SUSY1.e2822\_s1982\_s2008\_r5787\_r5853\_p1872/  &  0.5143  &  1.07  &  0.118  &  10000  &  154.0 \\
167789  &  mc14\_13TeV.167789.Sherpa\_CT10\_WenuMassiveCBPt500\_CJetFilterBVeto.merge.DAOD\_SUSY1.e2822\_s1982\_s2008\_r5787\_r5853\_p1872/  &  0.5208  &  1.07  &  0.289  &  10000  &  62.1 \\
167790  &  mc14\_13TeV.167790.Sherpa\_CT10\_WenuMassiveCBPt500\_CJetVetoBVeto.merge.DAOD\_SUSY1.e2822\_s1982\_s2008\_r5787\_r5853\_p1872/  &  0.5131  &  1.07  &  0.597  &  40000  &  122.0 \\
167791  &  mc14\_13TeV.167791.Sherpa\_CT10\_WmunuMassiveCBPt500\_BFilter.merge.DAOD\_SUSY1.e2822\_s1982\_s2008\_r5787\_r5853\_p1872/  &  0.5149  &  1.07  &  0.118  &  398000  &  6122.0 \\
167792  &  mc14\_13TeV.167792.Sherpa\_CT10\_WmunuMassiveCBPt500\_CJetFilterBVeto.merge.DAOD\_SUSY1.e2822\_s1982\_s2008\_r5787\_r5853\_p1872/  &  0.5132  &  1.07  &  0.280  &  100000  &  650.4 \\
167793  &  mc14\_13TeV.167793.Sherpa\_CT10\_WmunuMassiveCBPt500\_CJetVetoBVeto.merge.DAOD\_SUSY1.e2822\_s1982\_s2008\_r5787\_r5853\_p1872/  &  0.5130  &  1.07  &  0.603  &  199000  &  601.2 \\
167794  &  mc14\_13TeV.167794.Sherpa\_CT10\_WtaunuMassiveCBPt500\_BFilter.merge.DAOD\_SUSY1.e2822\_s1982\_s2008\_r5787\_r5853\_p1872/  &  0.5169  &  1.07  &  0.119  &  10000  &  151.9 \\
167795  &  mc14\_13TeV.167795.Sherpa\_CT10\_WtaunuMassiveCBPt500\_CJetFilterBVeto.merge.DAOD\_SUSY1.e2822\_s1982\_s2008\_r5787\_r5853\_p1872/  &  0.5136  &  1.07  &  0.291  &  10000  &  62.5 \\
167796  &  mc14\_13TeV.167796.Sherpa\_CT10\_WtaunuMassiveCBPt500\_CJetVetoBVeto.merge.DAOD\_SUSY1.e2822\_s1982\_s2008\_r5787\_r5853\_p1872/  &  0.5145  &  1.07  &  0.600  &  40000  &  121.1 \\
180534  &  mc14\_13TeV.180534.Sherpa\_CT10\_WenuMassiveCBPt40\_70\_BFilter.merge.DAOD\_SUSY1.e2822\_s1982\_s2008\_r5787\_r5853\_p1872/  &  1315.7000  &  1.07  &  0.041  &  350000  &  6.1 \\
180535  &  mc14\_13TeV.180535.Sherpa\_CT10\_WenuMassiveCBPt40\_70\_CJetFilterBVeto.merge.DAOD\_SUSY1.e2822\_s1982\_s2008\_r5787\_r5853\_p1872/  &  1316.8000  &  1.07  &  0.191  &  347500  &  1.3 \\
180536  &  mc14\_13TeV.180536.Sherpa\_CT10\_WenuMassiveCBPt40\_70\_CJetVetoBVeto.merge.DAOD\_SUSY1.e2822\_s1982\_s2008\_r5787\_r5853\_p1872/  &  1312.7000  &  1.07  &  0.767  &  499000  &  0.5 \\
180537  &  mc14\_13TeV.180537.Sherpa\_CT10\_WmunuMassiveCBPt40\_70\_BFilter.merge.DAOD\_SUSY1.e2822\_s1982\_s2008\_r5787\_r5853\_p1872/  &  1311.0000  &  1.07  &  0.041  &  10000  &  0.2 \\
180538  &  mc14\_13TeV.180538.Sherpa\_CT10\_WmunuMassiveCBPt40\_70\_CJetFilterBVeto.merge.DAOD\_SUSY1.e2822\_s1982\_s2008\_r5787\_r5853\_p1872/  &  1302.1000  &  1.07  &  0.184  &  10000  &  0.0 \\
180539  &  mc14\_13TeV.180539.Sherpa\_CT10\_WmunuMassiveCBPt40\_70\_CJetVetoBVeto.merge.DAOD\_SUSY1.e2822\_s1982\_s2008\_r5787\_r5853\_p1872/  &  1323.8000  &  1.07  &  0.772  &  40000  &  0.0 \\
180540  &  mc14\_13TeV.180540.Sherpa\_CT10\_WtaunuMassiveCBPt40\_70\_BFilter.merge.DAOD\_SUSY1.e2822\_s1982\_s2008\_r5787\_r5853\_p1872/  &  1316.1000  &  1.07  &  0.041  &  995000  &  17.2 \\
180541  &  mc14\_13TeV.180541.Sherpa\_CT10\_WtaunuMassiveCBPt40\_70\_CJetFilterBVeto.merge.DAOD\_SUSY1.e2822\_s1982\_s2008\_r5787\_r5853\_p1872/  &  1314.2000  &  1.07  &  0.189  &  349500  &  1.3 \\
180542  &  mc14\_13TeV.180542.Sherpa\_CT10\_WtaunuMassiveCBPt40\_70\_CJetVetoBVeto.merge.DAOD\_SUSY1.e2822\_s1982\_s2008\_r5787\_r5853\_p1872/  &  1314.4000  &  1.07  &  0.769  &  349000  &  0.3 \\
\hline
\hline
\end{tabular}}
\end{center}
\caption{List of simulated samples for
  top-related background processes as well as $W$+jets. The dataset ID, the generator
  cross-section $\sigma$, the $k$-Factor, the generator filter
  efficiency $\epsilon_{filter}$, the total number of
  generated events $N_{gen}$ and the equivalent luminosity ($L_{equiv}$) are shown.}
\label{tab:BGSamples1}
\end{sidewaystable}


\begin{sidewaystable}[h]
\begin{center}
\resizebox{\textwidth}{!}{
\tiny
\begin{tabular}{lllllll}
\hline
\hline
datasetID & Sample name & $\sigma$ [pb] & $k$-Factor & $\epsilon_{filter}$ & $N_{gen}$ & $L_{equiv}$\\
\hline
167749  &  mc14\_13TeV.167749.Sherpa\_CT10\_ZeeMassiveCBPt0\_BFilter.merge.DAOD\_SUSY1.e2798\_s1982\_s2008\_r5787\_r5853\_p1872/  &  1928.9000  &  1.09  &  0.038  &  500000  &  6.3 \\
167750  &  mc14\_13TeV.167750.Sherpa\_CT10\_ZeeMassiveCBPt0\_CFilterBVeto.merge.DAOD\_SUSY1.e2798\_s1982\_s2008\_r5787\_r5853\_p1872/  &  1926.8000  &  1.09  &  0.328  &  149500  &  0.2 \\
167751  &  mc14\_13TeV.167751.Sherpa\_CT10\_ZeeMassiveCBPt0\_CVetoBVeto.merge.DAOD\_SUSY1.e2798\_s1982\_s2008\_r5787\_r5853\_p1872/  &  1938.8000  &  1.09  &  0.636  &  150000  &  0.1 \\
167752  &  mc14\_13TeV.167752.Sherpa\_CT10\_ZmumuMassiveCBPt0\_BFilter.merge.DAOD\_SUSY1.e2798\_s1982\_s2008\_r5787\_r5853\_p1872/  &  1929.9000  &  1.09  &  0.038  &  499500  &  6.2 \\
167753  &  mc14\_13TeV.167753.Sherpa\_CT10\_ZmumuMassiveCBPt0\_CFilterBVeto.merge.DAOD\_SUSY1.e2798\_s1982\_s2008\_r5787\_r5853\_p1872/  &  1927.0000  &  1.09  &  0.326  &  150000  &  0.2 \\
167754  &  mc14\_13TeV.167754.Sherpa\_CT10\_ZmumuMassiveCBPt0\_CVetoBVeto.merge.DAOD\_SUSY1.e2798\_s1982\_s2008\_r5787\_r5853\_p1872/  &  1936.4000  &  1.09  &  0.636  &  150000  &  0.1 \\
167755  &  mc14\_13TeV.167755.Sherpa\_CT10\_ZtautauMassiveCBPt0\_BFilter.merge.DAOD\_SUSY1.e2798\_s1982\_s2008\_r5787\_r5853\_p1872/  &  1929.7000  &  1.09  &  0.038  &  499500  &  6.2 \\
167756  &  mc14\_13TeV.167756.Sherpa\_CT10\_ZtautauMassiveCBPt0\_CFilterBVeto.merge.DAOD\_SUSY1.e2798\_s1982\_s2008\_r5787\_r5853\_p1872/  &  1928.4000  &  1.09  &  0.327  &  150000  &  0.2 \\
167757  &  mc14\_13TeV.167757.Sherpa\_CT10\_ZtautauMassiveCBPt0\_CVetoBVeto.merge.DAOD\_SUSY1.e2798\_s1982\_s2008\_r5787\_r5853\_p1872/  &  1925.5000  &  1.09  &  0.634  &  149500  &  0.1 \\
167797  &  mc14\_13TeV.167797.Sherpa\_CT10\_ZeeMassiveCBPt70\_140\_BFilter.merge.DAOD\_SUSY1.e2798\_s1982\_s2008\_r5787\_r5853\_p1872/  &  66.7490  &  1.09  &  0.102  &  300000  &  40.4 \\
167798  &  mc14\_13TeV.167798.Sherpa\_CT10\_ZeeMassiveCBPt70\_140\_CFilterBVeto.merge.DAOD\_SUSY1.e2798\_s1982\_s2008\_r5787\_r5853\_p1872/  &  66.8320  &  1.09  &  0.394  &  99500  &  3.5 \\
167799  &  mc14\_13TeV.167799.Sherpa\_CT10\_ZeeMassiveCBPt70\_140\_CVetoBVeto.merge.DAOD\_SUSY1.e2798\_s1982\_s2008\_r5787\_r5853\_p1872/  &  66.7900  &  1.09  &  0.505  &  100000  &  2.7 \\
167800  &  mc14\_13TeV.167800.Sherpa\_CT10\_ZmumuMassiveCBPt70\_140\_BFilter.merge.DAOD\_SUSY1.e2798\_s1982\_s2008\_r5787\_r5853\_p1872/  &  66.7440  &  1.09  &  0.102  &  299500  &  40.4 \\
167801  &  mc14\_13TeV.167801.Sherpa\_CT10\_ZmumuMassiveCBPt70\_140\_CFilterBVeto.merge.DAOD\_SUSY1.e2798\_s1982\_s2008\_r5787\_r5853\_p1872/  &  66.6270  &  1.09  &  0.395  &  100000  &  3.5 \\
167802  &  mc14\_13TeV.167802.Sherpa\_CT10\_ZmumuMassiveCBPt70\_140\_CVetoBVeto.merge.DAOD\_SUSY1.e2798\_s1982\_s2008\_r5787\_r5853\_p1872/  &  66.9090  &  1.09  &  0.506  &  99500  &  2.7 \\
167803  &  mc14\_13TeV.167803.Sherpa\_CT10\_ZtautauMassiveCBPt70\_140\_BFilter.merge.DAOD\_SUSY1.e2798\_s1982\_s2008\_r5787\_r5853\_p1872/  &  66.8420  &  1.09  &  0.102  &  299500  &  40.3 \\
167804  &  mc14\_13TeV.167804.Sherpa\_CT10\_ZtautauMassiveCBPt70\_140\_CFilterBVeto.merge.DAOD\_SUSY1.e2798\_s1982\_s2008\_r5787\_r5853\_p1872/  &  66.8820  &  1.09  &  0.393  &  99500  &  3.5 \\
167805  &  mc14\_13TeV.167805.Sherpa\_CT10\_ZtautauMassiveCBPt70\_140\_CVetoBVeto.merge.DAOD\_SUSY1.e2798\_s1982\_s2008\_r5787\_r5853\_p1872/  &  66.9950  &  1.09  &  0.506  &  100000  &  2.7 \\
167809  &  mc14\_13TeV.167809.Sherpa\_CT10\_ZeeMassiveCBPt140\_280\_BFilter.merge.DAOD\_SUSY1.e2798\_s1982\_s2008\_r5787\_r5853\_p1872/  &  10.6360  &  1.09  &  0.118  &  200000  &  146.2 \\
167810  &  mc14\_13TeV.167810.Sherpa\_CT10\_ZeeMassiveCBPt140\_280\_CFilterBVeto.merge.DAOD\_SUSY1.e2798\_s1982\_s2008\_r5787\_r5853\_p1872/  &  10.6210  &  1.09  &  0.407  &  50000  &  10.6 \\
167811  &  mc14\_13TeV.167811.Sherpa\_CT10\_ZeeMassiveCBPt140\_280\_CVetoBVeto.merge.DAOD\_SUSY1.e2798\_s1982\_s2008\_r5787\_r5853\_p1872/  &  10.6170  &  1.09  &  0.473  &  50000  &  9.1 \\
167812  &  mc14\_13TeV.167812.Sherpa\_CT10\_ZmumuMassiveCBPt140\_280\_BFilter.merge.DAOD\_SUSY1.e2798\_s1982\_s2008\_r5787\_r5853\_p1872/  &  10.6290  &  1.09  &  0.118  &  199500  &  145.9 \\
167813  &  mc14\_13TeV.167813.Sherpa\_CT10\_ZmumuMassiveCBPt140\_280\_CFilterBVeto.merge.DAOD\_SUSY1.e2798\_s1982\_s2008\_r5787\_r5853\_p1872/  &  10.6500  &  1.09  &  0.411  &  50000  &  10.5 \\
167814  &  mc14\_13TeV.167814.Sherpa\_CT10\_ZmumuMassiveCBPt140\_280\_CVetoBVeto.merge.DAOD\_SUSY1.e2798\_s1982\_s2008\_r5787\_r5853\_p1872/  &  10.6750  &  1.09  &  0.476  &  50000  &  9.0 \\
167815  &  mc14\_13TeV.167815.Sherpa\_CT10\_ZtautauMassiveCBPt140\_280\_BFilter.merge.DAOD\_SUSY1.e2798\_s1982\_s2008\_r5787\_r5853\_p1872/  &  10.6260  &  1.09  &  0.118  &  200000  &  146.3 \\
167816  &  mc14\_13TeV.167816.Sherpa\_CT10\_ZtautauMassiveCBPt140\_280\_CFilterBVeto.merge.DAOD\_SUSY1.e2798\_s1982\_s2008\_r5787\_r5853\_p1872/  &  10.6270  &  1.09  &  0.409  &  50000  &  10.6 \\
167817  &  mc14\_13TeV.167817.Sherpa\_CT10\_ZtautauMassiveCBPt140\_280\_CVetoBVeto.merge.DAOD\_SUSY1.e2798\_s1982\_s2008\_r5787\_r5853\_p1872/  &  10.6690  &  1.09  &  0.475  &  50000  &  9.1 \\
167821  &  mc14\_13TeV.167821.Sherpa\_CT10\_ZeeMassiveCBPt280\_500\_BFilter.merge.DAOD\_SUSY1.e2798\_s1982\_s2008\_r5787\_r5853\_p1872/  &  0.8306  &  1.09  &  0.134  &  99000  &  816.0 \\
167822  &  mc14\_13TeV.167822.Sherpa\_CT10\_ZeeMassiveCBPt280\_500\_CFilterBVeto.merge.DAOD\_SUSY1.e2798\_s1982\_s2008\_r5787\_r5853\_p1872/  &  0.8350  &  1.09  &  0.423  &  40000  &  103.9 \\
167823  &  mc14\_13TeV.167823.Sherpa\_CT10\_ZeeMassiveCBPt280\_500\_CVetoBVeto.merge.DAOD\_SUSY1.e2798\_s1982\_s2008\_r5787\_r5853\_p1872/  &  0.8326  &  1.09  &  0.445  &  40000  &  99.0 \\
167824  &  mc14\_13TeV.167824.Sherpa\_CT10\_ZmumuMassiveCBPt280\_500\_BFilter.merge.DAOD\_SUSY1.e2798\_s1982\_s2008\_r5787\_r5853\_p1872/  &  0.8309  &  1.09  &  0.133  &  100000  &  830.2 \\
167825  &  mc14\_13TeV.167825.Sherpa\_CT10\_ZmumuMassiveCBPt280\_500\_CFilterBVeto.merge.DAOD\_SUSY1.e2798\_s1982\_s2008\_r5787\_r5853\_p1872/  &  0.8321  &  1.09  &  0.425  &  40000  &  103.8 \\
167826  &  mc14\_13TeV.167826.Sherpa\_CT10\_ZmumuMassiveCBPt280\_500\_CVetoBVeto.merge.DAOD\_SUSY1.e2798\_s1982\_s2008\_r5787\_r5853\_p1872/  &  0.8351  &  1.09  &  0.445  &  40000  &  98.7 \\
167827  &  mc14\_13TeV.167827.Sherpa\_CT10\_ZtautauMassiveCBPt280\_500\_BFilter.merge.DAOD\_SUSY1.e2798\_s1982\_s2008\_r5787\_r5853\_p1872/  &  0.8314  &  1.09  &  0.133  &  99500  &  825.5 \\
167828  &  mc14\_13TeV.167828.Sherpa\_CT10\_ZtautauMassiveCBPt280\_500\_CFilterBVeto.merge.DAOD\_SUSY1.e2798\_s1982\_s2008\_r5787\_r5853\_p1872/  &  0.8334  &  1.09  &  0.424  &  40000  &  103.9 \\
167829  &  mc14\_13TeV.167829.Sherpa\_CT10\_ZtautauMassiveCBPt280\_500\_CVetoBVeto.merge.DAOD\_SUSY1.e2798\_s1982\_s2008\_r5787\_r5853\_p1872/  &  0.8301  &  1.09  &  0.443  &  40000  &  99.8 \\
167833  &  mc14\_13TeV.167833.Sherpa\_CT10\_ZeeMassiveCBPt500\_BFilter.merge.DAOD\_SUSY1.e2798\_s1982\_s2008\_r5787\_r5853\_p1872/  &  0.0684  &  1.09  &  0.146  &  9500  &  872.7 \\
167834  &  mc14\_13TeV.167834.Sherpa\_CT10\_ZeeMassiveCBPt500\_CFilterBVeto.merge.DAOD\_SUSY1.e2798\_s1982\_s2008\_r5787\_r5853\_p1872/  &  0.0684  &  1.09  &  0.434  &  10000  &  309.0 \\
167835  &  mc14\_13TeV.167835.Sherpa\_CT10\_ZeeMassiveCBPt500\_CVetoBVeto.merge.DAOD\_SUSY1.e2798\_s1982\_s2008\_r5787\_r5853\_p1872/  &  0.0685  &  1.09  &  0.419  &  40000  &  1278.6 \\
167836  &  mc14\_13TeV.167836.Sherpa\_CT10\_ZmumuMassiveCBPt500\_BFilter.merge.DAOD\_SUSY1.e2798\_s1982\_s2008\_r5787\_r5853\_p1872/  &  0.0683  &  1.09  &  0.144  &  10000  &  932.8 \\
167837  &  mc14\_13TeV.167837.Sherpa\_CT10\_ZmumuMassiveCBPt500\_CFilterBVeto.merge.DAOD\_SUSY1.e2798\_s1982\_s2008\_r5787\_r5853\_p1872/  &  0.0690  &  1.09  &  0.441  &  10000  &  301.5 \\
167838  &  mc14\_13TeV.167838.Sherpa\_CT10\_ZmumuMassiveCBPt500\_CVetoBVeto.merge.DAOD\_SUSY1.e2798\_s1982\_s2008\_r5787\_r5853\_p1872/  &  0.0687  &  1.09  &  0.417  &  40000  &  1281.0 \\
167839  &  mc14\_13TeV.167839.Sherpa\_CT10\_ZtautauMassiveCBPt500\_BFilter.merge.DAOD\_SUSY1.e2798\_s1982\_s2008\_r5787\_r5853\_p1872/  &  0.0685  &  1.09  &  0.144  &  10000  &  930.1 \\
167840  &  mc14\_13TeV.167840.Sherpa\_CT10\_ZtautauMassiveCBPt500\_CFilterBVeto.merge.DAOD\_SUSY1.e2798\_s1982\_s2008\_r5787\_r5853\_p1872/  &  0.0688  &  1.09  &  0.441  &  10000  &  302.4 \\
167841  &  mc14\_13TeV.167841.Sherpa\_CT10\_ZtautauMassiveCBPt500\_CVetoBVeto.merge.DAOD\_SUSY1.e2798\_s1982\_s2008\_r5787\_r5853\_p1872/  &  0.0679  &  1.09  &  0.415  &  40000  &  1302.3 \\
\hline
\hline
\end{tabular}}
\end{center}
\caption{List of simulated samples for
   $Z\to\ell\ell$+jets. The dataset ID, the generator
  cross-section $\sigma$, the $k$-Factor, the generator filter
  efficiency $\epsilon_{filter}$, the total number of
  generated events $N_{gen}$ and the equivalent luminosity ($L_{equiv}$) are shown.}
\label{tab:BGSamples2}
\end{sidewaystable}


\begin{sidewaystable}[h]
\begin{center}
\resizebox{\textwidth}{!}{
\tiny
\begin{tabular}{lllllll}
\hline
\hline
datasetID & Sample name & $\sigma$ [pb] & $k$-Factor & $\epsilon_{filter}$ & $N_{gen}$ & $L_{equiv}$\\
\hline
187150  &  mc14\_13TeV.187150.PowhegPythia8\_AU2CT10\_WpWm\_ee.merge.DAOD\_SUSY1.e3059\_s1982\_s2008\_r5787\_r5853\_p1872/  &  1.1792  &  1.00  &  1.000  &  100000  &  84.8 \\
187151  &  mc14\_13TeV.187151.PowhegPythia8\_AU2CT10\_WpWm\_mue.merge.DAOD\_SUSY1.e3059\_s1982\_s2008\_r5787\_r5853\_p1872/  &  1.1790  &  1.00  &  1.000  &  200000  &  169.6 \\
187152  &  mc14\_13TeV.187152.PowhegPythia8\_AU2CT10\_WpWm\_taue.merge.DAOD\_SUSY1.e3059\_s1982\_s2008\_r5787\_r5853\_p1872/  &  1.1790  &  1.00  &  1.000  &  199500  &  169.2 \\
187153  &  mc14\_13TeV.187153.PowhegPythia8\_AU2CT10\_WpWm\_emu.merge.DAOD\_SUSY1.e3059\_s1982\_s2008\_r5787\_r5853\_p1872/  &  1.1790  &  1.00  &  1.000  &  199500  &  169.2 \\
187154  &  mc14\_13TeV.187154.PowhegPythia8\_AU2CT10\_WpWm\_mumu.merge.DAOD\_SUSY1.e3059\_s1982\_s2008\_r5787\_r5853\_p1872/  &  1.1792  &  1.00  &  1.000  &  100000  &  84.8 \\
187155  &  mc14\_13TeV.187155.PowhegPythia8\_AU2CT10\_WpWm\_taumu.merge.DAOD\_SUSY1.e3059\_s1982\_s2008\_r5787\_r5853\_p1872/  &  1.1790  &  1.00  &  1.000  &  199500  &  169.2 \\
187156  &  mc14\_13TeV.187156.PowhegPythia8\_AU2CT10\_WpWm\_etau.merge.DAOD\_SUSY1.e3059\_s1982\_s2008\_r5787\_r5853\_p1872/  &  1.1790  &  1.00  &  1.000  &  198500  &  168.4 \\
187157  &  mc14\_13TeV.187157.PowhegPythia8\_AU2CT10\_WpWm\_mutau.merge.DAOD\_SUSY1.e3059\_s1982\_s2008\_r5787\_r5853\_p1872/  &  1.1790  &  1.00  &  1.000  &  194500  &  165.0 \\
187158  &  mc14\_13TeV.187158.PowhegPythia8\_AU2CT10\_WpWm\_tautau.merge.DAOD\_SUSY1.e3059\_s1982\_s2008\_r5787\_r5853\_p1872/  &  1.1792  &  1.00  &  1.000  &  100000  &  84.8 \\
187401  &  mc14\_13TeV.187401.gg2vvPythia8\_AU2CT10\_WpWmenuenu.merge.DAOD\_SUSY1.e3131\_s1982\_s2008\_r5787\_r5853\_p1872/  &  0.3981  &  1.00  &  1.000  &  30000  &  75.4 \\
187402  &  mc14\_13TeV.187402.gg2vvPythia8\_AU2CT10\_WpWmenumunu.merge.DAOD\_SUSY1.e3131\_s1982\_s2008\_r5787\_r5853\_p1872/  &  0.3981  &  1.00  &  1.000  &  29500  &  74.1 \\
187403  &  mc14\_13TeV.187403.gg2vvPythia8\_AU2CT10\_WpWmenutaunu.merge.DAOD\_SUSY1.e3131\_s1982\_s2008\_r5787\_r5853\_p1872/  &  0.3981  &  1.00  &  1.000  &  29500  &  74.1 \\
187404  &  mc14\_13TeV.187404.gg2vvPythia8\_AU2CT10\_WpWmmunuenu.merge.DAOD\_SUSY1.e3131\_s1982\_s2008\_r5787\_r5853\_p1872/  &  0.3981  &  1.00  &  1.000  &  30000  &  75.4 \\
187405  &  mc14\_13TeV.187405.gg2vvPythia8\_AU2CT10\_WpWmmunumunu.merge.DAOD\_SUSY1.e3131\_s1982\_s2008\_r5787\_r5853\_p1872/  &  0.3981  &  1.00  &  1.000  &  30000  &  75.4 \\
187406  &  mc14\_13TeV.187406.gg2vvPythia8\_AU2CT10\_WpWmmunutaunu.merge.DAOD\_SUSY1.e3131\_s1982\_s2008\_r5787\_r5853\_p1872/  &  0.3981  &  1.00  &  1.000  &  30000  &  75.4 \\
187407  &  mc14\_13TeV.187407.gg2vvPythia8\_AU2CT10\_WpWmtaunuenu.merge.DAOD\_SUSY1.e3131\_s1982\_s2008\_r5787\_r5853\_p1872/  &  0.3981  &  1.00  &  1.000  &  30000  &  75.4 \\
187408  &  mc14\_13TeV.187408.gg2vvPythia8\_AU2CT10\_WpWmtaunumunu.merge.DAOD\_SUSY1.e3131\_s1982\_s2008\_r5787\_r5853\_p1872/  &  0.3981  &  1.00  &  1.000  &  30000  &  75.4 \\
187409  &  mc14\_13TeV.187409.gg2vvPythia8\_AU2CT10\_WpWmtaunutaunu.merge.DAOD\_SUSY1.e3131\_s1982\_s2008\_r5787\_r5853\_p1872/  &  0.3981  &  1.00  &  1.000  &  30000  &  75.4 \\ \hline
187160  &  mc14\_13TeV.187160.PowhegPythia8\_AU2CT10\_WmZ\_3e\_mll0p25\_DiLeptonFilter.merge.DAOD\_SUSY1.e3059\_s1982\_s2008\_r5787\_r5853\_p1872/  &  1.6880  &  1.00  &  0.285  &  198500  &  412.6 \\
187161  &  mc14\_13TeV.187161.PowhegPythia8\_AU2CT10\_WmZ\_e2mu\_mll0p4614\_DiLeptonFilter.merge.DAOD\_SUSY1.e3059\_s1982\_s2008\_r5787\_r5853\_p1872/  &  1.1680  &  1.00  &  0.341  &  158500  &  398.0 \\
187162  &  mc14\_13TeV.187162.PowhegPythia8\_AU2CT10\_WmZ\_e2tau\_mll3p804\_DiLeptonFilter.merge.DAOD\_SUSY1.e3059\_s1982\_s2008\_r5787\_r5853\_p1872/  &  0.2141  &  1.00  &  0.163  &  100000  &  2865.5 \\
187163  &  mc14\_13TeV.187163.PowhegPythia8\_AU2CT10\_WmZ\_mu2e\_mll0p25\_DiLeptonFilter.merge.DAOD\_SUSY1.e3059\_s1982\_s2008\_r5787\_r5853\_p1872/  &  1.7690  &  1.00  &  0.286  &  197500  &  390.4 \\
187164  &  mc14\_13TeV.187164.PowhegPythia8\_AU2CT10\_WmZ\_3mu\_mll0p4614\_DiLeptonFilter.merge.DAOD\_SUSY1.e3059\_s1982\_s2008\_r5787\_r5853\_p1872/  &  1.1510  &  1.00  &  0.342  &  199500  &  506.8 \\
187165  &  mc14\_13TeV.187165.PowhegPythia8\_AU2CT10\_WmZ\_mu2tau\_mll3p804\_DiLeptonFilter.merge.DAOD\_SUSY1.e3059\_s1982\_s2008\_r5787\_r5853\_p1872/  &  0.2141  &  1.00  &  0.164  &  100000  &  2848.0 \\
187166  &  mc14\_13TeV.187166.PowhegPythia8\_AU2CT10\_WmZ\_tau2e\_mll0p25\_DiLeptonFilter.merge.DAOD\_SUSY1.e3059\_s1982\_s2008\_r5787\_r5853\_p1872/  &  1.7690  &  1.00  &  0.147  &  99500  &  382.6 \\
187167  &  mc14\_13TeV.187167.PowhegPythia8\_AU2CT10\_WmZ\_tau2mu\_mll0p4614\_DiLeptonFilter.merge.DAOD\_SUSY1.e3059\_s1982\_s2008\_r5787\_r5853\_p1872/  &  1.1680  &  1.00  &  0.185  &  98500  &  455.8 \\
187168  &  mc14\_13TeV.187168.PowhegPythia8\_AU2CT10\_WmZ\_3tau\_mll3p804\_DiLeptonFilter.merge.DAOD\_SUSY1.e3059\_s1982\_s2008\_r5787\_r5853\_p1872/  &  0.2104  &  1.00  &  0.060  &  100000  &  7921.4 \\
187170  &  mc14\_13TeV.187170.PowhegPythia8\_AU2CT10\_WpZ\_3e\_mll0p25\_DiLeptonFilter.merge.DAOD\_SUSY1.e3059\_s1982\_s2008\_r5787\_r5853\_p1872/  &  2.3540  &  1.00  &  0.272  &  198500  &  310.0 \\
187171  &  mc14\_13TeV.187171.PowhegPythia8\_AU2CT10\_WpZ\_e2mu\_mll0p4614\_DiLeptonFilter.merge.DAOD\_SUSY1.e3059\_s1982\_s2008\_r5787\_r5853\_p1872/  &  1.5590  &  1.00  &  0.326  &  199500  &  392.5 \\
187172  &  mc14\_13TeV.187172.PowhegPythia8\_AU2CT10\_WpZ\_e2tau\_mll3p804\_DiLeptonFilter.merge.DAOD\_SUSY1.e3059\_s1982\_s2008\_r5787\_r5853\_p1872/  &  0.3129  &  1.00  &  0.164  &  100000  &  1948.7 \\
187173  &  mc14\_13TeV.187173.PowhegPythia8\_AU2CT10\_WpZ\_mu2e\_mll0p25\_DiLeptonFilter.merge.DAOD\_SUSY1.e3059\_s1982\_s2008\_r5787\_r5853\_p1872/  &  2.3710  &  1.00  &  0.272  &  198000  &  307.0 \\
187174  &  mc14\_13TeV.187174.PowhegPythia8\_AU2CT10\_WpZ\_3mu\_mll0p4614\_DiLeptonFilter.merge.DAOD\_SUSY1.e3059\_s1982\_s2008\_r5787\_r5853\_p1872/  &  1.5910  &  1.00  &  0.327  &  199500  &  383.5 \\
187175  &  mc14\_13TeV.187175.PowhegPythia8\_AU2CT10\_WpZ\_mu2tau\_mll3p804\_DiLeptonFilter.merge.DAOD\_SUSY1.e3059\_s1982\_s2008\_r5787\_r5853\_p1872/  &  0.3129  &  1.00  &  0.163  &  99500  &  1950.9 \\
187176  &  mc14\_13TeV.187176.PowhegPythia8\_AU2CT10\_WpZ\_tau2e\_mll0p25\_DiLeptonFilter.merge.DAOD\_SUSY1.e3059\_s1982\_s2008\_r5787\_r5853\_p1872/  &  2.3710  &  1.00  &  0.138  &  100000  &  305.6 \\
187177  &  mc14\_13TeV.187177.PowhegPythia8\_AU2CT10\_WpZ\_tau2mu\_mll0p4614\_DiLeptonFilter.merge.DAOD\_SUSY1.e3059\_s1982\_s2008\_r5787\_r5853\_p1872/  &  1.5590  &  1.00  &  0.174  &  100000  &  368.6 \\
187178  &  mc14\_13TeV.187178.PowhegPythia8\_AU2CT10\_WpZ\_3tau\_mll3p804\_DiLeptonFilter.merge.DAOD\_SUSY1.e3059\_s1982\_s2008\_r5787\_r5853\_p1872/  &  0.3081  &  1.00  &  0.060  &  99500  &  5382.5 \\ \hline
187180  &  mc14\_13TeV.187180.PowhegPythia8\_AU2CT10\_ZZ\_4e\_mll4\_DiLeptonFilter.merge.DAOD\_SUSY1.e3059\_s1982\_s2008\_r5787\_r5853\_p1872/  &  0.1324  &  1.00  &  0.919  &  293500  &  2412.2 \\
187181  &  mc14\_13TeV.187181.PowhegPythia8\_AU2CT10\_ZZ\_2e2mu\_mll4\_DiLeptonFilter.merge.DAOD\_SUSY1.e3059\_s1982\_s2008\_r5787\_r5853\_p1872/  &  0.2961  &  1.00  &  0.849  &  294500  &  1171.5 \\
187182  &  mc14\_13TeV.187182.PowhegPythia8\_AU2CT10\_ZZ\_2e2tau\_mll4\_DiLeptonFilter.merge.DAOD\_SUSY1.e3059\_s1982\_s2008\_r5787\_r5853\_p1872/  &  0.2961  &  1.00  &  0.609  &  300000  &  1663.7 \\
187183  &  mc14\_13TeV.187183.PowhegPythia8\_AU2CT10\_ZZ\_4mu\_mll4\_DiLeptonFilter.merge.DAOD\_SUSY1.e3059\_s1982\_s2008\_r5787\_r5853\_p1872/  &  0.1324  &  1.00  &  0.923  &  300000  &  2454.9 \\
187184  &  mc14\_13TeV.187184.PowhegPythia8\_AU2CT10\_ZZ\_2mu2tau\_mll4\_DiLeptonFilter.merge.DAOD\_SUSY1.e3059\_s1982\_s2008\_r5787\_r5853\_p1872/  &  0.2961  &  1.00  &  0.616  &  299500  &  1642.0 \\
187185  &  mc14\_13TeV.187185.PowhegPythia8\_AU2CT10\_ZZ\_4tau\_mll4\_DiLeptonFilter.merge.DAOD\_SUSY1.e3059\_s1982\_s2008\_r5787\_r5853\_p1872/  &  0.1324  &  1.00  &  0.113  &  300000  &  20051.9 \\
187186  &  mc14\_13TeV.187186.PowhegPythia8\_AU2CT10\_ZZ\_2e2nu\_mll4.merge.DAOD\_SUSY1.e3059\_s1982\_s2008\_r5787\_r5853\_p1872/  &  0.1032  &  1.00  &  1.000  &  100000  &  969.0 \\
187187  &  mc14\_13TeV.187187.PowhegPythia8\_AU2CT10\_ZZ\_2mu2nu\_mll4.merge.DAOD\_SUSY1.e3059\_s1982\_s2008\_r5787\_r5853\_p1872/  &  0.1032  &  1.00  &  1.000  &  100000  &  969.0 \\
187188  &  mc14\_13TeV.187188.PowhegPythia8\_AU2CT10\_ZZ\_2tau2nu\_mll4.merge.DAOD\_SUSY1.e3059\_s1982\_s2008\_r5787\_r5853\_p1872/  &  0.1032  &  1.00  &  1.000  &  99500  &  964.1 \\ \hline
200920  &  mc14\_13TeV.200920.Sherpa\_CT10\_llll.merge.DAOD\_SUSY1.e3213\_s1982\_s2008\_r5787\_r5853\_p1872/  &  1.0467  &  1.00  &  1.000  &  1000000  &  955.4 \\
200982  &  mc14\_13TeV.200982.Sherpa\_CT10\_lllvjj\_EW6.merge.DAOD\_SUSY1.e3618\_s1982\_s2008\_r5787\_r5853\_p1872/  &  0.04184  &  1.00  &  1.000  &  99500  &  2378.1 \\
200983  &  mc14\_13TeV.200983.Sherpa\_CT10\_lllljj\_EW6.merge.DAOD\_SUSY1.e3618\_s1982\_s2008\_r5787\_r5853\_p1872/  &  0.01307  &  1.00  &  1.000  &  96000  &  7345.1 \\
200922  &  mc14\_13TeV.200922.Sherpa\_CT10\_llvv.merge.DAOD\_SUSY1.e3213\_s1982\_s2008\_r5787\_r5853\_p1872/  &  13.8200  &  1.00  &  1.000  &  4972000  &  359.8 \\
200980  &  mc14\_13TeV.200980.Sherpa\_CT10\_llvvjj\_ss\_EW4.merge.DAOD\_SUSY1.e3618\_s1982\_s2008\_r5787\_r5853\_p1872/  &  0.02595  &  1.00  &  1.000  &  100000  &  3853.6 \\
200981  &  mc14\_13TeV.200981.Sherpa\_CT10\_llvvjj\_ss\_EW6.merge.DAOD\_SUSY1.e3618\_s1982\_s2008\_r5787\_r5853\_p1872/  &  0.04295  &  1.00  &  1.000 &  100000  &  2328.3\\
\hline
\hline
\end{tabular}}
\end{center}
\caption{List of simulated samples for
  di-boson background processes. The dataset ID, the generator
  cross-section $\sigma$, the $k$-Factor, the generator filter
  efficiency $\epsilon_{filter}$, the total number of
  generated events $N_{gen}$ and the equivalent luminosity ($L_{equiv}$) are shown.}
\label{tab:BGSamples3}
\end{sidewaystable}

\subsection{Detector simulation}

The detector simulation is performed with a full ATLAS detector
simulation~\cite{Aad:2010ah} based on {\sc Geant4}~\cite{Agostinelli:2002hh}. All
simulated samples are generated with a range of minimum-bias
interactions using {\sc Pythia 8}~\cite{Sjostrand:2007gs} with the MSTW2008LO PDF set~\cite{Sherstnev:2007nd} and the AUET2B tune 
overlaid on the hard-scattering event to
account for the multiple $pp$ interactions in the same bunch
crossing (in-time pileup) and neighbouring bunch crossing (out-of-time pileup). 
A bunch spacing of 25~ns is assumed, and for the bulk of the events a flat probability 
distribution for the average number of
interactions per bunch crossing is assumed with values ranging from
20 to 40 at most~\cite{dc14twiki}.


\FloatBarrier

\section{Object definition}
\label{sec:objects}
This section presents the definitions of the objects used in the analysis: jets, electrons, muons and $\met$. 
%Unless otherwise stated, the recommendations implemented in SUSYTools-00-05-06-23-03 tag and analysis release SUSY,2.3.24a 
Unless otherwise stated, the recommendations implemented in SUSYTools-00-07-17 tag and analysis release SUSY,2.3.38b 
are used for all the objects; most of the studies that led to these definitions were however performed with earlier tags. 
Note that the electron identification scale factors released on 15/01/16 within the ElectronEfficiencyCorrection-00-01-39 
package are used for AtlFast2 samples in the final results since they differ by up to 10\% in the barrel region with respect 
to those included in analysis release SUSY,2.3.38b.

\subsection{Jets}
\label{sec:objects_jets}

The jet selection is summarized in Table~\ref{tab:jetsdef}. 
Jets are reconstructed using the anti-$k_{t}$ jet algorithm~\cite{Cacciari:2008gp} 
with the distance parameter $R$ set to $0.4$ and topological clusters as input. 
Jets are calibrated with the EMTopo scheme applying the jet area pile up corrections. 
The jets are kept only if they have $p_\mathrm{T}>20$~GeV and lie within $|\eta|<2.8$. 
To mitigate the effects of pileup, Jet Vertex Tagger ({\sc JVT})~\cite{ATLAS-CONF-2014-018} requirements 
are applied to the selected jets as recommended by the JetEtMiss group (reject jets after the overlap removal procedure with $\pt<50$~GeV, $|\eta|<2.4$ and JVT$<$0.64). 
The gain in stability with respect to pile-up after
applying this set of cut is illustrated in Figure~\ref{Figure:JVF_Dependency_Run2}.
In order to remove events with fake \met, an event is vetoed 
when a jet ($|\eta|<4.9$) with quality judged as bad according to the {\tt VeryLoose} criterion is present. 

%%%%%%%%%%%%%%%%%%%%%%%%%%%%%%%%%%%%%%%%%%%%%%%%%%%%%%%%%%%%%
\begin{table}[htb!]
\caption{Summary of the jet selection criteria. }
\label{tab:jetsdef}
\begin{center}
    \begin{tabular}{|l|c|}
      \hline
      \hline
      \multicolumn{2}{|c|}{\textbf{Pre-selected jet}}\\
    \hline
      \hline
      Collection     & AntiKt4EMTopo \\
      \hline
      Acceptance     & $\pt > 20\,\GeV, |\eta | < 2.8$ \\
      \hline
      Overlap        & see section~\ref{sec:objects_overlap_removal} \\ % $\Delta{}R({\mathrm jet},e)$~$>$~0.2
\hline
 Jet vertex tagger  &  reject jets with $\pt<50$ GeV, $|\eta|<2.4$ \\
                    &   and JVT$<$0.64 after overlap removal\\
      \hline\hline
%      \multicolumn{2}{|c|}{\textbf{Signal jet}}\\
%      \hline
%     Acceptance     & $\pt > 40\,\GeV,$ \\
%      & $|\eta | < 2.8$ \\ 
%     \hline
%      \hline
      \multicolumn{2}{|c|}{\textbf{b-jets}}\\
    \hline
      \hline
     Acceptance     & $\pt > 20\,\GeV,$ \\
      & $|\eta | < 2.5$ \\ 
      \hline
   $b$-tagging  &  MV2c20 algorithm 70\% OP\\
                &  MV2c20 algorithm 80\% OP for overlap removal\\

      \hline   
\end{tabular}
\end{center}
\end{table}

\begin{figure}[h!]
\begin{center}
\includegraphics[trim=0.0cm 0cm 0cm 0cm,width=0.47\columnwidth]{FIGURES/Frac_njets_1and2}
\includegraphics[trim=0.0cm 0cm 0cm 0cm,width=0.47\columnwidth]{FIGURES/Frac_njets_3to5}
\vspace{-0.2cm}
\end{center}
\caption{Fraction of events [\%] in data with at least 1 or 2 jets (left) and with at least 3, 4 or 5 jets (right) with respect to average number of interactions per bunch crossing with and without a cut on the JVT. Event selection requires a pair of leptons, and all jets have \pt~$>$~25\GeV. $L$~=~3.2~fb$^{-1}$ and $\sqrt s$~=~13\TeV.}
\label{Figure:JVF_Dependency_Run2} 
\end{figure}
%%%%%%%%%%%%%%%%%%%%%%%%%%%%%%%%%%%%%%%%%%%%%%%%%%%%%%%%%%%%

\subsubsection{$b$-tagging}

Tagging of $b$-jets is done using the MV2c20 algorithm with the 70\% efficiency 
operating point. This algorithm is based on a neural network using 
the output weights of the JetFitter+IP3D, IP3D and SV1 algorithms as input.
This efficiency working point was favored by optimisation studies performed with MC15 simulated signal and background samples as described below.
Figure~\ref{fig:btagging} shows the $b$-jet multiplicity for the three tagging efficiency working points. 
Monte Carlo background distributions are shown after same-sign lepton pair requirement.

\begin{figure}[htb!!]
\begin{center}
\includegraphics[width=0.3\textwidth]{FIGURES/BTAGGING/02_CutLEP_ll_NB1JET_signals.pdf} 
\includegraphics[width=0.3\textwidth]{FIGURES/BTAGGING/02_CutLEP_ll_NB2JET_signals.pdf}
\includegraphics[width=0.3\textwidth]{FIGURES/BTAGGING/02_CutLEP_ll_NB3JET_signals.pdf}\\ 
\includegraphics[width=0.3\textwidth]{FIGURES/BTAGGING/05_CutSS_ll_NB1JET.pdf}
\includegraphics[width=0.3\textwidth]{FIGURES/BTAGGING/05_CutSS_ll_NB2JET.pdf}
\includegraphics[width=0.3\textwidth]{FIGURES/BTAGGING/05_CutSS_ll_NB3JET.pdf} 
\end{center}
\vspace{-0.2cm}
\caption{$b$-jet multiplicity for three different tagging efficiency working points, left 70\%, center 80\%, right 90\%.
Signals shapes are shown on the top figures, and stacked background expectations for an integrated luminosity of 3~fb$^{-1}$ are shown on the bottom plots.}
\label{fig:btagging}
\end{figure}

Different signal models described earlier on this document are expected to contain different heavy flavour jet multiplicities.
Therefore the choice of the most performant $b$-jet tagging efficiency working point was made by looking into regions that emulate the signal regions that would eventually be used in the analysis.
Table~\ref{tab:btaggingSR} describes the six different signal-like regions used on this optimisation. 

%%%%%%%%%%%%%%%%%%%%%%%%%%%%%%%%%%%%%%%%%%%%%%%%%%%%%%%%%%%%%
\begin{table}[htb!]
\caption{Signal-like region selections used for the $b$-jet efficiency working point optimisation.}
\label{tab:btaggingSR}
\begin{center}
\begin{tabular}{l|c|c|l}
                & \#$\ell$ & \#$b$-jets & Other cuts                          \\ \hline \hline
  SR-3$b$       & $\geq 2$ & $\geq 3$   & $N^{40}_j \geq 6$                    \\
  SR-3$b$ Soft  & $\geq 2$ & $\geq 3$   & $N^{20}_j \geq 7$ \&\& MET$>$150~GeV \\\hline
  SR-1$b$       & $\geq 2$ & $\geq 1$   & $N^{50}_j \geq 4$ \&\& MET$>$150~GeV (!SR3b)\\
  SR-1$b$ Incl  & $\geq 2$ & $\geq 1$   & $N^{40}_j \geq 4$ \&\& MET$>$150~GeV \\\hline
  SR-0$b$- 5j   & $\geq 2$ & $== 0$     & $N^{50}_j \geq 5$ \&\& MET$>$100~GeV \\
  SR-0$b$- 3j   & $\geq 2$ & $== 0$     & $N^{40}_j \geq 3$ \&\& MET$>$200~GeV \\ \hline
\end{tabular}
\end{center}
\end{table}
%%%%%%%%%%%%%%%%%%%%%%%%%%%%%%%%%%%%%%%%%%%%%%%%%%%%%%%%%%%%
         
\begin{figure}[htb!!]
\begin{center}
\includegraphics[width=0.45\textwidth]{FIGURES/BTAGGINGMC15/L2fb/plot_significance_gtt_CutSR3BS60_all.pdf}\\
%\includegraphics[width=0.3\textwidth]{FIGURES/BTAGGINGMC15/L2fb/plot_significance_bb1step_CutSR1BI60_all.pdf}\\
\includegraphics[width=0.45\textwidth]{FIGURES/BTAGGINGMC15/L2fb/plot_significance_gg1step_CutSR0BL60_all.pdf}        
\includegraphics[width=0.45\textwidth]{FIGURES/BTAGGINGMC15/L2fb/plot_significance_gg2step_CutSR0BL60_all.pdf}          
\end{center}
\vspace{-0.2cm}
\caption{Gaussian discovery signal significance for the various signal-like regions where the different models are sensitive.
Three signal models are tested: the gluino-stop off-shell (top), the gluino production mediated by charginos (bottom left) and the 2-step gluinos via charginos (bottom right).
$b$-jet tagging efficiency working points supported by the heavy flavour working group are 60\%(black), 70\%(red), 77\%(blue) and 85\%(green).
Expectations are estimated assuming an integrated luminosity of 2~fb$^{-1}$ of 13~TeV.}
\label{fig:btaggingGrid}
\end{figure}

Figure~\ref{fig:btaggingGrid} shows the discovery signal significance using a simple Gaussian approximation ($S/\sqrt{S+(\Delta B)^2}$) for different choices of the $b$-tagging working point (60\%, 70\%, 77\% and 85\%) with a flat assumption on the background uncertainty of 40\%. The signal region for each of the models was chosen based on the highest sensitivity.
The 3$b$ Soft signal-like region was found to be the most performant for the gluino-stop off-shell production, 
and the 0$b$-5j signal-like region for the gluino production mediated by charginos as well as the 2-step 
gluino production mediated by gauginos. 
%As more signal simulations become available similar studies would be performed for sbottom direct production and gluinos (and squarks) 2-step models mediated via sleptons.
The 70\% $b$-tagging efficiency working point is chosen since it is the best compromise allowing to reach good sensitivity in the different tested signal models.

%Similar studies with detailed information comparing the performance of previous MV1 (rel-19) tagger and MV2c20 (rel-20) are shown in Appendix~\ref{app_btagging}.

\subsection{Leptons}
\label{sec:objects_leptons}

This section summarizes the electron and muon object selection, as well as developments done in the optimization of the
lepton isolation and electron acceptance cuts.

\subsubsection{Electrons}
\label{sec:objects_electrons}

The electron selection is summarized in Table~\ref{tab:lepdef}. The Egamma CP group recommends the likelihood-based electron identification~\cite{ATLAS-CONF-2014-032} for Run-2, since it provides a factor of two better background rejection than the cut-based identification.
Four working points ({\tt VeryLooseLH}, {\tt LooseLH}, {\tt MediumLH}, {\tt TightLH}) are available for LH electrons.

Pre-selected electrons must satisfy the {\tt LooseLH} requirements and have $E_\mathrm{T}>10$~GeV and $|\eta|<2.47$. 
Electrons in the LAr crack region ($1.37<|\eta|<1.52$) are rejected to reduce the contribution from non-prompt electrons. 
A requirement on the transverse impact parameter of $|d_0/\sigma(d_0)|<5$ (as recommended by TrackingCP) is also applied to pre-selected electrons and helps reducing the contribution from charge mis-identification. 
Signal electrons are additionally required to pass the isolation cuts defined in~\ref{sec:isolation} 
as well as the {\tt TightLH} identification criteria and standard requirements of the longitudinal impact parameter ($|z_0 \cdot sin(\theta)|<$ 0.5 mm, as recommended by TrackingCP).

A multiplicative event weight is applied for each signal electron in MC to the overall event weight 
in order to correct for differences in efficiency between data and MC as recommended by the Egamma group.

\subsubsection{Muons}
\label{sec:objects_muons}

The muon selection is summarized in Table~\ref{tab:lepdef}. The Run-2 muon reconstruction is performed by the so-called \textit{third chain}: 
it has been designed to combine the best features of the Run-1 STACO and MUID chains to provide the best performance. 
The following muon selection working points are supported:
{\tt Tight}, {\tt Medium}, {\tt Loose} and {\tt VeryLoose}. 

Pre-selected muon candidates must pass the {\tt Medium} muon quality cuts and have  $p_\mathrm{T} > 10\GeV$ and $|\eta| < 2.4$.
A smearing procedure is applied to the muon $p_\mathrm{T}$.  
A multiplicative event weight is applied for each selected muon in MC to the overall event weight in order to correct for differences in efficiency between data and MC as recommended by the Muon CP group. 

Finally, signal muon candidates are required to pass the isolation cuts defined in~\ref{sec:isolation} as well as the requirements on the impact parameter of $|d_0/\sigma(d_0)|<3$ and $|z_0 \cdot sin(\theta)|<$ 0.5 mm, as recommended by TrackingCP.


Note that events with ``cosmic'' muons, or ``Bad'' muon are vetoed as described in Section~\ref{sec:presel}.


%%%%%%%%%%%%%%%%%%%%%%%%%%%%%%%%%%%%%%%%%%%%%%%%%%%%%%%%%%%%%
\begin{table}[htb!]
\caption{Summary of the electron and muon selection criteria. The signal
  selection requirements are applied on top of the preselection. The
  lepton-jet isolation requirement is applied after electron-jet overlap
  removal.}
\label{tab:lepdef}
\begin{center}
    \begin{tabular}{|l|c|c|}
      \hline
      \hline
       & \textbf{Pre-selected Electron} & \textbf{Pre-selected Muon} \\
      \hline
      \hline
      Acceptance     & $\pt > 10\,\GeV, |\eta^\mathrm{clust}| < 2.47$  & $\pt > 10\,\GeV, |\eta| < 2.5$ \\
                     &  except $1.37<|\eta^\mathrm{clust}|<1.52$       & \\
      \hline
      Quality & {\tt LooseLLH} & {\tt{xAOD::Muon::Medium}} \\
%      \hline
%      Impact parameter &  $|d_0/\sigma(d_0)|<$ 5.0 & - \\
      \hline
      $\ell$-jet Isolation      & $\Delta{}R(e,jet)$~$>$~0.4 & $\Delta{}R(\mu,jet)$~$>$~0.4 \\
      \hline
      Impact parameter & $|d_0/\sigma(d_0)|<5.0$  & \\
      \hline\hline
       & \textbf{Signal Electron} & \textbf{Signal Muon} \\
      \hline
      \hline
      Quality & {\tt TightLLH} & -\\
       & $|\eta|<2.0$  & -\\
            \hline
      Isolation                & ``FixedCutTight '' & ``FixedCutTightTrackOnly'' \\
      \hline
      Impact parameter & $|z_0 \cdot sin(\theta)|<$ 0.5 mm   & $|z_0 \cdot sin(\theta)|<$ 0.5 mm \\ 
                       &                                     & $|d_0/\sigma(d_0)| < 3.0$\\
     \hline
\end{tabular}
\end{center}
\end{table}
%%%%%%%%%%%%%%%%%%%%%%%%%%%%%%%%%%%%%%%%%%%%%%%%%%%%%%%%%%%%


\subsubsection{Lepton isolation}
\label{sec:isolation}

The isolation working points proposed from this analysis were accepted by the ATLAS Isolation forum and implemented as officially supported working points in the {\tt IsolationSelectionTool}. Unless otherwise stated, the following isolation working points are used in the analysis:
\begin{itemize}
\item Electrons: {\tt TightLLH}, ptvarcone20/\pt$<$0.06 and topoetcone20/\pt$<$0.06 (``FixedCutTight'')
\item Muons: ptvarcone30/\pt\ $<0.06$ (``FixedCutTightTrackOnly'')
\end{itemize}
%Note that no isolation scale factors are used so far in the analysis, but will be released by the EGamma and Muon combined performance groups with the whole 2015 dataset.
Isolation scale factors determined by the CP groups for these working points are applied. 

More details on the isolation studies performed in the context of this analysis can be found in Appendix~\ref{app:iso}.



\subsubsection{Electron acceptance}
\label{sec:eleAcceptance}

Dedicated studies have been conducted to evaluate the impact of a reduction of the $|\eta|$ coverage for electrons. 
This is motivated by the fact that the largest contributions from charge-flip and fake lepton backgrounds 
are observed at $|\eta|>2$ and in the LAr crack region ($1.37<|\eta|<1.52$), 
due to poorer detector resolution or increased amount of material in these regions. 

Table~\ref{tab:eleEta} shows the impact on the expected event yields when reducing the electron $|\eta|$ acceptance 
for the two leading electrons (taking a reference the case of $|\eta|<2.3$). The event selection applied to compute those numbers 
requires SS/3L, at least 3 jets with $\pt > 30$~GeV and $m_{ee}$ outside the 84-98~GeV range to reduce the contribution 
from charge flips. 

%%%%%%%%%%%%%%%%%%%%%%%%%%%%%%%%%%%%%%%%%%%%%%%%%%%%%%%%%%%%%
\begin{table}[htb]
\caption{Relative decrease in the event yield when reducing the $|\eta|$ acceptance for the two leading electrons with respect to $|\eta|<2.3$. The reference is considered to be $|\eta|$~$<$~2.47, and the crack region is excluded in all the cases. Numbers are shown for the event selection described in the text and separately for events with 1 or more $b$-jets.}
\label{tab:eleEta}
\begin{center}
    \begin{tabular}{|l|c|c|c|c|c|} \hline
                   & Prompt SS & $\ttbar +W$ & Gtt (G900\_L100) & Non-prompt leptons & Charge flip \\ \hline\hline
==1$b$, $|\eta|<1.37$ & 14\% & 16\% & 12\%  & 28\% & 69\% \\
==1$b$, $|\eta|<2.0$  & 5.4\%  & 6\%  & 7\%&  11\% & 37\% \\ 
==1$b$, $|\eta|<2.3$  & 0.4\%  & 1.5\%  & 4.7\%&  3.6\% &  11\% \\ \hline
$>$1$b$, $|\eta|<1.37$ & 13\% & 18\% & 10\%  & 44\% & 66\% \\
$>$1$b$, $|\eta|<2.0$  & 4.3\%  & 6.5\%  & 2.9\%&  21\% &  30\% \\ 
$>$1$b$, $|\eta|<2.3$  & 1.3\%  & 1.5\%  & 0.3\%&  8\% &  5.8\% \\  \hline
\end{tabular}
\end{center}
\end{table}
%%%%%%%%%%%%%%%%%%%%%%%%%%%%%%%%%%%%%%%%%%%%%%%%%%%%%%%%%%%%

As shown, reducing the acceptance to $|\eta|<2.0$ would reduce the contribution from processes with prompt 
SS/3L by 6.5\% or less, but the reduction in non-prompt leptons (11-21\%) and charge flip (30-37\%) is quite larger. 
Reducing the acceptance more severely for the two leading electrons to $|\eta|<1.37$ would reduce the non-prompt and 
charge flip components by 28-44\% and $\sim$69\%, respectively, with a reduction of only 13-18\% for prompt SS/3L processes. 

Therefore, this study proves that additional rejection against some of the main backgrounds affecting the analysis can be achieved 
by reducing the $|\eta|$ acceptance of the electrons. 
As illustrated in Table~\ref{tab:lepdef} we chose in conclusion to reject electrons with $|\eta|>2.0$ or in the EM calorimeter cracks ($1.37<|\eta_\text{cluster}|<1.52$). 
Note that these requirements are applied to electrons regardless of the lepton multiplicity of the event.

\subsection{Overlap removal}
\label{sec:objects_overlap_removal}

According to the above definitions, one single final state object may fall in more than one category, being therefore effectively double-counted. 
For example, one isolated electron is typically reconstructed both as an electron and as a jet. A procedure to remove overlaps between final state objects was therefore put in place, and applied on pre-selected objects. 
For this iteration, recommendations from the Harmonization effort~\cite{Harmonization} have been fully adopted. 
%The only exception is the usage a narrower cone for the muon-jet overlap removal (0.2 instead of 0.4).
%A gain of a few \% on signal acceptance and background rejection was observed with respect to the previous overlap-removal definitions, where no $b$-jet information, neither jet-track counting was used.
%The small gain is more evident on the muon channels, with up to 9\% higher acceptance for a Gtt sample, more details can be found in the appendix~\ref{app_overlapremoval} 

The procedure performed is described as follows:

\begin{itemize}

\item First, jets that are angularly close (cone of $\Delta R_y < 0.2$, note the usage of rapidity instead of pseudo-rapidity) 
to a pre-selected (non-isolated) electron are removed from the jet list, 
except if the $b$-tagging weight satisfies the 80\% working point (the electron is then likely coming from a semileptonic $b$-quark decay, so it is removed and the jet is kept). 
%Otherwise both the electron and the jet are kept. 

\item Following this, pre-selected electrons are removed if their distance to the closest jet is $\Delta R_y < 0.4$. 
Then if the $\Delta R_y$ between a jet and a muon is less than 0.4, 
one looks at the number of tracks associated to the jet ({\tt xAOD::JetAttribute::NumTrkPt500}): 
if strictly less than 3 the jet is removed (and the muon is kept),  
otherwise the muon is removed (and the jet is kept).
 
\item If there are an electron and a muon with $\Delta R_y<0.01$, the former is likely to originate from a bremsstrahlung 
of the muon. Such a muon momentum is not measured correctly so that in this case, both the electron
and the muon are removed.

\item Finally, if there are two electrons with $\Delta R_y<0.05$,  the electron with the highest \pT is kept and the other one is discarded.

\end{itemize}

Note that preliminary studies with DC14 data showed a few percent gain in sensitivity if using a narrower cone for the muon-jet overlap removal ($\Delta R_y<0.2$), 
but this was found to introduce features in the muon real and fake efficiency used in the matrix method (see Appendix~\ref{App:RealEfficiency}). 
Therefore, for simplicity and robustness in the analysis, it was decided to used the default value of $\Delta R_y<0.4$. 

%If the distance between two electrons is $\Delta R<0.1$, it is most likely to be that two
%tracks are associated with one calorimeter cluster mistakenly. In this case, lower $E_{\rm T}$ electron 
%is removed. Also 
% OLD TEXT:
%If there are an electron and muon with $\Delta R<0.1$, it is likely to be the bremsstrahlung
%of the muon, and it is therefore removed.

%Studies supporting the update to the latest recommendations (and choice of narrower cone mentioned above) are described below.
%
%Since signal samples in MC15 were not available at the time the study was performed, DC14 samples (signal and background) were used.
%Various signals models were used (Gtt 1300~GeV, $\tilde{\chi_0}$=100~GeV, Gtt 1300~GeV, $\tilde{\chi^0_1}$=700~GeV, sbottom 1300~TeV, 2-step (gluino) 700,400GeV) 
%to obtain significance estimates for different cone-distance used in the overlap-removal between jets and muons.
%Figure~\ref{fig:OR_significance} show the significance (using a Poissonian estimator) for the various flavour channels as well as their combination, after full event selection requirements.
% 
%\begin{figure}[htb!!]
%\begin{center}
%\includegraphics[width=0.45\textwidth]{OVERLAPREMOVAL/plot_significance_mm.pdf} 
%\includegraphics[width=0.45\textwidth]{OVERLAPREMOVAL/plot_significance_em.pdf} 
%\includegraphics[width=0.45\textwidth]{OVERLAPREMOVAL/plot_significance_ee.pdf} 
%\includegraphics[width=0.45\textwidth]{OVERLAPREMOVAL/plot_significance_ll.pdf} 
%\end{center}
%\vspace{-0.2cm}
%\caption{Discovery significance for an assumed integrated luminosity of 3~fb$^{-1}$, for different choices of cone separation used for the muon-jet object overlap removal in various signal models. 
%Top figures display the significance for the muon-related channels ($\mu\mu$ and $e\mu$) and bottom figues illustrate the effect on the $ee$ channel as well as the flavour combination.}
%\label{fig:OR_significance}
%\end{figure}
%
%The studies seem to favor a choice of a $\Delta R$ of 0.2.  


\subsection{Missing transverse energy}
\label{sec:objects_met}

The missing transverse energy (\met) is rebuilt using the xAOD container ``MET\_RefFinal'' as input and using the calibrated electron, muon and jet objects 
(and baseline photons according to SUSYTools definitions). 
In this version of the analysis, the track soft term is used for building the \met\ following the defaults in SUSYTools-00-07-17.


\subsection{Lepton truth matching}
\label{sec:truth_matching}

In several studies presented in this document, 
a lepton truth matching strategy is applied to distinguish the lepton originating from prompt decays of gauge boson or SUSY particles 
from non-prompt leptons originating from semileptonic $b$-decays, photon conversions or fakes. 
This strategy largely relies on the information from the {\tt MCTruthClassifier} tool. 
However, the latter does not always allow to properly classify electrons originating from a photon conversion, 
where it is important to fully understand the origin of the photon and its possible proximity to a prompt electron 
(FSR, brehmsstrahlung, which is the main mechanism leading to charge-flip electrons). 
In these cases, we complete this information by a matching within a cone of $\Delta R <0.1$ with truth prompt electrons, 
which are identified as the decays of heavy bosons ($W$, $Z$, $h$) or SUSY particles. 
This is however only possible in MC generators storing intermediate resonances in the event record, 
i.e. notably MC samples in which decay chains are handled by Pythia. 

\FloatBarrier



\section{Event selection}
\label{sec:evtsel}

\subsection{Trigger studies}
\label{sec:trigger}

\subsubsection{Final trigger strategy}

Based on the studies and efficiency measurements detailed in the rest of this section, the following trigger strategy is used in the analysis:

\begin{itemize}

\item If $\met$ < 250 GeV, only an \texttt{OR} combination of the following di-lepton triggers is used: \\HLT\_2e12\_lhloose\_L12EM10VH, HLT\_e17\_lhloose\_mu14, HLT\_mu18\_mu8noL1.

\item If $\met$ > 250 GeV, an \texttt{OR} between the di-lepton triggers and \texttt{HLT\_xe70} is used.

\end{itemize}

Trigger matching is applied as follows:
\begin{itemize}
\item only signal leptons with $\pt>20$~Gev are considered
\item for technical reasons, matching to muon triggers is performed only on the corresponding level-1 item (L1\_MU10 and L1\_MU15)
\item trigger matching for HLT\_mu18\_mu8noL1 is performed only on the mu18 leg; however, the presence of another signal muon in the event is required. 
\end{itemize}

\subsubsection{Detailed trigger studies (DC14, MC15)}

The trigger strategy for the analysis in Run-2 is similar to the one used in the Run-1 version. There, a combination of \met, single-lepton and di-lepton triggers was used for the selection of events in the signal regions as well as for background estimations. The triggers were checked consecutively starting with the \met\ trigger, followed by the single-lepton trigger and the di-lepton triggers, until one of the triggers is passed by the event. Offline cuts on the missing energy and the \pt\ of the triggered objects were applied to ensure to be on the efficiency plateau of the corresponding trigger.

For Run-2, the triggers to be considered are single-lepton, di-lepton triggers and a $\met$ trigger. The lepton trigger menu includes di-lepton triggers selecting same-flavour and mixed-flavour lepton events. Some single-lepton triggers have additional requirements on the isolation of the triggered lepton. The following triggers have been regarded as important for this analysis and were used for further studies on performance and efficiency:

\begin{itemize}
\item Single-lepton triggers: \texttt{HLT\_mu26, HLT\_mu50, HLT\_mu24\_imedium, HLT\_mu26\_imedium}, 

\texttt{HLT\_e20\_medium, HLT\_e60\_medium, HLT\_e24\_tight\_iloose, HLT\_e26\_tight\_iloose}

\item Di-lepton triggers: \texttt{HLT\_2e12\_loose\_L12EM10VH, HLT\_2e12\_lhloose\_L12EM10VH}

\texttt{HLT\_e17\_loose\_mu14, HLT\_e17\_lhloose\_mu14, HLT\_mu18\_mu8noL1}

\item \met\ trigger: \texttt{HLT\_xe100}

\end{itemize}

Note that the low-$\pt$ single-lepton triggers which would be run un-prescaled in 2015 contain requirements on the lepton isolation ({\tt iloose}, {\tt imedium}) 
and/or more stringent electron identification requirements ({\tt medium}) than those used in the baseline lepton definition. 
Therefore we can't rely on these chains to select the sample of event to be used for fake-lepton background estimation, 
although they might still be useful for signal region selection.

The performance of these triggers has been investigated by running the analysis on dedicated Monte Carlo samples. 
For each object triggered, an offline cut on the object \pt\ or the \met\ is applied to ensure the full efficiency of the trigger in the selected event. 
We ensure that the tested trigger was activated by one of the signal leptons found in the analysis, 
by requiring a geometrical $\Delta R$ matching between these signal leptons 
and the leptons reconstructed by the online version of the software, responsible for the trigger decision. 


\paragraph{Monte Carlo samples and software framework}

The analysis code was setup with the \texttt{AnalysisBase} framework (2.3.8 branch for rel.20 ATLAS software). 
The object selection was done by using the \texttt{SUSYTools-00-06-03} package 
which the recommended selections of signal and baseline objects at the time of the study. 

The Monte Carlo samples used for these studies are validation samples produced with the 20.1.4.3 MC15 ATLAS software release. 
%The samples used for this study do not contain pile-up simulation. 

\begin{itemize}

\item $t \bar{t}$ sample: 

\texttt{valid3.110401.PowhegPythia\_P2012\_ttbar\_nonallhad.recon.AOD.e3099\_s2578\_r6540}

\item $Z \rightarrow \mu \mu$ sample:  

\texttt{valid3.167826.Sherpa\_CT10\_ZmumuMassiveCBPt280\_500\_CVetoBVeto.recon.AOD.e3099\_s2578\_r6540}

\item $Z \rightarrow e e$ sample: 

\texttt{valid3.147406.PowhegPythia8\_AZNLO\_Zee.recon.AOD.e3099\_s2578\_r6540}

\end{itemize}

\paragraph{Total event yields}

The total event yields for the test Monte Carlo samples and different trigger configurations were investigated in order to understand the gain of the several trigger types and their combinations. The yields are measured in a $t \bar{t}$ Monte Carlo sample and for different trigger applications applied. The three configuration tested were:
\begin{enumerate}
\item Applying only di-lepton triggers ({\tt HLT\_2e12\_loose\_L12EM10VH}, {\tt HLT\_2e12\_lhloose\_L12EM10VH}, {\tt HLT\_mu18\_mu8noL1}, {\tt HLT\_e17\_loose\_mu14}, {\tt HLT\_e17\_lhloose\_mu14})
\item Applying a logical \texttt{OR} between di-lepton triggers and the \met trigger ({\tt HLT\_xe100})
\item Applying a logical \texttt{OR} between di-lepton, \met, and single-lepton triggers without explicit isolation requirements ({\tt HLT\_e60\_lhmedium}, {\tt HLT\_mu50})
\end{enumerate}

The results are shown in Figure~\ref{fig:triggerYields}, separately  for events with $\met < 200$ GeV and 	$\met > 200$ GeV. While for the low-\met events, the single-lepton triggers induce a enhancement on the yields of about 1.5\%, the measurement for high-\met events shows only a negligible gain by adding single-lepton triggers to the configuration. The most significant increase is induced by adding the \met trigger to the di-lepton trigger chain for events with $\met > 200$ GeV, with a 6\% increase in the event yields. 
Therefore, the logical \texttt{OR} of di-lepton and $\met$ triggers will be used in the analysis.
For simplicity and due to the small improvement they could provide, single-lepton triggers will not be used in the following. 

\begin{figure}[htb!]
\centering
\includegraphics[width=0.49\textwidth]{TRIGGER/Yields_noMet.pdf}
\includegraphics[width=0.49\textwidth]{TRIGGER/Yields_Met.pdf}
\caption{Total events yields of the $t\bar{t}$ sample for several trigger configurations. The yields are measured for events with $\met < 200$ GeV (left) and $\met > 200$ GeV (right) independently.}
\label{fig:triggerYields}
\end{figure}

\paragraph{Efficiencies}

The trigger efficiency in Monte Carlo can be obtained by dividing the number of triggered events by the total number of events. The generic cleaning cuts are applied in both cases. 


The efficiencies have been calculated separately for single-lepton, di-lepton triggers and for \met\ triggers. 
The results for some examples are shown in Fig.~\ref{fig:triggerEff} for di-lepton and \met\ triggers. 
%Further efficiency plots can be found in Appendix~\ref{app_trigger}. 
The turn-on curve for the \met trigger shows the expected evolution. 
The efficiency plateau is reached for $\met > 150-200$ GeV. At the trigger plateau, the efficiency is reaching $>99\%$. 
The combination of di-lepton and missing energy trigger shows an efficiency of $>90\%$ for di-muon events.

\begin{figure}[htb!]
\centering
\includegraphics[width=0.49\textwidth]{TRIGGER/Eff_HLT_xe100.pdf}
\includegraphics[width=0.49\textwidth]{TRIGGER/Eff_MET0_DILEPTON_1stMu.pdf}
\caption{Trigger efficiencies for \texttt{HLT\_xe100} (left) versus the missing energy and for the combination of di-lepton and \met-triggers (right) versus the \pt of the leading trigered muon.}
\label{fig:triggerEff}
\end{figure}



\paragraph{Choice of $\met$ trigger}

During the review of this analysis, it was pointed out that \texttt{xe\_70} trigger will be unprescaled during the 2015. 
Figure~\ref{fig:triggerEff_xe70_main} shows the turn-on curves of this trigger item for different requirements in the jet multiplicity,
and compared with \texttt{xe\_80} items. In all cases, the plateau of these trigger is reached at a $\met$ value of 250 GeV. More details can be found in App.~\ref{ap:met_update}.

\begin{figure}[htb!]
\centering
\includegraphics[width=0.50\textwidth]{TRIGGER/Eff_HLT_xe70_jets_MC.pdf}
\includegraphics[width=0.49\textwidth]{TRIGGER/Eff_HLT_xe70_HLT_xe80.pdf}
\caption{Trigger efficiencies in data for \texttt{HLT\_xe70} (left) and \texttt{HLT\_xe80}, \texttt{HLT\_xe80\_tc\_lcw} (right) versus $\met$. The different curves show the efficiency for events containing a di-lepton pair.}
 \label{fig:triggerEff_xe70_main}
\end{figure}


Based on these studies and efficiency measurements, a preliminary decision for the trigger selection has been made.

\begin{itemize}

\item If $\met$ < 250 GeV, only an \texttt{OR} combination of di-lepton triggers is used. 

\item If $\met$ > 250 GeV, an \texttt{OR} between the di-lepton triggers and \texttt{HLT\_xe70} is used.

\end{itemize}


\subsection{Event pre-selection}
\label{sec:presel}

%A few general selection criteria based on data quality, detector quality flags are the following:
\par A sample of two same-sign or three leptons is selected applying the following criteria:
\begin{itemize}
%\item{ \textbf{GRL}: The used GoodRunsList, providing a total
%    luminosity of $\lumi$~\ifb, was \newline
%\small{\texttt{data12\_8TeV.periodAllYear\_DetStatus-v58-pro14-01\_DQDefects-00-00-33\_PHYS\_StandardGRL\_All\_Good.xml}}.}
%\item{ \textbf{Trigger}: The data have been collected using lepton and \met\ triggers, as described in Section \ref{section:trigger}.}
%\item{ \textbf{LAr and Tile Error}: Impact from the noise bursts and data corruption in the LAr and
% Tile calorimeter is reduced by selecting events without an error from
% the quality checks (larError!=2 and tileError!=2). } 
%\item{ \textbf{Incomplete events}: Incomplete events due to the TTC restart procedure are rejected.}

%\item{ \textbf{Fake \met\ Veto}: Events with fake \met\ due to jets
%    pointing to dead TileCal and HEC regions are rejected. This is
%    achieved by vetoing events with at least one
%    jet~\footnote{Anti-k$_{\rm T}$ jets with $R=0.4$ before
%      electron-jet overlap removal are considered for the veto. } satisfying:
%    \newline 
%$\pt > 40\,\GeV\ \&\&\ C_{jet}  > 0.05\ \&\&\ \Delta\phi (\met,jet) <
%    0.3$ \newline where  $C_{jet}$ is the estimated correction factor
%    for the energy losses estimated using the jet profile. The veto is applied
%    to all events in data and simulation.}
    
\item{ \textbf{Jet Cleaning}: Events are required to pass the
  {\tt{VeryLooseBad}} set of cleaning requirements recommended by Jet-\met\ group \cite{jetcctwiki} 
  and implemented in the {\tt{JetCleaningTool}}. 
An event is rejected if at least one of the pre-selected jets (thus
after jet-electron overlap removal) fails the jet quality criteria. 
The cleaning requirements are intended to remove events where significant energy was deposited in
  the calorimeters due to instrumental effects such as cosmic rays,
  beam-induced (non-collision) particles, and noise. } 
  
\item{ \textbf{Primary Vertex}: events are required to have a primary vertex, that is, one of the reconstructed vertices
   must be labeled as {\tt{xAOD::VxType::PriVtx}}. No additional requirements on the number of tracks in the 
   vertex should be applied as recommended by the Tracking CP group~\cite{vertex_twiki}.
}

\item{ \textbf{Bad Muon Veto}:  Events containing at least one pre-selected muon satisfying $\sigma(q/p)/|q/p| >$ 0.2 before the overlap removal are rejected.}

\item{ \textbf{Cosmic Muon Veto}: Events containing a cosmic muon candidate are rejected. Cosmic muon candidates are selected among pre-selected muons, if they fail the requirements $|z_0| <$ 1.0\,mm and $|d_0| <$ 0.2\,mm, where the longitudinal and transverse impact paramaters $z_0$ and $d_0$ are calculated with respect to the primary vertex.}

\item{ \textbf{At least two leptons}: Events are required to contain
    at least two signal leptons (see Section~\ref{sec:objects} and
    Table \ref{tab:lepdef}) with $p_{\mathrm{T}}
    > 20\,\GeV$ for the two leading leptons. 
    If the event contains a third signal lepton with $p_{\mathrm{T}} > 10\,\GeV$ the event is regarded as three-lepton
    event otherwise as a two-lepton event. 
	Two of the leptons in the event are matched to the trigger objects 
    if the event is classified as passing a dilepton trigger (i.e. in the low-\met region).
    The data sample obtained is then divided into three channels depending on the flavor of the two 
    leptons forming a same-sign pair ($ee$, $\mu\mu$, $e\mu$). If more than one same-sign pairs can be built, 
    the one involving the leading lepton will be considered for the channel selection.}

\item{ \textbf{Same sign}: If the event is a two-lepton event the two leading leptons have to be of 
    same charge (same-sign).}
    
%\item{ \textbf{Z-veto}: If the event is a three-lepton event all possible opposite-charge same-flavor lepton
%    combinations are combined to calculate a mass $M_{\ell\ell}$. The event is rejected if one of the calculated mass
%    is close to the $Z$-mass, i.e.~$84 < M_{\ell\ell} <  98\,\mathrm{GeV}$.}

%\item{ \textbf{Invariant mass}: Events with $M_{\ell\ell} < 12\,\GeV$ are
%    rejected to avoid heavy flavor meson resonances. In three-lepton events, the invariant mass $M_{\ell\ell}$ is calculated from the 
%    two leading leptons. }
\end{itemize}


The following event variables are also used in the definition of the signal and validation regions in the analysis:

\begin{itemize}
\item The inclusive effective mass \meff~ defined as the scalar sum of
  the signal leptons \pt (see Table~\ref{tab:lepdef}), all signal jets \pt\ (see Table~\ref{tab:jetsdef}) and \met;
\item The transverse mass \mt\ computed from the leading lepton and
\met\ as \newline $\mt = \sqrt{2 \cdot \pt^\ell \cdot \met \cdot (1 - cos(\Delta \phi(\ell, \met )))}$.
\end{itemize}

\subsubsection{Optimisation of lepton pairing selection}
\label{sec:pairing}

In the previous version of this analysis performed during Run-1, different strategies have been used to select same-sign lepton pairs in an event. In order to optimize the event selection of the analysis, different types of lepton pairing selections were compared in terms of their sensitivity to the proposed signal regions:

\begin{itemize}
  \item \textbf{Inclusive 2 Leptons}: Two same-sign leptons with \pt $>$ 20 GeV are required, no cuts are applied on an eventual third lepton;
  \item \textbf{Inclusive 2 Leptons + 1}: Two same-sign leptons with \pt $>$ 20 GeV are required. A \pt cut of 10 GeV is applied to an eventual third lepton. If there are three leptons in the event the two leading leptons can also have opposite sign.
  \item \textbf{Inclusive 3 Leptons}: At least three leptons are required to be in the event. The two leading leptons need to have a \pt of 20 GeV while the third lepton needs to have at least 10 GeV.
  \item \textbf{Exclusive 2 Leptons}: Exactly two same-sign leptons are required with a \pt $>$ 20 GeV. If a third lepton occurs in the event, the event is vetoed.
\end{itemize}

%The background estimation for these optimisation studies is purely Monte Carlo driven. 

%Several Monte Carlo samples produced in the MC15 campaign have been used to model the SM background.
%
%\begin{itemize}
%  \item \textit{W/Z+jets}: \texttt{PowhegPythia8}, MC Id:361100-361108
%  \item \textit{$t\bar{t}$}: \texttt{PowhegPythiaEvtGen}, MC Id: 410000
%  \item \textit{SingleTop}: \texttt{PowhegPythiaEvtGen}, MC Id: 410011(2) t-chan, 410025(6) s-chan
%  \item \textit{t+W}: \texttt{PowhegPythia}, MC Id: 410015(6)
%  \item \textit{DiBoson}: \texttt{Sherpa}, MC Id: 361063-361087
%\end{itemize}
%
%For most of these samples SUSY1 derivations of xAOD's have been used to perform these studies (derivation tag p2372, produced in 20.1.5.4 cache). However, for particular samples which were initially not available in DxAOD format, the original xAOD files were used. For later studies also \textit{$t\bar{t}+V$} samples have been added in order to enhance the total background prediction. The signal samples originate from a private production starting from DC14 \texttt{evtgen} and running them through MC15 \texttt{fastsim} and \texttt{reco} (a766$\_$a767). Later these samples have been replaced by the official MC15 production. They cover several points from the one-step gluino $(\tilde{g} \rightarrow t \bar{t} \chi_{1}) $ grid and the 2-step gluino $(\tilde{g} \rightarrow qq W Z \chi_{1})$ grid. 

The background estimation for these optimization studies is purely Monte Carlo driven. 
In every Signal Region has been studied the related benchmark signal model. For this study, the object selection was done by using the \texttt{SUSYTools-00-06-24-01} package 
and release~\texttt{2.3.28} of the \texttt{AnalysisBase} software framework.

Studies were performed on several grid points (varying gluino and neutralino masses). 
All background and signal samples were scaled according to their cross-sections and the number of generated Monte Carlo events to an integrated luminosity of 4~fb$^{-1}$. 
The significance has been estimated using RooStats::NumberCountingUtils::BinomialObsZ, assuming 40$\%$ of background uncertainty, while the error on the significance is evaluated with 100 MC toys.\\ 
The different lepton pairing selections are tested on Run1-like Signal Regions.\\
\begin{figure}[!ht]
  \centering
  \includegraphics[width=0.49\textwidth]{PAIRING/SR1b_lepton_pairing}
  \includegraphics[width=0.49\textwidth]{PAIRING/SR3b_lepton_pairing}
  \caption{Left: Signal significances for a ``SR1b-like'' selection (cf text),  
  for several $\tilde{b}_{1}\tilde{b}_{1}^{*} \rightarrow t\bar{t}\tilde{\chi}_{1}^{+}\tilde{\chi}_{1}^{-}$ models with variable $m_{\tilde{g}}$ and fixed $m_{\chi}$ of 200 GeV. 
  Right: Signal significances for a ``SR3b-like'' selection (cf text), 
  for several $\tilde{g} \rightarrow t \bar{t} \chi_{1}$ models with variable $m_{\tilde{g}}$ and fixed $m_{\chi}$ of 400 GeV. 
  The significance is computed for different lepton pair selection strategies which are drawn in different colors in the plot.}
 \label{fig:Pairing_SR1b}
\end{figure}

\begin{figure}[!ht]
  \centering
  \includegraphics[width=0.49\textwidth]{PAIRING/SR0b3j_lepton_pairing}
  \includegraphics[width=0.49\textwidth]{PAIRING/SR0b5j_lepton_pairing}
  \caption{Left: Signal significances for a ``SR0b3j-like'' selection (cf text) 
  for several $\tilde{g} \rightarrow q\bar{q}(ll/l\nu)\tilde{\chi}_{1}^{0}$ models with variable $m_{\tilde{g}}$ and fixed $m_{\chi}$ of 300 GeV. 
Right: Signal significances for a ``SR0b5j-like'' selection (cf text) 
for several $\tilde{g} \rightarrow qqWZ\tilde{\chi}_{1}^{0}$ models with variable $m_{\tilde{g}}$ and fixed $m_{\chi}$ of 100 GeV. 
The significance is computed for different lepton pair selection strategies which are drawn in different colors in the plot.}
  \label{fig:Pairing_SR0b3j}
\end{figure}

Figure~\ref{fig:Pairing_SR1b} and~\ref{fig:Pairing_SR0b3j} show the signal significance for event selections
\footnote{All regions require $\ge 2 $ same-sign leptons; 
SR3b = $\ge 3$ $b$-jets + $\ge 6$ jets, SR1b = $\ge 1$ $b$-jet + $\ge 4$ jets + $\met>150$ GeV + $\meff>500$ GeV, 
SR0b3j = $b$-jet veto + $\ge 3$ jets + $\met>200$ GeV + $\meff>400$ GeV, SR0b5j = $b$-jet veto + $\ge 5$ jets + $\met>100$ GeV + $\meff>400$ GeV.} 
close to the final signal regions definitions (cf section~\ref{sec:SignalRegDef}), 
for the four different lepton pairing selections varying the gluino mass and a fixed neutralino mass. 
The ``inclusive 2L+1'' option shows the highest significance for almost all of the mass points investigated.
Since this pairing option is the most inclusive one in all signal regions, the event yields are higher 
with respect to the other options leading to a better exclusion strength for most signal scenarios.\\
The only exception is in SR0b3j where the Inclusive 3L is more efficient. 
This motivates the final decision of requiring at least 3 leptons in this signal region (cf section\~ref{sec:SignalRegDef}).\\ 

\subsubsection{Validation of the \pt cuts for the leading, sub-leading and 3rd lepton}

Since the ``inclusive 2L+1'' lepton pairing selection turned out to be the most significant one in most scenarios, this configuration has been used to investigate also the impact of changes in the \pt cuts of the lepton on the analysis sensitivity. Therefore several \pt cuts for the signal leptons have been tested independently. One additional configuration which was also tested in which is to apply only the baseline requirements on the third lepton if already two signal leptons have been found in the event. 

\begin{figure}[!ht]
  \centering
  \includegraphics[width=0.49\textwidth, page=1]{PAIRING/Pt_SR1b_SR3b.pdf}
  \includegraphics[width=0.49\textwidth, page=2]{PAIRING/Pt_SR1b_SR3b.pdf}
  \caption{Significances of the signal region SR1b and SR3b for $\tilde{g} \rightarrow t \bar{t} \chi_{1}$ models with variable $m_{\tilde{g}}$ and fixed $m_{\chi}$ of 100 GeV. The significance is computed for different \pt cuts on 1st, 2nd and 3rd lepton. The significances are also shown for $\pt(1st, \, 2nd, \, 3rd) = 20, \, 20, \, 10 $ GeV, relaxing the requirements on the third lepton to baseline-cuts.}
  \label{fig:Pt_SR1b_SR3b}
\end{figure}

Figure~\ref{fig:Pt_SR1b_SR3b} shows the significances of SR1b and SR3b (as defined in previous section) for different \pt cuts on first, second and third leptons. 
The significances are also shown for $\pt({\text 1st, 2nd, 3rd}) = 20, \, 20, \, 10 $ GeV but relaxing the requirements on the third lepton to baseline-cuts. 
The observed significances show only small dependencies on changes of the $p_T$ cuts. 
Also relaxing the requirements of the third lepton to the baseline definition has no big impact on the sensitivity of the analysis. 
Therefore a modification of the nominal lepton selection criteria would not bring a substantial benefit to the analysis sensitivity. 

%% \subsubsection{Validation of the jet \pt requirements in signal regions with a $b$ veto}
%% this section has been moved, to be after the SR definitions

\subsection{Data-MC comparisons}

In order to validate the various choices made regarding the object definitions and event selection, 
check their sensible behavior and their reasonable modelling in the simulations, 
we looked at the distributions of several kinematic variables obtained with the available 13 TeV data. 
Figures~\ref{fig:dataMC_2lep}-\ref{fig:dataMC_metmeff} show such selected distributions in data compared to MC. The background distributions are taken directly from MC with no data-driven estimation of the charge flip or non-prompt lepton backgrounds.

Figure~\ref{fig:dataMC_2lep} shows the dilepton invariant mass distributions for both OS and SS dilepton events, 
computed with the two leading $p_T$ leptons. 
A very good agreement with MC is observed in the OS channels, with a clear $Z$-boson mass peak in the $ee$ and $\mu\mu$ channels. 
In the SS channels, the $Z$-boson mass peak is also observed in the $ee$ channel due to electron charge mis-identification, with MC overestimating data by 20-30\%. 
An accumulation of events at the $Z$-boson mass is also observed in the SS $e\mu$ and $\mu\mu$ channels due to three-lepton events 
from either $Z$+jets with a fake lepton or from $WZ$ production.  

Lepton distributions are shown in Figures~\ref{fig:dataMC_lep1}-\ref{fig:dataMC_lep2}, with a reasonable data-MC agreement except at low lepton $\pt$ where some discrepancies and accumulation of events involving fake leptons ($Z$+jets, $W$+jets, $\ttbar$) are observed. Jet and $b$-jet distributions are shown in Figures~\ref{fig:dataMC_jet}-\ref{fig:dataMC_bjet} and Figure~\ref{fig:dataMC_metmeff} shows the $\met$ and $m_{\rm eff}$ distributions.

\begin{figure}[htb!]
\centering
{\includegraphics[width=0.49\textwidth]{DATAMC/Mll_afterlepton_OSee_0_physics.pdf}}
{\includegraphics[width=0.49\textwidth]{DATAMC/Mll_afterlepton_SSee_0_physics.pdf}}
{\includegraphics[width=0.49\textwidth]{DATAMC/Mll_afterlepton_OSem_0_physics.pdf}}
{\includegraphics[width=0.49\textwidth]{DATAMC/Mll_afterlepton_SSem_0_physics.pdf}}
{\includegraphics[width=0.49\textwidth]{DATAMC/Mll_afterlepton_OSmm_0_physics.pdf}}
{\includegraphics[width=0.49\textwidth]{DATAMC/Mll_afterlepton_SSmm_0_physics.pdf}}
\caption{Dilepton invariant mass distributions for opposite-sign (left) and same-sign (right) pairs for events selected in the $ee$ (top), $e\mu$ (center) and $\mu\mu$ (bottom) channels, computed with the two leading $p_T$ leptons. The background contribution is taken directly from MC with no data-driven estimation of the background with fake and non-prompt leptons or charge mis-identification. No low-mass Drell-Yan sample is included. Only luminosity and MC statistical uncertainties are included.
}
\label{fig:dataMC_2lep}
\end{figure}

\begin{figure}[htb!]
\centering
{\includegraphics[width=0.49\textwidth]{DATAMC/NLEP_afterlepton_SSee_0_physics.pdf}}
{\includegraphics[width=0.49\textwidth]{DATAMC/PTLEP1_afterlepton_SSee_0_physics.pdf}}
{\includegraphics[width=0.49\textwidth]{DATAMC/NLEP_afterlepton_SSem_0_physics.pdf}}
{\includegraphics[width=0.49\textwidth]{DATAMC/PTLEP1_afterlepton_SSem_0_physics.pdf}}
{\includegraphics[width=0.49\textwidth]{DATAMC/NLEP_afterlepton_SSmm_0_physics.pdf}}
{\includegraphics[width=0.49\textwidth]{DATAMC/PTLEP1_afterlepton_SSmm_0_physics.pdf}}
\caption{Lepton multiplicity (left) and leading lepton $\pt$ (right) for events selected in the $ee$ (top), $e\mu$ (center) and $\mu\mu$ (bottom) channels. The background contribution is taken directly from MC with no data-driven estimation of the background with fake and non-prompt leptons or charge mis-identification. Only luminosity and MC statistical uncertainties are included.
}
\label{fig:dataMC_lep1}
\end{figure}

\begin{figure}[htb!]
\centering
{\includegraphics[width=0.49\textwidth]{DATAMC/PTLEP2_afterlepton_SSee_0_physics.pdf}}
{\includegraphics[width=0.49\textwidth]{DATAMC/PTLEP3_afterlepton_SSee_0_physics.pdf}}
{\includegraphics[width=0.49\textwidth]{DATAMC/PTLEP2_afterlepton_SSem_0_physics.pdf}}
{\includegraphics[width=0.49\textwidth]{DATAMC/PTLEP3_afterlepton_SSem_0_physics.pdf}}
{\includegraphics[width=0.49\textwidth]{DATAMC/PTLEP2_afterlepton_SSmm_0_physics.pdf}}
{\includegraphics[width=0.49\textwidth]{DATAMC/PTLEP3_afterlepton_SSmm_0_physics.pdf}}
\caption{Sub-leading (left) and third leading lepton $\pt$ (right) for events selected in the $ee$ (top), $e\mu$ (center) and $\mu\mu$ (bottom) channels. The background contribution is taken directly from MC with no data-driven estimation of the background with fake and non-prompt leptons or charge mis-identification. Only luminosity and MC statistical uncertainties are included.
}
\label{fig:dataMC_lep2}
\end{figure}


\begin{figure}[htb!]
\centering
{\includegraphics[width=0.49\textwidth]{DATAMC/NJET30_afterlepton_SSee_0_physics.pdf}}
{\includegraphics[width=0.49\textwidth]{DATAMC/PTJET1_afterlepton_SSee_0_physics.pdf}}
{\includegraphics[width=0.49\textwidth]{DATAMC/NJET30_afterlepton_SSem_0_physics.pdf}}
{\includegraphics[width=0.49\textwidth]{DATAMC/PTJET1_afterlepton_SSem_0_physics.pdf}}
{\includegraphics[width=0.49\textwidth]{DATAMC/NJET30_afterlepton_SSmm_0_physics.pdf}}
{\includegraphics[width=0.49\textwidth]{DATAMC/PTJET1_afterlepton_SSmm_0_physics.pdf}}
\caption{Number of jets (left) and leading jet $\pt$ (right) for events selected in the $ee$ (top), $e\mu$ (center) and $\mu\mu$ (bottom) channels. The background contribution is taken directly from MC with no data-driven estimation of the background with fake and non-prompt leptons or charge mis-identification. Only luminosity and MC statistical uncertainties are included.
}
\label{fig:dataMC_jet}
\end{figure}


\begin{figure}[htb!]
\centering
{\includegraphics[width=0.49\textwidth]{DATAMC/NBJET70_20_afterlepton_SSee_0_physics.pdf}}
{\includegraphics[width=0.49\textwidth]{DATAMC/PTBJET70_1_afterlepton_SSee_0_physics.pdf}}
{\includegraphics[width=0.49\textwidth]{DATAMC/NBJET70_20_afterlepton_SSem_0_physics.pdf}}
{\includegraphics[width=0.49\textwidth]{DATAMC/PTBJET70_1_afterlepton_SSem_0_physics.pdf}}
{\includegraphics[width=0.49\textwidth]{DATAMC/NBJET70_20_afterlepton_SSmm_0_physics.pdf}}
{\includegraphics[width=0.49\textwidth]{DATAMC/PTBJET70_1_afterlepton_SSmm_0_physics.pdf}}
\caption{Number of $b$-jets (left) and leading $b$-jet $\pt$ (right) for events selected in the $ee$ (top), $e\mu$ (center) and $\mu\mu$ (bottom) channels. The background contribution is taken directly from MC with no data-driven estimation of the background with fake and non-prompt leptons or charge mis-identification. Only luminosity and MC statistical uncertainties are included.
}
\label{fig:dataMC_bjet}
\end{figure}


\begin{figure}[htb!]
\centering
{\includegraphics[width=0.49\textwidth]{DATAMC/MET_afterlepton_SSee_0_physics.pdf}}
{\includegraphics[width=0.49\textwidth]{DATAMC/MEFF_afterlepton_SSee_0_physics.pdf}}
{\includegraphics[width=0.49\textwidth]{DATAMC/MET_afterlepton_SSem_0_physics.pdf}}
{\includegraphics[width=0.49\textwidth]{DATAMC/MEFF_afterlepton_SSem_0_physics.pdf}}
{\includegraphics[width=0.49\textwidth]{DATAMC/MET_afterlepton_SSmm_0_physics.pdf}}
{\includegraphics[width=0.49\textwidth]{DATAMC/MEFF_afterlepton_SSmm_0_physics.pdf}}
\caption{Distributions of the $\met$ (left) and effective mass (right) for events selected in the $ee$ (top), $e\mu$ (center) and $\mu\mu$ (bottom) channels. The background contribution is taken directly from MC with no data-driven estimation of the background with fake and non-prompt leptons or charge mis-identification. Only luminosity and MC statistical uncertainties are included.
}
\label{fig:dataMC_metmeff}
\end{figure}

\FloatBarrier




\section{Signal region definition}
\label{sec:sr}
\label{sec:SignalRegDef}

The definitions of the signal regions have been studied to provide an optimal performance for $\sqrt s=13$ TeV collisions and a low integrated luminosity (2-4~\ifb). 
This optimization process was first performed with DC14 MC samples, and was then refined with the more accurate MC15 samples and close-to-final object definitions. 
We chose to categorize the signal regions based on their $b$-jet multiplicity, in continuation of the approach sustained in the Run-1 analysis: 
\begin{itemize}
\item[$\bullet$] Signal region(s) with at least one $b$-jet (``SR1b''): these selections target signal scenarios involving top or bottom quarks, 
mostly related to third-generation squarks, such as the benchmark process $\sbot\sbot^*\to t\bar t\tilde\chi_1^+\tilde\chi_1^-$. 
\item[$\bullet$] Signal region(s) with at least three $b$-jets (``SR3b''): these selections target signal scenarios involving many top or bottom quarks, 
such as the benchmark process $\gluino\gluino\to t\bar tt\bar t\ninoone\ninoone$, 
and with their intrinsically very low background are particularly well suited for scenarios with compressed mass spectra. 
\item[$\bullet$] Signal region(s) with a $b$-jet veto (``SR0b''): these selections allow to increase the sensitivity to signal scenarios without bottom quarks, 
by suppressing most of the top background -- the selections are then dominated by diboson background. 
\end{itemize}
One can notice that there is no dedicated selection for final states with $\ge 2$ $b$-jets: 
it is found to not be particularly useful, as the background is generally dominated by $t\bar t+X$ processes, 
which does not change substantially between $\ge 1$ and $\ge 2$ $b$-jets selections. 
By contrast the difference between $\ge 1$ and $\ge 3$ $b$-jets selections is very important. 

To this first classification we add minimal requirements on the inclusive jet multiplicity: 
\begin{center}
\begin{tabular}{c|c|c|c|c}
Signal region(s) & \multicolumn{2}{c|}{SR0b} & SR1b & SR3b\\\hline
Jets req. & $\ge 3$ ($p_T>50$~GeV) & $\ge 5$ ($p_T>50$~GeV) & $\ge 4$ ($p_T>50$~GeV) & $-$\\
\end{tabular}
\end{center}
As one can see, the SR0b selections were subdivided into two overlapping selections ($\ge 3$ or $\ge 5$ jets, 
also denoted as SR0b5j and SR0b3j) to cover various signal scenarios that lead to differently jet-enriched final states. 
The optimal minimal number of jets and the jet $p_T$ thresholds were defined as part of the DC14-based optimization, through a (\meff, \met, \#jets, jet $p_T$) scan similar to the one described below 
and focused on the few benchmark signal scenarios that were produced for DC14 studies. Only the $\pt$ threshold for SR0b3j was raised from 40 to 50~GeV for homogenization among the SRs since this change had very small impact in the sensitivity.

All these selections are inclusive in terms of leptons (``at least two same-sign leptons''), 
it was found that for these early results no substantial gain would be achieved by considering trilepton final states separately (as was done in the Run-1 analysis) except for SR0b3j, where a $\geq$3 lepton requirement was found to improve the sensitivity to slepton-mediated signals ($\gluino\to q\bar{q}(\ell\ell/\ell\nu)\ninoone$).

To complete the definition of the signal regions, we added requirements on the effective mass \meff{} and missing transverse momentum \met. 
We rely only on these two discriminant variables, well suited for generic SUSY searches, 
as one of the analysis strengths is to be sensitive to a broad range of BSM scenarios and we do not want to overtune it to a restricted set of benchmarks. 

\subsection{Optimization procedure and results}

The optimization of the signal region definitions was carried on with the MC15 samples. % (resp. the 50/25 ns configuration for background/signal). 
We scanned the $(\meff,\met)$ plane for the four selections detailed above, 
looking at the impact of the cuts on various signal benchmarks. 
We used as figure of merit the signal discovery significance (Zn), calculated with {\tt RooStats::NumberCountingUtils::BinominalObsZ} 
assuming an overall $40\%$ systematic uncertainty on the background prediction (as a compromise to the 50\% expected uncertainty on the fake-lepton background and 30\% on the prompt-lepton background). 
We discarded the cut configurations where the background projection was too imprecise, due to limited MC statistics; 
more precisely when the statistical error on the projected background exceeded $30\%$. 
We also focused on signal benchmarks that would provide at least 2 signal events for the considered luminosity. 

Figure~\ref{fig:OptimScan} shows as an example the $(\meff, \met)$ planes for two different signal regions and models.
The resulting maximum discovery significance across the signal grids and the corresponding $(\meff,\met)$ configurations 
are shown in Figure~\ref{fig:OptimSig1} for SR1b, SR3b and SR0b5j. As shown, with 2~\ifb\ of data we can have sensitivity beyond the existing Run-1 limits in some of the models. Note that the Run-1 limits shown in the figures correspond to the best ATLAS limit, not necessarily obtained by the SS/3L analysis.

\begin{figure}[!htb]
\centering
  \includegraphics[width=0.49\textwidth]{OPTIMIZATION/Scan1.pdf}
  \includegraphics[width=0.49\textwidth]{OPTIMIZATION/Scan2.pdf}
  \caption{Example of (\met, \meff) scans for SR0b5j (left) and SR3b (right). The configurations with maximum significance are highlighted as well as the outcome of the DC14 optimization studies.}
\label{fig:OptimScan}


\end{figure}


%\begin{figure}[!htb]
%\centering
%  \includegraphics[width=\textwidth]{OPTIMIZATION/Optimiz_SR0b5j_2fb.pdf}
%  \includegraphics[width=\textwidth]{OPTIMIZATION/Optimiz_SR0b3j_2fb.pdf}
%\caption{Maximum discovery significance (left) for 2~\ifb\, as well as the $\met$ (center) and $\meff$ (right) cuts needed to maximize the significance for SR0b5j in the $\gluino\gluino$ 2-step grid (top) and SR0b3j in the $\gluino\gluino$ 1-step grid (bottom). The Run-1 limits in those models are shown with a brown line.}
%\label{fig:OptimSig2}
%\end{figure}


\subsection{Signal regions}
 
The definition of the exact SR was done as a good compromise across the signal grids shown in 
Figure~\ref{fig:OptimSig1} with a single ($\met$, $\meff$) configuration. 
Tables~\ref{tab:SRdef2}-\ref{tab:SRdef4} show the optimized signal region definitions for scenarios with 2, 3 and 4~\ifb\, respectively. 
The final SR to be used for the 2015 analysis was determined by the luminosity available at the end of the data-taking period: 
if less than 2.5~\ifb\ had been available for analysis after GRL, we would have used the SRs for the 2~\ifb\ scenario; 
if more than 3.5~\ifb\ had been available, we would have used the SRs for the 4~\ifb\ scenario. 
But with the 3.2~\ifb\ eventually collected, we used the definitions corresponding to the intermediate scenario of 3~\ifb.
Figures~\ref{fig:OptimSig2} and \ref{fig:OptimSig4} show the significance values obtained for those signal regions in the SUSY models considered, 
with the 1.64$\sigma$ discovery contours extending beyond the Run-1 exclusions, even achieving a 3$\sigma$ sensitivity in certain regions of the mass parameter space.
 
 
\begin{table}[htb!]
\caption{Signal regions definition for the 2~\ifb\ scenario (to be used for $L <2.5$~fb$^{-1}$). The two leading leptons are required to have \pt~$>$~20~\GeV.}
\hspace{0.5cm}
\label{tab:SRdef2}
\centering
\begin{tabular}{|c|c|c|c|c|c|}
\hline 
\hline
Signal region  &  $N_{\rm{lept}}$   & $N_{b\rm{-jets}}^{20}$    & $N_{\rm{jets}}^{50}$  & \met\ [GeV] & \meff\ [GeV]   \\
\hline\hline
SR3b     &   $\ge$2  &   $\ge$3  &  - & $>$100 & $>$600   \\
\hline
SR1b     &  $\ge$2  &    $\ge$1  &  $\ge$4 &  $>$125 & $>$500 \\
\hline
SR0b5j &  $\ge$2  &    $==$0 &  $\ge$5 &  $>$100 & $>$600 \\
\hline
SR0b3j &  $\ge$3  &    $==$0 &  $\ge$3 &  $>$150 & $>$500 \\
\hline\hline
\end{tabular}
\end{table}

\begin{table}[htb!]
\caption{Signal regions definition for the 3~\ifb\ scenario (to be used for $2.5 \leq L <3.5$~fb$^{-1}$), 
the one eventually used in this analysis. The two leading leptons are required to have \pt~$>$~20~\GeV.}
\hspace{0.5cm}
\label{tab:SRdef3}
\centering
\begin{tabular}{|c|c|c|c|c|c|}
\hline 
\hline
Signal region  &  $N_{\rm{lept}}$   & $N_{b\rm{-jets}}^{20}$    & $N_{\rm{jets}}^{50}$  & \met\ [GeV] & \meff\ [GeV]   \\
\hline\hline
SR3b     &   $\ge$2  &   $\ge$3  &  - & $>$125 & $>$650   \\
\hline
SR1b     &  $\ge$2  &    $\ge$1  &  $\ge$4 &  $>$150 & $>$550 \\
\hline
SR0b5j &  $\ge$2  &    $==$0 &  $\ge$5 &  $>$125 & $>$650 \\
\hline
SR0b3j &  $\ge$3  &    $==$0 &  $\ge$3 &  $>$200 & $>$550 \\
\hline\hline
\end{tabular}
\end{table}

\begin{table}[htb!]
\caption{Signal regions definition for the 4~\ifb\ scenario (to be used for $L \geq 3.5$~fb$^{-1}$). The two leading leptons are required to have \pt~$>$~20~\GeV.}
\hspace{0.5cm}
\label{tab:SRdef4}
\centering
\begin{tabular}{|c|c|c|c|c|c|}
\hline 
\hline
Signal region  &  $N_{\rm{lept}}$   & $N_{b\rm{-jets}}^{20}$    & $N_{\rm{jets}}^{50}$  & \met\ [GeV] & \meff\ [GeV]   \\
\hline\hline
SR3b     &   $\ge$2  &   $\ge$3  &  - & $>$125 & $>$700   \\
\hline
SR1b     &  $\ge$2  &    $\ge$1  &  $\ge$4 &  $>$150 & $>$600 \\
\hline
SR0b5j &  $\ge$2  &    $==$0 &  $\ge$5 &  $>$125 & $>$700 \\
\hline
SR0b3j &  $\ge$3  &    $==$0 &  $\ge$3 &  $>$200 & $>$600 \\
\hline\hline
\end{tabular}
\end{table}

\begin{figure}[!htb]
\centering
  \includegraphics[width=0.9\textwidth]{OPTIMIZATION/Optimiz_SR1b_2fb.pdf}
  \includegraphics[width=0.9\textwidth]{OPTIMIZATION/Optimiz_SR3b_2fb.pdf}
  \includegraphics[width=0.9\textwidth]{OPTIMIZATION/Optimiz_SR0b5j_2fb.pdf}
  \includegraphics[width=0.9\textwidth]{OPTIMIZATION/Optimiz_SR0b3j_2fb.pdf}
  \caption{Maximum discovery significance (left) for 2~\ifb, as well as the $\met$ (center) and $\meff$ (right) cuts needed to maximize the significance for: (from top to bottom) SR1b in the $\sbot\sbot^*\to t\bar t\tilde\chi_1^+\tilde\chi_1^-$ grid, SR3b in the $\gluino\gluino\to t\bar tt\bar t\ninoone\ninoone$ grid, SR0b5j in the $\gluino\gluino$ with $\gluino\to q\bar{q}'WZ\ninoone$ grid and SR0b3j in the $\gluino\gluino$ with $\gluino\to q\bar{q}(\ell\ell/\ell\nu)\ninoone$ grid. The Run-1 limits in those models are shown with a brown line, and the 1.64$\sigma$ and 3$\sigma$ discovery contours from the proposed signal regions are shown in green and red, respectively.}
\label{fig:OptimSig1}
\end{figure}



\begin{figure}[!htb]
\centering
  \includegraphics[width=0.4\textwidth]{OPTIMIZATION/Optimiz_SR1b_2fb_125_500.pdf}
  \includegraphics[width=0.4\textwidth]{OPTIMIZATION/Optimiz_SR3b_2fb_100_600.pdf}\\
  \includegraphics[width=0.4\textwidth]{OPTIMIZATION/Optimiz_SR0b5j_2fb_100_600.pdf}
  \includegraphics[width=0.4\textwidth]{OPTIMIZATION/Optimiz_SR0b3j_2fb_150_500.pdf}
  \caption{Discovery significance for the SRs defined in Table~\ref{tab:SRdef2} (2~\ifb) for SR1b in the $\sbot\sbot^*\to t\bar t\tilde\chi_1^+\tilde\chi_1^-$ grid (top left), SR3b in the $\gluino\gluino\to t\bar tt\bar t\ninoone\ninoone$ grid (top right), SR0b5j in the $\gluino\gluino$ with $\gluino\to q\bar{q}'WZ\ninoone$ grid (bottom left) and SR0b3j in the $\gluino\gluino$ with $\gluino\to q\bar{q}(\ell\ell/\ell\nu)\ninoone$ grid (bottom right). The Run-1 limits in those models are shown with a brown line, and the 1.64$\sigma$ and 3$\sigma$ discovery contours from the proposed signal regions are shown in green and red, respectively.}
\label{fig:OptimSig2}
\end{figure}


\begin{figure}[!htb]
\centering
  \includegraphics[width=0.4\textwidth]{OPTIMIZATION/Optimiz_SR1b_4fb_150_600.pdf}
  \includegraphics[width=0.4\textwidth]{OPTIMIZATION/Optimiz_SR3b_4fb_125_700.pdf}\\
  \includegraphics[width=0.4\textwidth]{OPTIMIZATION/Optimiz_SR0b5j_4fb_125_700.pdf}
  \includegraphics[width=0.4\textwidth]{OPTIMIZATION/Optimiz_SR0b3j_4fb_200_600.pdf}
  \caption{Discovery significance for the SRs defined in Table~\ref{tab:SRdef4} (4~\ifb) for SR1b in the $\sbot\sbot^*\to t\bar t\tilde\chi_1^+\tilde\chi_1^-$ grid (top left), SR3b in the $\gluino\gluino\to t\bar tt\bar t\ninoone\ninoone$ grid (top right), SR0b5j in the $\gluino\gluino$ with $\gluino\to q\bar{q}'WZ\ninoone$ grid (bottom left) and SR0b3j in the $\gluino\gluino$ with $\gluino\to q\bar{q}(\ell\ell/\ell\nu)\ninoone$ grid (bottom right). The Run-1 limits in those models are shown with a brown line, and the 1.64$\sigma$ and 3$\sigma$ discovery contours from the proposed signal regions are shown in green and red, respectively.}
\label{fig:OptimSig4}
\end{figure}


%\FloatBarrier



\section{Background estimation}
\label{sec:bkg}
The main challenge, in this analysis, is to achieve reliable predictions of the low Standard Model background 
leading to the same-sign leptons + jets final state. 
This background is composed partly of rare processes such as the associate production of a top quark pair with a massive boson, 
or the production of multiple bosons. 
The other contribution consists in experimental backgrounds originating from the imperfect discrimination between prompt leptons and other objects, 
or the occasional misreconstruction of the electron charge. 
The following sections provide more details about the nature of these different categories of background, 
and the foreseen methods that will be used to estimate their contributions to the signal regions. 

\subsection{Backgrounds with prompt SS dilepton or three leptons}
\label{sec:bkg_prompt}

There are two main sources of Standard Model background leading to pairs of same-sign prompt leptons: 
\begin{itemize}
\item[$\bullet$] The associate production of top quark(s) and massive bosons, where the same-sign leptons pair 
is produced from leptonic decays of one of the top quarks and of the boson. 
These processes are characterized by a large jet multiplicity, the presence of $b$-jet(s), 
and always have intrinsic missing momentum. 
Therefore they generally represent the largest contribution to the signal regions. 
The dominant processes are $pp\to t\bar{t}W(j)$ and $pp\to t\bar{t}Z(j)$, 
while there are also minor contributions from $pp\to t\bar{t}H,\,tZbj,\,t\bar{t}WW$, and $pp\to t\bar{t}t\bar{t}$. 
\item[$\bullet$] The production of multiple massive bosons. 
These processes have generally low jet multiplicities. 
However, due to their larger cross-sections, they contribute in a significant way to the background entering signal regions without $b$-jets requirements. 
The dominant processes include $pp\to W^\pm W^\pm jj,\,WZ,\,ZZ$, 
with minor contributions from $pp\to\,WH,\,ZH,\,VVV$ and $pp \to\,H\to\,llll$.  
\end{itemize}

We plan to estimate the contributions of these various processes to the signal regions by relying on the Monte-Carlo predictions, 
normalized with the best known theoretical cross-sections : 
these processes are too rare to allow use of control regions until a significant integrated luminosity will be collected. 

Processes containing top quarks have cross-sections below 1~pb, which have consequently not yet been much constrained experimentally.
On the theoretical side, uncertainties on the cross sections are typically large: 30\% for $t\bar{t}W$ and 50\% for $t\bar{t}Z$
for the same-sign $\sqrt{s}=7$~TeV analysis~\cite{NoteSS3L_7TeV}, 22\% for both $t\bar{t}W$ and $t\bar{t}Z$ for the $\sqrt{s}=8$~TeV analysis~\cite{noteSS3L}. 

Cross-sections for diboson processes are known with a rather good accuracy, but only for the inclusive processes, 
while we are mostly interested in processes where several additional partons are produced. 

The validation regions described in section~\ref{sec:bkg_VR} will help us ensure that our understanding of these processes is sufficiently reliable, 
and that the systematic uncertainties assigned to the estimated rates are reasonable. 

\subsection{Charge flip leptons}
\label{sec:bkg_chflips}


Electrons are occasionally reconstructed with the wrong charge, 
generally because an irrelevant track is matched to the electron cluster during reconstruction instead of the real electron track. 
This may occur for example in the so-called trident process, where an electron emits a hard Brehmsstrahlung photon 
which further converts into an electron-positron pair, resulting in three close tracks. 
This confusion is the main reason for lepton charge flip, as errors on the track reconstruction itself are much rarer. 
Indeed, muon charge flip has been estimated negligible in past studies~\cite{noteSS3L}, and we consider only charge flip background originating from electrons. 

\begin{figure}[hbt]
\centering
\includegraphics[width=0.7\textwidth]{BKG/chargeFlipRate}
\caption{Expected electron charge flip rate determined in simulated $t\bar{t}\to e\mu\nu\nu$ events}
\label{fig:mc_chargeflip_rate}
\end{figure}


This type of background is particularly relevant to searches with same-sign leptons, 
as they may turn a pair of opposite-sign leptons from an abundant Standard Model process ($pp\to Z,\,t\bar{t},\,W^+W^-$\ldots) 
into a much rarer same-sign pair.
Because of the jets requirements characteristic of our analysis, 
the dominant source of charge flip electrons in most of the signal regions arise from leptonic decays of $t\bar{t}$ pairs. 
This type of background contributes only moderately to the signal regions -- for example in the Run-1 analysis it represented at most 10\% of the total background yield. 
But for looser event selections, such as the ones used to validate the background modelling, it can be a major component in electron channels. 
It is therefore important to be able to predict reliably this background. 

Figure~\ref{fig:mc_chargeflip_rate} shows the expected rate of charge flip for electrons 
satisfying the requirements listed in sections~\ref{sec:objects_electrons} and~\ref{sec:isolation}, 
determined in simulated $t\bar{t}\to e\mu\nu\nu$ events. 
The rate increases significantly at large pseudorapidity, reflecting the larger amount of material in front of the calorimeter. 
There is also a significant dependency to the transverse momentum, with larger rates obtained for energetic electrons. 
\\
\par{\bf Background estimation methodology\\}
We rely on a purely data-driven method to estimate yields of events with charge flipped electrons. 
Assuming one knows the electron charge flip rates $\xi(\eta,p_T)$, a simple way to estimate the yields is to select 
events with pairs of opposite-sign leptons and assign them a weight: 

\begin{align}
w = \xi\left(\eta^{1},p_T^{1}\right)\left[1-\xi\left(\eta^{2},p_T^{2}\right)\right] 
+ \xi\left(\eta^{2},p_T^{2}\right)\left[1-\xi\left(\eta^{1},p_T^{1}\right)\right] 
\end{align}
where $\xi=0$ for muons. 

The advantages of this method are a good statistical precision since the charge flip rate is rather low, 
and the lack of dependency on the simulation and related uncertainties. 
Obviously, it requires a precise measurement of the rates, which is described in the next paragraph. 
Another inconvenient is that the reconstructed electron energy (hence momentum as well) of charge flipped electrons 
tends to be too low by a few~\GeV, because of the hard Brehmsstrahlung at the origin of the charge flip. 
Simply reweighting electrons from opposite-sign lepton pairs therefore does not predict correctly the charge-flip background shape 
for energy-dependent variables. 
In the past this effect was simply neglected, as we do not rely on discriminant variables very sensitive to the electron energy. 
But we are now considering applying energy correction factors, or assign dedicated systematic uncertainties. 
\\
\par{\bf Measurement of the charge flip rates\\}
The simulation of the charge flip process is not accurate enough to be relied on (discrepancies up to a factor 2 were seen in Run-1 analyses), 
therefore the rates have to be measured in data. 
In Run-2, electron charge flip rate measurements will be centralized by the Egamma performance group (unlike up to now). 
One of our group members is involved in these activities and will provide the rates according to our customized electron definitions. 
The standard method which will be employed for these measurements is similar to the one described 
in the documentation of the Run-1 same-sign leptons + jets search~\cite{noteSS3L}. 
It relies on the observed numbers of opposite (OS) and same-sign (SS) electrons with an invariant mass close to the $Z$ mass, 
which provides a clean sample of electrons. Expressing these numbers as function of the electron charge flip rate $\xi(\eta,p_T)$: 
\begin{align}
N_\text{SS}\left(\eta^{1},p_T^{1},\eta^{2},p_T^{2}\right) \approx 
\left[\xi\left(\eta^{1},p_T^{1}\right) + \xi\left(\eta^{2},p_T^{2}\right)\right] 
N\left(\eta^{1},p_T^{1},\eta^{2},p_T^{2}\right)\,,\quad N=N_\text{OS}+N_\text{SS}
\end{align}
the rates are then obtained as the maximum likelihood estimators for the product of Poisson distributions $\mathcal{P}(N_\text{SS}| N)$ 
binned in $\eta$ and $p_T$ of the two electrons. 

Sources of systematic uncertainties on the measured rates account for the background subtraction, 
and closure tests performed on Monte-Carlo (differences between the computed and true rates) 
and data ($Z$ lineshape opposite-sign electrons pairs reweighted by the measured rates, compared to the distribution of same-sign pairs). 

\subsection{Backgrounds with fake leptons}
\label{sec:bkg_fakes}

The term "fake lepton" denotes here any reconstructed lepton not originating 
from the decay of massive gauge or Higgs bosons, or an electroweak initial/final state radiation. 
They can be non-prompt leptons produced in heavy flavor meson decays, converted photons, 
light hadrons faking the electron shower, in-flight decays of kaons or pions to muons, etc.  
Common properties shared by these objects are a bad response to electron identification cuts, 
non-zero impact parameters, and the reconstructed leptons are often not well isolated; 
these properties can be used to discriminate fake leptons against the prompt leptons we are interested in. 

In signal regions enriched in $b$-jets, Monte-Carlo simulations predict that the dominant source of fake leptons originates 
from semileptonic $t\bar{t}$ events, with a non-prompt lepton in the decay chain of one of the $B$ mesons. 
Contributions from photon conversions and hadron fakes are not to be neglected though, 
in particular in $b$-jet-depleted signal regions where the rate of $t\bar{t}$ events is lower 
and is competed by processes such as $W\gamma$ + jets. 

We will rely on at least two different methods to estimate the fake leptons yields in the signal regions, 
which are briefly described in the next paragraphs. They have both been employed successfully in the Run-1 analysis~\cite{noteSS3L}. 

\subsubsection{The matrix method}
\label{sec:bkg_fakes_matrixmethod}

The matrix method is a purely data-driven approach 
which relies on the different response of prompt and fake leptons to identification, isolation and impact parameters requirements: 
fake leptons have low probabilities to satisfy these requirements, unlike prompt leptons. 
No tentative is made to consider the different categories of fake leptons separately -- systematic uncertainties are assigned in consequence. 
\\
\par{\bf Methodology\\}
A combination of tight requirements on discriminant variables such as electron identification, lepton isolation and impact parameters, is defined (see Tabe~\ref{tab:lepdef}). 
Reconstructed leptons are then classified in two categories ("tight" and "loose"), depending on their success satisfying the tight requirements or not. 
If $\epsilon$ and $\zeta$ are respectively the probabilities for a prompt/fake lepton to satisfy the requirements, 
linear relationships can be established between the mean values of the rates of prompt/fakes and tight/loose leptons, which for the 1-lepton case are: 
\begin{align}
\label{eqn:matrixmethod}
<N_\text{tight}> &= \epsilon <N_\text{prompt}> + \zeta <N_\text{fake}> \\\notag
<N_\text{loose}> &=  (1-\epsilon) <N_\text{prompt}> + (1-\zeta) <N_\text{fake}>
\end{align}

This system of equations can be used to evaluate the number of prompt and fake leptons given the observed number of tight and loose leptons. 
A detailed explanation of the generalization of the method to handle events with arbitrary number of leptons, as used in the Run-1 analysis, 
can be found in~\cite{noteSS3L,TomThesis}. 

The method relies of the prior knowledge of the probabilities $\epsilon$ and $\zeta$, 
which need to be measured in dedicated samples enriched in prompt and fake leptons (see next paragraph). 
The uncertainties on the probabilities for fake leptons constitute the main source of uncertainties in the asymptotic regime. 
In the low statistics regime, one has to cope with the fact that the loose and tight leptons categories are not enough populated 
to provide reliable estimates: 
for example predictions of negative yields are a possible outcome. 
In general these estimates are accompanied by large statistical uncertainties. 

Finally, one should note that the charge flip electron background interferes with the matrix method 
as charged flipped electrons are notably more prone to fail impact parameter or tight identification requirements, 
and the related efficiencies are distinct from both those of prompt and fake electrons. 
They correspond so to speak to a third category of objects, while the matrix method is based on the assumption that only two categories are present. 
One way to solve the issue is to rely on the linearity of the matrix method estimate with respect to its input number of events; 
therefore one can simply subtract the estimated charge flip background in the tight and loose leptons categories, from observed data. 
This requires a dedicated measurement of the charge flip rate for electrons failing the tight requirement. 
\\
\par{\bf Measurement of the $\epsilon,\zeta$ probabilities\\}
These parameters are measured in dedicated samples enriched in prompt or fake leptons. 
Probabilities for prompt leptons are measured in $Z\to \ell\ell$ tag-and-probe selections, 
and are assigned systematic uncertainties determined in Monte-Carlo covering the 
differences between the lepton properties in the measurement regions, 
and in the signal regions which have much busier event topologies. 
Other systematic uncertainties should account for the less isolated leptons that can be 
found in signal scenarios where decay products are particularly boosted (for example the gluino-stop offshell model). 

Probabilities for fake leptons are much harder to determine, due to the difficulty to identify 
an event selection that would provide both a high purity, and enough statistics, especially for leptons with $p_T>40$~\GeV. 
In the 8 TeV versions of this analysis, such selections were requiring at least two same-sign leptons together with a jet, 
and the fake electron probabilities were determined separately for events with or without $b$-jets. 
Other analyses have been using inclusive selections with a single lepton, 
which have the advantage to be much more populated, but on the other hand 
are less representative of the properties of fake leptons that can be found in our signal regions. 
In general the probabilities vary largely with $p_T$ (see e.g. the isolation efficiencies in section~\ref{sec:isolation}) thus require binned measurements. 

The measurements of fake leptons probabilities are associated to large uncertainties, 
which cover for the nature of the fake leptons and the events that produce them 
being different between the measurement and signal regions. 

\subsubsection{The Monte-Carlo template method}
\label{sec:bkg_fakes_mctemplates}

This method is Monte-Carlo based, and makes use of the samples corresponding to the various processes expected to produce fake leptons. 
Correction factors of the fake lepton modelling in the simulation are determined 
in a combined fit of several control regions involving shapes of discriminant variables such as the missing transverse momentum or the jet multiplicity. 
There are five such correction factors (for fake electrons and muons originating from light or heavy flavor jets, and an additional factor for charge flip), 
which are applied on an event basis depending on the generator information about the origin of the fake lepton in the event. 
More details can be found in the documentation of the Run-1 analysis~\cite{noteSS3L}. 

\subsubsection{Statistical tools improving robustness in low statistics regime}
\label{sec:bkg_fakes_stattools}

Some alternatives to the matrix method have been considered, based on the same principles but with more formal constructions as probabilistic tools 
(hence a better behaviour in conditions far from the asymptotic regime), and are currently being developed. 
One of these methods~\cite{PseudoLikelihoodMatrixMethod} relies on the construction of a likelihood function 
which translates the equation system~(\ref{eqn:matrixmethod}) in terms of probabilities for single leptons, instead of relationships between mean values; 
this also allows to treat $\epsilon$ and $\zeta$ as nuisance parameters (according to their measurements uncertainties) instead of being frozen. 
One could then simply estimate the rates of prompt and fake leptons as the maximum likelihood estimators of the observations, 
but the large number of parameters (several leptons per event hence many possible loose/tight combinations, binned probabilities $\epsilon$ and $\zeta$\ldots) 
renders this approach impractical. 
A simplification is discussed in~\cite{PseudoLikelihoodMatrixMethod,TomThesis}, which preserves some advantages over the standard matrix method. 

Another approach has been proposed recently~\cite{TomThesis}, 
consisting in building confidence intervals on the rate of fake leptons 
from the associated Bayesian posterior, which may be numerically computed with Gibbs sampling. 
The claimed advantages are a better assessment of the uncertainties on the final estimate, 
since the whole posterior distribution is known, 
as well as the absence of issues with local minima which might affect likelihood-based methods. 
More details can be found in~\cite{TomThesis}. 

We are planning to check if these methods can be applied with all the complexity of a real world analysis, 
in which case they would bring quite a nice improvement to the current way of predicting fake lepton background in the signal regions. 

\subsection{Backgrounds with fake $b$-jets}
\label{sec:bkg_bjetfakes}

In the signal regions requiring at least three $b$-jets, an important part of the background originates from processes 
with two real $b$-jets, and a third jet not originating from a $b$-quark but satisfying the $b$-tagging requirements. 
Unlike the case of fake leptons, the $b$-tagging performance group usually provides scale factors (and associated uncertainties) 
to correct the simulation for mismodelling both of real and fake $b$-jets. 
The fake leptons and charge flip background, being predicted from data, obviously do not need any correction. 

An alternative data-driven method was used in Run-1 as cross-check, 
consisting in the equivalent of the lepton matrix method but applying on jets 
and replacing lepton tight requirements by the output weight of the $b$-tagging algorithm. 
This approach was primarily developed for the ATLAS SUSY search with 0/1 lepton and 3 $b$-jets~\cite{SUSY3bjetsRun1}. 
It could be considered again for Run-2. 

\subsection{Neglected backgrounds}
\label{sec:bkg_neglected}
Other sources of background such as cosmic rays or cavern background, as well as pile-up events where two distinct proton-proton pairs may interact and produce leptons, 
have been evaluated in the past and found to be negligible. 
Multiple scattering effects are included in the simulations but are also not expected to contribute enough to require in-depth studies. 

\subsection{Validation regions}
\label{sec:bkg_VR}
The different type of backgrounds will be validated by looking at the agreement between the data observation and the SM expectation in regions dominated by one type of background. As for Run-1, several validation regions will be considered, and their definition is a balance between high purity, large statistics and low signal contamination. The latter is assured by requiring an upper cut on \met\ 
($<150$~\GeV). If no validation region can be designed, the background will be checked by looking at different kinematic variables distributions. For a complete validation, several selections will be considered :
\begin{itemize}
	\item{Loose: requiring at least two signal leptons in the event.}
	\item{Intermediate: adding soft cuts i.e. at least one or two signal jets or $b$ jets.}
	\item{Hard: increase the number of jets in the event.}
\end{itemize}


It is known that the associated uncertainty on the SM background can be decreased by adding control regions enriched in a certain type of background and orthogonal on the signal regions. However, given the rare SM processes which dominate in the signal regions, the statistics expected with $\sqrt s=$ 13 \TeV\ and $L$~$\le$3 \ifb of data in potential control regions is too small to be included. Hence, only validation regions will be defined for early Run-2.

\paragraph{\ttbar\ + $V$ background\\}
To validate the \ttbar\ + $Z$ and \ttbar\ + $W$ backgrounds estimation, several tentative regions defined with at least 1 or 2 $b$-jets in the event were considered. To decrease the other type of backgrounds, several jets and a low \met\ cut were also required. However, given the low statistics, it was found that a combined \ttbar~+~ $V$ region will perform better. The region definition is given in Table~\ref{tab:ttV_VR}. The leading and sub-leading leptons are required to be energetic (20 \GeV) in order to reduce the fake lepton background, while the veto on the fourth baseline lepton is minimizing the contamination from $ZZ$ processes. Note that it is very hard to obtain a pure validation region in \ttbar\ + $V$, as the difference with the other prompt SS SM processes is mainly the jet multiplicity. The highest purity was found when considering at least four jets in the $ee$ and $e\mu$ channels, and at least three jets in the $\mu\mu$ channel. A high \meff\ cut is used to further reduce the detector background, wile an upper cut of 900 \GeV\ is used to reduce the signal contamination. Note that this cut can be tighten to further decrease the SUSY contamination from models like direct sbottom. The reached purity for $\sqrt s=$ 13 \TeV\ and $L$~$\le$3 \ifb of data is around 42$\%$ if the large eta region is excluded ($|\eta|_e$~$<$~1.37) . The purity can be increased to 60$\%$ if the inclusive 1 $b$ jet selection is replaced by inclusive 2 $b$ jet selection (and the 400 \GeV\ cut on \meff\ is relaxed to 350 \GeV). The signal contamination was studied by looking at several available SUSY models studied in this analysis. It can be up to 27$\%$ when the direct-sbottom model is considered and the sbottom mass is 550 \GeV, or 20$\%$ for direct squark (2 step) via sleptons with $W/Z$ bosons in the cascade decay, otherwise it is smaller than 10$\%$. 

\begin{table}[htb!]
\caption{\ttbar\ + $V$ validation region definition. The \pt\ threshold of the leading two leptons is 20 \GeV, otherwise 10 \GeV.}
\label{tab:ttV_VR}
\begin{center}
    \begin{tabular}{|c|cc|c|c|}
      \hline
      \hline
     VR& $ N_{lept}^{signal}$ & $N_{lept}^{baseline}$ & $N_{b-jets}^{20}$     & Other variables \\ \hline
    ttV & $\ge 2$ &$<4$  & $\ge$1  & 30 $<$ \met $<$ 200 \GeV, 400 $<$ \meff $<$ 900 \GeV, $|\eta|_e$~$<$~1.37, \\
   &  & && $N_{jets}^{30}$ $\ge$ 4 in $ee$ and $e\mu$, $N_{jets}^{25}$ $\ge$ 3 in $\mu\mu$ channels\\	
      \hline
\end{tabular}
\end{center}
\end{table}


\paragraph{$ZZ$ + jets validation region\\}
Even if this type of background is minor in several of the defined signal regions, its contribution can is significant in regions with no $b$ jet requirement and three leptons. Therefore, a validation region was defined as presented in Table~\ref{tab:ZZ_VR}. It requires at least three signal leptons with \pt~$>$~20 \GeV, and at least four baseline leptons. The reached purity for $\sqrt s=$ 13 \TeV\ and $L$~$\le$3 \ifb of data is up to 97$\%$. In order to be closer to the SRs, another validation region can be defined by asking at least two jets with \pt~$>$~25 \GeV\ with a purity of 84$\%$. The signal contamination is at most 20$\%$.

\begin{table}[htb!]
\caption{$ZZ$ validation region definition. The \pt\ threshold of the two leading leptons is 20 \GeV, otherwise 10 \GeV. To be closer to the SRs, at least two jets with \pt~$>$~25 \GeV\ can be required in the event (and $N_{b-jets}^{20}$~$==$~0 becomes $N_{b-jets}^{20}$~$\ge$~0), without affecting the validation region purity.}
\label{tab:ZZ_VR}
\begin{center}
    \begin{tabular}{|c|cc|c|c|}
      \hline
      \hline
     VR & $ N_{lept}^{signal}$ & $N_{lept}^{baseline}$ & $N_{b-jets}^{20}$     & Other variables \\ \hline
     ZZ& $==3$ & $\geq 4$  & $==$0  & \met $<$ 150 \GeV, \meff $>$ 100 \GeV, $|\eta|_e$~$<$~2.  \\
     \hline
\end{tabular}
\end{center}
\end{table}


\paragraph{$WZ$ + jets validation region\\}
Similarly to $ZZ$ type of background, it is negligible in most of signal regions. However, it is expected to populate the regions with 0 $b$-jets and low jet multiplicity. Therefore, its validation is important and the proposed region definition is presented in Table~\ref{tab:WZ_VR}. It requires exactly three leptons and a veto on the fourth-leading baseline lepton in order to reduce the $ZZ$ background contamination. The lower cut on \met\ (30 \GeV) is mainly reducing the charge flip contamination. The \mt\ cut of 50 \GeV, even if is not used to define the signal regions, is assuring a smaller fake lepton contamination. At least one and at most three jets with \pt~$>$~25 \GeV are required in the event. It assures a purity of 70$\%$ for $\sqrt s=$ 13 \TeV\ and $L$~$\le$3 \ifb of data. The signal contamination is found to be below 1$\%$.

\begin{table}[htb!]
\caption{$WZ$ validation region definition. The \pt\ threshold of the two leading leptons is 20 \GeV, otherwise 10 \GeV.}
\label{tab:WZ_VR}
\begin{center}
    \begin{tabular}{|c|cc|c|c|}
      \hline
      \hline
     VR & $N_{lept}^{signal}$ & $N_{lept}^{baseline}$ &  $N_{b-jets}^{20}$     & Other variables \\ \hline
     WZ & $==$3 & $<4$  & $==$0  & 30 $<$ \met $<$ 300 \GeV, \mt~$>$50 \GeV, 1 $\leq$ $N_{jets}^{25}$ $\leq$ 3, $|\eta|_e$~$<$~2. \\
     \hline
\end{tabular}
\end{center}
\end{table}

\paragraph{$WW$ + jets validation region\\}
Similarly to the other diboson processes, the $WW$ + jets (with a pair of same sign lepton in the final state) events contribute mainly in the signal regions with no $b$ jet requirement. Two validation regions are defined (Table~\ref{tab:WW_VR}). In the first region, a cut on the invariant mass of the first two leading jets in the event is considered. It selects forward jets from VBS processes which are not necessary populating the signal regions defined in the analysis. However it is used as a cross-check as a high purity (50$\%$) can be reached. A second region is defined by selection two energetic jets (50 \GeV) and decreasing the electron acceptance to $|\eta|_e$~$<$~1.37. The purity is only 31$\%$. In both validation regions the signal contamination is at most 15$\%$.

\begin{table}[htb!]
\caption{$WW$ validation region definition. The \pt\ threshold of the two leading leptons is 20 \GeV.}
\label{tab:WW_VR}
\begin{center}
    \begin{tabular}{|c|cc|c|c|}
      \hline
      \hline
     VR & $N_{lept}^{signal}$ & $N_{lept}^{baseline}$ &  $N_{b-jets}^{20}$     & Other variables \\ \hline
     WW1 & $==$2 & $<3$  & $==$0  & 40 $<$ \met $<$ 300 \GeV, $Z$ mass veto,\\ 
     &&&& \mt~$>$40 \GeV, $N_{jets}^{40}$ $\ge$ 2, $m_{jj}$ $>$ 500 \GeV, $|\eta|_e$~$<$~2. \\\hline
     WW1 & $==$2 & $<3$  & $==$0  & 40 $<$ \met $<$ 300 \GeV, $Z$ mass veto,\\
     &&&& \mt~$>$40 \GeV, $N_{jets}^{50}$ $\ge$ 2, $|\eta|_e$~$<$~1.37 \\
     \hline
\end{tabular}
\end{center}
\end{table}

\paragraph{Fake lepton validation regions\\}
Beside the usual validation plots shown for the Run-1 analysis, several fake lepton validation regions are defined, in order to increase the confidence on the background estimation methods. Generally, depending on the $b$-jet multiplicity two categories are considered: low and high \meff, as presented in Table~\ref{tab:Fake_VR}. To reduce the charge flip background contamination, a veto on the $Z$ boson mass ($80<m_{\ell\ell}<100$~\GeV) is imposed only in the $ee$ channel. The background from prompt SS processes is reduced by a veto on the third baseline lepton. The leading two leptons are required to have a \pt\ threshold of 15 \GeV, and the event should have at least one jet. For the low \meff\ regions, the highest purity was reached when \met~$<$~100 \GeV\ and \meff~$<$~400 \GeV\ was imposed. A lower cut on \meff\ will drastically reduce the fake lepton background in the low \meff\ validation region, while in the high \meff\ region a higher threshold will increase the level of prompt SS background. Note that, the two validation regions overlap and the reached purity varies between 80 and 90$\%$. The signal contamination is found to be maximum 10$\%$.

\begin{table}[htb!]
\caption{Fake lepton validation region definition. The \pt\ threshold of the two leading leptons is 15 \GeV. }
\label{tab:Fake_VR}
\begin{center}
    \begin{tabular}{|c|cc|c|c|}
      \hline
      \hline
      \multicolumn{5}{|c|}{\textbf{Low meff fake leptons VRs}}\\
      \hline
      \hline
     VR & $N_{lept}^{signal}$ & $N_{lept}^{baseline}$ & $N_{b-jets}^{20}$     & Other variables \\ \hline
     FL0b-L\meff& $==$2 & $<3$  & $==$0  & 20 $<$ \met $<$ 100 \GeV, \meff $<$ 400 \GeV, \mt~$<$60 \GeV, $N_{jets}^{25}\geq 1$ \\
     FL1b-L\meff& $==$2 & $<3$  & $==$1  & 20 $<$ \met $<$ 100 \GeV, \meff $<$ 400 \GeV, \mt~$<$60 \GeV\\
     FL2b-L\meff&$==$2  & $<3$ & $\ge$2  & 20 $<$ \met $<$ 100 \GeV, \meff $<$ 400 \GeV, \mt~$<$80 \GeV \\
     \hline
      \hline
      \multicolumn{5}{|c|}{\textbf{High meff fake leptons VRs}}\\
      \hline
      \hline
     VR & $N_{lept}^{signal}$ & $N_{lept}^{baseline}$ & $N_{b-jets}^{20}$     & Other variables \\ \hline
     FL0b-H\meff& $==$2 & $<3$ & $==$0  & 20 $<$ \met $<$ 100 \GeV, \meff $>$ 300 \GeV, \mt~$<$100 \GeV, $N_{jets}^{25}\geq 1$ \\
     FL1b-H\meff& $==$2 & $<3$ & $==$1  & 20 $<$ \met $<$ 50 \GeV, \meff $>$ 300 \GeV, \mt~$<$80 \GeV\\
     FL2b-H\meff&$==$2  & $<3$ & $\ge$2  & 20 $<$ \met $<$ 50 \GeV, \meff $>$ 300 \GeV, \mt~$<$50 \GeV\\
     \hline
\end{tabular}
\end{center}
\end{table}




\section{Systematic Uncertainties}
\label{sec:systematics}

\subsection{Experimental systematics}
\label{sec:syst_exp}

All the  experimental systematics provided by the SUSYTools {\tt getSystInfoList()} method have been considered. 
The list of sources of uncertainty and the corresponding names of the variations are:

\textbf{Jet energy scale ({\tt{JET\_GroupedNP\_\{1-3\}\_\_1\{up,down\}}})}  \\
One of the strongly reduced uncertainty sets provided by the JetEtMiss group for early Run-2 searches is used in this note. 
These sets are intended for use by analyses which are not sensitive to jet-by-jet correlations arising from changes to the jet energy scale 
(as expected for many early SUSY searches), 
and we use the scenario {\tt InsituJES2012\_3NP\_Scenario1.config} (as included in the JetUncertainties package). 
We checked that the uncertainties obtained from one of the 3 other scenarios did not lead to significant changes. 
The jet energy is scaled up and down (in a fully correlated way) by the $\pm 1\sigma$ uncertainty of each nuisance parameter.

\textbf{Jet energy resolution ({\tt{JET\_JER\_SINGLE\_NP\_\_1up}})} \\
An extra $\pt$ smearing is added to the jets based on their $\pt$ and $\eta$ to account for a possible underestimate of the jet energy resolution in the MC simulation. This is done by the {\tt JERSmearingTool} in the JetResolution package.

\textbf{Egamma resolution ({\tt{EG\_RESOLUTION\_ALL\_\_1\{up,down\}}})} \\
A nuisance filtering scheme to reduce the $\sim$8 NPs used to electron and photon resolution to only one as implemented in SUSYTools is used.

\textbf{Egamma scale ({\tt{EG\_SCALE\_ALL\_\_1\{up,down\}}})}\\
A nuisance filtering scheme to reduce the $\sim$16 NPs used to electron and photon resolution to only one as implemented in SUSYTools is used.

\textbf{Electron efficiency ({\tt{EL\_EFF\_\{ID,RECO,TRIGGER,Iso\}\_TotalCorrUncertainty\_\_1\{up,down\}}})} \\
These uncertainty sources are associated with the electron efficiency scale factors provided by the Egamma CP group.

\textbf{Muon efficiency ({\tt{MUON\_EFF\_\{STAT,SYS\}\_\_1\{up,down\}}}, \\
{\tt{MUON\_EFF\_Trig\{Stat,Syst\}Uncertainty\_\_1\{up,down\}}}, \\
{\tt{MUON\_ISO\_\{STAT,SYS\}\_\_1\{up,down\}}}) }\\
This uncertainty corresponds to the statiscal and systematic uncertainties in the muon efficiency scale factors provided by the Muon CP group on the muon reconstruction, trigger and isolation.

\textbf{Muon resolution uncertainty  ({\tt{MUONS\_ID\_\_1\{up,down\}}}, {\tt{MUONS\_MS\_\_1\{up,down\}}})} \\
This is evaluated as variations in the smearing of the inner detector and muon spectrometer tracks associated to the muon objects by $\pm 1\sigma$ their uncertainty

\textbf{Muon momentum scale ({\tt{MUONS\_SCALE\_\_1\{up,down\}}})} \\
This is evaluated as variations in the scale of the momentum of the muon objects

\textbf{\met\ soft term uncertainties  ({\tt{MET\_SoftTrk\_Reso\{Pare,Perp\}}}, {\tt{MET\_SoftCalo\_Scale\{Up,Down\}}})}\\
Note that the effect of the hard object uncertainties (most notably JES and JER) are also propagated to the $\met$.

\textbf{Flavor tagging ({\tt{FT\_EFF\_\{B,C,Light\}\_systematics\_\_1\{up,down\}}}, \\
{\tt{FT\_EFF\_\{B,C,Light\}\_extrapolation\_\_1\{up,down\}}}, \\
{\tt{FT\_EFF\_\{B,C,Light\}\_extrapolation from charm\_\_1\{up,down\}}})} \\
Similarly to the case of the JES, a significant reduction in the number of nuisance parameters was provided by the Flavour Tagging CP group at the beginning of Run-2.\\

\textbf{Pileup reweighting ({\tt{PRW\_DATASF\_\_1\{up,down\}}})}\\
This uncertainty is obtained by re-scaling the $\mu$ value in data by 1.00 and 1/1.23, covering the full difference between applying and not-applying the nominal $\mu$ correction of 1/1.16, as well as uncertainty on the luminosity measurement which is expected to dominate.\\


\subsection{Theoretical systematics}
\label{sec:syst_theo}

% In Run-1~\cite{paperSS3L,noteSS3L}, the systematic uncertainties associated with the $t \bar{t} + V$ and
% diboson MC were assessed by using samples where the factorization and
% renormalization scales have been varied. We plan to follow a similar approach, 
% following the recommendations by the Physics Modeling Group.
%once scale variation samples are available in mc15 productions.
%Other theoretical uncertainties were evaluated by comparing different generators and samples produced with different
%number of partons. A similar approach will be used in 2015, following the recommendations by the Physics Modeling Group.
The theoretical uncertainties on the $t \bar{t} + V$ production cross-sections are $22\%$ for $t\bar{t}W$ \cite{Campbell:2012dh} 
and $t\bar{t}Z$~\cite{Garzelli:2012bn} 
In addition, uncertainties on the signal region fiducial acceptance for these processes were assessed 
by using MC samples with varied factorization and renormalization scales. 
This led to overall uncertainties of $30\%$ on the $t \bar{t} + V$ contributions to the signal regions.  

For inclusive diboson production, cross-section uncertainties amount to $7\%$ (computed with MCFM \cite{Campbell:2011bn}. 
Delays in production of samples with scale variations prevented us from checking scale impact on the fiducial acceptance; 
however, after comparisons between {\sc Sherpa} and {\sc Powheg} predictions, 
we established an overall 30\% uncertainty as well for these processes. 

Normalisation uncertainties between 35\% and 100\% were applied to processes with smaller contributions (triboson 
production, $t\bar{t}h$, $t+Z$, etc.).

% For dibosons, the theory uncertainties will be evaluated by comparing the results from the nominal {\sc Sherpa} samples 
% with alternative samples generated with {\sc Powheg}. Note that currently {\sc Powheg} samples exist for all diboson 
% processes except same-sign $W^\pm W^\pm jj$ production, which has important contribution to the signal regions without $b$-jets.
% Also note that with an increase in production cross section of a factor $\sim$4 between $\sqrt{s}=8$~TeV and 13~TeV, $\ttbar H$ 
% production can be a relevant background in the signal regions containing $b$-jets. The theory uncertainties for this process 
% will be evaluated in a similar way as in other analyses in the Higgs working group.




\section{Simultaneous fit method and results}
\label{sec:fit}
The sensitivity of the signal regions proposed in Section~\ref{sec:sr} is evaluated with more refined statistical tools in this section, 
by using the HistFitter framework~\cite{HistFitter} to perform discovery hypothesis tests for the different signal scenarios of interest. 
The event yields assumed in these test are the ones predicted by the set of Monte-Carlo samples described in section~\ref{sec:samples}, 
and the object selections detailed in Sections~\ref{sec:objects} and~\ref{sec:evtsel}. All the results in this section are presented for an integrated luminosity of 3 fb$^{-1}$. 

The fit configuration is setup in the \textit{exclusion} mode, i.e. to perform hypothesis testing with a known BSM signal. 
Of course here the rejected hypothesis is the background-only hypothesis. 
Only one signal region is fitted at the time; furthermore, there is no control region defined in the analysis. 
We plan to investigate at a later stage potential benefits of combined fits of several signal regions, 
which was the setup in use for the Run-1 analysis. 
Distinct global systematic uncertainties are assigned to each category of background, amounting to 30\% for $t\bar t+V$ processes, $35\%$ for diboson ,100\% for minor processes such as $t\bar tH$, and 50\% for the reducible background ($t\bar t$, $V$ + jets, $W^\pm W^\mp$). 
\\
\par{\bf Event yields in the signal regions\\}

\begin{figure}[htb!]
\centering
\subfigure{\includegraphics[width=0.49\textwidth]{HISTFITTER/can_SR3b3j20_binned_nJets20Jvf025_beforeFit}}
\subfigure{\includegraphics[width=0.49\textwidth]{HISTFITTER/can_SR3b6j40_cuts_beforeFit}}
\caption{Expected yields for Standard Model processes and a few characteristic signal scenarios, 
in selections with at least two same-sign leptons and three $b$-jets. The selection on the right side further requires at least six jets with $p_T>40$~\GeV. 
}
\label{fig:histfitter_sr3b}
\end{figure}

\begin{figure}[htb!]
\centering
\subfigure[$\met>100$, $\geq 4$ jets ($p_T>50$)]{\includegraphics[width=0.49\textwidth]{HISTFITTER/can_SR1b100_binned_meff_beforeFit}}
\subfigure[$\met>100$, $\geq 3$ jets ($p_T>40$)]{\includegraphics[width=0.49\textwidth]{HISTFITTER/can_SR0b3jmet100_binned_meff_beforeFit}}
\subfigure[$\met>150$, $\geq 4$ jets ($p_T>50$]{\includegraphics[width=0.49\textwidth]{HISTFITTER/can_SR1b150_binned_meff_beforeFit}}
\subfigure[$\met>100$, $\geq 4$ jets ($p_T>40$)]{\includegraphics[width=0.49\textwidth]{HISTFITTER/can_SR0b4jmet100_cuts_beforeFit}}
\subfigure[$\met>200$, $\geq 4$ jets ($p_T>50$]{\includegraphics[width=0.49\textwidth]{HISTFITTER/can_SR1b200_binned_meff_beforeFit}}
\subfigure[$\met>100$, $\geq 5$ jets ($p_T>40$)]{\includegraphics[width=0.49\textwidth]{HISTFITTER/can_SR0b5jmet100_cuts_beforeFit}}
\caption{Expected yields for Standard Model processes and a few characteristic signal scenarios, 
in selections with at least two same-sign leptons, at least one (left) or no $b$-jet (right), and the other cuts that define each of these signal region candidates. 
}
\label{fig:histfitter_sr0and1b}
\end{figure}


Figures~\ref{fig:histfitter_sr3b} and~\ref{fig:histfitter_sr0and1b} present the event yields of the background and a few characteristic signal scenarios, 
in various signal region candidates. The regions that have enough statistics to allow it are binned in the most relevant 
variable to discriminate between signal and background, either the effective mass or the number of jets (the latter for the 3 $b$-jets selection). 
Binnings are chosen such that the background yield in one bin remains at the level of around one event. 
One should note that they correspond to the binning used when performing shape fits as described in the next paragraph. 
\\
\par{\bf Sensitivity for the gluino stop offshell model\\}

\begin{figure}[htb!]
\centering
\subfigure{\includegraphics[width=0.49\textwidth]{HISTFITTER/gtt1300}}
\subfigure{\includegraphics[width=0.49\textwidth]{HISTFITTER/gtt_lsp100}}
\caption{Expected signal significance for 3 fb$^{-1}$ for the gluino offshell stop model $\gluino\to t\bar{t}\neut$, 
varying the neutralino mass with a fixed gluino mass of 1300~\GeV (left) 
or extrapolating to different gluino masses from a reference point with gluino and neutralino masses of 1300 and 100~\GeV respectively (right). }
\label{fig:signif_gtt}
\end{figure}

\begin{figure}[htb!]
\centering
\subfigure{\includegraphics[width=0.49\textwidth]{HISTFITTER/sbottom}}
\subfigure{\includegraphics[width=0.49\textwidth]{HISTFITTER/rpv_tbs}}
\caption{Expected signal significance for 3 fb$^{-1}$ for the direct sbottom $\sbot\to t\chargino$ (left) 
or the gluino RPV $\gluino\to tbs$ (right) models. The neutralino mass is fixed to 100~\GeV in the sbottom scenario. }
\label{fig:signif_sbottom_and_rpv}
\end{figure}

\begin{figure}[htb!]
\centering
\subfigure{\includegraphics[width=0.49\textwidth]{HISTFITTER/firstgen_gq}}
\subfigure{\includegraphics[width=0.49\textwidth]{HISTFITTER/firstgen_lsp}}
\caption{Expected signal significance for 3 fb$^{-1}$ for the gluino and squark 2-step decay models $\gluino/\tilde{q}\to q(q')WZ\neut$, 
varying the gluino or squark masses for a fixed neutralino mass of 100 GeV (left), 
or varying the neutralino mass for fixed gluino and squark masses of respectively 1000 and 650~\GeV. 
}
\label{fig:signif_firstgen}
\end{figure}

Figure~\ref{fig:signif_gtt} presents the signal significance obtained with several signal region candidates, for the gluino offshell stop model. 
For some of the regions, a shape fit of the effective mass or jet multiplicity distributions is performed. 
While it can provide additional benefit over simple cuts, the main reason for its use here was rather simply to avoid tuning cuts for each signal point. 
One can see indeed on the figure that for each mass, the significance obtained with an effective mass shape fit is rather close to the best 
of the significances computed with different choices of fixed cuts. 
As expected, at large neutralino mass signal regions based on $\geq 3$ $b$-jets perform better than the $\geq 1$ $b$-jet signal regions. 
There is not much difference in significance between the $\geq 3b$ selection requiring 6 hard jets and the one based on softer jets, 
possibly due to the fact that the mass spectra of the available grid points are not very compressed. 
One would probably see a larger difference for points lying on the diagonal $m_{\gluino}=2m_t+m_{\neut}$.

As only few signal samples have been generated, the projections 
Figure~\ref{fig:signif_gtt} (right) provides an estimate of the evolution of the sensitivity in a finer way, 
by simply using the kinematic distributions and yields obtained in a reference sample (gluino mass 1300~\GeV, neutralino mass 100~\GeV) 
weighted by the ratio of inclusive cross-sections between the mass to be extrapolated to, and the reference.  
It has been checked that the significance obtained from the next available sample (at a gluino mass of 1500~\GeV) matches rather closely the extrapolated value. 

One can conclude from these plots that at low neutralino mass and with such as low integrated luminosity, 
there is no sensitivity beyond the Run-1 exclusion. 
At higher neutralino mass however, the analysis would be sensitive to an area not yet excluded.  
\\
\par{\bf Sensitivity for the direct sbottom model\\}
Figure~\ref{fig:signif_sbottom_and_rpv} (left) shows the signal significance for several sbottom masses and a light neutralino. 
Here, the advantage of keeping a moderate $\met$ cut in the definition of the $\geq 1b$ signal region is clear. 
One can see that $2\sigma$ sensitivity up to masses of 650~\GeV might be expected, which is quite a nice improvement over the Run-1 reach 
(450~\GeV, although the expected limit was 50~\GeV higher). 
\\
\par{\bf Sensitivity for the gluino model with RPV stop decay\\}
The expected sensitivity for this model can be found on Fig.~\ref{fig:signif_sbottom_and_rpv} (right). 
The most powerful signal region for this model is as expected the selection with at least 3 $b$-jets and 6 hard jets. 
One can notice however that the signal regions with moderate $\met>100$~\GeV cut are quite sensitive as well : 
the absence of neutralinos is compensated by the presence of boosted tops decaying leptonically. 
\\
\par{\bf Sensitivity for the gluino and squark models with 2-step decays\\}
Finally, the performances of signal region definitions vetoing $b$-jets are probed with the gluino and squarks models with 2-step decays in gauge bosons. 
The signal significances for varied masses of gluino/squark or neutralino are provided on Fig~\ref{fig:signif_firstgen}. 
The signal region requiring 5 hard jets appears as the best compromise for both models, and in different regions of the phase space. 
For the gluino model, not much sensitivity beyong the Run-1 exclusion is gained at high gluino mass; 
at high neutralino mass, however, a significant non-excluded region up to neutralino masses of 650~\GeV seems to be at reach. 
For the squark model, on the other hand, one will be able to probe a new region of the phase space at high squark mass, up to 900~\GeV, which was not excluded in Run-1. 


\section{Conclusion}  
\label{sec:conclusion}

This note summarized the Run-2 preparation activities for the SUSY search using same-sing dileptons or three leptons, 
and the current status as of the end of April 2015 for the ``SUSY walkthroughs'' to be held on April 28th. 
These studies were conducted with DC14 Monte Carlo samples, and despite the known caveats in these samples (missing samples 
for some backgrounds, sub-optimal $b$-tagging, etc.), progress has been reported about object selection, trigger studies, 
signal region optimization, background estimation strategy and validation regions, as well as first sensitivity results 
including experimental systematics. 


\FloatBarrier
\clearpage

%\appendix


\bibliographystyle{atlasBibStyleWithTitle}
\bibliography{ss3l2015}

\clearpage
\appendix

%\input{appendix_pMSSM}
%\input{appendix_SS_SqSq}

\clearpage 
\section{Details on the isolation optimization}
\label{app_isolation}

For optimizing the isolation requirements in the analysis, and following the conclusions in Section~\ref{sec:isolation} 
about which of the available isolation variables to use, the following benchmark isolation cuts are defined for electrons:
\begin{itemize}
\item {\tt ELE Lose}: MediumLH, ptvarcone20/\pt$<$0.10, topoetcone20/\pt$<$0.10
\item {\tt ELE Medium}: MediumLH, ptvarcone20/\pt$<$0.06, topoetcone20/\pt$<$0.06
\item {\tt ELE Tight}: for \pt$<$60~GeV: MediumLH, ptvarcone20/\pt$<$0.06, topoetcone20/\pt$<$0.06;
           for \pt$>$60~GeV: TightLH, ptvarcone20/\pt$<$0.06, topoetcone20/\pt$<$0.06
\item {\tt ELE Very Tight}: TightLH, ptvarcone20/\pt$<$0.06, topoetcone20/\pt$<$0.06
\item {\tt ELE Very Very Tight}: TightLH, ptvarcone20/\pt$<$0.04, topoetcone20/\pt$<$0.04
\end{itemize}

Similarly, the following isolation cuts are defined for muons:
\begin{itemize}
\item {\tt MUO Loose}: for \pt$<$40~GeV: ptvarcone30/\pt$<$0.12; for \pt$>$40~GeV: ptvarcone30/\pt$<$0.06 
\item {\tt MUO Tight}: ptvarcone30/\pt$<$0.06 
\item {\tt MUO Very Tight}: ptvarcone30/\pt$<$0.04 
\end{itemize}

For each combination of the electron and muon isolation cuts, the signal significance is computed using RooStats::NumberCountingUtils::BinomialExpZ, assuming 30\% background uncertainty and a luminosity of 3~fb$^{-1}$. 
For simplicity, only $\ttbar$ and $\ttbar +V$ is considered for the background. Four different signal points are used, 
covering mass spectra with large and also with small mass differences.

Table~\ref{tab:isoOpt1} shows the results obtained in a loose event selection, requiring either same-sign dilepton with 4 jets (\pt$>$40~GeV) and $\met>100$~GeV, or $\geq$3 leptons, $\geq$3 $b$-jets ($\pt>40$~GeV). No $m_{\text{eff}}$ requirement is applied. 
These results point to a preference for tight isolation cuts, with {\tt ELE Very Very Tight} giving the maximum significance 
in the four signal models considered. 

Since the signal and background composition can be different after the full signal region selection is applied, 
Table~\ref{tab:isoOpt2} shows the same numbers after applying the event selection requirements in Run-1 SR1b: 
same-sign dilepton, 3 jets (\pt$>$40~GeV), $\geq$1 $b$-jets, 
$\met >150$~GeV and $\mt >100$~GeV. In this case, {\tt MUO Tight} is preferred, with no strong preference between 
{\tt ELE Very Very Tight} and {\tt ELE Very Tight}.

As outcome of these studies, {\tt MUO Tight} and {\tt ELE Very Tight} was selected to be used for lepton isolation in the analysis.


%%%%%%%%%%%%%%%%%%%%%%%%%%%%%%%%%%%%%%%%%%%%%%%%%%%%%%%%%%%%%
\begin{table}[htb]
\caption{Number of background and signal events after the selection explained in the text normalized to 3\ifb\ and 
corresponding signal significance for four
signal models and different scenarios for the lepton isolation. The largest significance values is highlighted in bold.}
\label{tab:isoOpt1}
\begin{center}
\begin{tabular}{|cc|c|cc|cc|cc|cc|}
\hline
  &  &  & \multicolumn{2}{c|}{SMGG2WWZZ}  & \multicolumn{2}{c|}{Gtt}  & \multicolumn{2}{c|}{SMGG2WWZZ} & \multicolumn{2}{c|}{Gtt} \\ 
  &  &   & \multicolumn{2}{c|}{1000\_700\_400}  & \multicolumn{2}{c|}{G1300\_L100}  & \multicolumn{2}{c|}{1000\_900\_800} & \multicolumn{2}{c|}{G1300\_L900} \\ \hline
ele  & mu & BKG & $n_{sig}$ & Signif & $n_{sig}$ & Signif & $n_{sig}$ & Signif & $n_{sig}$ & Signif \\ \hline
L  & L & 27.04 & 16.76 & 1.319 & 5.09 & 0.307  & 2.48 & 0.052 & 2.63 & 0.068  \\  
M  & L & 24.22 & 16.11 & 1.392 & 4.91 & 0.335  & 2.19 & 0.045 & 2.46 & 0.074  \\  
T  & L & 23.45 & 15.94 & 1.415 & 4.81 & 0.338  & 2.15 & 0.047 & 2.41 & 0.075  \\  
VT & L & 23.09 & 15.94 & 1.433 & 4.75 & 0.338  & 2.07 & 0.040 & 2.34 & 0.071  \\  
VVT& L & 21.33 & 14.97 & 1.437 & 4.56 & 0.351  & 2.07 & \textbf{0.056} & 2.25 & 0.078  \\  \hline
L  & T & 25.99 & 16.60 & 1.351 & 5.03 & 0.317  & 2.27 & 0.039 & 2.55 & 0.068  \\  
M  & T & 23.17 & 16.11 & 1.444 & 4.84 & 0.347  & 1.99 & 0.030 & 2.38 & 0.074  \\  
T  & T & 22.40 & 15.94 & 1.469 & 4.73 & 0.349  & 1.95 & 0.032 & 2.33 & 0.077  \\  
VT & T & 22.04 & 15.94 & 1.489 & 4.67 & 0.350  & 1.87 & 0.025 & 2.27 & 0.073  \\  
VVT& T & 19.85 & 14.97 & 1.522 & 4.48 & 0.373  & 1.87 & 0.045 & 2.18 & \textbf{0.086}  \\  \hline
L  &VT & 27.67 & 16.43 & 1.268 & 4.96 & 0.285  & 2.23 & 0.023 & 2.50 & 0.050  \\  
M  &VT & 24.13 & 15.94 & 1.382 & 4.77 & 0.322  & 1.95 & 0.018 & 2.32 & 0.059  \\  
T  &VT & 22.23 & 15.78 & 1.463 & 4.66 & 0.345  & 1.91 & 0.028 & 2.28 & 0.072  \\  
VT &VT & 21.20 & 15.78 & 1.521 & 4.60 & 0.359  & 1.83 & 0.027 & 2.21 & 0.074  \\  
VVT&VT & 19.20 & 14.81 & \textbf{1.546} & 4.42 & \textbf{0.380}  & 1.83 & 0.046 & 2.12 & 0.085  \\ \hline
\end{tabular}
\end{center}
\end{table}
%%%%%%%%%%%%%%%%%%%%%%%%%%%%%%%%%%%%%%%%%%%%%%%%%%%%%%%%%%%%


%%%%%%%%%%%%%%%%%%%%%%%%%%%%%%%%%%%%%%%%%%%%%%%%%%%%%%%%%%%%%
\begin{table}[htb]
\caption{Number of background and signal events in the Run-1 SR1b normalized to 3\ifb\ and corresponding signal significance for two
signal models and different scenarios for the lepton isolation. The largest significance values is highlighted in bold.}
\label{tab:isoOpt2}
\begin{center}
\begin{tabular}{|cc|c|cc|cc|cc|cc|}
\hline
  &  &  & \multicolumn{2}{c|}{SMGG2WWZZ}  & \multicolumn{2}{c|}{Gtt} \\ 
  &  &   & \multicolumn{2}{c|}{1000\_700\_400}  & \multicolumn{2}{c|}{G1300\_L100} \\ \hline
ele  & mu & BKG & $n_{sig}$ & Signif & $n_{sig}$ & Signif \\ \hline
L  & L & 11.82 & 4.68 & 0.671 & 1.63 & 0.114\\  
M  & L & 9.96 & 4.49 & 0.740   & 1.57 & 0.142 \\  
T  & L & 9.07 & 4.29 & 0.759   & 1.51 & 0.149 \\  
VT & L & 8.97 & 4.29 & 0.766   & 1.49 & 0.149 \\  
VVT& L & 8.37 & 3.90 & 0.724   & 1.42 & 0.147 \\  \hline
L  & T & 10.90 & 4.68 & 0.721  & 1.61 & 0.130\\  
M  & T & 9.04 & 4.49 & 0.801   & 1.56 & 0.162 \\  
T  & T & 8.15 & 4.29 & 0.827   & 1.49 & 0.171 \\  
VT & T & 8.06 & 4.29 & \textbf{0.834}   & 1.48 & 0.171 \\  
VVT& T & 7.35 & 3.90 & 0.803   & 1.41 & \textbf{0.174} \\  \hline
L  &VT & 13.46 & 4.49 & 0.565  & 1.57 & 0.076\\  
M  &VT & 10.69 & 4.29 & 0.660  & 1.52 & 0.114\\  
T  &VT & 9.01 & 4.10 & 0.723   & 1.45 & 0.138 \\  
VT &VT & 8.24 & 4.10 & 0.777   & 1.44 & 0.155 \\  
VVT&VT & 7.16 & 3.71 & 0.772   & 1.37 & 0.171 \\ \hline
\end{tabular}
\end{center}
\end{table}
%%%%%%%%%%%%%%%%%%%%%%%%%%%%%%%%%%%%%%%%%%%%%%%%%%%%%%%%%%%%

\clearpage
\section{Trigger efficiencies}
\label{app_trigger}

Studies on performance and efficiency have been done for several single-lepton, dilepton and \met\ triggers Results are shown here.

\begin{figure}[h!]
\centering
\subfigure{\includegraphics[width=0.45\textwidth]{TRIGGER/Eff_HLT_e60_medium}}
\subfigure{\includegraphics[width=0.45\textwidth]{TRIGGER/Eff_HLT_e24_tight_iloose}}
\caption{Trigger efficiencies for \texttt{HLT\_e60\_medium} and \texttt{HLT\_e24\_tight\_iloose} versus \pt\ of the leading electron.}
\label{fig:eff_trigger_el1}
\end{figure}

\begin{figure}[h!]
\centering
\subfigure{\includegraphics[width=0.45\textwidth]{TRIGGER/Eff_HLT_2e12_loose}}
\subfigure{\includegraphics[width=0.45\textwidth]{TRIGGER/Eff_HLT_2e12_loose2}}
\caption{Trigger efficiency for \texttt{HLT\_2e12\_loose\_L12EM10VH} plotted against \pt\ of the leading electron (left-hand side) and the subleading electron (right-hand side).}
\label{fig:eff_trigger_el2}
\end{figure}


\begin{figure}[h!]
\centering
\subfigure{\includegraphics[width=0.45\textwidth]{TRIGGER/Eff_HLT_xe60}}
\subfigure{\includegraphics[width=0.45\textwidth]{TRIGGER/Eff_HLT_xe70}}
\caption{Trigger efficiencies for \texttt{HLT\_xe60} and \texttt{HLT\_xe70} versus the missing transverse energy.}
\label{fig:eff_trigger_met}
\end{figure}


\begin{figure}[h!]
\centering
\subfigure{\includegraphics[width=0.6\textwidth]{TRIGGER/Comparison_Met}}
\caption{Events selected by different \met\ triggers and without any trigger selection versus the missing transverse energy.}
\label{fig:dummy_label}
\end{figure}

\begin{figure}[h!]
\centering
\subfigure{\includegraphics[width=1\textwidth]{TRIGGER/Comparison_e2D}}
\caption{Leading electron versus subleading electron for events selected by \texttt{HLT\_e20\_medium} (left-hand side) and by \texttt{HLT\_e60\_medium} (right-hand side).}
\label{fig:eff_trigger_el2D}
\end{figure}


\input{appendix_3l3b}

\end{document}
