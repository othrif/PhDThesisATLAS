
%A few general selection criteria based on data quality, detector quality flags are the following:
\par A sample of two same-sign or three leptons is selected applying the following criteria:
\begin{itemize}
%\item{ \textbf{GRL}: The used GoodRunsList, providing a total
%    luminosity of $\lumi$~\ifb, was \newline
%\small{\texttt{data12\_8TeV.periodAllYear\_DetStatus-v58-pro14-01\_DQDefects-00-00-33\_PHYS\_StandardGRL\_All\_Good.xml}}.}
%\item{ \textbf{Trigger}: The data have been collected using lepton and \met\ triggers, as described in Section \ref{section:trigger}.}
%\item{ \textbf{LAr and Tile Error}: Impact from the noise bursts and data corruption in the LAr and
% Tile calorimeter is reduced by selecting events without an error from
% the quality checks (larError!=2 and tileError!=2). } 
%\item{ \textbf{Incomplete events}: Incomplete events due to the TTC restart procedure are rejected.}

%\item{ \textbf{Fake \met\ Veto}: Events with fake \met\ due to jets
%    pointing to dead TileCal and HEC regions are rejected. This is
%    achieved by vetoing events with at least one
%    jet~\footnote{Anti-k$_{\rm T}$ jets with $R=0.4$ before
%      electron-jet overlap removal are considered for the veto. } satisfying:
%    \newline 
%$\pt > 40\,\GeV\ \&\&\ C_{jet}  > 0.05\ \&\&\ \Delta\phi (\met,jet) <
%    0.3$ \newline where  $C_{jet}$ is the estimated correction factor  
%    for the energy losses estimated using the jet profile. The veto is applied
%    to all events in data and simulation.}
    
\item{ \textbf{Jet Cleaning}: Events are required to pass the
  {\tt{VeryLooseBad}} set of cleaning requirements recommended by Jet-\met\ group \cite{jetcctwiki} 
  and implemented in the {\tt{JetCleaningTool}}. 
An event is rejected if at least one of the pre-selected jets (thus
after jet-electron overlap removal) fails the jet quality criteria. 
The cleaning requirements are intended to remove events where significant energy was deposited in
  the calorimeters due to instrumental effects such as cosmic rays,
  beam-induced (non-collision) particles, and noise. } 
  
\item{ \textbf{Primary Vertex}: events are required to have a primary vertex, that is, one of the reconstructed vertices
   must be labeled as {\tt{xAOD::VxType::PriVtx}}. No additional requirements on the number of tracks in the 
   vertex should be applied as recommended by the Tracking CP group~\cite{vertex_twiki}.
}

\item{ \textbf{Bad Muon Veto}:  Events containing at least one pre-selected muon satisfying $\sigma(q/p)/|q/p| >$ 0.2 before the overlap removal are rejected.}

\item{ \textbf{Cosmic Muon Veto}: Events containing a cosmic muon candidate are rejected. Cosmic muon candidates are selected among pre-selected muons, if they fail the requirements $|z_0| <$ 1.0\,mm and $|d_0| <$ 0.2\,mm, where the longitudinal and transverse impact paramaters $z_0$ and $d_0$ are calculated with respect to the primary vertex.}

\item{ \textbf{At least two leptons}: Events are required to contain
    at least two signal leptons (see Section~\ref{sec:objects} and
    Table \ref{tab:lepdef}) with $p_{\mathrm{T}}
    > 20\,\GeV$ and $p_{\mathrm{T}} > 15\,\GeV$ for the leading and subleading lepton, respectively. 
    If the lepton contains a third lepton with $p_{\mathrm{T}} > 15\,\GeV$ the event is regarded as three-lepton
    event otherwise as a two-lepton event. Once the trigger is fully operative in the MC samples, 
    one or two of the three leptons will be 
    matched to the trigger object
    if the event is classified as passing a single or dilepton trigger.
    The data sample obtained is then divided into three channels depending on the flavor of the two 
    leading leptons ($ee$, $\mu\mu$, $e\mu$).}

\item \textbf{Electron-muon overlap}: For any events containing at least one of
  each of a signal electron and muon, the event is vetoed if $\Delta R(e, \mu) < 0.1$.

\item{ \textbf{Same sign}: If the event is a two-lepton event the two leading leptons have to be of 
    same charge (same-sign).}
\item{ \textbf{Z-veto}: If the event is a three-lepton event all possible opposite-charge same-flavor lepton
    combinations are combined to calculate a mass $M_{\ell\ell}$. The event is rejected if one of the calculated mass
    is close to the $Z$-mass, i.e.~$84 < M_{\ell\ell} <  98\,\mathrm{GeV}$.}
\item{ \textbf{Invariant mass}: Events with $M_{\ell\ell} < 12\,\GeV$ are
    rejected to avoid heavy flavor meson resonances. In three-lepton events, the invariant mass $M_{\ell\ell}$ is calculated from the 
    two leading leptons. }
\end{itemize}


The following event variables are also used in the definition of the signal and validation regions in the analysis:

\begin{itemize}
\item The inclusive effective mass \meff~ defined as the scalar sum of
  the signal leptons \pt (see Table~\ref{tab:lepdef}), all signal jets \pt\ (see Table~\ref{tab:jetsdef}) and \met;
\item The transverse mass \mt\ computed from the leading lepton and
\met\ as \newline $\mt = \sqrt{2 \cdot \pt^l \cdot \met \cdot (1 - cos(\Delta \phi(l, \met )))}$;
\end{itemize}