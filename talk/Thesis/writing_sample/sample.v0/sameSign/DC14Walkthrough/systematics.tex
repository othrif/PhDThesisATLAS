
\subsection{Experimental systematics}
\label{sec:syst_exp}

All the  experimental systematics provided by the SUSYTools {\tt getSystInfoList()} method have been considered. 
The list of sources of uncertainty and the corresponding names of the variations are:

\paragraph{Jet energy scale ({\tt{JET\_GroupedNP\_\{1-3\}\_\_1\{up,down\}}})}  
One of the strongly reduced uncertainty sets provided by the JetEtMiss group for early Run-2 searches is used in this note. These sets are intended for use by analyses which are not sensitive to jet-by-jet correlations arising from changes to the jet energy scale (as expected for many early SUSY searches). Although the recommendation is to evaluate 4 different reduced sets, only one of them is used for the moment: {\tt InsituJES2012\_3NP\_Scenario1.config} (as included in the JetUncertainties-00-09-19 package tag).
The jet energy is scaled up and down (in a fully correlated way) by the $\pm 1\sigma$ uncertainty of each nuisance parameter.

\paragraph{Jet energy resolution ({\tt{JER\_\_1up}})}  An extra $\pt$ smearing is added to the jets based on their $\pt$ and $\eta$ to account for a possible underestimate of the jet energy resolution in the MC simulation. This is done by the {\tt JERSmearingTool} in the JetResolution-03-00-28 package tag.

\paragraph{Egamma resolution ({\tt{EG\_RESOLUTION\_ALL\_\_1\{up,down\}}})} A nuisance filtering scheme to reduce the $\sim$8 NPs used to electron and photon resolution to only one as implemented in SUSYTools is used.

\paragraph{Egamma scale ({\tt{EG\_SCALE\_ALL\_\_1\{up,down\}}})}
A nuisance filtering scheme to reduce the $\sim$16 NPs used to electron and photon resolution to only one as implemented in SUSYTools is used.

\paragraph{Electron efficiency ({\tt{EL\_EFF\_UncorrUncertainty\_\_1\{up,down\}}})} 
This is uncertainty is associated with the electron efficiency scale factors provided by the Egamma CP group.

\paragraph{Muon efficiency ({\tt{MUONSFSTAT\_\_1\{up,down\}}}, {\tt{MUONSFSYS\_\_1\{up,down\}}})}
This is uncertainty corresponds to the statiscal and systematic uncertainties in the muon efficiency scale factors provided by the Muon CP group.

\paragraph{Muon resolution uncertainty  ({\tt{MUONS\_ID\_\_1\{up,down\}}}, {\tt{MUONS\_MS\_\_1\{up,down\}}})} 
This is evaluated as variations in the smearing of the inner detector and muon spectrometer tracks associated to the muon objects by $\pm 1\sigma$ their uncertainty

\paragraph{Muon momentum scale ({\tt{MUONS\_SCALE\_\_1\{up,down\}}})} 
This is evaluated as variations in the scale of the momentum of the muon objects

\paragraph{\met\ calorimeter soft term uncertainties  ({\tt{MET\_SoftCalo\_Reso}}, {\tt{MET\_SoftCalo\_Scale\{Up,Down\}}})}
Note that the effect of the hard object uncertainties (most notably JES and JER) are also propagated to the $\met$.

\paragraph{Flavor tagging ({\tt{FT\_Eigen\_B\_\{0-9\}\_\_1\{up,down\}}}, {\tt{FT\_Eigen\_C\_\{0-3\}\_\_1\{up,down\}}},\\ {\tt{FT\_Eigen\_Light\_\{0-11\}\_\_1\{up,down\}}})} 
The full eigenvector variation approach~\cite{flavTagg} is used for all jet flavours ($B$, $C$ and light), while keeping track of uncertainties that are common to multiple jet flavours. Similarly to the case of the JES, a significant reduction in the number of nuisance parameters (currently 52 variations) is expected to be provided by the Flavour Tagging CP group before the start of data taking.\\

All these experimental systematics have been included in the production of flat ROOT ntuples, as mentioned in Section~\ref{sec:ntuples}, although due to time limitations at the moment of writing this note, its effect was not included 
in the results shown in Section~\ref{sec:fit}, where flat uncertainty values were used instead.

\subsection{Theoretical systematics}
\label{sec:syst_theo}

In Run-1~\cite{paperSS3L,noteSS3L}, the systematic uncertainties associated with the $t \bar{t} + V$ and
diboson MC were assessed by using samples where the factorization and
renormalization scales have been varied. We plan to follow a similar approach once 
scale variation samples are available in mc15 productions.
Other theoretical uncertainties were evaluated by comparing different generators and samples produced with different
number of partons. A similar approach will be used in 2015, following the recommendations by the Physics Modeling Group.

In addition, the theoretical uncertainties on the cross-sections were $22\%$ for $t\bar{t}W$ \cite{Campbell:2012dh} 
and $t\bar{t}Z$ \cite{Garzelli:2012bn} and $7\%$ for 	diboson production (computed with MCFM \cite{Campbell:2011bn}, 
considering scales, parton distribution functions and $\alpha_s$ uncertainties). 
Normalisation uncertainties between 35\% and 100\% were applied to processes with smaller contributions (triboson 
production, $t+Z$, etc.).

