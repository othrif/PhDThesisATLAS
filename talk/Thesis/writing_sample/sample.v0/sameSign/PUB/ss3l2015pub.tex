\newcommand*{\ATLASLATEXPATH}{latex/}
\documentclass[UKenglish,texlive=2013,PAPER,coverpage]{\ATLASLATEXPATH atlasdoc}
% The following command is needed by arXiv to ensure use of pdflatex.
% It should be included in the first 5 lines of the preamble.
% \pdfoutput=1
%  cernpreprint          Create front page for a CERN preprint.
%                        See atlas-preprint-cover.tex for a list of variables that should be defined.
%  txfonts=true|false    Use txfonts rather than the default newtx - needed for arXiv submission.
\usepackage[subcaption]{\ATLASLATEXPATH atlaspackage}
\usepackage{\ATLASLATEXPATH atlasprocess}
\usepackage{\ATLASLATEXPATH atlasmisc}
\usepackage{\ATLASLATEXPATH atlasbiblatex}
\usepackage{\ATLASLATEXPATH atlascontribute}
\usepackage{\ATLASLATEXPATH atlasphysics}
\usepackage{\ATLASLATEXPATH atlasbsm}

\addbibresource{newSUSYRefs.bib}
\addbibresource{ss3l2015pub.bib}
\addbibresource{bibtex/bib/ATLAS.bib}
\addbibresource{bibtex/bib/PubNotes.bib}
\addbibresource{bibtex/bib/ConfNotes.bib}

\graphicspath{{logos/}{FIGURES/}}

\usepackage{ss3l2015pub-defs}
\usepackage[normalem]{ulem}
\usepackage{xparse}
\captionsetup[subfigure]{labelformat=simple}
\renewcommand*{\thesubfigure}{(\alph{subfigure})}

%-------------------------------------------------------------------------------
% This file contains the title, author and abstract.
% It also contains all relevant document numbers used by the different cover pages.
%-------------------------------------------------------------------------------

% Title
\AtlasTitle{Search for supersymmetry at $\sqrt{s}=13$~TeV in final states with jets and two same-sign leptons or three leptons with the ATLAS detector}

% Author - this does not work with revtex (add it after \begin{document})
\author{The ATLAS Collaboration}

% Authors and list of contributors to the analysis
% \AtlasAuthorContributor also adds the name to the author list
% Include package latex/atlascontribute to use this
% Use authblk package if there are multiple authors, which is included by latex/atlascontribute
% \usepackage{authblk}
% \renewcommand\Authands{, } % avoid ``. and'' for last author
% \renewcommand\Affilfont{\itshape\small} % affiliation formatting
% \AtlasAuthorContributor{First AtlasAuthorContributor}{a}{Author's contribution.}
% \AtlasAuthorContributor{Second AtlasAuthorContributor}{b}{Author's contribution.}
% \AtlasAuthorContributor{Third AtlasAuthorContributor}{a}{Author's contribution.}
% \AtlasContributor{Fourth AtlasContributor}{Contribution to the analysis.}
% \author[a]{First Author}
% \author[a]{Second Author}
% \author[b]{Third Author}
% \affil[a]{One Institution}
% \affil[b]{Another Institution}

% If a special author list should be indicated via a link use the following code:
% Include the two lines below if you do not use atlasstyle:
% \usepackage[marginal,hang]{footmisc}
% \setlength{\footnotemargin}{0.5em}
% Use the following lines in all cases:
% \usepackage{authblk}
% \author{The ATLAS Collaboration%
% \thanks{The full author list can be found at:\newline
%   \url{https://atlas.web.cern.ch/Atlas/PUBNOTES/ATL-PHYS-PUB-2014-007/authorlist.pdf}}
% }

\AtlasVersion{3.2}

% ATLAS reference code, to help ATLAS members to locate the paper
\AtlasRefCode{SUSY-2015-09}
%\AtlasRefCode{ATLAS-CONF-2015-078}


% CERN preprint number
% \PreprintIdNumber{CERN-PH-2014-XX}

% ATLAS date - arXiv submission; to be filled in by the Physics Office
% \AtlasDate{\today}

% arXiv identifier
% \arXivId{14XX.YYYY}

% HepData record
% \HepDataRecord{ZZZZZZZZ}

% Submission journal and final reference
\AtlasJournal{Eur.\ Phys.\ J.\ C}
% \AtlasJournalRef{\PLB 789 (2014) 123}
% \AtlasDOI{}

% Abstract - % directly after { is important for correct indentation
\AtlasAbstract{A search for strongly produced supersymmetric particles is conducted using signatures involving multiple energetic jets and either two isolated leptons ($e$ or $\mu$) with the same electric charge or at least three isolated leptons. The search also utilises $b$-tagged jets, missing transverse momentum and other observables to extend its sensitivity. The analysis uses a data sample of proton--proton collisions at $\sqrt{s}= 13$~TeV recorded with the ATLAS detector at the Large Hadron Collider in 2015 corresponding to a total integrated luminosity of 3.2~fb$^{-1}$. No significant excess over the Standard Model expectation is observed.  
%and the results are interpreted in several simplified supersymmetric models extending the exclusion limits from previous searches. 
The results are interpreted in several simplified supersymmetric models and extend the exclusion limits from previous searches.
In the context of exclusive production and simplified decay modes, gluino masses are excluded at $95\%$ confidence level 
up to 1.1--1.3~TeV for light neutralinos (depending on the decay channel), 
and bottom squark masses are also excluded up to 540 GeV. 
In the former scenarios, neutralino masses are also excluded up to 550-850 GeV for gluino masses around 1 TeV. 
}

%-------------------------------------------------------------------------------
% The following information is needed for the cover page. The commands are only defined
% if you use the coverpage option in atlasdoc or use the atlascover package
%-------------------------------------------------------------------------------

% List of supporting notes  (leave as null \AtlasCoverSupportingNote{} if you want to skip this option)
\AtlasCoverSupportingNote{Supporting note}{https://cds.cern.ch/record/2052581}
% \AtlasCoverSupportingNote{Short title note 2}{https://cds.cern.ch/record/YYYYYYY}
%
% OR (the 2nd option is deprecated, especially for CONF and PUB notes)
%
% Supporting material TWiki page  (leave as null \AtlasCoverTwikiURL{} if you want to skip this option)
% \AtlasCoverTwikiURL{https://twiki.cern.ch/twiki/bin/view/Atlas/WebHome}

% Comment deadline
\AtlasCoverCommentsDeadline{February 25th, 2016}

% Analysis team members - contact editors should no longer be specified
% as there is a generic email list name for the editors
\AtlasCoverAnalysisTeam{B.~Abbott, J.-F.~Arguin, S.~Berlendis, F.~Cardillo, G.~Carrillo-Montoya, A.~Di~Simone, O.~Ducu, D. Gerbaudo, F.~Hubaut, G.~Herten, S.~Kahn, A.~Lleres, J.~Maurer, E.~Monnier, T.~Nguyen, A.~Paramonov, J.~Poveda, P.~Pralavorio, H.~Ren, O.~Rifki, Y.-T.~Shen, P.~Skubic, A.~Taffard, P.~Tornamb\'{e}, H.~Tr\'{e}panier, X.~Zhuang}

% Editorial Board Members - indicate the Chair by a (chair) after his/her name
% Give either all members at once (then they appear on one line), or separately
\AtlasCoverEdBoardMember{Jamie Boyd~(chair), Sara~Strandberg, Kerim~Suruliz}
% \AtlasCoverEdBoardMember{EdBoard~Chair~(chair)}
% \AtlasCoverEdBoardMember{EB~Member~1}
% \AtlasCoverEdBoardMember{EB~Member~2}
% \AtlasCoverEdBoardMember{EB~Member~3}

% A PUB note has readers and not an EdBoard -- give their names here (one line or several entries)
% \AtlasCoverReaderMember{Reader~1, Reader~2}
% \AtlasCoverReaderMember{Reader~1}
% \AtlasCoverEdBoardMember{Reader~2}

% Editors egroup
\AtlasCoverEgroupEditors{atlas-SUSY-2015-09-editors@cern.ch} % was atlas-conf-susy-2015-06-editors@cern.ch

% EdBoard egroup
\AtlasCoverEgroupEdBoard{atlas-SUSY-2015-09-editorial-board@cern.ch} % was atlas-conf-susy-2015-06-editorial-board@cern.ch




%-------------------------------------------------------------------------------
% Content
%-------------------------------------------------------------------------------
\begin{document}

\maketitle

%-------------------------------------------------------------------------------
\section{Introduction}
\label{sec:intro}
%-------------------------------------------------------------------------------

Supersymmetry (SUSY)~\cite{Golfand:1971iw,Volkov:1973ix,Wess:1974tw,Wess:1974jb,Ferrara:1974pu,Salam:1974ig} is one of the most studied frameworks to extend the Standard Model (SM) beyond the electroweak scale; a general review can be found in Ref.~\cite{Martin:1997ns}. 
In its minimal realisation (MSSM)~\cite{Fayet:1976et,Fayet:1977yc} 
it predicts a new bosonic (fermionic) partner for each fundamental SM fermion (boson), 
as well as an additional Higgs doublet. 
If $R$-parity is conserved~\cite{Farrar:1978xj} the lightest supersymmetric particle (LSP) is stable and is typically the lightest neutralino\footnote{The SUSY partners of the Higgs and electroweak gauge bosons mix to form
the mass eigenstates known as charginos ($\tilde{\chi}^{\pm}_{l}$, $l = 1, 2$ ordered by
increasing mass) and neutralinos ($\tilde{\chi}^{0}_{m}$, $m = 1, \ldots, 4$ ordered by increasing
mass).} \ninoone. 
Only such scenarios are considered in this paper. 
In many models, the LSP can be a viable dark matter candidate~\cite{Goldberg:1983nd,Ellis:1983ew} 
and produce collider signatures with large missing transverse momentum. 


In order to address the SM hierarchy problem with SUSY models
~\cite{Sakai:1981gr,Dimopoulos:1981yj,Ibanez:1981yh,Dimopoulos:1981zb}, 
TeV-scale masses are required~\cite{Barbieri:1987fn,deCarlos:1993yy} 
for the partners of the gluons (gluinos~\gluino) 
and of the top quark chiral degrees of freedom (top squarks \stopL and \stopR), 
due to the large top Yukawa coupling. 
The latter also favours significant $\stopL$--$\stopR$ mixing, 
so that the lighter mass eigenstate $\stopone$ is in many scenarios lighter than the other squarks~\cite{Inoue:1982pi,Ellis:1983ed}. 
Bottom squarks may also be light, being bound to top squarks by $SU(2)_{\rm L}$ invariance. 
This leads to potentially large production cross-sections 
for gluino pairs ($\gluino\gluino$), top--antitop squark pairs ($\stopone\stoponebar$) and bottom--antibottom squark pairs ($\sbottomone\sbottomonebar$) at the Large Hadron Collider (LHC)~\cite{Borschensky:2014cia}. 
Production of isolated leptons may arise in the cascade decays of those superpartners to SM quarks and neutralinos~\ninoone, 
via intermediate neutralinos~$\tilde\chi^0_{2,3,4}$ or charginos~$\tilde\chi^\pm_{1,2}$ 
that in turn lead to $W$, $Z$ or Higgs bosons, or to lepton superpartners (sleptons). 
Lighter third-generation squarks would also enhance $\gluino\to t\stoponebar$ or $\gluino\to b\sbottomonebar$ branching ratios over the generic decays involving light-flavour squarks,
favouring the production of heavy flavour quarks and, in the case of top quarks, additional leptons. 

This paper 
presents a search for SUSY in final states 
with two leptons (electrons or muons) of the same electric charge (referred to as same-sign (SS) leptons) or three leptons (3L) in any charge combination, 
jets and missing transverse momentum (${\vec p}^{\rm miss}_{\rm T}$, whose magnitude is referred to as \met). 
It is an extension to $\sqrt s=13$ TeV of an earlier search performed by ATLAS with $\sqrt s=8$ TeV data~\cite{paperSS3L}, 
and uses the data collected by the ATLAS experiment~\cite{PERF-2007-01} in proton--proton ($pp$) collisions during 2015. 
Despite the much lower integrated luminosity collected at $\sqrt s=13$~TeV compared to that collected at $\sqrt s=8$~TeV, 
a similar or improved sensitivity at $\sqrt{s}=13$~TeV is expected 
due to the much larger cross-sections predicted for the production of SUSY particles with masses at the TeV scale. 
A similar search for SUSY in this topology was also performed by the CMS Collaboration~\cite{Chatrchyan:2013fea} at $\sqrt s=8$~TeV.
While the same-sign leptons signature is present in many scenarios of physics beyond the SM (BSM), 
SM processes leading to such final states have very small cross-sections. 
Compared to many other BSM searches, analyses based on same-sign leptons therefore allow 
the use of looser kinematic requirements (for example, on \met or the momentum of jets and leptons), 
preserving sensitivity to scenarios with small mass differences between gluinos/squarks and the LSP, or in which $R$-parity is not conserved~\cite{paperSS3L}. 

The sensitivity to a wide range of models is illustrated by the interpretation of the results  
in the context of four different SUSY benchmark processes that may lead to same-sign or three-lepton signatures. 
The first two scenarios focus on gluino pair production with generic decays into light quarks and multiple leptons, 
either involving light sleptons, $\gluino\to q\bar q\ninotwo\to q\bar q \ell\slepton^*\to q\bar q\ell^+\ell^-\ninoone$ (Fig.~\ref{fig:feynman_gg2sl}), 
or mediated by a cascade involving $\chinoonepm$ and $\ninotwo$, 
$\gluino \to q\bar q'\chinoonepm\to q\bar q'W^\pm\ninotwo\to q\bar q'W^\pm Z\ninoone$  (Fig.~\ref{fig:feynman_gg2WZ}). 
The other two scenarios are motivated by the expectation that the third-generation squarks are lighter than the other squarks 
and target the direct production of $\sbottomone\sbottomonebar$ pairs 
with subsequent chargino-mediated $\sbottomone\to tW^-\ninoone$ decays (Fig.~\ref{fig:feynman_b1b1})  
or the production of $\gluino\gluino$ pairs decaying as $\gluino\to t\bar t\ninoone$ via an off-shell top squark (Fig.~\ref{fig:feynman_gtt}). 

\begin{figure}[t!]
\centering
\begin{subfigure}[t]{0.24\textwidth}\includegraphics[width=\textwidth]{FIGURES/gogo-qqqqllllN1N1-N2}\caption{}\label{fig:feynman_gg2sl}
\end{subfigure}
\begin{subfigure}[t]{0.24\textwidth}\includegraphics[width=\textwidth]{FIGURES/gogo-qqqqWWZZN1N1-C1N2}\caption{}\label{fig:feynman_gg2WZ}\end{subfigure}
\begin{subfigure}[t]{0.24\textwidth}\includegraphics[width=\textwidth]{FIGURES/sbsb-ttWWN1N1}\caption{}\label{fig:feynman_b1b1}\end{subfigure}
\begin{subfigure}[t]{0.24\textwidth}\includegraphics[width=\textwidth]{FIGURES/gogo-ttttN1N1}\caption{}\label{fig:feynman_gtt}\end{subfigure}
\caption{SUSY processes featuring gluino (a, b, d) or bottom squark (c) pair production considered in this analysis.}
\label{fig:feynman}
\end{figure}

Four signal regions (SRs) are designed to achieve good sensitivity for these SUSY scenarios, mainly characterised by the number of $b$-tagged jets or reconstructed leptons. 
They are detailed in Section~\ref{sec:selection}, preceded by descriptions of the experimental apparatus~(Section~\ref{sec:detector}) and
the simulated samples~(Section~\ref{sec:dataMC}).
Section~\ref{sec:bkg} is devoted to the estimation of the
contribution from SM processes to the signal regions, 
validated by comparisons with data in dedicated regions. 
The results are presented in Section~\ref{sec:result}, 
together with the statistical tests used to interpret the results in the context of the SUSY benchmark scenarios. 
Finally, Section~\ref{sec:conclusion} summarises the main conclusions of this paper.

%-------------------------------------------------------------------------------
\section{The ATLAS detector}
\label{sec:detector}
%-------------------------------------------------------------------------------

The ATLAS experiment~\cite{PERF-2007-01} is a multi-purpose particle detector with a forward-backward symmetric cylindrical
geometry and nearly $4\pi$ coverage in solid angle.\footnote{ATLAS uses
  a right-handed coordinate system with its origin at the nominal
  interaction point (IP) in the centre of the detector and the
  $z$-axis along the beam pipe. The $x$-axis points from the IP to the
  centre of the LHC ring, and the $y$-axis points upward. Cylindrical
  coordinates ($r$, $\phi$) are used in the transverse plane, $\phi$
  being the azimuthal angle around the beam pipe. The pseudorapidity
  is defined in terms of the polar angle $\theta$ as $\eta = -\ln
  \tan(\theta/2)$. Rapidity is defined as $y=0.5 \ln\left[(E + p_z )/(E - p_z )\right]$ where $E$ denotes the energy and $p_z$ is the component of the momentum along the beam direction.}
The interaction point is surrounded by an inner detector (ID), a
calorimeter system, and a muon spectrometer.

The ID provides precision tracking of charged particles for
pseudorapidities $|\eta| < 2.5$ and is surrounded by a superconducting solenoid providing a \SI{2}{T} axial magnetic field.
It consists of pixel and silicon-microstrip detectors inside a
transition radiation tracker.  
One significant upgrade for the $\sqrt{s}=13$~TeV running period is the presence of the
Insertable B-Layer~\cite{CERN-LHCC-2010-013}, an additional pixel layer close to the interaction point, which 
provides high-resolution hits at small radius to improve the tracking performance.

In the pseudorapidity region $|\eta| < 3.2$, high-granularity lead/liquid-argon (LAr)
electromagnetic (EM) sampling calorimeters are used.
A steel/scintillator tile calorimeter measures hadron energies for
$|\eta| < 1.7$.
The endcap and forward regions, spanning $1.5<|\eta| <4.9$, are 
instrumented with LAr calorimeters for both the EM and hadronic
measurements.

The muon spectrometer consists of three large superconducting toroids
with eight coils each, 
a system of trigger and precision-tracking chambers, 
which provide triggering and tracking capabilities in the
ranges $|\eta| < 2.4$ and $|\eta| < 2.7$, respectively.

A two-level trigger system is used to select events. The first-level
trigger is implemented in hardware and uses a subset of the detector
information. This is followed by the software-based High-Level Trigger stage,
which can run offline reconstruction and
calibration software, reducing the event rate to about \SI{1}{kHz}.



%-------------------------------------------------------------------------------
\section{Dataset and simulated event samples}
\label{sec:dataMC}
%-------------------------------------------------------------------------------

\input{dataMC}

%-------------------------------------------------------------------------------
\section{Event selection}
\label{sec:selection}
%-------------------------------------------------------------------------------

Candidate events are required to have a reconstructed vertex~\cite{ATL-PHYS-PUB-2015-026}, 
with at least two associated tracks with $\pt >400$~MeV, 
and the vertex with the largest $\Sigma \pt^2$ of the associated tracks is chosen as the primary vertex of the event.

For the data-driven background estimations, two categories of electrons and muons are used: 
``candidate'' and ``signal'' (the latter being a subset of the ``candidate'' leptons satisfying tighter selection criteria). 
Electron candidates are reconstructed from energy depositions in the electromagnetic calorimeter that have been matched to an ID track 
and are required to have $|\eta|<2.47$, a transverse momentum $\pT>\SI{10}{\GeV}$,
and to pass a Loose likelihood-based identification requirement~\cite{ATLAS-CONF-2016-024}. 
The likelihood input variables include measurements of calorimeter shower shapes and track properties from the ID. 
Candidates within the transition region between the barrel and endcap electromagnetic calorimeters,
$1.37<|\eta|<1.52$, are removed. The track matched with the electron must have a significance of the transverse impact parameter 
with respect to the reconstructed primary vertex, $d_0$, of $\vert d_0\vert/\sigma(d_0) < 5$.
Muon candidates are reconstructed in the region $|\eta|<2.5$ 
from muon spectrometer tracks matching ID tracks.
All muon candidates must have $\pT>\SI{10}{\GeV}$ and must pass the Medium identification requirements~\cite{Aad:2016jkr}, 
based on selections on the number of hits in the different ID and muon spectrometer subsystems, 
and the significance of the charge-to-momentum ratio $q/p$~\cite{Aad:2016jkr}.

Jets are reconstructed with the anti-$k_t$
algorithm~\cite{Cacciari:2008} with radius parameter $R=0.4$, using three-dimensional topological energy
clusters in the calorimeter~\cite{PERF-2014-07} as input. 
All jets must have $\pT>\SI{20}{\GeV}$ and $|\eta|<2.8$.
Jets are calibrated as described in Ref.~\cite{ATLAS-PHYS-PUB-2015-015}.
Furthermore, for all jets the expected average energy contribution from
pile-up is subtracted according to the jet area~\cite{ATLAS-PHYS-PUB-2015-015}.
In order to reduce the effects of pile-up, jets with $\pt<\SI{60}{GeV}$ and $|\eta|<2.4$ are required to have a significant fraction of their 
associated tracks originating from the primary vertex, 
as defined by the jet vertex tagger~\cite{ATLAS-CONF-2014-018}. 

Identification of jets containing $b$-hadrons ($b$-tagging) is performed with the MV2c10 algorithm, 
a multivariate discriminant making use of track impact parameters 
and reconstructed secondary vertices~\cite{Aad:2015ydr,ATL-PHYS-PUB-2015-022}.
A requirement is chosen corresponding to a 70\% average efficiency 
for tagging $b$-jets in simulated $\ttbar$ events. 
The rejection factors for light-quark/gluon jets, $c$-quark jets and hadronically decaying $\tau$ leptons in simulated $\ttbar$ events 
are approximately 380, 12 and 54, respectively~\cite{ATL-PHYS-PUB-2015-022,ATL-PHYS-PUB-2016-012}. 
Jets with $|\eta|<2.5$ which satisfy the $b$-tagging requirement are identified as $b$-jets. 
To compensate for differences between data and MC simulation in the $b$-tagging efficiencies and mis-tagging rates, 
correction factors derived in data control regions are applied to the simulated samples~\cite{ATL-PHYS-PUB-2015-022}. 

After object identification, overlaps between the different objects are resolved. 
Any jet within a distance $\Delta R_y =\sqrt{(\Delta y)^2+(\Delta\phi)^2}$ = 0.2 of a lepton candidate is discarded, 
unless the jet is $b$-tagged\footnote{In this case the $b$-tagging operating point corresponding to an efficiency of 85\% is used.},
in which case the lepton is discarded since it is likely originating from a semileptonic $b$-hadron decay. 
Any remaining lepton within $\Delta R_y= \operatorname{min}\{0.4, 0.1 + \SI{9.6}{GeV}/\pt(\ell)\}$ of a jet is discarded. 
However, if the jet has fewer than three associated tracks or the muon \pt exceeds 70\% of the sum of the transverse momenta 
of the associated tracks and 50\% of the \pt of the jet, the muon is retained and the jet is discarded. This avoids 
inefficiencies for high-energy muons undergoing significant energy loss in the calorimeter. 
%Any calo-tagged muon\footnote{A calo-tagged muon is a muon identified solely by calorimeter based identification 
%(so no signal in MS) and are relevant only at $|\eta|<0.1$.} sharing an ID track with an electron is removed. 
Finally, any electron sharing an ID track with the remaining muons is also removed.


Electrons that are likely to be reconstructed with an incorrect charge assignment are rejected using electron
cluster and track properties: the track impact parameter, the track curvature significance, the cluster width and the
quality of the matching between the cluster and its associated track, both in terms of energy and position. These
variables, as well as the electron pt and eta,
are combined into a single classifier using a boosted decision tree (BDT). A selection requirement
on the BDT output is chosen to achieve a rejection factor between 7 and 8 for electrons with a wrong charge
assignment while selecting properly measured electrons with an efficiency of 97\%. Correction
factors to account for differences of the selection efficiency between data
and MC have been applied to the selected electrons in the MC simulation.

Signal electrons must satisfy a Medium likelihood-based identification requirement~\cite{ATLAS-CONF-2016-024}.
In region with high amount of material in the tracker, an electron may emit a hard brems\-strah\-lung photon which subsequently 
converted to an electron--positron pair, causing its charge to be mis-measured (later refered to as ``charge-flip''). 
To reduce the impact of charge mis-identification, electrons must satisfy $|\eta|<2.0$. Furthermore, electrons that 
are likely to be reconstructed with an incorrect charge assignment are rejected using electron
cluster and track properties: the track impact parameter, the track curvature significance, the cluster width and the
quality of the matching between the cluster and its associated track, both in terms of energy and position. These
variables, as well as the electron pt and eta, are combined into a single classifier using a boosted decision tree (BDT). 
A selection requirement on the BDT output is chosen to achieve a rejection factor between 7 and 8 for electrons 
with a wrong charge assignment while selecting properly measured electrons with an efficiency of 97\%. Correction
factors to account for differences of the selection efficiency between data and MC have been applied to the selected 
electrons in the MC simulation.
%Correction factors to account for differences of the selection efficiency between data 
%and MC have been applied to the selected electrons in the MC simulation.
Signal muons must fulfil the requirement of $\vert d_0\vert/\sigma(d_0) < 3$. 
The track associated with the signal leptons must have a longitudinal impact parameter with respect to the 
reconstructed primary vertex, $z_0$, satisfying $\vert z_0 \sin\theta\vert  < 0.5$~mm. 
Isolation requirements are applied to both the signal electrons and muons. 
The  scalar sum of the \pt of tracks within a variable-size cone around the lepton, 
excluding its own track, must be less than 6\% of the lepton \pt. 
The track isolation cone radius for electrons (muons) 
$\Delta R_\eta=\sqrt{(\Delta\eta)^2+(\Delta\phi)^2}$ 
is given by the smaller of $\Delta R_\eta = \SI{10}{~GeV}/\pt$ and $\Delta R_\eta = 0.2\,(0.3)$. 
That is, a cone of size $0.2\,(0.3)$ at low $\pt$ but narrower for high-\pt leptons. 
In addition, in the case of electrons the energy of calorimeter clusters in a cone of $\Delta R_\eta = 0.2$ around the electron 
(excluding the deposition from the electron itself) must be less than 6\% of the electron \pt. 
Simulated events are corrected to account for minor differences in the lepton trigger, reconstruction, 
identification and isolation efficiencies between data and MC simulation.

The missing transverse momentum is defined as the negative vector sum of the transverse momenta 
of all identified physics objects (electrons, photons~\cite{ATLAS-CONF-2012-123}, muons, jets) and an additional soft term. 
The soft term is constructed from all tracks that are not associated with any physics object, 
but that are associated with the primary vertex. 
In this way, the $\met$ is adjusted for the best calibration of the jets and the other identified physics objects above, 
while maintaining pile-up independence in the soft term~\cite{ATL-PHYS-PUB-2015-027, ATL-PHYS-PUB-2015-023}.

Events are selected using a combination (logical OR) of dilepton and $\met$ triggers, 
the latter being used only for events with $\met>\SI{250}{GeV}$. 
The trigger-level requirements on $\met$ and the leading and subleading lepton \pt are looser than those applied offline 
to ensure that trigger efficiencies are constant in the relevant phase space. 
The event selection requires at least two signal leptons with $\pt>20$~\GeV~(apart from two SRs where 
the second lepton \pt should be greater than 10~\GeV, see Table~\ref{tab:SRdef3}). 
If the event contains exactly two signal leptons, they must have the same electric charge. 
Events are discarded if they contain any jet failing basic quality selection criteria in order to 
reject detector noise and non-collision background events~\cite{ATLAS-CONF-2010-038}.

To maximise the sensitivity to the signal models of Figure~\ref{fig:feynman}, 
nineteen overlapping signal regions are defined as shown in Table~\ref{tab:SRdef3}, with requirements on the number of signal leptons 
($N_{\ell}^{\rm{signal}}$), the number of $b$-jets with $\pt>\SI{20}{\GeV}$ ($N_{b\rm{-jets}}$), 
the number of jets above a given \pt thresholds (25, 40 or \SI{50}{GeV}) regardless of their flavour ($N_{\rm{jets}}$), 
\met, the effective mass (\meff), defined as the scalar sum of the $\pt$ of the signal leptons and jets (regardless of their flavour) 
in the event plus the \met, and the charge of the leptons in the event. 
The values of acceptance times efficiency of the SR selections for the SUSY signal models 
typically range between \textcolor{red}{[UPDATE]}\% for models with a light \ninoone and \textcolor{red}{[UPDATE]}\% for models with a heavy \ninoone. 
\textcolor{red}{UPDATE: Check that this sentence makes sense when we have all numbers}.


\begin{table}[tbh!]
%\rotatebox{90}
\resizebox{\textwidth}{!}
{
\hspace{0.5cm}
\def\arraystretch{1.2}
\centering
\small
\begin{tabular}{|c|c|c|c|c|c|c|c|c|c|c}
\hline
Signal region  &  $N_{\rm{leptons}}^{\rm{signal}}$   & $N_{b\rm{-jets}}$ & $N_{\rm{jets}}$  & $p^{}_{\rm{T,jet}}$ & \met\ & \meff\ & \met/\meff  & Other & Targeted  \\
 Name          &                                  &                   &                  &    [GeV]             & [GeV] & [GeV] &   &  & Signal  \\
\hline\hline

%(\textit{ex-SR3b1}) 
Rpc2L3bS         & $\ge 2$SS  & $\ge 3$ & $\ge 6$ & $>25$ & $>150$ & --  & $>0.2$    & - 			        & Fig.~\ref{fig:feynman_gtt}\\ 
%(\textit{ex-SR3b2})
Rpc2L3bH         & $\ge 2$SS  & $\ge 3$ & $\ge 6$ & $>25$ & $>250$ & $>1200$  & -	  & - 			        & Fig.~\ref{fig:feynman_gtt}, NUHM2\\ 
\hline
%(\textit{ex-SRhigh})
Rpc2Lsoft1b    & $\ge 2$SS  & $\ge 1$ & $\ge 6$ & $>25$ & $>100$ &          & $>0.3$    & 20,10 $<$\ptlone,\ptltwo $<$ 100 GeV & Fig.~\ref{fig:feynman_gttOffshell}\\ 
%(\textit{ex-SRlow})
Rpc2Lsoft2b      & $\ge 2$SS  & $\ge 2$ & $\ge 6$ & $>25$ & $>200$ & $>600$   & $>0.25$   & 20,10 $<$\ptlone,\ptltwo $<$ 100 GeV & Fig.~\ref{fig:feynman_gttOffshell} \\ 
\hline
%(\textit{ex-SR0b1})
Rpc2L0bS         & $\ge 2$SS  & $=0$    & $\ge 6$ & $>25$ & $>150$ & $>600$   & $>0.25$   & - 				& Fig.~\ref{fig:feynman_gg2WZ}\\
%(\textit{ex-SR0b2}) 
Rpc2L0bH         & $\ge 2$SS  & $=0$    & $\ge 6$ & $>40$ & $>250$ & $>900$   & -	  & - 				& Fig.~\ref{fig:feynman_gg2WZ}\\
\hline
%(\textit{ex-SR3L0b1})
Rpc3L0bS       & $\ge 3$    & $=0$    & $\ge 4$ & $>40$ & $>200$ &  --   & -	  & - 				& Fig.~\ref{fig:feynman_gg2sl}\\ 
%(\textit{ex-SR3L0b2})
Rpc3L0bH       & $\ge 3$    & $=0$    & $\ge 4$ & $>40$ & $>200$ & $>1600$  & -	  & - 				& Fig.~\ref{fig:feynman_gg2sl}\\
%(\textit{ex-SR3L1b1})
Rpc3L1bS       & $\ge 3$    & $\ge 1$ & $\ge 4$ & $>40$ & $>200$ & $>600$   & -         & - 				& Fig.~\ref{fig:feynman_gg2sl} ?? \\ 
%(\textit{ex-SR3L1b2})
Rpc3L1bH       & $\ge 3$    & $\ge 1$ & $\ge 4$ & $>40$ & $>200$ & $>1600$  & -	  & - 				& Fig.~\ref{fig:feynman_gg2sl} ?? \\
\hline
%(\textit{ex-SR1b1})
Rpc2L1bS         & $\ge 2$SS  & $\ge 1$ & $\ge 6$ & $>25$ & $>150$ & $>600$   & $>0.25$   & - 				& Fig.~\ref{fig:feynman_b1b1}\\
%(\textit{ex-SR1b2})
Rpc2L1bH         & $\ge 2$SS  & $\ge 1$ & $\ge 6$ & $>25$ & $>250$ & $>1250$  & $>0.2$    & - 				& Fig.~\ref{fig:feynman_b1b1}\\ 
\hline
%(\textit{ex-SR1b-3LSS})
Rpc3LSS1b    & $\ge \ell^\pm\ell^\pm\ell^\pm$ & $\ge 1$ & - & -    & -  & -        & -& veto 81$<$\mee$<$101 GeV 	& Fig.~\ref{fig:feynman_t1t1}\\ 
\hline
%(\textit{ex-SR1b-GG})
Rpv2L1bH       & $\ge 2$SS  & $\ge 1$ & $\ge 6$ & $>50$ & -      & $>2000$  & -         &  - 				& Figs.~\ref{fig:feynm_rpv_gl313},\ref{fig:feynm_rpv_gl321}\\
%(\textit{ex-SRRPV0b})
Rpv2L0b        & $=2$SS     & $=0$    & $\ge 6$ & $>40$ & -      & $>1800$  & -         &  veto 81$<$\mee$<$101 GeV 		& Fig.~\ref{fig:feynm_rpv_glprime} \\
%(\textit{ex-SRRPV3b})
Rpv2L3bH       & $\ge 2$SS  & $\ge 3$ & $\ge 6$ & $>40$ & -      & $>1800$  & -        & veto 81$<$\mee$<$101 GeV		& Fig.~\ref{fig:feynm_rpv_gl112} \\
%(\textit{ex-SR3b-DD})
Rpv2L3bS       & $\ge \ell^-\ell^-$   & $\ge 3$ & $\ge 3$ & $>50$ & -      & $>1200$   & -                           & & Fig.~\ref{fig:feynm_rpv_sd313}\\
%(\textit{ex-SR1b-DD-low})
Rpv2L1bS   & $\ge \ell^-\ell^-$  & $\ge 1$ & $\ge 4$ & $>50$ & -      & $>1200$  & -         & -  			        & Fig.~\ref{fig:feynm_rpv_sd321}\\
%(\textit{ex-SR1b-DD-high})
Rpv2L1bM  & $\ge \ell^-\ell^-$  & $\ge 1$ & $\ge 4$ & $>50$ & -      & $>1800$  & -         & - 			        & Fig.~\ref{fig:feynm_rpv_sd321}\\
\hline
\end{tabular}
}
\caption{Summary of the signal region definition. Unless explicitely stated in the table, at least two signal leptons with 
$\pt>$20 GeV and same charge (SS) are required in each signal region. Requirements 
are placed on the number of signal leptons ($N_{\rm{lept}}^{\rm{signal}}$), the number of jets ($N_{\rm{jets}}$) or the number 
of $b$-jets with $\pt>\SI{20}{GeV}$ ($N_{b\rm{-jets}}$), \pt of the lepton or jet, \met, \meff\ or \met/\meff. The last column indicates the 
targeted signal model.}
\label{tab:SRdef3}
\end{table}

%The SR3L1-SR3L2 and SR0b1-SR0b2 signal regions are sensitive to squarks of the first and second generations directly produced or appearing in gluino decays in RPC models, 
%leading to final states particularly rich in leptons (Fig.~\ref{fig:feynman_gg2sl}) or in jets (Fig.~\ref{fig:feynman_gg2WZ}), 
%but with no enhancement of the production of $b$-quarks. 
%Third-generation squark RPC models resulting in final states with two $b$-quarks, 
%such as direct bottom squark production (Fig.~\ref{fig:feynman_b1b1}), are targeted by the SR1b signal region. 
%Finally, the signal region SR3b targets production of gluinos decaying via a top squark resulting in final states with four $b$-quarks (Fig.~\ref{fig:feynman_gtt}). 

%Signal regions targeting RPV models contain no $\met$ requirement. 
%The SR1b-DD and SR3b-DD signal regions are sensitive to direct down squark production 
%with a $\lambda''_{321}$ (Fig.~\ref{fig:feynm_rpv_sd321}) and $\lambda''_{313}$ (Fig.~\ref{fig:feynm_rpv_sd313}) coupling, respectively. 
%Models featuring gluino production decaying via top squark including a $\lambda''_{313}$ (Fig.~\ref{fig:feynm_rpv_gl313}) or $\lambda''_{321}$ (Fig.~\ref{fig:feynm_rpv_gl321}) coupling are explored with the SR1b-GG signal region.



%-------------------------------------------------------------------------------
\section{Background estimation}
\label{sec:bkg}
%-------------------------------------------------------------------------------

The main challenge, in this analysis, is to achieve reliable predictions of the low Standard Model background 
leading to the same-sign leptons + jets final state. 
This background is composed partly of rare processes such as the associate production of a top quark pair with a massive boson, 
or the production of multiple bosons. 
The other contribution consists in experimental backgrounds originating from the imperfect discrimination between prompt leptons and other objects, 
or the occasional misreconstruction of the electron charge. 
The following sections provide more details about the nature of these different categories of background, 
and the foreseen methods that will be used to estimate their contributions to the signal regions. 

\subsection{Backgrounds with prompt SS dilepton or three leptons}
\label{sec:bkg_prompt}

There are two main sources of Standard Model background leading to pairs of same-sign prompt leptons: 
\begin{itemize}
\item[$\bullet$] The associate production of top quark(s) and massive bosons, where the same-sign leptons pair 
is produced from leptonic decays of one of the top quarks and of the boson. 
These processes are characterized by a large jet multiplicity, the presence of $b$-jet(s), 
and always have intrinsic missing momentum. 
Therefore they generally represent the largest contribution to the signal regions. 
The dominant processes are $pp\to t\bar{t}W(j)$ and $pp\to t\bar{t}Z(j)$, 
while there are also minor contributions from $pp\to t\bar{t}H,\,tZbj,\,t\bar{t}WW$, and $pp\to t\bar{t}t\bar{t}$. 
\item[$\bullet$] The production of multiple massive bosons. 
These processes have generally low jet multiplicities. 
However, due to their larger cross-sections, they contribute in a significant way to the background entering signal regions without $b$-jets requirements. 
The dominant processes include $pp\to W^\pm W^\pm jj,\,WZ,\,ZZ$, 
with minor contributions from $pp\to\,WH,\,ZH,\,VVV$ and $pp \to\,H\to\,llll$.  
\end{itemize}

We plan to estimate the contributions of these various processes to the signal regions by relying on the Monte-Carlo predictions, 
normalized with the best known theoretical cross-sections : 
these processes are too rare to allow use of control regions until a significant integrated luminosity will be collected. 

Processes containing top quarks have cross-sections below 1~pb, which have consequently not yet been much constrained experimentally.
On the theoretical side, uncertainties on the cross sections are typically large: 30\% for $t\bar{t}W$ and 50\% for $t\bar{t}Z$
for the same-sign $\sqrt{s}=7$~TeV analysis~\cite{NoteSS3L_7TeV}, 22\% for both $t\bar{t}W$ and $t\bar{t}Z$ for the $\sqrt{s}=8$~TeV analysis~\cite{noteSS3L}. 

Cross-sections for diboson processes are known with a rather good accuracy, but only for the inclusive processes, 
while we are mostly interested in processes where several additional partons are produced. 

The validation regions described in section~\ref{sec:bkg_VR} will help us ensure that our understanding of these processes is sufficiently reliable, 
and that the systematic uncertainties assigned to the estimated rates are reasonable. 

\subsection{Charge flip leptons}
\label{sec:bkg_chflips}


Electrons are occasionally reconstructed with the wrong charge, 
generally because an irrelevant track is matched to the electron cluster during reconstruction instead of the real electron track. 
This may occur for example in the so-called trident process, where an electron emits a hard Brehmsstrahlung photon 
which further converts into an electron-positron pair, resulting in three close tracks. 
This confusion is the main reason for lepton charge flip, as errors on the track reconstruction itself are much rarer. 
Indeed, muon charge flip has been estimated negligible in past studies~\cite{noteSS3L}, and we consider only charge flip background originating from electrons. 

\begin{figure}[hbt]
\centering
\includegraphics[width=0.7\textwidth]{BKG/chargeFlipRate}
\caption{Expected electron charge flip rate determined in simulated $t\bar{t}\to e\mu\nu\nu$ events}
\label{fig:mc_chargeflip_rate}
\end{figure}


This type of background is particularly relevant to searches with same-sign leptons, 
as they may turn a pair of opposite-sign leptons from an abundant Standard Model process ($pp\to Z,\,t\bar{t},\,W^+W^-$\ldots) 
into a much rarer same-sign pair.
Because of the jets requirements characteristic of our analysis, 
the dominant source of charge flip electrons in most of the signal regions arise from leptonic decays of $t\bar{t}$ pairs. 
This type of background contributes only moderately to the signal regions -- for example in the Run-1 analysis it represented at most 10\% of the total background yield. 
But for looser event selections, such as the ones used to validate the background modelling, it can be a major component in electron channels. 
It is therefore important to be able to predict reliably this background. 

Figure~\ref{fig:mc_chargeflip_rate} shows the expected rate of charge flip for electrons 
satisfying the requirements listed in sections~\ref{sec:objects_electrons} and~\ref{sec:isolation}, 
determined in simulated $t\bar{t}\to e\mu\nu\nu$ events. 
The rate increases significantly at large pseudorapidity, reflecting the larger amount of material in front of the calorimeter. 
There is also a significant dependency to the transverse momentum, with larger rates obtained for energetic electrons. 
\\
\par{\bf Background estimation methodology\\}
We rely on a purely data-driven method to estimate yields of events with charge flipped electrons. 
Assuming one knows the electron charge flip rates $\xi(\eta,p_T)$, a simple way to estimate the yields is to select 
events with pairs of opposite-sign leptons and assign them a weight: 

\begin{align}
w = \xi\left(\eta^{1},p_T^{1}\right)\left[1-\xi\left(\eta^{2},p_T^{2}\right)\right] 
+ \xi\left(\eta^{2},p_T^{2}\right)\left[1-\xi\left(\eta^{1},p_T^{1}\right)\right] 
\end{align}
where $\xi=0$ for muons. 

The advantages of this method are a good statistical precision since the charge flip rate is rather low, 
and the lack of dependency on the simulation and related uncertainties. 
Obviously, it requires a precise measurement of the rates, which is described in the next paragraph. 
Another inconvenient is that the reconstructed electron energy (hence momentum as well) of charge flipped electrons 
tends to be too low by a few~\GeV, because of the hard Brehmsstrahlung at the origin of the charge flip. 
Simply reweighting electrons from opposite-sign lepton pairs therefore does not predict correctly the charge-flip background shape 
for energy-dependent variables. 
In the past this effect was simply neglected, as we do not rely on discriminant variables very sensitive to the electron energy. 
But we are now considering applying energy correction factors, or assign dedicated systematic uncertainties. 
\\
\par{\bf Measurement of the charge flip rates\\}
The simulation of the charge flip process is not accurate enough to be relied on (discrepancies up to a factor 2 were seen in Run-1 analyses), 
therefore the rates have to be measured in data. 
In Run-2, electron charge flip rate measurements will be centralized by the Egamma performance group (unlike up to now). 
One of our group members is involved in these activities and will provide the rates according to our customized electron definitions. 
The standard method which will be employed for these measurements is similar to the one described 
in the documentation of the Run-1 same-sign leptons + jets search~\cite{noteSS3L}. 
It relies on the observed numbers of opposite (OS) and same-sign (SS) electrons with an invariant mass close to the $Z$ mass, 
which provides a clean sample of electrons. Expressing these numbers as function of the electron charge flip rate $\xi(\eta,p_T)$: 
\begin{align}
N_\text{SS}\left(\eta^{1},p_T^{1},\eta^{2},p_T^{2}\right) \approx 
\left[\xi\left(\eta^{1},p_T^{1}\right) + \xi\left(\eta^{2},p_T^{2}\right)\right] 
N\left(\eta^{1},p_T^{1},\eta^{2},p_T^{2}\right)\,,\quad N=N_\text{OS}+N_\text{SS}
\end{align}
the rates are then obtained as the maximum likelihood estimators for the product of Poisson distributions $\mathcal{P}(N_\text{SS}| N)$ 
binned in $\eta$ and $p_T$ of the two electrons. 

Sources of systematic uncertainties on the measured rates account for the background subtraction, 
and closure tests performed on Monte-Carlo (differences between the computed and true rates) 
and data ($Z$ lineshape opposite-sign electrons pairs reweighted by the measured rates, compared to the distribution of same-sign pairs). 

\subsection{Backgrounds with fake leptons}
\label{sec:bkg_fakes}

The term "fake lepton" denotes here any reconstructed lepton not originating 
from the decay of massive gauge or Higgs bosons, or an electroweak initial/final state radiation. 
They can be non-prompt leptons produced in heavy flavor meson decays, converted photons, 
light hadrons faking the electron shower, in-flight decays of kaons or pions to muons, etc.  
Common properties shared by these objects are a bad response to electron identification cuts, 
non-zero impact parameters, and the reconstructed leptons are often not well isolated; 
these properties can be used to discriminate fake leptons against the prompt leptons we are interested in. 

In signal regions enriched in $b$-jets, Monte-Carlo simulations predict that the dominant source of fake leptons originates 
from semileptonic $t\bar{t}$ events, with a non-prompt lepton in the decay chain of one of the $B$ mesons. 
Contributions from photon conversions and hadron fakes are not to be neglected though, 
in particular in $b$-jet-depleted signal regions where the rate of $t\bar{t}$ events is lower 
and is competed by processes such as $W\gamma$ + jets. 

We will rely on at least two different methods to estimate the fake leptons yields in the signal regions, 
which are briefly described in the next paragraphs. They have both been employed successfully in the Run-1 analysis~\cite{noteSS3L}. 

\subsubsection{The matrix method}
\label{sec:bkg_fakes_matrixmethod}

The matrix method is a purely data-driven approach 
which relies on the different response of prompt and fake leptons to identification, isolation and impact parameters requirements: 
fake leptons have low probabilities to satisfy these requirements, unlike prompt leptons. 
No tentative is made to consider the different categories of fake leptons separately -- systematic uncertainties are assigned in consequence. 
\\
\par{\bf Methodology\\}
A combination of tight requirements on discriminant variables such as electron identification, lepton isolation and impact parameters, is defined (see Tabe~\ref{tab:lepdef}). 
Reconstructed leptons are then classified in two categories ("tight" and "loose"), depending on their success satisfying the tight requirements or not. 
If $\epsilon$ and $\zeta$ are respectively the probabilities for a prompt/fake lepton to satisfy the requirements, 
linear relationships can be established between the mean values of the rates of prompt/fakes and tight/loose leptons, which for the 1-lepton case are: 
\begin{align}
\label{eqn:matrixmethod}
<N_\text{tight}> &= \epsilon <N_\text{prompt}> + \zeta <N_\text{fake}> \\\notag
<N_\text{loose}> &=  (1-\epsilon) <N_\text{prompt}> + (1-\zeta) <N_\text{fake}>
\end{align}

This system of equations can be used to evaluate the number of prompt and fake leptons given the observed number of tight and loose leptons. 
A detailed explanation of the generalization of the method to handle events with arbitrary number of leptons, as used in the Run-1 analysis, 
can be found in~\cite{noteSS3L,TomThesis}. 

The method relies of the prior knowledge of the probabilities $\epsilon$ and $\zeta$, 
which need to be measured in dedicated samples enriched in prompt and fake leptons (see next paragraph). 
The uncertainties on the probabilities for fake leptons constitute the main source of uncertainties in the asymptotic regime. 
In the low statistics regime, one has to cope with the fact that the loose and tight leptons categories are not enough populated 
to provide reliable estimates: 
for example predictions of negative yields are a possible outcome. 
In general these estimates are accompanied by large statistical uncertainties. 

Finally, one should note that the charge flip electron background interferes with the matrix method 
as charged flipped electrons are notably more prone to fail impact parameter or tight identification requirements, 
and the related efficiencies are distinct from both those of prompt and fake electrons. 
They correspond so to speak to a third category of objects, while the matrix method is based on the assumption that only two categories are present. 
One way to solve the issue is to rely on the linearity of the matrix method estimate with respect to its input number of events; 
therefore one can simply subtract the estimated charge flip background in the tight and loose leptons categories, from observed data. 
This requires a dedicated measurement of the charge flip rate for electrons failing the tight requirement. 
\\
\par{\bf Measurement of the $\epsilon,\zeta$ probabilities\\}
These parameters are measured in dedicated samples enriched in prompt or fake leptons. 
Probabilities for prompt leptons are measured in $Z\to \ell\ell$ tag-and-probe selections, 
and are assigned systematic uncertainties determined in Monte-Carlo covering the 
differences between the lepton properties in the measurement regions, 
and in the signal regions which have much busier event topologies. 
Other systematic uncertainties should account for the less isolated leptons that can be 
found in signal scenarios where decay products are particularly boosted (for example the gluino-stop offshell model). 

Probabilities for fake leptons are much harder to determine, due to the difficulty to identify 
an event selection that would provide both a high purity, and enough statistics, especially for leptons with $p_T>40$~\GeV. 
In the 8 TeV versions of this analysis, such selections were requiring at least two same-sign leptons together with a jet, 
and the fake electron probabilities were determined separately for events with or without $b$-jets. 
Other analyses have been using inclusive selections with a single lepton, 
which have the advantage to be much more populated, but on the other hand 
are less representative of the properties of fake leptons that can be found in our signal regions. 
In general the probabilities vary largely with $p_T$ (see e.g. the isolation efficiencies in section~\ref{sec:isolation}) thus require binned measurements. 

The measurements of fake leptons probabilities are associated to large uncertainties, 
which cover for the nature of the fake leptons and the events that produce them 
being different between the measurement and signal regions. 

\subsubsection{The Monte-Carlo template method}
\label{sec:bkg_fakes_mctemplates}

This method is Monte-Carlo based, and makes use of the samples corresponding to the various processes expected to produce fake leptons. 
Correction factors of the fake lepton modelling in the simulation are determined 
in a combined fit of several control regions involving shapes of discriminant variables such as the missing transverse momentum or the jet multiplicity. 
There are five such correction factors (for fake electrons and muons originating from light or heavy flavor jets, and an additional factor for charge flip), 
which are applied on an event basis depending on the generator information about the origin of the fake lepton in the event. 
More details can be found in the documentation of the Run-1 analysis~\cite{noteSS3L}. 

\subsubsection{Statistical tools improving robustness in low statistics regime}
\label{sec:bkg_fakes_stattools}

Some alternatives to the matrix method have been considered, based on the same principles but with more formal constructions as probabilistic tools 
(hence a better behaviour in conditions far from the asymptotic regime), and are currently being developed. 
One of these methods~\cite{PseudoLikelihoodMatrixMethod} relies on the construction of a likelihood function 
which translates the equation system~(\ref{eqn:matrixmethod}) in terms of probabilities for single leptons, instead of relationships between mean values; 
this also allows to treat $\epsilon$ and $\zeta$ as nuisance parameters (according to their measurements uncertainties) instead of being frozen. 
One could then simply estimate the rates of prompt and fake leptons as the maximum likelihood estimators of the observations, 
but the large number of parameters (several leptons per event hence many possible loose/tight combinations, binned probabilities $\epsilon$ and $\zeta$\ldots) 
renders this approach impractical. 
A simplification is discussed in~\cite{PseudoLikelihoodMatrixMethod,TomThesis}, which preserves some advantages over the standard matrix method. 

Another approach has been proposed recently~\cite{TomThesis}, 
consisting in building confidence intervals on the rate of fake leptons 
from the associated Bayesian posterior, which may be numerically computed with Gibbs sampling. 
The claimed advantages are a better assessment of the uncertainties on the final estimate, 
since the whole posterior distribution is known, 
as well as the absence of issues with local minima which might affect likelihood-based methods. 
More details can be found in~\cite{TomThesis}. 

We are planning to check if these methods can be applied with all the complexity of a real world analysis, 
in which case they would bring quite a nice improvement to the current way of predicting fake lepton background in the signal regions. 

\subsection{Backgrounds with fake $b$-jets}
\label{sec:bkg_bjetfakes}

In the signal regions requiring at least three $b$-jets, an important part of the background originates from processes 
with two real $b$-jets, and a third jet not originating from a $b$-quark but satisfying the $b$-tagging requirements. 
Unlike the case of fake leptons, the $b$-tagging performance group usually provides scale factors (and associated uncertainties) 
to correct the simulation for mismodelling both of real and fake $b$-jets. 
The fake leptons and charge flip background, being predicted from data, obviously do not need any correction. 

An alternative data-driven method was used in Run-1 as cross-check, 
consisting in the equivalent of the lepton matrix method but applying on jets 
and replacing lepton tight requirements by the output weight of the $b$-tagging algorithm. 
This approach was primarily developed for the ATLAS SUSY search with 0/1 lepton and 3 $b$-jets~\cite{SUSY3bjetsRun1}. 
It could be considered again for Run-2. 

\subsection{Neglected backgrounds}
\label{sec:bkg_neglected}
Other sources of background such as cosmic rays or cavern background, as well as pile-up events where two distinct proton-proton pairs may interact and produce leptons, 
have been evaluated in the past and found to be negligible. 
Multiple scattering effects are included in the simulations but are also not expected to contribute enough to require in-depth studies. 

\subsection{Validation regions}
\label{sec:bkg_VR}
The different type of backgrounds will be validated by looking at the agreement between the data observation and the SM expectation in regions dominated by one type of background. As for Run-1, several validation regions will be considered, and their definition is a balance between high purity, large statistics and low signal contamination. The latter is assured by requiring an upper cut on \met\ 
($<150$~\GeV). If no validation region can be designed, the background will be checked by looking at different kinematic variables distributions. For a complete validation, several selections will be considered :
\begin{itemize}
	\item{Loose: requiring at least two signal leptons in the event.}
	\item{Intermediate: adding soft cuts i.e. at least one or two signal jets or $b$ jets.}
	\item{Hard: increase the number of jets in the event.}
\end{itemize}


It is known that the associated uncertainty on the SM background can be decreased by adding control regions enriched in a certain type of background and orthogonal on the signal regions. However, given the rare SM processes which dominate in the signal regions, the statistics expected with $\sqrt s=$ 13 \TeV\ and $L$~$\le$3 \ifb of data in potential control regions is too small to be included. Hence, only validation regions will be defined for early Run-2.

\paragraph{\ttbar\ + $V$ background\\}
To validate the \ttbar\ + $Z$ and \ttbar\ + $W$ backgrounds estimation, several tentative regions defined with at least 1 or 2 $b$-jets in the event were considered. To decrease the other type of backgrounds, several jets and a low \met\ cut were also required. However, given the low statistics, it was found that a combined \ttbar~+~ $V$ region will perform better. The region definition is given in Table~\ref{tab:ttV_VR}. The leading and sub-leading leptons are required to be energetic (20 \GeV) in order to reduce the fake lepton background, while the veto on the fourth baseline lepton is minimizing the contamination from $ZZ$ processes. Note that it is very hard to obtain a pure validation region in \ttbar\ + $V$, as the difference with the other prompt SS SM processes is mainly the jet multiplicity. The highest purity was found when considering at least four jets in the $ee$ and $e\mu$ channels, and at least three jets in the $\mu\mu$ channel. A high \meff\ cut is used to further reduce the detector background, wile an upper cut of 900 \GeV\ is used to reduce the signal contamination. Note that this cut can be tighten to further decrease the SUSY contamination from models like direct sbottom. The reached purity for $\sqrt s=$ 13 \TeV\ and $L$~$\le$3 \ifb of data is around 42$\%$ if the large eta region is excluded ($|\eta|_e$~$<$~1.37) . The purity can be increased to 60$\%$ if the inclusive 1 $b$ jet selection is replaced by inclusive 2 $b$ jet selection (and the 400 \GeV\ cut on \meff\ is relaxed to 350 \GeV). The signal contamination was studied by looking at several available SUSY models studied in this analysis. It can be up to 27$\%$ when the direct-sbottom model is considered and the sbottom mass is 550 \GeV, or 20$\%$ for direct squark (2 step) via sleptons with $W/Z$ bosons in the cascade decay, otherwise it is smaller than 10$\%$. 

\begin{table}[htb!]
\caption{\ttbar\ + $V$ validation region definition. The \pt\ threshold of the leading two leptons is 20 \GeV, otherwise 10 \GeV.}
\label{tab:ttV_VR}
\begin{center}
    \begin{tabular}{|c|cc|c|c|}
      \hline
      \hline
     VR& $ N_{lept}^{signal}$ & $N_{lept}^{baseline}$ & $N_{b-jets}^{20}$     & Other variables \\ \hline
    ttV & $\ge 2$ &$<4$  & $\ge$1  & 30 $<$ \met $<$ 200 \GeV, 400 $<$ \meff $<$ 900 \GeV, $|\eta|_e$~$<$~1.37, \\
   &  & && $N_{jets}^{30}$ $\ge$ 4 in $ee$ and $e\mu$, $N_{jets}^{25}$ $\ge$ 3 in $\mu\mu$ channels\\	
      \hline
\end{tabular}
\end{center}
\end{table}


\paragraph{$ZZ$ + jets validation region\\}
Even if this type of background is minor in several of the defined signal regions, its contribution can is significant in regions with no $b$ jet requirement and three leptons. Therefore, a validation region was defined as presented in Table~\ref{tab:ZZ_VR}. It requires at least three signal leptons with \pt~$>$~20 \GeV, and at least four baseline leptons. The reached purity for $\sqrt s=$ 13 \TeV\ and $L$~$\le$3 \ifb of data is up to 97$\%$. In order to be closer to the SRs, another validation region can be defined by asking at least two jets with \pt~$>$~25 \GeV\ with a purity of 84$\%$. The signal contamination is at most 20$\%$.

\begin{table}[htb!]
\caption{$ZZ$ validation region definition. The \pt\ threshold of the two leading leptons is 20 \GeV, otherwise 10 \GeV. To be closer to the SRs, at least two jets with \pt~$>$~25 \GeV\ can be required in the event (and $N_{b-jets}^{20}$~$==$~0 becomes $N_{b-jets}^{20}$~$\ge$~0), without affecting the validation region purity.}
\label{tab:ZZ_VR}
\begin{center}
    \begin{tabular}{|c|cc|c|c|}
      \hline
      \hline
     VR & $ N_{lept}^{signal}$ & $N_{lept}^{baseline}$ & $N_{b-jets}^{20}$     & Other variables \\ \hline
     ZZ& $==3$ & $\geq 4$  & $==$0  & \met $<$ 150 \GeV, \meff $>$ 100 \GeV, $|\eta|_e$~$<$~2.  \\
     \hline
\end{tabular}
\end{center}
\end{table}


\paragraph{$WZ$ + jets validation region\\}
Similarly to $ZZ$ type of background, it is negligible in most of signal regions. However, it is expected to populate the regions with 0 $b$-jets and low jet multiplicity. Therefore, its validation is important and the proposed region definition is presented in Table~\ref{tab:WZ_VR}. It requires exactly three leptons and a veto on the fourth-leading baseline lepton in order to reduce the $ZZ$ background contamination. The lower cut on \met\ (30 \GeV) is mainly reducing the charge flip contamination. The \mt\ cut of 50 \GeV, even if is not used to define the signal regions, is assuring a smaller fake lepton contamination. At least one and at most three jets with \pt~$>$~25 \GeV are required in the event. It assures a purity of 70$\%$ for $\sqrt s=$ 13 \TeV\ and $L$~$\le$3 \ifb of data. The signal contamination is found to be below 1$\%$.

\begin{table}[htb!]
\caption{$WZ$ validation region definition. The \pt\ threshold of the two leading leptons is 20 \GeV, otherwise 10 \GeV.}
\label{tab:WZ_VR}
\begin{center}
    \begin{tabular}{|c|cc|c|c|}
      \hline
      \hline
     VR & $N_{lept}^{signal}$ & $N_{lept}^{baseline}$ &  $N_{b-jets}^{20}$     & Other variables \\ \hline
     WZ & $==$3 & $<4$  & $==$0  & 30 $<$ \met $<$ 300 \GeV, \mt~$>$50 \GeV, 1 $\leq$ $N_{jets}^{25}$ $\leq$ 3, $|\eta|_e$~$<$~2. \\
     \hline
\end{tabular}
\end{center}
\end{table}

\paragraph{$WW$ + jets validation region\\}
Similarly to the other diboson processes, the $WW$ + jets (with a pair of same sign lepton in the final state) events contribute mainly in the signal regions with no $b$ jet requirement. Two validation regions are defined (Table~\ref{tab:WW_VR}). In the first region, a cut on the invariant mass of the first two leading jets in the event is considered. It selects forward jets from VBS processes which are not necessary populating the signal regions defined in the analysis. However it is used as a cross-check as a high purity (50$\%$) can be reached. A second region is defined by selection two energetic jets (50 \GeV) and decreasing the electron acceptance to $|\eta|_e$~$<$~1.37. The purity is only 31$\%$. In both validation regions the signal contamination is at most 15$\%$.

\begin{table}[htb!]
\caption{$WW$ validation region definition. The \pt\ threshold of the two leading leptons is 20 \GeV.}
\label{tab:WW_VR}
\begin{center}
    \begin{tabular}{|c|cc|c|c|}
      \hline
      \hline
     VR & $N_{lept}^{signal}$ & $N_{lept}^{baseline}$ &  $N_{b-jets}^{20}$     & Other variables \\ \hline
     WW1 & $==$2 & $<3$  & $==$0  & 40 $<$ \met $<$ 300 \GeV, $Z$ mass veto,\\ 
     &&&& \mt~$>$40 \GeV, $N_{jets}^{40}$ $\ge$ 2, $m_{jj}$ $>$ 500 \GeV, $|\eta|_e$~$<$~2. \\\hline
     WW1 & $==$2 & $<3$  & $==$0  & 40 $<$ \met $<$ 300 \GeV, $Z$ mass veto,\\
     &&&& \mt~$>$40 \GeV, $N_{jets}^{50}$ $\ge$ 2, $|\eta|_e$~$<$~1.37 \\
     \hline
\end{tabular}
\end{center}
\end{table}

\paragraph{Fake lepton validation regions\\}
Beside the usual validation plots shown for the Run-1 analysis, several fake lepton validation regions are defined, in order to increase the confidence on the background estimation methods. Generally, depending on the $b$-jet multiplicity two categories are considered: low and high \meff, as presented in Table~\ref{tab:Fake_VR}. To reduce the charge flip background contamination, a veto on the $Z$ boson mass ($80<m_{\ell\ell}<100$~\GeV) is imposed only in the $ee$ channel. The background from prompt SS processes is reduced by a veto on the third baseline lepton. The leading two leptons are required to have a \pt\ threshold of 15 \GeV, and the event should have at least one jet. For the low \meff\ regions, the highest purity was reached when \met~$<$~100 \GeV\ and \meff~$<$~400 \GeV\ was imposed. A lower cut on \meff\ will drastically reduce the fake lepton background in the low \meff\ validation region, while in the high \meff\ region a higher threshold will increase the level of prompt SS background. Note that, the two validation regions overlap and the reached purity varies between 80 and 90$\%$. The signal contamination is found to be maximum 10$\%$.

\begin{table}[htb!]
\caption{Fake lepton validation region definition. The \pt\ threshold of the two leading leptons is 15 \GeV. }
\label{tab:Fake_VR}
\begin{center}
    \begin{tabular}{|c|cc|c|c|}
      \hline
      \hline
      \multicolumn{5}{|c|}{\textbf{Low meff fake leptons VRs}}\\
      \hline
      \hline
     VR & $N_{lept}^{signal}$ & $N_{lept}^{baseline}$ & $N_{b-jets}^{20}$     & Other variables \\ \hline
     FL0b-L\meff& $==$2 & $<3$  & $==$0  & 20 $<$ \met $<$ 100 \GeV, \meff $<$ 400 \GeV, \mt~$<$60 \GeV, $N_{jets}^{25}\geq 1$ \\
     FL1b-L\meff& $==$2 & $<3$  & $==$1  & 20 $<$ \met $<$ 100 \GeV, \meff $<$ 400 \GeV, \mt~$<$60 \GeV\\
     FL2b-L\meff&$==$2  & $<3$ & $\ge$2  & 20 $<$ \met $<$ 100 \GeV, \meff $<$ 400 \GeV, \mt~$<$80 \GeV \\
     \hline
      \hline
      \multicolumn{5}{|c|}{\textbf{High meff fake leptons VRs}}\\
      \hline
      \hline
     VR & $N_{lept}^{signal}$ & $N_{lept}^{baseline}$ & $N_{b-jets}^{20}$     & Other variables \\ \hline
     FL0b-H\meff& $==$2 & $<3$ & $==$0  & 20 $<$ \met $<$ 100 \GeV, \meff $>$ 300 \GeV, \mt~$<$100 \GeV, $N_{jets}^{25}\geq 1$ \\
     FL1b-H\meff& $==$2 & $<3$ & $==$1  & 20 $<$ \met $<$ 50 \GeV, \meff $>$ 300 \GeV, \mt~$<$80 \GeV\\
     FL2b-H\meff&$==$2  & $<3$ & $\ge$2  & 20 $<$ \met $<$ 50 \GeV, \meff $>$ 300 \GeV, \mt~$<$50 \GeV\\
     \hline
\end{tabular}
\end{center}
\end{table}





%-------------------------------------------------------------------------------
\section{Results}
\label{sec:result}
%-------------------------------------------------------------------------------

\begin{table}[htb!]
\begin{center}
\setlength{\tabcolsep}{0.0pc}
\caption{The number of observed data events and expected background contributions in the signal regions. 
The $p$-value of the observed events for the background-only hypothesis is denoted by $p(s = 0)$. 
The ``Rare'' category contains the contributions from associated production of $\ttbar$ with $h/WW/t/\ttbar$, 
as well as $tZ$, $Wh$, $Zh$, and triboson production. 
Background categories shown as ``$-$'' denote that they cannot contribute to a given region (charge flips or $W^\pm W^\pm jj$ in 3-lepton regions). 
The individual uncertainties can be correlated and therefore do not necessarily add up in quadrature to the total systematic uncertainty. 
}
\label{tab:SR_yields}
{\small
\begin{tabular*}{\textwidth}{@{\extracolsep{\fill}}lcccc}
\noalign{\smallskip}\hline\hline\noalign{\smallskip}
         & SR0b3j         & SR0b5j     & SR1b & SR3b     \\[-0.05cm]
\noalign{\smallskip}\hline\hline\noalign{\smallskip}
Observed events         & $3$     &  $3$  & $7$  & $1$            \\
\noalign{\smallskip}\hline\noalign{\smallskip}
Total background events & $1.5 \pm 0.4$ & $0.88 \pm 0.29$ & $4.5 \pm 1.0$ & $0.80 \pm 0.25$\\
$p(s = 0)$                &  0.13  &  0.04  &  0.15  &   0.36   \\
\noalign{\smallskip}\hline\noalign{\smallskip}
Fake/non-prompt leptons & $<0.2$ & $0.05\pm 0.18$ & $0.8 \pm 0.8$ & $0.13 \pm 0.17$\\
Charge-flip & $-$ & $0.02 \pm 0.01$ & $0.60 \pm 0.12$ & $0.19 \pm 0.06$\\
$t\bar{t}W$ & $0.02 \pm 0.01$ & $0.08 \pm 0.04$ & $1.1 \pm 0.4$ & $0.10 \pm 0.05$\\
$t\bar{t}Z$ & $0.10 \pm 0.04$ & $0.05 \pm 0.03$ & $0.92 \pm 0.31$ & $0.14 \pm 0.06$\\
$WZ$ & $1.2 \pm 0.4$ & $0.48 \pm 0.20$ & $0.18 \pm 0.11$ & $<0.02$\\
$W^\pm W^\pm jj$ & $-$ & $0.12 \pm 0.07$ & $0.03 \pm 0.02$ & $<0.01$\\
$ZZ$ & $<0.03$ & $<0.04$ & $<0.03$ & $<0.03$\\
Rare & $0.14 \pm 0.08$ & $0.07 \pm 0.05$ & $0.8 \pm 0.4$ & $0.24 \pm 0.14$\\  
\noalign{\smallskip}\hline\hline\noalign{\smallskip}
\end{tabular*}
}
\end{center}
\end{table}

\begin{figure}[t!]
\centering
\begin{subfigure}[t]{0.49\textwidth}\includegraphics[width=\textwidth]{FIGURES/CONF_SR0b3j.pdf}
\caption{}\label{fig:Results_SR0b3j}\end{subfigure}
\begin{subfigure}[t]{0.49\textwidth}\includegraphics[width=\textwidth]{FIGURES/CONF_SR0b5j.pdf}
\caption{}\label{fig:Results_SR0b5j}\end{subfigure}
\begin{subfigure}[t]{0.49\textwidth}\includegraphics[width=\textwidth]{FIGURES/CONF_SR1b.pdf}
\caption{}\label{fig:Results_SR1b}\end{subfigure}
\begin{subfigure}[t]{0.49\textwidth}\includegraphics[width=\textwidth]{FIGURES/CONF_SR3b.pdf}
\caption{}\label{fig:Results_SR3b}\end{subfigure}
\caption{Missing transverse momentum distributions after (a) SR0b3j, (b) SR0b5j, (c) SR1b and (d) SR3b selection, beside the \met requirement. 
The results in the signal regions are shown in the last (inclusive) bin of each plot. 
The statistical uncertainties in the background prediction are included in the uncertainty band, 
as well as the theory uncertainties for the backgrounds with prompt SS/3L, 
and the full systematic uncertainties for data-driven backgrounds. 
The ``Fake leptons'' category corresponds to FNP leptons (see text), 
and the ``Rare'' category contains the contributions from associated production of $\ttbar$ with $h/WW/t/\ttbar$, 
as well as $tZ$, $Wh$, $Zh$, and triboson production. 
}
\label{fig:Results_SR_metD} 
\end{figure} 


Figure~\ref{fig:Results_SR_metD} shows the data \met distributions after the signal region selections (beside that on \met) in data 
together with the expected contributions from all the SM backgrounds with their total statistical and systematic uncertainties. 
For illustration, a typical SUSY signal distribution corresponding to the most relevant benchmark scenario 
in each SR is displayed.
The detailed yields for data and the different sources of SM background in the signal regions 
are presented in Table~\ref{tab:SR_yields}. 
The uncertainties amount to 22--34\% of the total background depending on the signal region. 
In all four SRs the number of data events exceeds the expectation but is consistent within the uncertainties, 
the smallest $p$-value for the SM-only hypothesis being 0.04 for SR0b5j. 
Out of the 14 events in the SRs, 2 of the events in SR1b and the 3 events in SR0b3j contain three leptons. 
None of those events contain three leptons of equal charge. 

In the absence of any significant deviations from the SM predictions, upper limits on possible BSM contributions to the signal regions are computed, 
in particular in the context of the SUSY benchmark scenarios described in Section~\ref{sec:intro}. 
The HistFitter framework~\cite{Baak:2014wma}, which utilises a profile-likelihood-ratio test~\cite{Cowan:2010js}, 
is used to establish 95\% confidence intervals using the CL$_\mathrm{s}$ prescription~\cite{Read_CLs}. 
The likelihood is built as the product of a Poisson probability density function describing the observed number of events in the signal region 
and Gaussian distributions constraining the nuisance parameters 
associated with the systematic uncertainties whose widths correspond to the sizes of these uncertainties; 
Poisson distributions are used instead for MC statistical uncertainties. 
Correlations of a given nuisance parameter across the different sources of backgrounds and the signal are taken into account when relevant. 
The statistical tests are performed independently for each of the signal regions. 

Table~\ref{tab:upperlimits} presents 95\% confidence level (CL) model-independent upper limits 
on the number of BSM events, $N_\mathrm{BSM}$, that may contribute to the signal regions. 
Normalising these by the integrated luminosity $L$ of the data sample, they can be interpreted as upper limits on the visible BSM cross-section $\sigma_{\rm{vis}}$, 
defined as the product $\sigma_{\rm{prod}}\times A \times\epsilon=N_\mathrm{BSM}/L$ of production cross-section, acceptance and reconstruction efficiency. 

\begin{table}[htb!]
\centering
\caption{Signal model-independent upper limits on the number of BSM events ($N_{\rm{BSM}}$) 
  and the visible signal cross-section ($\sigma_{\rm{vis}}$) in the four SRs. 
  The numbers (in parentheses) give the observed (expected under the SM hypothesis) 95\% CL upper
  limits. Calculations are performed with pseudo-experiments.
  The $\pm$1$\sigma$ variations on the expected limit due to the statistical and systematic uncertainties in the background prediction are also shown. 
}
\label{tab:upperlimits}
{\small
\renewcommand{\arraystretch}{1.4}
\begin{tabular*}{\textwidth}{@{\extracolsep{\fill}}lrrrr}
\noalign{\smallskip}\hline\hline\noalign{\smallskip}
         & SR0b3j         & SR0b5j     & SR1b & SR3b     \\[-0.05cm]
\noalign{\smallskip}\hline\hline\noalign{\smallskip}
$N_{\rm{BSM}}^{\rm{obs}}$ ($N_{\rm{BSM}}^{\rm{exp}}$) 
 & $5.9$  $({4.1}^{+1.6}_{-0.8})$
 & $6.4$ $({3.6}^{+1.2}_{-1.1})$
 & $8.8$ $({6.0}^{+2.6}_{-1.6})$ 
 & $3.8$ $({3.7}^{+1.1}_{-0.5})$ \\
$\sigma_{\rm{vis}}^{\rm{obs}}$ [fb] & 1.8 & 2.0 & 2.8 & 1.2\\
\noalign{\smallskip}\hline\hline\noalign{\smallskip}
  \end{tabular*}
}
\end{table} 

\begin{figure}[htb!]
\centering
\begin{subfigure}[t]{0.49\textwidth}\includegraphics[width=\textwidth]{exclusion2015SameSign_SR0b3j.pdf}
\caption{$\gluino\to q\bar q \ell\ell\ninoone$ scenario, SR0b3j}\label{fig:limits_SR0b3j}\end{subfigure}
\begin{subfigure}[t]{0.49\textwidth}\includegraphics[width=\textwidth]{exclusion2015SameSign_SR0b5j.pdf}
\caption{$\gluino\to q\bar q' WZ\ninoone$ scenario, SR0b5j}\label{fig:limits_SR0b5j}\end{subfigure}
\par\bigskip
\begin{subfigure}[t]{0.49\textwidth}\includegraphics[width=\textwidth]{exclusion2015SameSign_SR1b.pdf}
\caption{$\sbottomone\to tW^-\ninoone$ scenario, SR1b}\label{fig:limits_SR1b}\end{subfigure}
\begin{subfigure}[t]{0.49\textwidth}\includegraphics[width=\textwidth]{exclusion2015SameSign_SR3b.pdf}
\caption{$\gluino\to t\bar t\ninoone$ scenario, SR3b}\label{fig:limits_SR3b}\end{subfigure}
\caption{
Observed and expected exclusion limits on the \gluino, \sbottomone and \ninoone masses 
in the context of SUSY scenarios with simplified mass spectra 
featuring $\gluino\gluino$ or $\sbottomone\sbottomonebar$ pair production with exclusive decay modes. 
The signal region used to obtain the limits is specified for each scenario. 
The contours of the band around the expected limit are the $\pm$1$\sigma$ results, 
  including all uncertainties except theoretical uncertainties on the signal cross-section. The dotted lines around the observed
    limit illustrate the change in the observed limit as the nominal signal cross-section is scaled up and down
    by the theoretical uncertainty. All limits are computed at 95\% CL. 
    The diagonal lines indicate the kinematic limit for the decays in each specified scenario.  
For figures (b) and (d), results are compared with the observed limits obtained by previous ATLAS searches~\cite{paperSS3L,Aad:2015iea,Aad:2015pfx}. 
For figures (a) and (c), a direct comparison with earlier searches is not possible, due to differing model assumptions. 
}
\label{fig:Results_Limits} 
\end{figure} 

Exclusion limits are also set on the masses of the superpartners involved in the four SUSY benchmark scenarios considered in this analysis. 
Simplified models corresponding to a single production mode and with 100\% branching ratio to a specific decay chain are used, 
with the masses of the SUSY particles not involved in the process set to very high values. 
Figure~\ref{fig:Results_Limits} shows the limits 
on the mass of the $\ninoone$ as a function of the $\gluino$ or $\sbottomone$ mass. 
%For these results, asymptotic formulas~\cite{Cowan:2010js} are used to model the probability distribution of the test statistic. 
In some cases, the new limits set by this analysis can be compared 
with the existing limits set by the combination of ATLAS SUSY searches with 8 TeV data~\cite{Aad:2015iea,Aad:2015pfx}. 
For parts of the parameter space, the sensitivity reached with the 13 TeV dataset exceeds that of the 8 TeV dataset,
and additional parameter space regions can be excluded, especially for large neutralino masses. 

Signal models featuring gluino pair production with a subsequent gluino decay via $\ninotwo$ and light sleptons\\ 
($\gluino\to q\bar q\ninotwo\to q\bar q (\ell\slepton^*/\nu\tilde{\nu}^*)\to q\bar q(\ell^+\ell^-/\nu\nu)\ninoone$) 
are probed using SR0b3j (Fig.~\ref{fig:limits_SR0b3j}).
In this simplified model, the gluinos decay into $u\bar u$, $d\bar d$, $s\bar s$ or $c\bar c$ with equal probabilities, 
and the six types of leptons are also produced in the $\tilde\chi_2^0$ decays with equal probabilities. 
The $\ninotwo$ mass is set to $m_{\ninotwo}=(m_{\gluino} + m_{\ninoone})/2$, 
with the $\slepton$ and $\tilde{\nu}$ masses set to $m_{\slepton,\tilde{\nu}}=(m_{\ninotwo} + m_{\ninoone})/2$.
Gluino masses up to $m_{\gluino}\approx\SI{1.3}{TeV}$ for a light \ninoone and \ninoone masses up to $m_{\ninoone}\approx\SI{850}{GeV}$ for gluinos with $m_{\gluino}\approx\SI{1.1}{TeV}$ are excluded in this scenario. 

Similarly, models with gluino production  with a subsequent two-step gluino decay via $\chinoonepm$ and $\ninotwo$\\ 
($\gluino\to q\bar q \chinoonepm \to q\bar q W\ninotwo \to  q\bar q W Z \ninoone$) 
are probed with SR0b5j (Fig.~\ref{fig:limits_SR0b5j}).
In this simplified model, the gluinos decay into $u\bar u$, $d\bar d$, $s\bar s$ or $c\bar c$ with equal probabilities. 
The $\chinoonepm$ mass is set to $m_{\chinoonepm}=(m_{\gluino} + m_{\ninoone})/2$ and
the $\ninotwo$ mass is set to $m_{\ninotwo}=(m_{\chinoonepm} + m_{\ninoone})/2$; 
$W$ and $Z$ bosons produced in the decay chain are not necessarily on-shell. 
The exclusion limits in this scenario reach $m_{\gluino}\approx\SI{1.1}{TeV}$ (for light $\ninoone$) and $m_{\ninoone}\approx\SI{550}{GeV}$ (for $m_{\gluino}\approx\SI{1.0}{TeV}$).

Exclusion limits in a simplified model of bottom squark production with chargino-mediated $\sbottomone\to tW^-\ninoone$ decays are 
obtained with SR1b (Fig.~\ref{fig:limits_SR1b}).
In this model the $\chinoonepm$ mass is set to $m_{\chinoonepm}=m_{\ninoone} + \SI{100}{GeV}$.
The limits can reach mass values of $m_{\sbottomone}\approx\SI{540}{GeV}$ for a light $\ninoone$, 
while $m_{\ninoone}\lesssim\SI{140}{GeV}$ are also excluded for $m_{\sbottomone}\approx\SI{425}{GeV}$, 
significantly extending the previous limits obtained at $\sqrt{s}=8$~TeV~\cite{Aad:2015pfx} 
which excluded $m_{\sbottomone}\lesssim\SI{470}{GeV}$ for $m_{\ninoone}\approx\SI{60}{GeV}$ for a similar model.

Finally, SR3b is used to set limits on masses in a simplified model with 
gluino pair production and $\gluino\to t\bar t\ninoone$ decays 
via an off-shell top squark (Fig.~\ref{fig:limits_SR3b}). 
In that case, gluino masses of $m_{\gluino}\lesssim\SI{1.2}{TeV}$ are excluded for $m_{\ninoone}\lesssim\SI{600}{GeV}$, 
and $\ninoone$ masses up to $m_{\ninoone}\approx\SI{680}{GeV}$ are also excluded for $m_{\gluino}\approx\SI{1.05}{TeV}$. 


\FloatBarrier

%-------------------------------------------------------------------------------
\section{Conclusion}
\label{sec:conclusion}
%-------------------------------------------------------------------------------

A search for supersymmetry in events with exactly two same-sign leptons or at least three leptons, multiple jets, 
$b$-jets and $\met$ is presented. 
The analysis is performed with proton--proton collision data at $\sqrt{s}=13\TeV$ collected with the ATLAS detector at the Large Hadron Collider corresponding to an integrated luminosity of 3.2~fb$^{-1}$. 
With no significant excess over the Standard Model expectation observed,
results are interpreted in the framework of simplified models featuring gluino and bottom squark production.
In the $\gluino\gluino$ simplified models considered, $m_{\gluino}\lesssim 1.1$--$1.3~\TeV$ and $m_{\ninoone}\lesssim 550$--$850~\GeV$ 
are excluded at 95\% confidence level depending on the model parameters. Bottom squark masses of $m_{\sbottomone}\lesssim 540~\GeV$ 
are also excluded for a light $\ninoone$ in a $\sbottomone\sbottomonebar$ simplified model with $\sbottomone\to tW^-\ninoone$.
These results are complementary to those of previous searches and extend the exclusion limits they set.


%-------------------------------------------------------------------------------
\section*{Acknowledgements}
%-------------------------------------------------------------------------------

\input{acknowledgements/Acknowledgements}

\printbibliography

\clearpage
\appendix
\part*{Auxiliary material}
\addcontentsline{toc}{part}{Auxiliary material}
\input{auxPlots}

\end{document}
