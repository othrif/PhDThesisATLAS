The data were collected by the ATLAS detector during 2015 with a peak 
instantaneous luminosity of $L=5.2\times 10^{33}$~cm$^{-2}$s$^{-1}$, 
a bunch spacing of 25~ns, and a mean number of additional $pp$ interactions per bunch crossing 
(pile-up) in the dataset of $\langle \mu \rangle = 14$.
After the application of beam, detector and data quality requirements, 
the integrated luminosity considered in this analysis corresponds to 3.2~fb$^{-1}$ with an uncertainty of $\pm5\%$. 
The luminosity and its uncertainty are derived following a methodology similar to that detailed in
Ref.~\cite{DAPR-2011-01} from a preliminary calibration of the
luminosity scale using a pair of $x$--$y$ beam separation scans performed in
August 2015.

Monte Carlo (MC) simulated event samples are used to aid in the
estimation of the background from SM processes and to model the SUSY signal. 
The MC samples are processed through an ATLAS detector simulation~\cite{Aad:2010ah} based on 
{\sc Geant4}~\cite{Agostinelli:2002hh} 
or a fast simulation using a parameterisation of the calorimeter response 
and {\sc Geant4} for the other parts of the detector~\cite{ATL-PHYS-PUB-2010-013}   
and are reconstructed in the same manner as the data. 

Diboson processes with four charged leptons ($\ell$), three charged leptons and one neutrino, or 
two charged leptons and two neutrinos are simulated using the \SHERPA~v2.1.1 generator~\cite{Gleisberg:2008ta}, 
and are described in detail in Ref.~\cite{pubnote_mc_multiboson}. 
The matrix elements contain the doubly resonant $WW$, $WZ$ and $ZZ$ processes and all other diagrams with
four or six electroweak vertices (such as same-electric-charge $W$ boson production in association with two jets, $W^\pm W^\pm jj$). 
Fully leptonic triboson processes ($WWW$, $WWZ$, $WZZ$ and $ZZZ$) with up to six charged leptons are also simulated using \SHERPA~v2.1.1 
and described in Ref.~\cite{pubnote_mc_multiboson}. 
The $4\ell$ and $2\ell+2\nu$ processes are calculated at next-to-leading order (NLO) for up to one additional parton; 
final states with two and three additional partons are calculated at leading order (LO). 
The $WWZ\to 4\ell+2\nu$ or $2\ell+4\nu$ processes are calculated at LO with up to two additional partons. 
The $3\ell +1\nu$ process is calculated at 
NLO and up to three extra partons at LO using the Comix~\cite{Gleisberg:2008fv} 
and OpenLoops~\cite{Cascioli:2011va} 
matrix element generators and merged with the \SHERPA parton shower~\cite{Schumann:2007mg} 
using the ME+PS@NLO prescription~\cite{Hoeche:2012yf}. 
The $WWZ/WZZ\to 3\ell+3\nu$, $ZZZ\to 6\ell+0\nu$, $4\ell+2\nu$ or $2\ell+4\nu$ processes 
are calculated with the same configuration but with up to only two extra partons at LO. 
The CT10~\cite{Lai:2010vv} parton distribution function (PDF) set is used for all \SHERPA samples in conjunction with 
a dedicated tuning of the parton shower parameters developed by the \SHERPA authors. 
The generator cross-sections (at NLO for most of the processes) are used when normalising these backgrounds.

Samples of $t\bar{t} V$ (with $V=W$ and $Z$, including non-resonant $Z/\gamma^*$ contributions) 
and $t\bar{t}WW$ production are generated at LO with \MADGRAPH v2.2.2~\cite{Alwall:2011uj} interfaced to the \PYTHIA 8.186~\cite{Sjostrand:2007gs} parton shower model,
with up to two ($\ttbar W$), one ($\ttbar Z$) or no ($\ttbar WW$) extra partons included in the matrix element; 
they are described in detail in~\cite{pubnote_mc_ttv}. 
\MADGRAPH is also used to simulate the $tZ$, $\ttbar\ttbar$ and $\ttbar t$ processes. 
The {\sc A14} set of tuned parameters (tune)~\cite{pub-2014-021} is used together with the {\sc NNPDF23LO} PDF set~\cite{Ball:2012cx}. 
The $\ttbar W$, $\ttbar Z$, $\ttbar WW$ and $\ttbar\ttbar$ events are normalised to their NLO cross-section~\cite{Alwall:2014hca} 
while the generator cross-section is used for $tZ$ and $\ttbar t$.

Production of a Higgs boson in association with a $\ttbar$ pair is simulated using \AMCATNLO~\cite{Alwall:2014hca} 
(in \MADGRAPH v2.2.2) interfaced to \HERWIG 2.7.1~\cite{Corcella:2000bw}.  
The UEEE5 underlying-event tune is used together with the CTEQ6L1~\cite{Pumplin:2002vw} (matrix element) and CT10~\cite{Lai:2010vv} (parton shower) PDF sets.
Simulated samples of SM Higgs boson production in association with a $W$ or $Z$ boson are produced with \PYTHIA 8.186, using the {\sc A14} tune and the {\sc NNPDF23LO} PDF set. 
Events are normalised with cross-sections calculated at NLO~\cite{Dittmaier:2012vm}.

The signal SUSY processes are generated from LO matrix elements with up to two extra partons, 
using the \MADGRAPH v2.2.3 generator interfaced to \PYTHIA 8.186 with the {\sc A14} tune for the modelling of the SUSY decay chain, 
parton showering, hadronisation and the description of the underlying event. 
Parton luminosities are provided by the {\sc NNPDF23LO} PDF set. 
Jet--parton matching is realised following the CKKW-L prescription~\cite{Lonnblad:2011xx}, with a matching scale set to one quarter of the pair-produced superpartner mass. 
Signal cross-sections are calculated to NLO in the strong coupling constant, 
adding the resummation of soft gluon emission at next-to-leading-logarithmic accuracy (NLO+NLL)~\cite{Beenakker:1996ch,Kulesza:2008jb,Kulesza:2009kq,Beenakker:2009ha,Beenakker:2011fu}. 
The nominal cross-section and the uncertainty are taken from an envelope of cross-section predictions using different PDF sets and factorisation and renormalisation scales, 
as described in Ref.~\cite{Kramer:2012bx}. 
The production cross-section of gluino pairs with a mass of \SI{1.2}{\TeV} is \SI{86}{\fb} at $\sqrt{s}=13$~TeV (compared with \SI{4.4}{\fb} at $\sqrt{s}=8$~TeV), 
while the production cross-section of pairs of bottom squarks with a mass of \SI{500}{\GeV} is \SI{520}{\fb} at $\sqrt{s}=13$~TeV (compared with \SI{86}{\fb} at $\sqrt{s}=8$~TeV). 

In all MC samples, except those produced by \SHERPA, the {\sc EvtGen}~v1.2.0 program~\cite{EvtGen} is used to model the properties of the bottom and charm hadron decays. 
To simulate the effects of additional $pp$ collisions in the same and nearby bunch crossings, 
additional interactions are generated using the soft QCD processes of \PYTHIA 8.186 
with the A2 tune~\cite{ATLAS-PHYS-PUB-2012-003} and the MSTW2008LO PDF~\cite{Martin:2009iq}, 
and overlaid onto the simulated hard scatter event. 
The Monte Carlo samples are reweighted so that the distribution of the number of reconstructed vertices matches the distribution observed in the data.
 