
Supersymmetry (SUSY)~\cite{Miyazawa:1966,Ramond:1971gb,Golfand:1971iw,Neveu:1971rx,Neveu:1971iv,Gervais:1971ji,Volkov:1973ix,Wess:1973kz,Wess:1974tw}
is a theoretically favoured extension of the Standard Model (SM),
which for each degree of freedom of the SM predicts another degree of
freedom with a different spin. These degrees of freedom combine into
physical superpartners of the SM particles: scalar partners of quarks
and leptons (squarks (\sq) and sleptons), fermionic
partners of gauge and Higgs bosons (gluinos (\gl), charginos
($\tilde{\chi}^{\pm}_{i}$, with $i$ = 1,2) and neutralinos
($\tilde{\chi}^{0}_{i}$ with $i$ = 1,2,3,4)), all with identical quantum numbers to their SM partners, except spin.
Since no superpartner of any of the SM particles has been observed yet,
SUSY must be a broken symmetry at some higher energy scale.

The discovery (or exclusion) of weak-scale SUSY is one of the highest
physics priorities for the LHC. The primary target for early
supersymmetry searches in proton-proton ($pp$) collisions at
a centre-of-mass energy of 13~TeV at the LHC, given their large
expected cross-section, is the strong production of gluinos and squarks.
In order for supersymmetry to provide a solution to the hierarchy problem of the SM, the supersymmetric partners of the top and bottom quarks are 
expected to be light and within reach of the LHC. 

Under the hypothesis of $R$-parity conservation
\cite{Fayet:1976et,Fayet:1977yc,Farrar:1978xj,Fayet:1979sa}, SUSY
partners are produced in pairs and decay to the Lightest
Supersymmetric Particle (LSP) which is stable and, in large variety of
models, is assumed to be the lightest neutralino (\neut) which escapes detection.
The undetected $\neut$ would result in substantial missing
transverse momentum ($\met$), while the rest of the cascade, originating from the decays of
squarks and gluinos, would yield final states with multiple jets and
possibly leptons.

In this analysis, events containing multiple jets and either two leptons (electrons or muons) 
of the same electric charge (same-sign leptons, SS) or at least three leptons (3L) 
are used to search for strongly produced 
supersymmetric particles. Signatures with SS or 3L are predicted in many SUSY scenarios.
Gluinos produced in pairs or in association with a squark  
can lead to SS signatures when decaying to any final state that includes leptons because gluinos are Majorana fermions.
Squark production, directly in pairs
or through $\gluino \gluino$ or $\gluino \tilde q$  production with 
subsequent $\gluino \to q\tilde q$ decay, can also lead to
SS or 3L signatures when the squarks decay in cascades involving top quarks ($t$), charginos, 
neutralinos or sleptons, which subsequently decay as $t\to bW$,
$\tilde\chi^{\pm}_i\to W^{\pm (*)}\tilde\chi^0_j$,
$\tilde\chi^0_i\to h/Z^{(*)} \tilde\chi^0_j$, or
$\tilde\ell\to \ell\ninoone$, respectively.
Since this search benefits from low 
SM backgrounds, it allows the use of relatively loose kinematic
requirements on
$\met$, increasing the sensitivity to scenarios with
small mass differences between SUSY particles (compressed scenarios)
or where $R$-parity is violated. This search is thus sensitive to a wide variety 
of models based on very different assumptions.

This note presents the analysis of the 2015 dataset. 
Previous Run-2 preparation studies performed with DC14 samples can be found in~\cite{NoteDC14}. 
The Run-1 search for strongly produced SUSY particles with SS/3L 
was conducted using data from the full 2012 data-taking period (20.3 \ifb\ at $\rts$=8 TeV)~\cite{paperSS3L,noteSS3L}. 
In that result, exclusion limits were placed on 15 different models.
Gluino-mediated top squark scenarios, favoured by naturalness arguments, were excluded
for $m_{\gluino}<[600\mbox{--}1000] \GeV$, largely independently of the top
squark mass and decay mode. Similar limits were placed on gluino-mediated production of
first- and second-generation quarks for $m_{\ninoone}<[300\mbox{--}600] \GeV$. Limits were
also placed on pair-production  of bottom squarks and squarks of the first
and second generations decaying in long cascades. 
In this new set of results using 2015 data, no significant excess is observed, 
and we extend the Run-1 exclusion limits for several SUSY models. 

%This note summarizes the studies performed by the analysis team with DC14 Monte Carlo samples in preparation for Run-2. 
%The Run-1 search for strongly produced SUSY particles with SS/3L was conducted using data from the full 2012 data-taking period (20.3 \ifb\ at $\rts$=8 TeV)~\cite{paperSS3L,noteSS3L}. In that result, exclusion limits were placed on 15 different models.
%Gluino-mediated top squark scenarios, favoured by naturalness arguments, were excluded
%for $m_{\gluino}<[600\mbox{--}1000] \GeV$, largely independently of the top
%squark mass and decay mode. Similar limits were placed on gluino-mediated production of
%first- and second-generation quarks for $m_{\ninoone}<[300\mbox{--}600] \GeV$. Limits were
%also placed on pair-production  of bottom squarks and squarks of the first
%and second generations decaying in long cascades.

