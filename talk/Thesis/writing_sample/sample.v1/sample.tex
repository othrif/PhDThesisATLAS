% \documentclass[linenumbers,summarypage,hyperlinks]{outhesis}
\documentclass{outhesis}


% For a bibliography style, you must have the appropriate .bst file
% \bibliographystyle{apj}
\bibliographystyle{prsty}

% Provide the correct margins
\usepackage[top=1in, bottom=1in, left=1.6in, right=1.2in]{geometry}
% If you want a double-sided copy for yourself, uncomment the next line
 % \usepackage[twoside,top=1in, bottom=1in, left=1.6in, right=1.2in]{geometry}

\graphicspath{{/Users/othmane/Documents/Thesis/writing_sample/sample.v0/FIGURES/}}
\graphicspath{{logos/}{figures/}}
\usepackage{atlasphysics}
\usepackage{susydefs}
\usepackage{hepparticles}
\usepackage{indentfirst}
\usepackage{subfigure}
\usepackage[makeroom]{cancel}
\usepackage{float}
\usepackage{setspace}
%\usepackage{subcaption}
\usepackage{subcaption}

\begin{document}

%% Place Dissertation information here
%% Follow the convention for the use of capital letters 
%% or else the font will not be formatted properly
\author{Othmane Rifki}
\university{UNIVERSITY OF OKLAHOMA}
\college{GRADUATE COLLEGE}
\department{HOMER L. DODGE DEPARTMENT OF PHYSICS AND ASTRONOMY}
\title{Searching for Supersymmetric Particles at the Large Hadron Collider using the ATLAS Detector}
\address{Norman, Oklahoma}
\yr{2016}
\dgname{DOCTOR OF PHILOSOPHY}
%% List your committee members here
\committee{{Dr. Brad Abbott, Chair}, {Dr. S. Lakshmivarahan, Outside Member}, Dr. Mike Strauss, Dr. Chung Kao, Dr. Eric Abraham}

%% Put your dedication here. This is completely optional. Delete it if you don't need it.
%\begin{dedication}
 % 
%\end{dedication}

%% Put your acknowledgements here. This is completely optional. Delete it if you don't need it.
%\begin{acknowledgements}
%  
% \end{acknowledgements}

%% Put your abstract here.

\frontmatter

\maketitle

\mainmatter

%\singlespacing


\section{Overview}
%\label{sec:intro}


As part of our quest to understand and describe the physical world we live in, human beings embarqued on a journey to search for the  most 
fundamental constituents of matter, describe their interactions, and predict their behavior. In short, we want to answer the question 
``what is matter made of?''
The last few decades represented a revolutionary period in the study of fundamental physics.  Several particle accelerators became operational 
and allowed us to disentangle the structure of matter down to a handful of constituents we call elementary particles. The 
most powerful of these accelerators is is the Large Hadron Collider (LHC) at CERN . 
The basic idea of the LHC and other accelerators is to collide particles, 
in our case protons,  at very high energies and take snapshots of these collisions. The goal is to guess the form of the interactions happening 
in each collision and compare the resulting theoretical predictions with the results of our snapshots of the collision, called the experimental data. 
This is one of the strategies that led us to formulate a mathematical framework that describes the elementary particles and their interactions
which is known as the Standard Model of elementary particles (SM). 
When subjected to experimental tests, the SM successfully describes three of the four fundamental forces: electromagnetic, weak, and strong interactions. On the other hand, the SM is not believed to be complete since it fails to explain a number of problems that are still facing today's physics community. First, the SM does not incorporate the fourth fundamental force of gravity. Moreover, It does not provide insight on the nature of the ``invisible'' matter that is holding galaxies together, which constitutes $\sim$ 26\% of the energy density of the universe and is known as \emph{Dark Matter}. In addition, the SM does not account for the different masses and mixing of the 12 leptons
 known as the \emph{flavor problem}, and the predominance of matter over antimatter. In order to solve these problems, searches for physics not accounted for by the SM have been pursued at the LHC. The aim is to conduct 
searches for new states of matter from the experimental data and compare it to theoretical models that predict the existence of these new states of matter.
 The experimental data used in 
our analysis has been collected by the ATLAS detector which is one of the most complex particle detectors ever designed. ATLAS takes snapshots of the 
collisions that are happening 40 million times per second and reconstruct what happened in the collision. We analyze this information to test 
if the theoretical models that we are considering are compatible with our experimental data. One of the theories that predict the existence of 
new states of matter is supersymmetry. Supersymmetry is based on a fundamental symmetry between the elementary particles and the particles that 
mediate their interactions. \\

In the remainder of this document, I present an anlysis of data collected by the ATLAS experiment with the aim of comparing it with theoretical predictions 
based on supersymmetric theories. The level of detail is intended for the professional scientific audience. 


%-------------------------------------------------------------------------------
\section{Introduction}
\label{sec:intro}
%-------------------------------------------------------------------------------

Supersymmetry (SUSY)~\cite{Golfand:1971iw,Volkov:1973ix,Wess:1974tw,Wess:1974jb,Ferrara:1974pu,Salam:1974ig} is one of the most popular extension
of the Standard Model (SM). A general review can be found in Ref.~\cite{Martin:1997ns}. In its minimal realisation 
(the MSSM)~\cite{Fayet:1976et,Fayet:1977yc} it predicts a new bosonic (fermionic) partner for each fundamental SM fermion (boson), 
as well as an additional Higgs doublet. If $R$-parity~\cite{Farrar:1978xj} is conserved (RPC) the lightest supersymmetric particle (LSP) is 
stable and is typically the lightest neutralino\footnote{The SUSY partners of the Higgs and electroweak gauge bosons mix to form
the mass eigenstates known as charginos ($\tilde{\chi}^{\pm}_{l}$, $l = 1, 2$ ordered by increasing mass) and neutralinos 
($\tilde{\chi}^{0}_{m}$, $m = 1, \ldots, 4$ ordered by increasing mass).} \ninoone. In many models, the LSP can be a viable dark 
matter candidate~\cite{Goldberg:1983nd,Ellis:1983ew} and produce collider signatures with large missing transverse momentum at the Large Hadron 
Collider (LHC). On the contrary, if $R$-parity is violated (RPV), high jet and lepton multiplicity events are expected. 
Both RPC and RPV scenarios can match the final state signature considered in this article.

In order to address the SM hierarchy problem with SUSY models~\cite{Sakai:1981gr,Dimopoulos:1981yj,Ibanez:1981yh,Dimopoulos:1981zb}, 
TeV-scale masses are required~\cite{Barbieri:1987fn,deCarlos:1993yy} for the partners of the gluons (gluinos~$\tilde{g}$) 
and of the top quark chiral degrees of freedom (top squarks \stopL and \stopR), due to the large top Yukawa coupling.\footnote{The 
partners of the left-handed (right-handed) quarks are labelled $\tilde{q}_{L(R)}$. In the case where there is significant $L$/$R$ 
mixing (as is the case for third generation squarks) the mass eigenstates of these squarks are labelled $\tilde{q}_{1,2}$ ordered 
in increasing mass.} The latter also favours significant $\stopL$--$\stopR$ mixing, 
so that the lighter mass eigenstate $\stopone$ is in many scenarios lighter than the other squarks~\cite{Inoue:1982pi,Ellis:1983ed}. 
Bottom squarks ($\tilde{b}$) may also be light, being bound to top squarks by $SU(2)_{\rm L}$ invariance. 
This leads to potentially large production cross-sections 
for gluino pairs ($\gluino\gluino$), top--antitop squark pairs ($\stopone\stoponebar$) and bottom--antibottom squark pairs 
($\sbottomone\sbottomonebar$) at the LHC~\cite{Borschensky:2014cia}. 
Production of isolated leptons may arise in the cascade decays of those superpartners to SM quarks and neutralinos~\ninoone, 
via intermediate neutralinos~$\tilde\chi^0_{2,3,4}$ or charginos~$\tilde\chi^\pm_{1,2}$ 
that in turn lead to $W$, $Z$ or Higgs bosons, or to lepton superpartners (sleptons). 
Lighter third-generation squarks would also enhance gluino decays to top or bottom quarks 
over the generic decays involving light-flavour squarks,
favouring the production of heavy flavour quarks and, in the case of top quarks, additional leptons. 

This article presents a search for SUSY in final states with two leptons (electrons or muons) of the same electric charge 
(referred to as same-sign (SS) leptons) or three leptons (3L), jets and in some cases also missing transverse momentum 
(whose magnitude is referred to as \met). 
It is an extension of an earlier search performed by ATLAS with $\sqrt s=13$ TeV data~\cite{paperSS3L}, 
and uses the data collected by the ATLAS experiment~\cite{PERF-2007-01} in proton--proton ($pp$) collisions during 2015 and 2016.  
A similar search for SUSY in this topology was also performed by the CMS Collaboration at $\sqrt s=13$~TeV~\cite{Khachatryan:2016kod, Khachatryan:2017qgo}.
While the same-sign leptons signature is present in many scenarios of physics beyond the SM (BSM), 
SM processes leading to such final states have very small cross-sections. 
Compared to other BSM searches, analyses based on same-sign leptons or three lepton therefore allow 
the use of looser kinematic requirements (for example, on \met or the momentum of jets and leptons), 
adopting sensitivity to scenarios with small mass differences between gluinos/squarks and the LSP, or in which 
$R$-parity is not conserved.%~\cite{paperSS3L}. 

The sensitivity to a wide range of BSM physics processes is illustrated by the interpretation of the results  
in the context of twelve different SUSY simplified models~\cite{Alwall:2008ve,Alwall:2008ag,Alves:2011wf} 
that may lead to same-sign or three-lepton signatures. 
For RPC models, the first four scenarios focus on gluino pair production with generic decays into on-shell (Fig.~\ref{fig:feynman_gtt}) 
or off-shell (Fig.~\ref{fig:feynman_gttOffshell}) top quarks or light quarks. The latter are accompanied by a cascade decay involving 
a $\chinoonepm$ and a $\ninotwo$ (Fig.~\ref{fig:feynman_gg2WZ}) or a $\ninotwo$ and light sleptons (Fig.~\ref{fig:feynman_gg2sl}). 
The other two RPC scenarios target the direct production of third generation squark pairs 
with subsequent electroweakino-mediated (Fig.~\ref{fig:feynman_b1b1} and ~\ref{fig:feynman_t1t1}). 
The former is a generic same-sign lepton search for bottom squark production. The latter, addressed here by looking for a three same-sign lepton final state, 
is a tentative model that could explain the excesses seen in same-sign signatures during Run1~\cite{Huang:2015fba}.
Finally, a low fine-tuned generic SUSY model, the non-universal Higgs model with two extra parameters (NUHM2)~\cite{Ellis:2002iu,Ellis:2002wv}, 
is also considered. 


In the case of non-zero RPV couplings in the baryonic sector ($\lambda''_{ijk}\neq 0$),
as proposed in minimal flavor violation scenarios~\cite{Nikolidakis:2007fc,Smith:2008ju,Csaki:2011ge},
gluino and squarks may decay directly to top quarks, leading to final states with same-sign leptons~\cite{Durieux:2013uqa,Berger:2013sir} 
and $b$-tagged jets (Figs.~\ref{fig:feynm_rpv_gl313},~\ref{fig:feynm_rpv_gl321}).
Alternatively a gluino decaying to a neutralino LSP, that further decays to SM particles via a non-zero $\lambda'$ or $\lambda''$ RPV coupling, 
is also possible (Figs.~\ref{fig:feynm_rpv_glprime},~\ref{fig:feynm_rpv_gl112}).  
Lower $\met$ is expected in these scenarios, as there is no stable LSP, and the only \met originates from neutrinos produced in the top quark decays.
Pair-production of right-handed\footnote{These RPV baryon-number-violating couplings only couple to $SU(2)$ singlets.} 
like-sign down squarks (Figs.~\ref{fig:feynm_rpv_sd313},~\ref{fig:feynm_rpv_sd321}) are also considered.
In all of these scenarios, antisquarks decay into the charge-conjugate final states of those indicated for the corresponding squarks, 
and gluinos decay with equal probabilities into the given final state or its charge conjugate.

After describing the experimental apparatus~(Section~\ref{sec:detector}) and
the simulated event samples~(Section~\ref{sec:dataMC}), the 19 signal regions (SRs) designed to achieve 
good sensitivity for the 12 SUSY scenarios of Fig.~\ref{fig:feynman} are presented (Section~\ref{sec:selection}).
Section~\ref{sec:bkg} describes the estimation of the contribution from SM processes to the signal regions, 
validated by comparisons with data in dedicated regions. Systematics are described in Section~\ref{sec:syst}. 
The results are presented in Section~\ref{sec:result} together with the statistical procedure used to interpret 
the results in the context of the SUSY benchmark scenarios. Finally, Section~\ref{sec:conclusion} summarises the main conclusions.

%-------------------------------------------------------------------------------
\section{The ATLAS detector}
\label{sec:detector}
%-------------------------------------------------------------------------------

The ATLAS experiment~\cite{PERF-2007-01} is a multi-purpose particle detector with a forward-backward symmetric cylindrical
geometry and nearly $4\pi$ coverage in solid angle.\footnote{ATLAS uses
  a right-handed coordinate system with its origin at the nominal
  interaction point (IP) in the centre of the detector and the
  $z$-axis along the beam pipe. The $x$-axis points from the IP to the
  centre of the LHC ring, and the $y$-axis points upward. Cylindrical
  coordinates ($r$, $\phi$) are used in the transverse plane, $\phi$
  being the azimuthal angle around the beam pipe. The pseudorapidity
  is defined in terms of the polar angle $\theta$ as $\eta = -\ln
  \tan(\theta/2)$. Rapidity is defined as $y=0.5 \ln\left[(E + p_z )/(E - p_z )\right]$ 
  where $E$ denotes the energy and $p_z$ is the component of the momentum along the beam direction. 
  The transverse momentum \pt, the transverse energy \et and the missing transverse momentum \met 
  are defined in the $x-y$ plane.}
The interaction point is surrounded by an inner detector (ID), a
calorimeter system, and a muon spectrometer (MS).

The ID provides precision tracking of charged particles with
pseudorapidities $|\eta| < 2.5$ and is surrounded by a superconducting solenoid providing a \SI{2}{T} axial magnetic field.
It consists of pixel and silicon-microstrip detectors inside a
transition radiation tracker. One significant upgrade for the $\sqrt{s}=13$~TeV running period is the presence of the
Insertable B-Layer~\cite{CERN-LHCC-2010-013}, an additional pixel layer close to the interaction point, which 
provides high-resolution hits at small radius to improve the tracking performance.

In the pseudorapidity region $|\eta| < 2.5$, high-granularity lead/liquid-argon 
electromagnetic sampling calorimeters are used.
A steel/scintillator tile calorimeter measures hadron energies for $|\eta| < 1.7$.
The endcap and forward regions, spanning $1.5<|\eta| <4.9$, are 
instrumented with liquid-argon calorimeters 
for both the electromagnetic and hadronic measurements. 

The MS consists of three large superconducting toroids
with eight coils each, 
a system of trigger and precision-tracking chambers, 
which provide triggering and tracking capabilities in the
ranges $|\eta| < 2.4$ and $|\eta| < 2.7$, respectively.

A two-level trigger system is used to select events~\cite{Aad:2012xs,ATL-DAQ-PUB-2016-001,Aaboud:2016leb}. The first-level
trigger is implemented in hardware. This is followed by the software-based High-Level Trigger stage,
which can run offline-like reconstruction, reducing the event rate to about \SI{1}{kHz}.





\end{document}


