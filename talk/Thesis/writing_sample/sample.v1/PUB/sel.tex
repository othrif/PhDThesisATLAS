Candidate events are required to have a reconstructed vertex~\cite{ATL-PHYS-PUB-2015-026}, 
with at least two associated tracks with $\pt >400$~MeV, 
and the vertex with the largest $\Sigma \pt^2$ of the associated tracks is chosen as the primary vertex of the event.

For the data-driven background estimations, two categories of electrons and muons are used: 
``candidate'' and ``signal'' (the latter being a subset of the ``candidate'' leptons satisfying tighter selection criteria). 
Electron candidates are reconstructed from energy depositions in the electromagnetic calorimeter that have been matched to an ID track 
and are required to have $|\eta|<2.47$, a transverse momentum $\pT>\SI{10}{\GeV}$,
and to pass a Loose likelihood-based identification requirement~\cite{ATLAS-CONF-2016-024}. 
The likelihood input variables include measurements of calorimeter shower shapes and track properties from the ID. 
Candidates within the transition region between the barrel and endcap electromagnetic calorimeters,
$1.37<|\eta|<1.52$, are removed. The track matched with the electron must have a significance of the transverse impact parameter 
with respect to the reconstructed primary vertex, $d_0$, of $\vert d_0\vert/\sigma(d_0) < 5$.
Muon candidates are reconstructed in the region $|\eta|<2.5$ 
from muon spectrometer tracks matching ID tracks.
All muon candidates must have $\pT>\SI{10}{\GeV}$ and must pass the Medium identification requirements~\cite{Aad:2016jkr}, 
based on selections on the number of hits in the different ID and muon spectrometer subsystems, 
and the significance of the charge-to-momentum ratio $q/p$~\cite{Aad:2016jkr}.

Jets are reconstructed with the anti-$k_t$
algorithm~\cite{Cacciari:2008} with radius parameter $R=0.4$, using three-dimensional topological energy
clusters in the calorimeter~\cite{PERF-2014-07} as input. 
All jets must have $\pT>\SI{20}{\GeV}$ and $|\eta|<2.8$.
Jets are calibrated as described in Ref.~\cite{ATLAS-PHYS-PUB-2015-015}.
Furthermore, for all jets the expected average energy contribution from
pile-up is subtracted according to the jet area~\cite{ATLAS-PHYS-PUB-2015-015}.
In order to reduce the effects of pile-up, jets with $\pt<\SI{60}{GeV}$ and $|\eta|<2.4$ are required to have a significant fraction of their 
associated tracks originating from the primary vertex, 
as defined by the jet vertex tagger~\cite{ATLAS-CONF-2014-018}. 

Identification of jets containing $b$-hadrons ($b$-tagging) is performed with the MV2c10 algorithm, 
a multivariate discriminant making use of track impact parameters 
and reconstructed secondary vertices~\cite{Aad:2015ydr,ATL-PHYS-PUB-2015-022}.
A requirement is chosen corresponding to a 70\% average efficiency 
for tagging $b$-jets in simulated $\ttbar$ events. 
The rejection factors for light-quark/gluon jets, $c$-quark jets and hadronically decaying $\tau$ leptons in simulated $\ttbar$ events 
are approximately 380, 12 and 54, respectively~\cite{ATL-PHYS-PUB-2015-022,ATL-PHYS-PUB-2016-012}. 
Jets with $|\eta|<2.5$ which satisfy the $b$-tagging requirement are identified as $b$-jets. 
To compensate for differences between data and MC simulation in the $b$-tagging efficiencies and mis-tagging rates, 
correction factors derived in data control regions are applied to the simulated samples~\cite{ATL-PHYS-PUB-2015-022}. 

After object identification, overlaps between the different objects are resolved. 
Any jet within a distance $\Delta R_y =\sqrt{(\Delta y)^2+(\Delta\phi)^2}$ = 0.2 of a lepton candidate is discarded, 
unless the jet is $b$-tagged\footnote{In this case the $b$-tagging operating point corresponding to an efficiency of 85\% is used.},
in which case the lepton is discarded since it is likely originating from a semileptonic $b$-hadron decay. 
Any remaining lepton within $\Delta R_y= \operatorname{min}\{0.4, 0.1 + \SI{9.6}{GeV}/\pt(\ell)\}$ of a jet is discarded. 
However, if the jet has fewer than three associated tracks or the muon \pt exceeds 70\% of the sum of the transverse momenta 
of the associated tracks and 50\% of the \pt of the jet, the muon is retained and the jet is discarded. This avoids 
inefficiencies for high-energy muons undergoing significant energy loss in the calorimeter. 
%Any calo-tagged muon\footnote{A calo-tagged muon is a muon identified solely by calorimeter based identification 
%(so no signal in MS) and are relevant only at $|\eta|<0.1$.} sharing an ID track with an electron is removed. 
Finally, any electron sharing an ID track with the remaining muons is also removed.


Electrons that are likely to be reconstructed with an incorrect charge assignment are rejected using electron
cluster and track properties: the track impact parameter, the track curvature significance, the cluster width and the
quality of the matching between the cluster and its associated track, both in terms of energy and position. These
variables, as well as the electron pt and eta,
are combined into a single classifier using a boosted decision tree (BDT). A selection requirement
on the BDT output is chosen to achieve a rejection factor between 7 and 8 for electrons with a wrong charge
assignment while selecting properly measured electrons with an efficiency of 97\%. Correction
factors to account for differences of the selection efficiency between data
and MC have been applied to the selected electrons in the MC simulation.

Signal electrons must satisfy a Medium likelihood-based identification requirement~\cite{ATLAS-CONF-2016-024}.
In region with high amount of material in the tracker, an electron may emit a hard brems\-strah\-lung photon which subsequently 
converted to an electron--positron pair, causing its charge to be mis-measured (later refered to as ``charge-flip''). 
To reduce the impact of charge mis-identification, electrons must satisfy $|\eta|<2.0$. Furthermore, electrons that 
are likely to be reconstructed with an incorrect charge assignment are rejected using electron
cluster and track properties: the track impact parameter, the track curvature significance, the cluster width and the
quality of the matching between the cluster and its associated track, both in terms of energy and position. These
variables, as well as the electron pt and eta, are combined into a single classifier using a boosted decision tree (BDT). 
A selection requirement on the BDT output is chosen to achieve a rejection factor between 7 and 8 for electrons 
with a wrong charge assignment while selecting properly measured electrons with an efficiency of 97\%. Correction
factors to account for differences of the selection efficiency between data and MC have been applied to the selected 
electrons in the MC simulation.
%Correction factors to account for differences of the selection efficiency between data 
%and MC have been applied to the selected electrons in the MC simulation.
Signal muons must fulfil the requirement of $\vert d_0\vert/\sigma(d_0) < 3$. 
The track associated with the signal leptons must have a longitudinal impact parameter with respect to the 
reconstructed primary vertex, $z_0$, satisfying $\vert z_0 \sin\theta\vert  < 0.5$~mm. 
Isolation requirements are applied to both the signal electrons and muons. 
The  scalar sum of the \pt of tracks within a variable-size cone around the lepton, 
excluding its own track, must be less than 6\% of the lepton \pt. 
The track isolation cone radius for electrons (muons) 
$\Delta R_\eta=\sqrt{(\Delta\eta)^2+(\Delta\phi)^2}$ 
is given by the smaller of $\Delta R_\eta = \SI{10}{~GeV}/\pt$ and $\Delta R_\eta = 0.2\,(0.3)$. 
That is, a cone of size $0.2\,(0.3)$ at low $\pt$ but narrower for high-\pt leptons. 
In addition, in the case of electrons the energy of calorimeter clusters in a cone of $\Delta R_\eta = 0.2$ around the electron 
(excluding the deposition from the electron itself) must be less than 6\% of the electron \pt. 
Simulated events are corrected to account for minor differences in the lepton trigger, reconstruction, 
identification and isolation efficiencies between data and MC simulation.

The missing transverse momentum is defined as the negative vector sum of the transverse momenta 
of all identified physics objects (electrons, photons~\cite{ATLAS-CONF-2012-123}, muons, jets) and an additional soft term. 
The soft term is constructed from all tracks that are not associated with any physics object, 
but that are associated with the primary vertex. 
In this way, the $\met$ is adjusted for the best calibration of the jets and the other identified physics objects above, 
while maintaining pile-up independence in the soft term~\cite{ATL-PHYS-PUB-2015-027, ATL-PHYS-PUB-2015-023}.

Events are selected using a combination (logical OR) of dilepton and $\met$ triggers, 
the latter being used only for events with $\met>\SI{250}{GeV}$. 
The trigger-level requirements on $\met$ and the leading and subleading lepton \pt are looser than those applied offline 
to ensure that trigger efficiencies are constant in the relevant phase space. 
The event selection requires at least two signal leptons with $\pt>20$~\GeV~(apart from two SRs where 
the second lepton \pt should be greater than 10~\GeV, see Table~\ref{tab:SRdef3}). 
If the event contains exactly two signal leptons, they must have the same electric charge. 
Events are discarded if they contain any jet failing basic quality selection criteria in order to 
reject detector noise and non-collision background events~\cite{ATLAS-CONF-2010-038}.

To maximise the sensitivity to the signal models of Figure~\ref{fig:feynman}, 
nineteen overlapping signal regions are defined as shown in Table~\ref{tab:SRdef3}, with requirements on the number of signal leptons 
($N_{\ell}^{\rm{signal}}$), the number of $b$-jets with $\pt>\SI{20}{\GeV}$ ($N_{b\rm{-jets}}$), 
the number of jets above a given \pt thresholds (25, 40 or \SI{50}{GeV}) regardless of their flavour ($N_{\rm{jets}}$), 
\met, the effective mass (\meff), defined as the scalar sum of the $\pt$ of the signal leptons and jets (regardless of their flavour) 
in the event plus the \met, and the charge of the leptons in the event. 
The values of acceptance times efficiency of the SR selections for the SUSY signal models 
typically range between \textcolor{red}{[UPDATE]}\% for models with a light \ninoone and \textcolor{red}{[UPDATE]}\% for models with a heavy \ninoone. 
\textcolor{red}{UPDATE: Check that this sentence makes sense when we have all numbers}.


\begin{table}[tbh!]
%\rotatebox{90}
\resizebox{\textwidth}{!}
{
\hspace{0.5cm}
\def\arraystretch{1.2}
\centering
\small
\begin{tabular}{|c|c|c|c|c|c|c|c|c|c|c}
\hline
Signal region  &  $N_{\rm{leptons}}^{\rm{signal}}$   & $N_{b\rm{-jets}}$ & $N_{\rm{jets}}$  & $p^{}_{\rm{T,jet}}$ & \met\ & \meff\ & \met/\meff  & Other & Targeted  \\
 Name          &                                  &                   &                  &    [GeV]             & [GeV] & [GeV] &   &  & Signal  \\
\hline\hline

%(\textit{ex-SR3b1}) 
Rpc2L3bS         & $\ge 2$SS  & $\ge 3$ & $\ge 6$ & $>25$ & $>150$ & --  & $>0.2$    & - 			        & Fig.~\ref{fig:feynman_gtt}\\ 
%(\textit{ex-SR3b2})
Rpc2L3bH         & $\ge 2$SS  & $\ge 3$ & $\ge 6$ & $>25$ & $>250$ & $>1200$  & -	  & - 			        & Fig.~\ref{fig:feynman_gtt}, NUHM2\\ 
\hline
%(\textit{ex-SRhigh})
Rpc2Lsoft1b    & $\ge 2$SS  & $\ge 1$ & $\ge 6$ & $>25$ & $>100$ &          & $>0.3$    & 20,10 $<$\ptlone,\ptltwo $<$ 100 GeV & Fig.~\ref{fig:feynman_gttOffshell}\\ 
%(\textit{ex-SRlow})
Rpc2Lsoft2b      & $\ge 2$SS  & $\ge 2$ & $\ge 6$ & $>25$ & $>200$ & $>600$   & $>0.25$   & 20,10 $<$\ptlone,\ptltwo $<$ 100 GeV & Fig.~\ref{fig:feynman_gttOffshell} \\ 
\hline
%(\textit{ex-SR0b1})
Rpc2L0bS         & $\ge 2$SS  & $=0$    & $\ge 6$ & $>25$ & $>150$ & $>600$   & $>0.25$   & - 				& Fig.~\ref{fig:feynman_gg2WZ}\\
%(\textit{ex-SR0b2}) 
Rpc2L0bH         & $\ge 2$SS  & $=0$    & $\ge 6$ & $>40$ & $>250$ & $>900$   & -	  & - 				& Fig.~\ref{fig:feynman_gg2WZ}\\
\hline
%(\textit{ex-SR3L0b1})
Rpc3L0bS       & $\ge 3$    & $=0$    & $\ge 4$ & $>40$ & $>200$ &  --   & -	  & - 				& Fig.~\ref{fig:feynman_gg2sl}\\ 
%(\textit{ex-SR3L0b2})
Rpc3L0bH       & $\ge 3$    & $=0$    & $\ge 4$ & $>40$ & $>200$ & $>1600$  & -	  & - 				& Fig.~\ref{fig:feynman_gg2sl}\\
%(\textit{ex-SR3L1b1})
Rpc3L1bS       & $\ge 3$    & $\ge 1$ & $\ge 4$ & $>40$ & $>200$ & $>600$   & -         & - 				& Fig.~\ref{fig:feynman_gg2sl} ?? \\ 
%(\textit{ex-SR3L1b2})
Rpc3L1bH       & $\ge 3$    & $\ge 1$ & $\ge 4$ & $>40$ & $>200$ & $>1600$  & -	  & - 				& Fig.~\ref{fig:feynman_gg2sl} ?? \\
\hline
%(\textit{ex-SR1b1})
Rpc2L1bS         & $\ge 2$SS  & $\ge 1$ & $\ge 6$ & $>25$ & $>150$ & $>600$   & $>0.25$   & - 				& Fig.~\ref{fig:feynman_b1b1}\\
%(\textit{ex-SR1b2})
Rpc2L1bH         & $\ge 2$SS  & $\ge 1$ & $\ge 6$ & $>25$ & $>250$ & $>1250$  & $>0.2$    & - 				& Fig.~\ref{fig:feynman_b1b1}\\ 
\hline
%(\textit{ex-SR1b-3LSS})
Rpc3LSS1b    & $\ge \ell^\pm\ell^\pm\ell^\pm$ & $\ge 1$ & - & -    & -  & -        & -& veto 81$<$\mee$<$101 GeV 	& Fig.~\ref{fig:feynman_t1t1}\\ 
\hline
%(\textit{ex-SR1b-GG})
Rpv2L1bH       & $\ge 2$SS  & $\ge 1$ & $\ge 6$ & $>50$ & -      & $>2000$  & -         &  - 				& Figs.~\ref{fig:feynm_rpv_gl313},\ref{fig:feynm_rpv_gl321}\\
%(\textit{ex-SRRPV0b})
Rpv2L0b        & $=2$SS     & $=0$    & $\ge 6$ & $>40$ & -      & $>1800$  & -         &  veto 81$<$\mee$<$101 GeV 		& Fig.~\ref{fig:feynm_rpv_glprime} \\
%(\textit{ex-SRRPV3b})
Rpv2L3bH       & $\ge 2$SS  & $\ge 3$ & $\ge 6$ & $>40$ & -      & $>1800$  & -        & veto 81$<$\mee$<$101 GeV		& Fig.~\ref{fig:feynm_rpv_gl112} \\
%(\textit{ex-SR3b-DD})
Rpv2L3bS       & $\ge \ell^-\ell^-$   & $\ge 3$ & $\ge 3$ & $>50$ & -      & $>1200$   & -                           & & Fig.~\ref{fig:feynm_rpv_sd313}\\
%(\textit{ex-SR1b-DD-low})
Rpv2L1bS   & $\ge \ell^-\ell^-$  & $\ge 1$ & $\ge 4$ & $>50$ & -      & $>1200$  & -         & -  			        & Fig.~\ref{fig:feynm_rpv_sd321}\\
%(\textit{ex-SR1b-DD-high})
Rpv2L1bM  & $\ge \ell^-\ell^-$  & $\ge 1$ & $\ge 4$ & $>50$ & -      & $>1800$  & -         & - 			        & Fig.~\ref{fig:feynm_rpv_sd321}\\
\hline
\end{tabular}
}
\caption{Summary of the signal region definition. Unless explicitely stated in the table, at least two signal leptons with 
$\pt>$20 GeV and same charge (SS) are required in each signal region. Requirements 
are placed on the number of signal leptons ($N_{\rm{lept}}^{\rm{signal}}$), the number of jets ($N_{\rm{jets}}$) or the number 
of $b$-jets with $\pt>\SI{20}{GeV}$ ($N_{b\rm{-jets}}$), \pt of the lepton or jet, \met, \meff\ or \met/\meff. The last column indicates the 
targeted signal model.}
\label{tab:SRdef3}
\end{table}

%The SR3L1-SR3L2 and SR0b1-SR0b2 signal regions are sensitive to squarks of the first and second generations directly produced or appearing in gluino decays in RPC models, 
%leading to final states particularly rich in leptons (Fig.~\ref{fig:feynman_gg2sl}) or in jets (Fig.~\ref{fig:feynman_gg2WZ}), 
%but with no enhancement of the production of $b$-quarks. 
%Third-generation squark RPC models resulting in final states with two $b$-quarks, 
%such as direct bottom squark production (Fig.~\ref{fig:feynman_b1b1}), are targeted by the SR1b signal region. 
%Finally, the signal region SR3b targets production of gluinos decaying via a top squark resulting in final states with four $b$-quarks (Fig.~\ref{fig:feynman_gtt}). 

%Signal regions targeting RPV models contain no $\met$ requirement. 
%The SR1b-DD and SR3b-DD signal regions are sensitive to direct down squark production 
%with a $\lambda''_{321}$ (Fig.~\ref{fig:feynm_rpv_sd321}) and $\lambda''_{313}$ (Fig.~\ref{fig:feynm_rpv_sd313}) coupling, respectively. 
%Models featuring gluino production decaying via top squark including a $\lambda''_{313}$ (Fig.~\ref{fig:feynm_rpv_gl313}) or $\lambda''_{321}$ (Fig.~\ref{fig:feynm_rpv_gl321}) coupling are explored with the SR1b-GG signal region.

