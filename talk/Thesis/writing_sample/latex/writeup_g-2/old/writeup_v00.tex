
\documentclass[]{article}
\addtolength{\oddsidemargin}{-.875in}
\addtolength{\evensidemargin}{-.875in}
\addtolength{\textwidth}{1.75in}
\addtolength{\topmargin}{-.875in}
\addtolength{\textheight}{1.75in}
\include{math}



\begin{document}
\thispagestyle{empty}
\pagestyle{empty}


\begin{title}
\Large{Measurement of the Muon Magnetic Moment anomaly by the g-2 Experiment at Fermilab}
\end{title}


\begin{abstract}
Every particle has an intrinsic magnetic moment due to its spin angular momentum characterized by a constant g, called the gyromagnetic ratio, that is very close to the value 2. The experimental measurement of this quantity to a very high accuracy has made it one of the most precisely measured quantities in particle physics. This measurement when compared with the theoretical predictions of the Standard Model of particle physics shows a difference in the order of 3 standard deviation which suggests the possibility of new physics. The goal of this review is to describe the measurements done in the past, and the improvements that the Fermilab g-2 experiment will achieve.

\end{abstract}


Plan:\\
Introduction\\
Why g-2 theory\\
Measurement of g-2\\
Improvements with fermilab\\
Possiblities: if theory and experiment are different ... \\


\section*{Introduction}

The projects main areas are: accelerator, storage ring, and detectors.\\
A team responsible for the disassembly and transport of E821 equipment from BNL to Fermilab. g-2 uses existing elements used in the equivalent experiment at BNL. Such parts are 
Accelerator: high-purity muon beam of 3.1GeV/c to g-2\\
Storage Rings: The measurement of the magnetic field via NMR and calibrations should be better than 0.1ppm with an averaged uncertainty of 0.07ppm.  \\
Detectors: The detectors need to measure the spin precession of the muon to a statistical error of 0.1 ppm and systematics on $\omega_a$ to 0.07ppm. \\
g-2 uses superconducting coils from E821, the Brookhaven experiment. These 50 ft in diameter coils had to be transported from Brookhaven in Long Island to Fermilab near Chicago via ground transportation and a barge that travelled along the Atlantic cost, the Gold of Mexico, then the Illinois Waterway to the point of closest approach to the lab.\\
Muon g-2 data in 2016\\






\section*{Chapter 1: }
\subsection*{}
\subsection*{}
\section*{Chapter 2: }
\subsection*{}
\subsection*{}
\section*{Conclusion}
\section*{Appendix}



\begin{enumerate}
\item {Appendix 1:}
\item {Appendix 2:}
\item {Appendix 3:}
\end{enumerate}

\section*{Terminology}
wbs work breakdown structure\\
storage ring: circular particle accelerator that maintains particles at the same energy for a long period of time. Radio-frequency cavities are used to replace the lost energy.
cryogenic: production of very low temperature and the behavior of material in them\\
Septum (pl septa): A magnetic or electrostatic device used to deflect charged particles along one of two paths.//
Inflector : A magnetic or electrostatic device to apply a transverse force to a beam. Most often the term is applied to the pulsed septa which bend the injected beam onto the equilibrium orbit of a circular accelerator. \\
yoke ?
\begin{enumerate}
\item Magnet
\item Inflector
\item Storage ring Vacuum
\item Kickers: electromagnetic device that place the injected muon beam onto a central orbit. 
\item Quadrupoles: a magnet consisting of four poles that focuses a beam in one plane. 
\item Collimator : A movable, solid block of material which can be used to limit beam size or to stop it altogether. 

\end{enumerate}



\end{document}