The reconstructed objects (leptons, photons, $b$-jets, etc.) in a collision event are used to perform a wide range of SM measurements 
or searches for evidence of BSM physics. The assumption is that these objects are `real` representing the desired particles 
in the final state used in the analysis. 
In practice, the reconstructed objects might not be always `real'. In fact, they may be something completely different that
were mistakenly reconstructed as the desired objects, called `fake' objects.
While these occurrences are rare, they do affect some analyses more than others.
The analysis presented in this dissertation is highly affected by 
 `fake' leptons. 
To illustrate the problem, a hadronic jet may deposit more energy in the electromagnetic calorimeter than the hadronic calorimeter, 
or it may leave a narrow deposit of energy, leading the reconstruction algorithms to mistake this jet for an electron.
From the analysis point of view, the `fake' electron will pass all the selection criteria and will be indistinguishable from 
a `real' electron. 
It is important for the analysis that it requires a reconstructed electron to model the fake electron background to get a sound 
result. This example was given with electrons, but can be generalized to muons as well. 
In short, any analysis that uses leptons in the final state must account for the `fake' lepton background. 
This background can be more or less important depending on the detector, the analysis selection, and the number of leptons required. 
To estimate this background it is important to first understand what type of processes lead to fake leptons.
