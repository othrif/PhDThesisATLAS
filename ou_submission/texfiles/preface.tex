Over the past five years, 
I have always been asked by family, friends, 
and people I meet, what do you do.
I have written the introduction of Chapter 1 with a general audience in mind to answer this
question.

I conducted  my research in order to  better understand the fundamental interactions of elementary particles 
and to search for unknown physics phenomena by studying proton-proton collisions recorded by the ATLAS detector at the world's
most powerful particle accelerator, the Large Hadron Collider (LHC) at CERN.
I have performed this work over five years while being part of the  experimental particle physics group at the University of Oklahoma.
I spent one year at Argonne National Laboratory as an ATLAS Support Center fellow and I spent the last two years of my PhD at CERN. 

During my PhD, I contributed to two standard model measurements with the LHC Run-1 data of proton-proton collisions at 
a center-of-mass energy of 8 \TeV.
I performed the higher order calculations of the photon production rate which was found to agree with data over ten orders of 
magnitude. The result helped reduce the uncertainties associated with the dynamic structure of the proton impacting
every physics result at the LHC \cite{paper-2015-Photon}.
I also participated in the analysis that made the first experimental observation of the  associated production of a top quark pair
and a vector boson \cite{paper-2015-ttV}, an important background for the search of unknown physics that I present in this dissertation.

The search for supersymmetry is one of the highest priorities of the 
LHC program. The LHC experiments has carried out a 
vigorous search program to analyze the fast incoming data at the higher 
center of mass energy of 13 \TeV. 
The main part of my research is in analyzing the LHC proton-proton collision data at 13 \TeV~
in final states with two same sign leptons or three leptons and jets to search for supersymmetry
which has been the subject of several publications \cite{Aaboud:2017dmy,conf-2016-SS3L,paper-2015-SS3L,conf-2015-SS3L}. 
In this dissertation, I describe in detail the work I performed in searching for supersymmetry in this final state.
I present my own work, except where explicit reference is made to the work of others.
It is worth noting that the regulations of the ATLAS collaboration require me to include 
the ``ATLAS'' label in the plots that have used ATLAS data or ATLAS simulation. 


The conclusions of this search provide relevant results to help guide the particle physics 
community in setting constraints on a large variety of supersymmetric models. 
This search, along with other complementary searches performed by ATLAS, has
put stringent constraints on the masses of the strongly produced 
supersymmetric particles.


In addition to analyzing the physics data, I have carried out other projects as part of the ATLAS collaboration.
I participated in the evolution of the Region of Interest Builder (RoIB), a system that processes 
every event recorded by ATLAS \cite{pcroib_orifki}. 
I increased the ATLAS data acquisition system flexibility and reduced the operational overload associated 
with custom electronics 
by migrating the functionality of the RoIB from a custom multi-card crate of VME-based electronics to a single custom PCI-Express 
(PCIe) card in a commodity-computer. I have helped install the new system in ATLAS at the start of 2016.
This new system has been used to collect all the data analyzed in this dissertation.

I have also insured the good operation of the ATLAS detector by actively participating in data-taking and fulfilling various 
supporting roles for the ATLAS data acquisition system. I have taken shift leader shifts at the ATLAS control room 
where I was responsible for coordinating the activities of the different detectors of ATLAS to have ATLAS ready for collisions
and for communicating to the LHC operators the readiness of ATLAS for 
proton collisions.
I have also provided 
operational support as an on-call expert for two critical elements of the ATLAS data acquisition system; the RoIB and 
the readout system that buffers all the ATLAS data.

