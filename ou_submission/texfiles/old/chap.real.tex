\section{Real lepton efficiency}
\graphicspath{{figures/real_lepton_efficiency/}}
The real lepton efficiency parameter should be measured in a region dominated by real leptons, usually with the same topology as the real leptons dominating the signal regions. It can be a control region dominated by $Z\to\ell\ell$ or ``$W\to\ell\nu$'' or \ttbar, etc. events.
This section is dedicated to the real lepton efficiency measurement using the Tag and Probe method. 
The real efficiency is obtained by taking the ratio between the number of probe leptons passing the signal requirement 
($n_\mathrm{signal}$) and the number of probe candidate leptons ($n_\mathrm{candidate}$).
The estimated number of the background events ($N_\mathrm{candidate}^{bkg}$) associated with the low $p_\mathrm{T}$ leptons 
(as low as 20 \GeV) are considered and estimated using a background template method.
Equation~\ref{eq:RLE_efficiency_formula} shows the real lepton efficiency calculation.

\begin{equation}
\epsilon = \frac{N_{\mathrm{signal}}}{N_{\mathrm{candidate}} - N_{\mathrm{candidate}}^{bkg}}
\label{eq:RLE_efficiency_formula}
\end{equation}

%%%
\subsection{The $Z$ tag-and-probe method}
\label{subsubsec:RLE_ZTandP_method}

The leptons used for the efficiency measurements are extracted from data using the $Z$ tag-and-probe method.
Events are required to have at least two candidate leptons.
One of the two leptons, the tag lepton, is required to have a transverse momentum $\pt>25 \GeV$ and pass the signal lepton requirements
specified in Table ??.
In the case of the $Z\to ee$ events, an additional $|\eta|<2$ requirement is used for the tag and probe selection.
The probe lepton, which is used for the real efficiency measurements, is required to pass the candidate lepton requirements and have to 
carry opposite charge and same flavor with respect to the corresponding tag lepton. This requirement ensures the selection of 
real leptons to allow the real lepton efficiency measurement.
The tag and the probe leptons are required to match the single lepton triggers or dilepton triggers listed in 
Table~\ref{tab:RLE_single_lepton_triggers}.

\begin{table}[htbp]
\begin{center}
\begin{tabular}{cccc}
\hline
\hline
Trigger & lepton & 2015 & 2016\\
\hline
\multirow{2}{*}{Single lepton trigger} & electron & \texttt{e24\_lhmedium\_iloose\_L1EM20VH} & \texttt{e26\_lhtight\_nod0\_ivarloose}\\
& muon & \texttt{mu20\_iloose\_L1MU15} & \texttt{mu26\_ivarmedium}\\
\hline
\multirow{2}{*}{Dilepton trigger} & electron & \texttt{2e12\_lhloose\_L12EM10VH} & \texttt{2e17\_lhvloose\_nod0}\\
& muon & \texttt{mu18\_mu8noL1} & \texttt{mu22\_mu8noL1}\\
\hline
\hline
\end{tabular}
\end{center}
\caption{The list of single lepton and dilepton triggers used for the real lepton efficiency measurements.
The dilepton triggers are used for studying the systematic uncertainties causing by the trigger.}
\label{tab:RLE_single_lepton_triggers}
\end{table}

All possible combinations of the tag-and-probe pairs are considered to avoid any bias and to increase the statistics.
The invariant mass of the tag-and-probe pair system should satisfy $80<m_{\ell\ell}<100 \GeV$.
The data-to-MC comparison of the tag-and-probe pair invariant mass distributions are shown in Figure~\ref{fig:RLE_mll_distributions}, and indicates the need of subtracting the background, especially when the probe electron has $\pt<20 \GeV$.

\begin{figure}[htbp]
\includegraphics[width=0.48\textwidth]{signal_level_Mee_pt1015_ratio_plot_MC_normalized}
\includegraphics[width=0.48\textwidth]{signal_level_Mee_pt1520_ratio_plot_MC_normalized}\\
\includegraphics[width=0.48\textwidth]{signal_level_Mmumu_pt1015_ratio_plot_MC_normalized}
\includegraphics[width=0.48\textwidth]{signal_level_Mmumu_pt1520_ratio_plot_MC_normalized}
\caption{The distributions of the tag-and-probe pair invariant mass ($m_{\ell\ell}$) computed using $Z+jets$ MC and 2015 + 2016 data.
The blue color represents the $Z$ truth matched events, the red color stands for the $Z$ tag-and-probe events, and the black dots are data.
The top row shows the $m_{ee}$ computed with probe electrons in the $10<\pt<15 \GeV$ (left) and $15<\pt<20 \GeV$ (right) intervals.
The bottom row shows the $m_{\mu\mu}$ cases.
The MC distributions are scaled to the data using a Gaussian fit of the $Z$ mass peak ($85<m_{\ell\ell}<95 \GeV$).}
\label{fig:RLE_mll_distributions}
\end{figure}

The background contamination associated with these low $\pt$ electrons is estimated using a background template method similar to the one used by the $e/\gamma$ performance group for their efficiency measurements~\cite{ATLAS-CONF-2014-018}.
Because the real lepton efficiency is obtained by computing the ratio between the number of probe leptons passing the signal requirements ($N_{\textrm{signal}}$) and the number of probe leptons passing the candidate requirements ($N_{\textrm{candidate}}$), the estimated background contamination ($N_{\textrm{candidate}}^{bkg}$) in the candidate probe leptons needs to be subtracted.
As the background contamination is found to be negligible for signal leptons, no background subtraction is performed at the numerator.
Equation~\ref{eq:RLE_efficiency_formula} shows the real lepton efficiency calculation.

\begin{equation}
\epsilon = \frac{N_{\textrm{signal}}}{N_{\textrm{candidate}} - N_{\textrm{candidate}}^{bkg}}
\label{eq:RLE_efficiency_formula}
\end{equation}



\subsection{>Background subtraction}
\label{subsubsec:RLE_bkg_contamination}

The background contamination has been evaluated on data using a background template method.
A sample enriched in background is obtained by inverting the calorimeter isolation and track isolation cuts and requesting the electron object to fail the medium LH identification.
In order to asses a systematic on the background template definition, three variations of background template definitions are considered.
Table~\ref{tab:RLE_bkg_templates} summarizes the background template definitions and the Figure~\ref{fig:RLE_bkg_templates} shows the background template $m_{ee}$ distributions computed with probe electrons in $10<\pt<15 \GeV$ and $15<\pt<20 \GeV$, respectively.
The invariant mass distribution of the template events ($m_{ee}^{\textrm{template}}$) is then used to estimate the amount of background in the measurement region ($80<m_{\ell\ell}<100 \GeV$).

\begin{table}[htbp]
\begin{center}
\begin{tabular}{cccc}
\hline
\hline
cut & variation 1 template & candidate template & variation 2 template\\
\hline
Identification & - & fail medium LH & fail medium LH\\
Calorimeter isolation & $E_{\text T}^{topocone20} /\pt>6\%$ & $E_{\text T}^{topocone20} /\pt>15\%$ & $E_{\text T}^{topocone20} /\pt>20\%$\\
Track isolation & $p_{\text T}^{varcone20} /\pt>6\%$ & $E_{\text T}^{topocone20} /\pt>8\%$ & $E_{\text T}^{topocone20} /\pt>15\%$\\
\hline
\hline
\end{tabular}
\end{center}
\caption{The definition of the background templates used to estimate the background contamination associated with the $Z$ tag-and-probe method.
The candidate template is used to estimate the background contamination and the variation 1 and 2 templates, which have looser and tighter requirements, are used to assess the systematic caused by the background contamination.}
\label{tab:RLE_bkg_templates}
\end{table}

\begin{figure}[htbp]
\includegraphics[width=0.48\textwidth]{bkg_template_electron_pt_1015_eta0201.pdf}
\includegraphics[width=0.48\textwidth]{bkg_template_electron_pt_1520_eta0201.pdf}
\caption{The $m_{ee}$ distributions of the three different background templates.
The $m_{ee}$ distributions in the left and right plots are computed using probe electrons with $10<\pt<15 \GeV$ and $15<\pt<20 \GeV$, respectively.
Because the variation 1 template has looser calorimeter and track isolation requirements, we see a peak in the $Z$ mass region.
Compared with the variation 1 template, the candidate and the variation 2 templates have tighter cuts so no peak can be found in the $Z$ mass region.}
\label{fig:RLE_bkg_templates}
\end{figure}

As the template cuts remove background objects, the background template has to be normalized to provide the correct background estimation.
The $120<m_{ee}<150 \GeV$ region is used for this normalization as a smaller prompt electron contribution is expected in this region.
The background in this tail region is estimated by integrating the candidate $m_{ee}$ distribution in the tail region after subtracting the prompt electron contamination which is estimated by integrating the $m_{ee}$ distribution in the tail region using the $Z\to ee$ MC simulation.
Equation~\ref{eq:RLE_bkg_in_the_tail} shows the estimation of the number of background events in the tail region using the candidate electrons.

\begin{equation}
N_{bkg}^{\textrm{tail}} = N_{\textrm{candidate}}^{\textrm{tail}} - N_{\textrm{MC, prompt}}^{\textrm{tail}}
\label{eq:RLE_bkg_in_the_tail}
\end{equation}

The candidate electron selection criteria already provides a relatively pure sample of prompt electrons.
Therefore, the background template suffers from low statistics in the tail region.
To avoid any bias in the normalization factor due to statistical fluctuations, the template is fitted using an exponential function.
The fitting range is $60<m_{ee}^{\textrm{template}}<120 \GeV$.
In order to minimize the prompt lepton contamination arising from $Z\to ee$ events, the $80< m_{ee}^{\textrm{template}}<100 \GeV$ region is excluded.
Because of the low statistics in the tail, the fit is mostly driven by the $60< m_{ee}^{\textrm{template}}<80 \GeV$.
After applying the fit, the template in the tail region, $N_{\textrm{template}}^{\textrm{tail}}$, is normalized to the background in the tail, $N_{bkg}^{\textrm{tail}}$, to get the correct estimated number of background.
Figure~\ref{fig:RLE_bkg_estimations} shows the candidate $m_{ee}$ distributions before and after applying the background subtraction using the background template.
The data after applying the background subtraction is compared to the MC simulations and the background template distribution and their corresponding fit are also shown.
The simulated $m_{ee}$ distribution of $Z\to ee$ MC are normalized to the data after background subtraction using a Gaussian fit in $85<m_{ee}<95 \GeV$ region.
The top and the bottom rows correspond to the $10<\pt<15 \GeV$ and $15<\pt<20 \GeV$, respectively.
The left, middle, and right columns correspond to the $0<|\eta|<0.8$, $0.8<|\eta|<1.37$, and $1.52<|\eta|<2.0$ regions, respectively.
After applying the background subtraction, the data agree with the MC simulation within the statistical uncertainties.

\begin{figure}[htbp]
\includegraphics[width=0.33\textwidth]{bkg_subtraction_baseline_template_range_baseline_mll80_100_pt10_15_eta0_80_tag_trigger_matched.pdf}
\includegraphics[width=0.33\textwidth]{bkg_subtraction_baseline_template_range_baseline_mll80_100_pt10_15_eta80_137_tag_trigger_matched.pdf}
\includegraphics[width=0.33\textwidth]{bkg_subtraction_baseline_template_range_baseline_mll80_100_pt10_15_eta151_200_tag_trigger_matched.pdf}\\
\includegraphics[width=0.33\textwidth]{bkg_subtraction_baseline_template_range_baseline_mll80_100_pt15_20_eta0_80_tag_trigger_matched.pdf}
\includegraphics[width=0.33\textwidth]{bkg_subtraction_baseline_template_range_baseline_mll80_100_pt15_20_eta80_137_tag_trigger_matched.pdf}
\includegraphics[width=0.33\textwidth]{bkg_subtraction_baseline_template_range_baseline_mll80_100_pt15_20_eta151_200_tag_trigger_matched.pdf}
\caption{Plots illustrate the background subtraction procedure.
The top row represents the $10<\pt<15 \GeV$ region and the the bottom row shows the $15<\pt<20 \GeV$ region.
The three columns, from left to right, are the $0<|\eta|<0.8$, $0.8<|\eta|<1.37$, and $1.52<|\eta|<2.0$ regions, respectively.
The $m_{ee}$ distributions for data before the background subtraction (full black dots) and after the background subtraction (blue squares) are shown along with the corresponding $Z\to ee$ MC distributions (open black circles).
The MC distribution is normalized to the data after the background subtraction using a Gaussian fit of the $Z$ peak region ($85<m_{ee}<95 \GeV$).
The lower panels are the ratio between data after the background subtraction and the MC.
The corresponding background templates (red triangles) and their respective fitting results (green lines) are also shown.
}
\label{fig:RLE_bkg_estimations}
\end{figure}

Then, the background contamination in the $Z$ mass region ($80<m_{ee}<100 \GeV$) is calculated using:

\begin{equation}
N_{bkg}^{80<m_{ee}<100 \GeV} = \int_{80}^{100} N_{\textrm{template}} \ dm_{ee} \cdot \frac{N_{bkg}^{\textrm{tail}}}{N_{\textrm{template}}^{\textrm{tail}}}
\label{eq:RLE_bkg_in_80_mll_100}
\end{equation}

The background estimations are summarized in Table~\ref{tab:RLE_bkg_estimations}.
The largest improvements are observed in the $10<\pt<15 \GeV$ range which a sizeable background contamination is subtracted.
The background contamination in $15<\pt<20 \GeV$ is relatively small and indicates the $Z$ tag-and-probe method provides high purity sample of prompt leptons.
The real electron efficiencies before and after applying the background subtraction are shown in Table~\ref{tab:RLE_efficiency_before_and_after_background_subtraction}.

\begin{table}[htbp]
\begin{center}
\begin{tabular}{cccc}
\hline
\hline
& $0<|\eta|<0.8$ & $0.8<|\eta|<1.37$ & $1.52<|\eta|<2.0$\\
\hline
$10<\pt<15 \GeV$ & 4.04\% & 2.10\% & 3.17\%\\
$15<\pt<20 \GeV$ & 0.44\% & 0.58\% & 0.76\%\\
\hline
\hline
\end{tabular}
\end{center}
\caption{The estimated background contamination using the background template.
The $\pt$ and $|\eta|$ binnings correspond to the one used for the final measurements.}
\label{tab:RLE_bkg_estimations}
\end{table}

\begin{table}[htbp]
\begin{center}
\begin{tabular}{ccccc}
\hline
\hline
& background subtraction & $0<|\eta|<0.8$ & $0.8<|\eta|<1.37$ & $1.52<|\eta|<2.0$\\
\hline
\multirow{2}{*}{$10<\pt<15 \GeV$} & before & $57.4 \pm 0.9$ & $66.6 \pm 0.8$ & $53.2 \pm 0.9$\\
& after & $59.9 \pm 1.9$ & $68.0 \pm 1.8$ & $55.0 \pm 1.7$\\
\hline
\multirow{2}{*}{$15<\pt<20 \GeV$} & before & $64.5 \pm 0.2$ & $69.4 \pm 0.2$ & $62.0 \pm 0.3$\\
& after & $64.8 \pm 0.5$ & $69.8 \pm 0.5$ & $62.5 \pm 0.6$\\
\hline
\hline
\end{tabular}
\end{center}
\caption{The real electron efficiencies in percentage before and after applying the background subtraction.
The efficiency in three different $|\eta|$ regions for $10<\pt<15 \GeV$ and $15<\pt<20 \GeV$ are shown.}
\label{tab:RLE_efficiency_before_and_after_background_subtraction}
\end{table}





%%%
\subsection{Real lepton efficiencies}
\label{subsubsec:RLE_electron_efficiency}

\subsection{>Cut efficiencies}
\label{subsubsec:RLE_cut_efficiencies}

The real lepton efficiency is defined as the ratio between the number of prompt leptons passing the signal lepton definitions and the number of prompt leptons passing the candidate lepton definitions (as defined in~\cite{SS3L-Moriond2017}). Thus, it measures the efficiency of the signal lepton definition cuts with respect to the candidate ones.
The efficiencies associated to each signal cut with respect to candidate definitions are shown in Figure~\ref{fig:RLE_cut_efficiencies}.

\begin{figure}[htbp]
\begin{center}
\includegraphics[width=0.48\textwidth]{cut_efficiency_electron}
\includegraphics[width=0.48\textwidth]{cut_efficiency_muon}
\caption{Efficiencies of the signal electron (left) and muon (right) definition cuts as a function of $\pt$.
The leptons are selected from data samples using $Z$ tag-and-probe method. The black points correspond to the total real lepton efficiency.
The red squares represent the loose to medium likelihood cut efficiency. The green triangles stand for the calorimeter isolation cut efficiency and the blue triangles are the distribution of track isolation cut efficiency.
The longitudinal and tranverse impact parameters cut efficiencies are presented by magenta diamonds and cyan crosses, respectively.
}
\label{fig:RLE_cut_efficiencies}
\end{center}
\end{figure}

The left plot in Figure~\ref{fig:RLE_cut_efficiencies} shows that the prompt electron efficiency increases from $\sim$62\% in the $10<\pt<15 \GeV$ region to $\sim$98\% when $\pt>80 \GeV$.
The dominant contribution to the electron efficiency losses is the calorimeter isolation.
The loose to medium likelilihood (LH) cut efficiency increases from $\sim$92\% to $\sim$96\% in the $10<\pt<30 \GeV$ region then reaches a plateau for $30<\pt<50 \GeV$ and increases again to $\sim$98\% when $\pt>60 \GeV$.
The calorimeter isolation cut efficiency increases from $\sim$ 69\% in the $10<\pt<15 \GeV$ region to $\sim$98\% when $\pt>70 \GeV$.
The efficiency associated to the track isolation cut is $\sim$89\% for $10<\pt<15 \GeV$ and increases up to $\sim$100\% when $\pt>60 \GeV$.
The contribution of the longitudinal impact parameter cut to the real efficiency is $\sim$98\% for $10<\pt<15 \GeV$ and increases up to $\sim$100\% when $\pt>15 \GeV$.

As the same muon identification is used for the candidate and the signal definitions, the muon efficiencies are much higher than the electron ones (Figure~\ref{fig:RLE_cut_efficiencies}, right hand side).
The associated efficiencies computed using $Z\to \mu \mu$ events increase from $\sim$80\% for $10<\pt<15 \GeV$ to $\sim$98\% when $\pt>50 \GeV$.
The dominant contribution to the muon efficiency is the track isolation cut. The associated efficiency increases from $\sim$82\% to 98\% when $\pt>50 \GeV$.
Another difference with the electron case is that a transverse impact parameter cut is used in the muon signal definitions while this cut is already applied at the candidate level in the electron case.
The contributions of the longitudinal and transverse impact parameter cuts on muon efficiency are very small because the cut efficiency of transverse impact parameter is $\sim$99\% and the longitudinal one is 100\%.



\subsection{>Real lepton efficiencies}
\label{subsubsec:RLE_results}

Figure~\ref{fig:RLE_real_efficiency_total_systematics} shows the real lepton efficiencies as a function of \pt and $|\eta|$ used by the matrix method.
The three $|\eta|$ binnings used for the electron case are driven by the electromagnetic calorimeter geometry.
The background subtraction has been applied on the $10<\pt<15 \GeV$ and $15<\pt<20 \GeV$ for the electron case.
Because the crack region, $1.37<|\eta|<1.52$, is not considered in the analysis, this region is removed from the real electron efficiency study.
The uncertainties are the quadratic sum of the statistical uncertainties and the measurement systematic uncertainties.
The electron efficiencies in $1.52<|\eta|<2.01$ are lower than the other two $|\eta|$ regions.
This is expected as the electron identification is better in the central region of the calorimeter. 

\begin{figure}[htbp]
\includegraphics[width=0.48\textwidth]{real_electron_efficiency_total_systematics.pdf}
\includegraphics[width=0.48\textwidth]{real_muon_efficiency_total_systematics.pdf}
\caption{The real lepton efficiencies as a function of \pt and $|\eta|$ measured using the $Z$ tag-and-probe method.
The left plot corresponds to the real electron efficiencies in three $|\eta|$ regions and the right plot stands for the real muon efficiencies in four $|\eta|$ regions.
The $|\eta|$ binning used for the real electron efficiencies measurement corresponds to the geometry of the electromagnetic calorimeter and removing the creak region.
A homogeneous $|\eta|$ binning has been chosen for the muon case.}
\label{fig:RLE_real_efficiency_total_systematics}
\end{figure}



\subsection{>Tag-and-probe method and truth matching comparisons}
\label{subsubsec:RLE_truth_matched}

Although the $Z$ tag-and-probe method could select probe leptons accuratly, we would like to verify the accuracy of the $Z$ tag-and-probe method by comparing it with the truth matched information using the $Z\to\ell\ell$ MC samples.
Figure~\ref{fig:RLE_TandP_truth_match_comparisons} shows the real lepton efficiencies computed using $Z$ tag-and-probe method and truth matching as a function of $\pt$, $|\eta|$, and $\Delta R(\ell, jet)$, respectively.
The uncertainties shown in the plots correspond to the statistical uncertainties only.
There are as large as 7\% differences in the low $\pt$ region for the real electron efficiencies as function of $\pt$ computed by these two different methods.
The differences decrease when $\pt$ increases and no differences can be seen when $\pt>50 \GeV$.
The largest differences is about 3\% in the real electron efficiencies as a function of $|\eta|$.
Larger differences between these two methods can be seen in the real electron efficiencies as a function of $\Delta R(e, jet)$ when the $\Delta R(e, jet)<0.4$ because the overlap removal requirement has been applied on the candidate electrons.
The overlap removal removes the electrons so the $\Delta R(e, jet)<0.4$ region lacks statistics.
The efficiency differences between $Z$ tag-and-probe method and truth matching are less then 1\% for the muon case except the real muon efficiencies as a function of $\Delta R(\mu, jet)$ which has larger differences between these two methods when $\Delta R(\mu, jet)<0.4$ also because of the overlap removal requirement.
The comparisons showing small differences between methods indicate the probe leptons selected by the $Z$ tag-and-probe method are reliable and the differences may be considered as the systematic uncertainties.

\begin{figure}[htbp]
\includegraphics[width=0.33\textwidth]{Compare_TandP_truth_match_electron_pt.pdf}
\includegraphics[width=0.33\textwidth]{Compare_TandP_truth_match_electron_eta.pdf}
\includegraphics[width=0.33\textwidth]{Compare_TandP_truth_match_electron_dRjet.pdf}\\
\includegraphics[width=0.33\textwidth]{Compare_TandP_truth_match_muon_pt.pdf}
\includegraphics[width=0.33\textwidth]{Compare_TandP_truth_match_muon_eta.pdf}
\includegraphics[width=0.33\textwidth]{Compare_TandP_truth_match_muon_dRjet.pdf}
\caption{
The real lepton efficiencies computed using $Z$ tag-and-probe method and truth matching.
The top row is the electron case and the bottom row is the muon case.
The three columns from the left to the right are the real lepton efficiencies as a function $\pt$, $|\eta|$, and $\Delta R(\ell, jet)$, respectively.
The real lepton efficiencies computed using $Z$ tag-and-probe are denoted by the red dots and the one calculated using truth matching are indicated by the blue triangles.
The lower pads show the ratio with respect to the $Z$ tag-and-probe method.
}
\label{fig:RLE_TandP_truth_match_comparisons}
\end{figure}



\subsection{>Data-to-MC comparisons}
\label{subsubsec:RLE_data_to_mc_comparisons}

The real lepton efficiencies computed using the $Z$ tag-and-probe method in data are compared to those using the simulated $Z\to \ell\ell$ MC processes.
All 2015 and 2016 data are considered, and corresponds to an integrated luminosity of 36.5 fb$^{-1}$ after good run list requirements.
All the MC lepton scale factors provided by the CP group are applied and the simulation is reweighted to the pile-up observed in the data.
%Besides, an additional truth match is added for the MC lepton selection.
Figure~\ref{fig:RLE_real_efficiency_pt_eta_dRjet} shows the electron and the muon real efficiencies as a function of $\pt$, $|\eta|$ and $\Delta R(\ell, jet)$ measured on data and with simulated $Z\to ee$ and $Z\to \mu\mu$ MC processes, respectively.
The associated uncertainties correspond to the statistical uncertainties only.
A reasonable data to MC agreement is observed except the low $\Delta R(\ell, jet)$ region because of lacking statistics.
The real electron efficiencies as a function of $|\eta|$ computed using the $Z\to ee$ MC are slightly lower than the one computed using data.
The differences come from the efficiencies drop of $Z\to ee$ MC in the $40<\pt<50 \GeV$. 

\begin{figure}[htbp]
%\includegraphics[width=0.96\textwidth]{real_efficiency.pdf}
\includegraphics[width=0.33\textwidth]{real_efficiency_ratio_plot_electron_pt.pdf}
\includegraphics[width=0.33\textwidth]{real_efficiency_ratio_plot_electron_eta.pdf}
\includegraphics[width=0.33\textwidth]{real_efficiency_ratio_plot_electron_dRjet.pdf}\\
\includegraphics[width=0.33\textwidth]{real_efficiency_ratio_plot_muon_pt.pdf}
\includegraphics[width=0.33\textwidth]{real_efficiency_ratio_plot_muon_eta.pdf}
\includegraphics[width=0.33\textwidth]{real_efficiency_ratio_plot_muon_dRjet.pdf}
\caption{The real lepton efficiencies as a function of $\pt$, $|\eta|$ and $\Delta R(\ell, jet)$ measured on data and MC using the $Z$ tag-and-probe method.
The plots on the top row correspond to the real electron efficiencies and the plots on the bottown row correspond to the real muon efficiencies.
The 2015 + 2016 data are denoted by the black dots, the $Z\to\ell\ell$ MC are shown using red squares and are reweighted to the data pile-up.
The uncertainties shown in the plots are corresponding to the statistical uncertainties only.
Some differences between data and $Z\to \ell\ell$ MC can be seen in the low $\Delta R(\ell, jet)$ region.
The binning used for the real electron efficiencies as a function of $\eta$ corresponds to the geometry of the electromagnetic calorimeter.}
\label{fig:RLE_real_efficiency_pt_eta_dRjet}
\end{figure}



\subsection{>Real lepton efficiency versus pileup}
\label{subsubsec:RLE_vs_pileup}
 
The relations between the real lepton efficiencies and the pileup are also studied.
The efficiencies calculated using 2015 + 2016 data are compared with those obtained using $Z$ tag-and-probe method and truth matching MC samples.
The \ttbar and $\tilde{g}\to\ttbar\tilde{\chi^{0}_{1}}$ MC samples are also considered because we are interested in the behavior of the efficiencies with different event topologies.
Figure~\ref{fig:RLE_vs_pileup} shows the electron and muon real efficiencies as a function of the average interactions per crossing $<\mu>$.
The real electron efficiencies are about 92\% to 93\% at low $<\mu>$ and decrease when $<\mu>$ increase.
Comparing with the real lepton efficiencies calculated by data samples, the efficiencies computed using $\tilde{g}\to\ttbar\tilde{\chi^{0}_{1}}$ MC sample have lower efficiencies in both electron and muon cases.
But the real lepton efficiencies computed by the \ttbar have similar efficiencies with the data one in the electron case and have lower efficiencies in the muon case.
This is because the \ttbar MC has lower real muon efficiencies when $\pt<40 \GeV$ with respect to data.
If we require a $\pt>40 \GeV$ requirement on the probe leptons, then the real muon efficiencies for \ttbar MC agree with data.
The electron and muon real efficiencies as a function of $\pt$ compared with data and $Z\to\ell\ell$ are shown in Figure~\ref{fig:RLE_real_efficiency_ttbar_gtt}.
From which we notice the real lepton efficiencies of \ttbar process is lower than the data one when $\pt<40 \GeV$ and the $\tilde{g}\to\ttbar\tilde{\chi^{0}_{1}}$ process has lower real lepton efficiencies in both electron and muon cases.

\begin{figure}[htbp]
\includegraphics[width=0.48\textwidth]{real_efficiency_vs_AvgMu_elec.pdf}
\includegraphics[width=0.48\textwidth]{real_efficiency_vs_AvgMu_muon.pdf}
\caption{
The real lepton efficiencies as a function of the average interactions per crossing $<\mu>$.
The data is shown as black dots and $Z\to \ell\ell$ tag-and-probe and truth matching are in red squares and blue triangles, respectively.
The real lepton efficiencies calculated using \ttbar and $\tilde{g}\to\ttbar\tilde{\chi^{0}_{1}}$ MC are also shown in magenta diamonds and yellow crosses.
And they show the real lepton efficiencies with different event topoloties.
The $|\eta|<2$ requirement has been applied on the \ttbar and $\tilde{g}\to\ttbar\tilde{\chi^{0}_{1}}$ MC samples for the electron case.
}
\label{fig:RLE_vs_pileup}
\end{figure}

\begin{figure}[htbp]
\includegraphics[width=0.48\textwidth]{real_efficiency_ratio_plot_electron_pt_ttbar_gtt.pdf}
\includegraphics[width=0.48\textwidth]{real_efficiency_ratio_plot_muon_pt_ttbar_gtt.pdf}
\caption{
The electron and muon real efficiencies as a function of $\pt$.
The black dots stands for data, the red squares and blue triangles indicate $Z\to\ell\ell$ tag-and-probe and truth matching, respectively.
The \ttbar and $\tilde{g}\to\ttbar\tilde{\chi^{0}_{1}}$ are represented by magenta diamonds and yellow crosses, respectively.
All of the real lepton efficiencies computed by MC processes are agree with the data one when $\pt>40 \GeV$.
The differences in the $\pt<40 \GeV$ region come from the different event topologies. 
}
\label{fig:RLE_real_efficiency_ttbar_gtt}
\end{figure}

%%% 
\subsection{Sources of systematic uncertainties}
\label{subsec:RLE_sources_of_systematic_uncertainties}

\subsection{>Measurement systematics}
\label{subsubsec:RLE_bkg_systematics}

The systematic uncertainties associated with the $Z$ tag-and-probe mothod have been studied by varying the background template definitions, the template fitting ranges, and the $m_{\ell\ell}$ windows used for the real lepton efficiency measurements.
The variations of the background template definitions are presented in Table~\ref{tab:RLE_bkg_templates} and the additional template fitting ranges are [60 \textendash 70] $\cup$ [100 \textendash 120] GeV and [65 \textendash 75] $\cup$ [100 \textendash 120] GeV.
The two other measurement $m_{\ell\ell}$ windows considered are $75<m_{ee}<105 \GeV$ and $85<m_{ee}<95 \GeV$.
In total 27 variations of the measurement methods are considered in $\pt<20 \GeV$ and 3 variations when $\pt>20 \GeV$ for the electron case.
Because the background subtraction is applied on the electron case only, no background templates and template fitting ranges are used in the muon case.
Therefore, the systematic uncertainties of real muon efficiency are studied by varying the 3 different $m_{\ell\ell}$ windows only.
Table~\ref{tab:RLE_bkg_systematics_elec} and Table~\ref{tab:RLE_bkg_systematics_muon} show the measurement uncertainties for the electron and muon cases, respectively.

\begin{center}
\begin{table}[htbp]
\resizebox{\textwidth}{!}{%
\begin{tabular}{cccc}
\hline
\hline
\multicolumn{4}{c}{Electrons (measurement)}\\
\hline
$|\eta|$ & [0, 0.8] & [0.8, 1.37] & [1.52, 2.0]\\
\hline
$10<\pt<15 \GeV$ & 2.32\%(t) / 2.85\%(f) / 5.06\%(m) & 4.84\%(t) / 1.99\%(f) / 5.66\%(m) & 5.90\%(t) / 0.28\%(f) / 10.31\%(m)\\
$15<\pt<20 \GeV$ & 1.39\%(t) / 0.00\%(f) / 3.55\%(m) & 2.01\%(t) / 0.01\%(f) / 3.94\%(m) & 2.05\%(t) / 0.15\%(f) / 6.19\%(m)\\
$20<\pt<25 \GeV$ & 2.46\% & 3.34\% & 3.88\%\\
$25<\pt<30 \GeV$ & 1.69\% & 2.17\% & 2.66\%\\
$30<\pt<35 \GeV$ & 1.19\% & 1.75\% & 2.07\%\\
$35<\pt<40 \GeV$ & 0.70\% & 1.23\% & 1.32\%\\
$40<\pt<50 \GeV$ & 0.20\% & 0.30\% & 0.42\%\\
$50<\pt<60 \GeV$ & 0.15\% & 0.17\% & 0.20\%\\
$60<\pt<70 \GeV$ & 0.13\% & 0.14\% & 0.21\%\\
$70<\pt<80 \GeV$ & 0.10\% & 0.17\% & 0.18\%\\
$80<\pt<120 \GeV$ & 0.12\% & 0.11\% & 0.19\%\\
$120<\pt<150 \GeV$ & 0.10\% & 0.16\% & 0.06\%\\
$150<\pt<200 \GeV$ & 0.11\% & 0.03\% & 0.19\%\\
\hline
\hline
\end{tabular}
}
\caption{
The measurement systematic uncertainties in percentage on the real electron efficiencies.
The background subtraction is applied on the first two $\pt$ bins on the electron case so there are three different sources of the measurement systematic uncertainties, varying templates, varying fitting ranges, and varying $m_{\ell\ell}$ windows denoted by t, f, and m, respectively.
Only the variations of $m_{\ell\ell}$ windows are applied on the electron case when $\pt>20 \GeV$.}
\label{tab:RLE_bkg_systematics_elec}
\end{table}
\end{center}

\begin{table}[htbp]
\begin{center}
%\resizebox{\textwidth}{!}{%
\begin{tabular}{ccccc}
\hline
\hline
\multicolumn{5}{c}{Muon (measurement)}\\
\hline
$|\eta|$ & [0, 0.6] & [0.6, 1.2] & [1.2, 1.8] & [1.8, 2.5]\\
\hline
$10<\pt<15 \GeV$ & 1.29\% & 1.06\% & 0.96\% & 0.98\%\\
$15<\pt<20 \GeV$ & 0.44\% & 0.38\% & 0.56\% & 0.64\%\\
$20<\pt<25 \GeV$ & 0.19\% & 0.22\% & 0.38\% & 0.56\%\\
$25<\pt<30 \GeV$ & 0.09\% & 0.12\% & 0.22\% & 0.36\%\\
$30<\pt<35 \GeV$ & 0.06\% & 0.11\% & 0.23\% & 0.32\%\\
$35<\pt<40 \GeV$ & 0.05\% & 0.07\% & 0.13\% & 0.26\%\\
$40<\pt<50 \GeV$ & 0.04\% & 0.04\% & 0.05\% & 0.07\%\\
$50<\pt<60 \GeV$ & 0.06\% & 0.06\% & 0.09\% & 0.07\%\\
$60<\pt<70 \GeV$ & 0.06\% & 0.07\% & 0.08\% & 0.08\%\\
$70<\pt<80 \GeV$ & 0.06\% & 0.08\% & 0.11\% & 0.05\%\\
$80<\pt<120 \GeV$ & 0.07\% & 0.07\% & 0.12\% & 0.07\%\\
$120<\pt<150 \GeV$ & 0.06\% & 0.06\% & 0.15\% & 0.05\%\\
$150<\pt<200 \GeV$ & 0.09\% & 0.11\% & 0.15\% & 0.06\%\\
\hline
\hline
\end{tabular}
%}
\caption{
The measurement systematic uncertainties in percentage on the real muon efficiencies.
Because the background subtraction is not applied on the muon case, only the variations of $m_{\ell\ell}$ windows are applied on the muon case for all $\pt$ regions.
}
\label{tab:RLE_bkg_systematics_muon}
\end{center}
\end{table}

The largest contribution to the systematic uncertainties arises from the $m_{\ell\ell}$ window variations.
This result is expected as electrons extracted from the $m_{\ell\ell}$ tail region are affected by bremsstrahlung effects.
Thus, the efficiency computed with electrons extracted with a large $m_{\ell\ell}$ window will be lower than the one computed using a tigher $m_{\ell\ell}$ window.
In the $10<\pt<15 \GeV$ region, the order of magnitude of the $m_{\ell\ell}$ window variation is 9\% whereas the background subtraction one is 5\%.
This result shows the robustness of the background subtraction method.



\subsection{>Trigger bias}
\label{subsubsec:RLE_trigger_bias}

Besides the measurement systematic uncertainties, the different trigger strategies for the analysis are also considered as sources of systematic uncertainties.
The leptons entering in the signal regions are required to fire one of the di-lepton triggers.
For example, if the event fires the $E_{\text T}^{miss}$ trigger or the considered lepton is the third leading lepton, then no trigger matching should be applied for the efficiency computation.
On the other hand, if the event fires the di-lepton trigger and the considered lepton is the leading lepton or the sub-leading lepton, then a trigger matching should be applied before the real lepton efficiency measurement.
The nominal value of the real lepton effciencies are measured with events triggered by the single lepton triggers listed in Table~\ref{tab:RLE_single_lepton_triggers} and the tag lepton must match to the corresponding single lepton trigger.
The tag trigger matching is required in order to provide unbiased probe leptons for the real lepton efficiency measurements.
The systematics uncertainties of the real lepton efficiencies are assigned as the differences between the nominal values and the values measured with different trigger.
Moreover, as a $\pt>20 \GeV$ requirement is applied on the two leading leptons, the leptons with $\pt<20 \GeV$ will never be trigger matched to the di-lepton trigger.
Therefore, no systematics are assigned in the $10<\pt<20 \GeV$ range.
Figure~\ref{fig:RLE_trigger_bias_electron} shows the real electron efficiencies computed with the different trigger strategies as a function of $\pt$ in different $|\eta|$ regions.
Because the crack region, $1.37<|\eta|<1.52$, is not considered in the analysis, this region is removed from the real electron efficiency study.
And Figure~\ref{fig:RLE_trigger_bias_muon} shows the real muon efficiencies computed with the different trigger strategies as a function of $\pt$ in different $|\eta|$ regions.
These plots indicate that the trigger strategy does not affect the real muon efficiency measurement.
Table~\ref{tab:RLE_trigger_syst_elec} and Table~\ref{tab:RLE_trigger_syst_muon} show the systematic uncertainties in percentage due to the different trigger stratagies.

\begin{figure}[htbp]
\includegraphics[width=0.33\textwidth]{trigger_uncertainty_electron_eta080_ratio_plot.pdf}
\includegraphics[width=0.33\textwidth]{trigger_uncertainty_electron_eta80137_ratio_plot.pdf}
\includegraphics[width=0.33\textwidth]{trigger_uncertainty_electron_eta152201_ratio_plot.pdf}
\caption{The real electron efficiencies as a function of $\pt$ in 3 different $|\eta|$ regions.
Four different trigger strategies are applied.
The nominal values of the real electron efficiencies are measured using the single lepton trigger with tag trigger matched.
The differences between the nominal values and the values measured using other strategies are assigned as the systematic uncertainties.}
\label{fig:RLE_trigger_bias_electron}
\end{figure}

\begin{figure}[htbp]
\includegraphics[width=0.48\textwidth]{trigger_uncertainty_muon_eta060_ratio_plot.pdf}
\includegraphics[width=0.48\textwidth]{trigger_uncertainty_muon_eta60120_ratio_plot.pdf}\\
\includegraphics[width=0.48\textwidth]{trigger_uncertainty_muon_eta120180_ratio_plot.pdf}
\includegraphics[width=0.48\textwidth]{trigger_uncertainty_muon_eta180250_ratio_plot.pdf}
\caption{The real muon efficiencies as a function of $\pt$ in 4 different $|\eta|$ regions.
Four different trigger strategies are applied.
The nominal values of the real muon efficiencies are measured using the single lepton trigger with tag trigger matched.
The differences between the nominal values and the values measured using other strategies are assigned as the systematic uncertainties.}
\label{fig:RLE_trigger_bias_muon}
\end{figure}

\begin{table}[htbp]
\begin{center}
\begin{tabular}{cccc}
\hline
\hline
\multicolumn{4}{c}{Electrons (trigger)}\\
\hline
$|\eta|$ & [0, 0.8] & [0.8, 1.37] & [1.52, 2.0]\\
\hline
$10<\pt<15 \GeV$ & 2.46\% & 1.32\% & 3.02\%\\
$15<\pt<20 \GeV$ & 0.16\% & 0.78\% & 1.33\%\\
$20<\pt<25 \GeV$ & 0.29\% & 0.84\% & 1.18\%\\
$25<\pt<30 \GeV$ & 1.53\% & 2.07\% & 2.20\%\\
$30<\pt<35 \GeV$ & 1.28\% & 1.63\% & 1.81\%\\
$35<\pt<40 \GeV$ & 0.98\% & 1.19\% & 1.42\%\\
$40<\pt<50 \GeV$ & 0.73\% & 0.90\% & 1.05\%\\
$50<\pt<60 \GeV$ & 0.68\% & 0.81\% & 1.05\%\\
$60<\pt<70 \GeV$ & 0.61\% & 0.70\% & 1.13\%\\
$70<\pt<80 \GeV$ & 0.65\% & 0.77\% & 1.27\%\\
$80<\pt<120 \GeV$ & 0.60\% & 0.66\% & 1.11\%\\
$120<\pt<120 \GeV$ & 0.38\% & 0.40\% & 0.79\%\\
$150<\pt<200 \GeV$ & 0.43\% & 0.22\% & 0.25\%\\
\hline
\hline
\end{tabular}
\caption{The systematic uncertainties in percentage due to the different trigger strategies for the real electron efficiencies.
The uncertainties of each trigger strategy are computed with respect to the one applied single lepton trigger and required tag trigger matched.
And the total uncertainties are the quadratic sum of the uncertainties of each trigger strategy.
}
\label{tab:RLE_trigger_syst_elec}
\end{center}
\end{table}

\begin{table}[htbp]
\begin{center}
\begin{tabular}{ccccc}
\hline
\hline
\multicolumn{5}{c}{Muons (trigger)}\\
\hline
$|\eta|$ & [0, 0.6] & [0.6, 1.2] & [1.2, 1.8] & [1.8, 2.5]\\
\hline
$10<\pt<15 \GeV$ & 0.11\% & 0.15\% & 0.34\% & 0.19\%\\
$15<\pt<20 \GeV$ & 0.14\% & 0.50\% & 0.75\% & 0.77\%\\
$20<\pt<25 \GeV$ & 0.30\% & 0.63\% & 1.01\% & 0.93\%\\
$25<\pt<30 \GeV$ & 0.90\% & 1.38\% & 2.12\% & 1.83\%\\
$30<\pt<35 \GeV$ & 0.58\% & 0.84\% & 1.27\% & 0.99\%\\
$35<\pt<40 \GeV$ & 0.33\% & 0.37\% & 0.57\% & 0.46\%\\
$40<\pt<50 \GeV$ & 0.16\% & 0.13\% & 0.16\% & 0.13\%\\
$50<\pt<60 \GeV$ & 0.06\% & 0.06\% & 0.07\% & 0.07\%\\
$60<\pt<70 \GeV$ & 0.04\% & 0.04\% & 0.04\% & 0.04\%\\
$70<\pt<80 \GeV$ & 0.05\% & 0.06\% & 0.04\% & 0.06\%\\
$80<\pt<120 \GeV$ & 0.03\% & 0.03\% & 0.03\% & 0.11\%\\
$120<\pt<150 \GeV$ & 0.05\% & 0.13\% & 0.10\% & 0.08\%\\
$150<\pt<200 \GeV$ & 0.04\% & 0.03\% & 0.02\% & 0.19\%\\
\hline
\hline
\end{tabular}
\caption{The systematic uncertainties in percentage due to the different trigger strategies for the real muon efficiencies.
The uncertainties of each trigger strategy are computed with respect to the one applied single lepton trigger and required tag trigger matched.
And the total uncertainties are the quadratic sum of the uncertainties of each trigger strategy.
}
\label{tab:RLE_trigger_syst_muon}
\end{center}
\end{table}

\subsection{>Extrapolation to signal regions}
\label{subsubsec:RLE_extrapolation_to_signal_region}

The real lepton efficiencies are measured with a sample enriched in $Z\to \ell\ell$ events characterized by well isolated leptons.
The different processes entering in the signal region are accompanied by many ($b$-)jets and with a different event topology.
Thus, the leptons presented in the final state are not necessary well isolated.
As tighter isolation cuts are used for the signal lepton definitions, their associated real efficiencies can be smaller.
A SUSY benchmark model 
$\tilde{g}\to\ttbar\tilde{\chi}^{0}_{1}$ 
is used and the boosted event topoloties are selected by applying the requirement 
$\Delta m=m_{\tilde{g}}-m_{\chi^{0}_{1}}>1 \TeV$ 
on the model.
These differences in the real lepton efficiencies between the $Z\to \ell\ell$ processes and the $\tilde{g}\to\ttbar\tilde{\chi}^{0}_{1}$ process are assigned as system uncertainties.
As the topology of one of the main irreducible backgrounds $\ttbar V$ is close to the \ttbar one, the efficiencies measured with the leptons from \ttbar are also considered.
The efficiency comparisons are made for each $\pt$ bin considering different $\Delta R(\ell, jets)$ ranges.

The kinematic distributions of the candidate leptons from the $Z\to \ell\ell$, \ttbar, and $\tilde{g}\to\ttbar\tilde{\chi}^{0}_{1}$ are showned in Figure~\ref{fig:RLE_kinematic}.
The top row shows the \pt distributions for the candidate electrons on the left hand side and muons on the right hand side.
The bottom row shows the $|\eta|$ distributions.
These plots show that the leptons from SUSY process are more boosted and more central than the ones from $Z$ and \ttbar processes.
The $\Delta R(\ell, jet)$ and the $N_{jets}$ distributions of the candidate leptons extracted from $Z\to\ell\ell$, \ttbar, and $\tilde{g}\to\ttbar\tilde{\chi}^{0}_{1}$ processes are shown in Figure~\ref{fig:RLE_dRjet_Njet}.
The left hand side is the electron case and the right hand side is the muon case.
The $\Delta R(\ell, jet)$ distribution associated to the SUSY signals peak at 0.5 and most of the statistics are contained in $\Delta R(\ell, jet)<1$ region.
In comparison, the leptons from the $Z\to\ell\ell$ processes are not accompanied with a signal jet and the $\Delta R(\ell, jet)$ distribution peak about $\Delta R(\ell, jet)=3$.
The jet multiplicity of the $Z\to\ell\ell$ peak at 4 jets for the electron case, 3 jets for the muon case, but for the SUSY signal peaks at 9 jets.
These plots confirm that the leptons produced in the SUSY signal are accompained with many more jets and are therefore less isolated than the $Z\to\ell\ell$ processes.
This extreme topology enables us to assess a conservative SUSY signal extrapolation systematic uncertainty that should cover all SUSY signal processes considered by the analysis.

\begin{figure}[htbp]
\includegraphics[width=0.96\textwidth]{baseline_kinematics.pdf}
\caption{The kinematic distributions of the candidate lepton from the processes considered for the systematic uncertainty study.
The top row shows the \pt distributions for the candidate electrons on the left hand side and muons on the right hand side.
The bottom row shows the $|\eta|$ distributions.
The $\tilde{g}\to\ttbar\tilde{\chi}^{0}_{1}$ process is more boosted and centralized than the $Z\to\ell\ell$ and \ttbar processes.
}
\label{fig:RLE_kinematic}
\end{figure}

\begin{figure}[htbp]
\includegraphics[width=0.96\textwidth]{baseline_deltaR_and_NJets.pdf}
\caption{The $\Delta R(\ell, jet)$ and the $N_{jets}$ distributions of the candidate leptons extracted from $Z\to\ell\ell$, \ttbar, and $\tilde{g}\to\ttbar\tilde{\chi}^{0}_{1}$ processes.
Most of the statistics of $\tilde{g}\to\ttbar\tilde{\chi}^{0}_{1}$ are located in $\Delta R(\ell, jet)<1$ region and higher $N_{jets}$ region.
In the contrat, the statistics of $Z\to\ell\ell$ processes are populated at higher $\Delta R(\ell, jet)<1$ region and lower $N_{jets}$ region.
}
\label{fig:RLE_dRjet_Njet}
\end{figure}

The real lepton efficiencies as a functionof \pt using $Z\to\ell\ell$, \ttbar, and $\tilde{g}\to\ttbar\tilde{\chi^{0}_{1}}$ are shown in Figure~\ref{fig:RLE_real_efficiency_ttbar_gtt}.
The lower panel shows the ratio with respect to the data.
In the left hand side plot shows the real electron efficiencies as a function of \pt and the right hand side shows the real muon efficiencies as a function of \pt. 
The left hand side plot shows that the real electron efficiencies are \pt dependent in the $\pt<50 \GeV$ region and become stable when $\pt>50 \GeV$.
The efficiencies computed using $\tilde{g}\to\ttbar\tilde{\chi^{0}_{1}}$ are about 8\% lower than the efficiencies calculated using $Z\to ee$.
The observed differences in the low \pt region are mostly due to the calorimeter isolation and the track isolation cuts.
And the track isolation and $d_{0}/\sigma_{d_{0}}$ cuts are the main reasons cause the efficiency differences in the muon real efficiencies.

The average efficiencies of $Z\to\ell\ell$ are calculated and the relative efficiency differences are computed with respect to the average efficiencies.
The relative efficiency differences as a function of $\Delta R(\ell, jet)$ are considered for each measured \pt bins and are shown in Table~\ref{tab:RLE_syst_busy}.

\begin{center}
\begin{table}
\resizebox{\textwidth}{!}{%
\begin{tabular}{ccccccccc}
\hline
\hline
\multicolumn{9}{c}{electrons (busy environments)}\\
\hline
$\Delta R(e, jet)$ & [0, 0.1] & [0.1, 0.15] & [0.15, 0.2] & [0.2, 0.3] & [0.3, 0.35] & [0.35, 0.4] & [0.4, 0.6] & [0.6, 4]\\
\hline
10 GeV $< p_{\text T} <$ 20 GeV & - & - & - & - & - & - & 25.31\% & 6.5\%\\
20 GeV $< p_{\text T} <$ 30 GeV & - & - & - & - & - & 73.37\% & 10.21\% & 0.37\%\\
30 GeV $< p_{\text T} <$ 40 GeV & - & - & - & 97.71\% & 48.22\% & 15.54\% & 7.29\% & 0.58\%\\
40 GeV $< p_{\text T} <$ 50 GeV & - & - & - & 52.81\% & 22.80\% & 16.73\% & 7.68\% & 1.10\%\\
50 GeV $< p_{\text T} <$ 60 GeV & - & - & - & 29.96\% & 21.49\% & 20.23\% & 6.99\% & 2.78\%\\
60 GeV $< p_{\text T} <$ 80 GeV & - & - & 55.89\% & 24.31\% & 17.40\% & 24.77\% & 6.20\% & 2.87\%\\
80 GeV $< p_{\text T} <$ 150 GeV & - & 57.52\% & 30.24\% & 16.45\% & 12.73\% & 20.92\% & 4.44\% & 2.73\%\\
150 GeV $< p_{\text T} <$ 200 GeV & 88.54\% & 40.16\% & 19.34\% & 8.45\% & 14.66\% & 16.57\% & 2.57\% & 1.90\%\\
\hline
\hline
%\end{tabular}
%}
%\resizebox{\textwidth}{!}{%
%\begin{tabular}{ccccccccccc}
%\hline
%\hline
\multicolumn{9}{c}{muons (busy environments)}\\
\hline
$\Delta R(\mu, jet)$ & [0, 0.1] & [0.1, 0.15] & [0.15, 0.2] & [0.2, 0.3] & [0.3, 0.35] & [0.35, 0.4] & [0.4, 0.6] & [0.6, 4]\\
\hline
10 GeV $< p_{\text T} <$ 20 GeV & - & - & - & - & - & - & 33.59\% & 5.18\%\\
20 GeV $< p_{\text T} <$ 30 GeV & - & - & - & - & - & 82.34\% & 22.27\% & 3.39\%\\
30 GeV $< p_{\text T} <$ 40 GeV & - & - & -  & 98.54\% & 56.36\% & 31.89\% & 14.22\% & 2.24\%\\
40 GeV $< p_{\text T} <$ 50 GeV & - & - & - & 53.10\% & 21.33\% & 13.90\% & 6.81\% & 1.45\%\\
50 GeV $< p_{\text T} <$ 60 GeV & - & - & - & 24.98\% & 13.72\% & 9.62\% & 3.83\% & 0.79\%\\
60 GeV $< p_{\text T} <$ 80 GeV & - & - & 44.41\% & 13.75\% & 6.14\% & 4.76\% & 2.04\% & 0.15\%\\
80 GeV $< p_{\text T} <$ 150 GeV & - & 29.94\% & 7.14\% & 3.16\% & 1.30\% & 1.04\% & 0.07\% & 0.57\%\\
150 GeV $< p_{\text T} <$ 200 GeV & 82.26\% & 4.14\% & 1.02\% & 0.17\% & 0.29\% & 0.62\% & 1.02\% & 1.13\%\\
\hline
\hline
\end{tabular}
}
\caption{
Systematic uncertainties on the measured real lepton efficiency due to extrapolation to busy environments using 
%$\gluino \to \ttbar \tilde{\chi_1^0}$ events. 
}
\label{tab:RLE_syst_busy}
\end{table}
\end{center}


%Note for Yu-Ting : %
%Motivate here why we had to change the radius of the \pt-dependent cone :
%\begin{itemize}
%\item $\Delta R=\operatorname{min}\left\{0.4, 0.1+9.6 \GeV/\pt(\ell)\right\}$ (in our analysis)
%\item $\Delta R=\operatorname{min}\left\{0.4, 0.04+10 \GeV/\pt(\ell)\right\}$ (default in the {\ttfamily AssociationUtils} package)
%\end{itemize} 
%(keep this table in this section, it's quoted in the Overlap removal section!)%

\subsection{>Final uncertainties}
\label{subsubsec:RLE_final_uncertainties}

The final uncertainties of the real lepton efficiency study are the quadratic sum of the statistical uncertainties and the systematic uncertainties.
The systematic uncertainties under the consideration include the measurement uncertainties, the trigger uncertainties, the uncertainties come from the differences between truth matching and tag-and-probe methods, and the uncertainties in the busy environment.
Because the uncertainties in the busy environment are a function of \pT and $\Delta R$ and the other systematic uncertainties are a function of \pT and $|\eta|$, we keep the uncertainties in busy environment in Table~\ref{tab:RLE_syst_busy} and do not combine it with other uncertainties.
The Table~\ref{tab:RLE_final_uncertainties_elec} and Table~\ref{tab:RLE_final_uncertainties_muon} show the final uncertainties.

\begin{table}[htbp]
\begin{center}
\begin{tabular}{cccc}
\hline
\hline
\multicolumn{4}{c}{Electrons (final uncertainties)}\\
\hline
$|\eta|$ & [0, 0.8] & [0.8, 1.37] & [1.52, 2.0]\\
\hline
$10<\pt<15 \GeV$ & 0.047 & 0.063 & 0.089\\
$15<\pt<20 \GeV$ & 0.027 & 0.042 & 0.062\\
$20<\pt<25 \GeV$ & 0.018 & 0.031 & 0.041\\
$25<\pt<30 \GeV$ & 0.029 & 0.024 & 0.027\\
$30<\pt<35 \GeV$ & 0.023 & 0.021 & 0.023\\
$35<\pt<40 \GeV$ & 0.014 & 0.018 & 0.018\\
$40<\pt<50 \GeV$ & 0.007 & 0.010 & 0.010\\
$50<\pt<60 \GeV$ & 0.008 & 0.010 & 0.010\\
$60<\pt<70 \GeV$ & 0.007 & 0.010 & 0.010\\
$70<\pt<80 \GeV$ & 0.008 & 0.011 & 0.012\\
$80<\pt<120 \GeV$ & 0.010 & 0.010 & 0.011\\
$120<\pt<120 \GeV$ & 0.005 & 0.005 & 0.011\\
$150<\pt<200 \GeV$ & 0.005 & 0.003 & 0.020\\
\hline
\hline
\end{tabular}
\caption{The final uncertainties of the real electron efficiencies.
The final uncertainties are calculated using the quadratic sum of the statistical uncertainties and the systematic uncertainties.
The systematic uncertainties contain the measurement uncertainties, the trigger uncertainties, and the uncertainties come from the difference between truth matching and tag-and-probe methods.
The trigger uncertainties are included for $\pT>20 \GeV$ only.
The uncertainties in the busy environment do not incorporate in the final uncertainties calculation because it is a function of \pT and $\Delta R$.
}
\label{tab:RLE_final_uncertainties_elec}
\end{center}
\end{table}

\begin{table}[htbp]
\begin{center}
\begin{tabular}{ccccc}
\hline
\hline
\multicolumn{5}{c}{Muons (final uncertainties)}\\
\hline
$|\eta|$ & [0, 0.6] & [0.6, 1.2] & [1.2, 1.8] & [1.8, 2.5]\\
\hline
$10<\pt<15 \GeV$ & 0.014 & 0.010 & 0.008 & 0.011\\
$15<\pt<20 \GeV$ & 0.005 & 0.006 & 0.008 & 0.011\\
$20<\pt<25 \GeV$ & 0.003 & 0.006 & 0.010 & 0.010\\
$25<\pt<30 \GeV$ & 0.011 & 0.015 & 0.022 & 0.019\\
$30<\pt<35 \GeV$ & 0.007 & 0.009 & 0.014 & 0.011\\
$35<\pt<40 \GeV$ & 0.004 & 0.004 & 0.006 & 0.006\\
$40<\pt<50 \GeV$ & 0.002 & 0.001 & 0.002 & 0.001\\
$50<\pt<60 \GeV$ & 0.001 & 0.001 & 0.001 & 0.001\\
$60<\pt<70 \GeV$ & 0.001 & 0.001 & 0.001 & 0.002\\
$70<\pt<80 \GeV$ & 0.002 & 0.001 & 0.001 & 0.002\\
$80<\pt<120 \GeV$ & 0.004 & 0.002 & 0.002 & 0.002\\
$120<\pt<150 \GeV$ & 0.006 & 0.005 & 0.005 & 0.005\\
$150<\pt<200 \GeV$ & 0.005 & 0.005 & 0.005 & 0.006\\
\hline
\hline
\end{tabular}
\caption{The final uncertainties of the real muon efficiencies.
The final uncertainties are calculated using the quadratic sum of the statistical uncertainties and the systematic uncertainties.
The systematic uncertainties contain the measurement uncertainties, the trigger uncertainties, and the uncertainties come from the difference between truth matching and tag-and-probe methods.
The uncertainties in the busy environment do not incorporate in the final uncertainties calculation because it is a function of \pT and $\Delta R$.
}
\label{tab:RLE_final_uncertainties_muon}
\end{center}
\end{table}


