The goal of the analysis is to maximize the information that can be 
extracted from comparing the observed data to the background 
prediction in the signal regions designed to search for new physics 
topologies. Statistical tools are essential to tell in the most 
powerful way and to the best of our knowledge if there is a new physics 
signal beyond what is already known in the observed data. 
At the same time, it is important to properly treat the systematic 
uncertainties associated with the complexity of the experimental 
apparatus (the ATLAS detector) and the background predictions when 
presenting an interpretation of the results. 
%%  we would like to answer the following questions:
%% \begin{itemize}
%% \item Does the data contain evidence for new particles or interactions not 
%% accounted for by the background prediction?
%% \item What is the most powerful test statistic that can differentiate 
%%   the Standard Model expectation from a hypothesized new signal?
%% \item If there is a signal, what is its statistical significance?
%% \item In the absence of a signal, can it be excluded at a given confidence 
%%   level? 
%% \end{itemize} 
This chapter describes the statistical methodology employed to 
test the compatibility between data and prediction while taking into 
account the systematic uncertainties. 
The analysis' possible outcomes are represented by a likelihood function 
that combines observations, predictions, and associated uncertainties. 
At this point the 
hypothesis testing is performed with the corresponding one-sided profile 
likelihood ratio~\cite{Cowan:2010js}, 
and upper limits are provided as one-sided $95\%$ confidence level intervals in the CL$_\text{s}$ formalism~\cite{Read:2002}. 
The statistical tool used to perform the quantification of the significance 
of hypothetical excesses seen in data
or upper limits setting on new physics contributions as implemented in 
this analysis will be described in this chapter.


