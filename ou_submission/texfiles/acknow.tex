While I have not found evidence for supersymmetry in the work presented in this dissertation, 
my search for it led me to meet some talented, passionate, and knowledgeable physicists, 
many of whom became close friends and mentors to me during this exciting journey. 
I am deeply grateful for their guidance and encouragement to think critically and independently,
to take initiatives, and to appreciate the elegance in the world of elementary particles 
and their interactions.

First and foremost, thanks to my advisor, Brad Abbott, for his continuous support and encouragement over these past five years. 
Brad gave me enough freedom and space to pursue the projects I felt most passionate about and at the same time be always
there to guide me through all the decisions I had to make during my PhD. I could always rely on his perspective to set realistic 
expectations, and for that I am very grateful.

Thanks to the faculty of the High Energy Physics (HEP) group at OU, Michael Strauss, Patrick Skubic, Phillip Gutierrez, 
and John Stupak,
for their useful feedback on my many presentations and for financially supporting my research.
I would like to thank my doctoral committee for evaluating my progress and reviewing my work especially towards the end 
of my PhD. 
Thanks to the OU HEP students and postdocs, Ben Peasron, Callie Bradley, David Bertsche, David Shope, Muhammad Alhroob, 
Muhammad Saleem, Scarlet Norberg, and Yu-Ting Shen, who made my experience 
either at CERN or OU richer and more enjoyable.

Thanks to the whole Argonne National Laboratory HEP group. 
%The one year I spent with the group was one of the  best periods of my graduate school experience.
I would like to particularly thank Bob Blair, Jeremy Love, and Jinlong Zhang in working together to make the evolution of the ATLAS 
Region of Interest Builder (RoIB) a success. 
We replaced the two-decade-old RoIB that ``discovered the Higgs'', as Jeremy likes to say, with a modern solution 
that worked flawlessly.
%The new system integration and commissioning in ATLAS went so smoothly that barely anyone noticed, a testament of a job well done,
%keeping in mind that a failure of this system would have hindered the data-taking ability of ATLAS.
I would like to thank Sergei Chekanov with whom I made my first contribution to an ATLAS paper.
I am particularly grateful for having worked with Sasha Paramonov who showed me how to think like an experimental particle physicist.
%Sasha would always find simple ways to convey his ideas and take the time to answer my many questions. 
I would like to also thank the theorists in the group, in particular Ahmed Ismail, Carlos Wagner, and Ian Low for the many 
discussions we had about new signatures of supersymmetry.
I am thankful to James Proudfoot who continued offering his advice and guidance throughout my stay at Argonne and afterwards.

Thanks to William Vazquez and Wainer Vandelli in their invaluable help while developing the RoIB software.
I still remember my excitement the Friday afternoon at CERN when we finally figured out how to get the RoIB to operate at 
over the targeted 100 kHz rate. It was one of the highlights of the project.

Thanks to my collaborators in the ``SS/3L'' supersymmetry seach group. In particular, I am thankful to 
Otilia Ducu, Julien Maurer, Ximo Poveda, Peter Tornambe, and Fabio Cardillo for the countless hours of debugging, 
checking and cross-checking results, and the long Skype conversations. 
I have not only learned from them the intricate details of searching for rare signals and estimating 
the difficult detector backgrounds but also crisis management skills with the three conference rushes we had to endure. 
We had a very pleasant and efficient work environment which translated into being among the first groups to 
publish their results.
%leading to very early publications of 
%which allowed us to put out the second LHC SUSY paper 
%with 2015 dataset, the first ATLAS conference note for ICHEP 2016, 
%and the second ATLAS SUSY paper with the full 2015 and 2016 datasets. 
It was a real pleasure working with this team.

Thanks to all my teachers and professors who had an impact on my trajectory.
Thanks to my graduate school professors, 
in particular Ronald Kantowski, Chung Kao, and Howard Baer, who made the subjects of 
electrodynamics, particle physics, and quantum field theory much more understandable.
Thanks to my undergraduate professors at Drexel University,
Chuck Lane, David Goldberg, Michael Vogeley, and Robert Gilmore, who gave me a solid foundation
in physics and played an important role in my orientation.
Thanks to my high school physics teacher, Mohammed El Baki, who 
decided in of our classes not to follow 
the curriculum and instead took us on a journey to the fundamental 
constituents of matter down to the level of quarks. 

Thanks to the many friends at CERN and the Geneva area for the good times in the mountains, on the slopes, aboard boats, 
and under water. The hiking, skiing, sailing, and diving experiences made going through graduate school much more enjoyable and fun.

Thanks to my family and friends in Morocco, Philadelphia, Norman, Chicago, and around the world for their continuous support and care. 
The list of people that I would like to thank is very long.
I would like to particularly mention Abdelfattah, Debbie and Mohamed, 
Heidar and his family, Mak, Salah, Sameed, and many who should be in this list but aren't. 

Finally, I want to thank my parents, Latifa and Chouaib, and my sister, Hafsa, for their  
endless support and love during this adventure. This dissertation is dedicated to them.




%Edoardo Farina
%Ondrej Hladik
%Artem Basalaev
%Giulia Ripellion
%Sameed Mohammad
%Nicolas Hoffmann
%Philipp Millet
%Guy Crockford
%Denis Regat


