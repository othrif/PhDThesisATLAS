% \documentclass[linenumbers,summarypage,hyperlinks]{outhesis}
\documentclass{outhesis}

% For a bibliography style, you must have the appropriate .bst file
%\bibliographystyle{prsty}
\bibliographystyle{bibtex/bst/atlasBibStyleWoTitle}

% Provide the correct margins
%\usepackage[top=1in, bottom=1in, left=1.6in, right=1.2in]{geometry}
% If you want a double-sided copy for yourself, uncomment the next line
% \usepackage[twoside,top=1in, bottom=1in, left=1.6in, right=1.2in]{geometry}

\usepackage[singlespacing]{setspace}

% Specify where ATLAS LaTeX style files can be found.
\newcommand*{\ATLASLATEXPATH}{latex/}
\usepackage{\ATLASLATEXPATH atlasmisc}
\usepackage{\ATLASLATEXPATH atlasbsm}
\usepackage{\ATLASLATEXPATH atlasphysics}
\usepackage{\ATLASLATEXPATH myadditions}

\begin{document}

\begin{equation}
%\begin{aligned}
 L(\{N^\text{SS,obs}_\varpi\}|\{\xi(\eta,\pt)\}) 
= 
 \prod_{\varpi} \mathcal P\left(N^\text{SS,obs}_\varpi|w_\text{flip}(\xi(\eta_1,p_{\mathrm{T},1}),\xi(\eta_2,p_{\mathrm{T},2}))\times N^\text{OS+SS,obs}_\varpi\right)
\label{eqn:chargeflip_likelihood}
%\end{aligned}
\end{equation}


\begin{align}
\begin{pmatrix}n_\text{T}\\n_\text{L}\end{pmatrix} 
= \begin{pmatrix}\varepsilon_r & \varepsilon_f\\ 1-\varepsilon_r & 1-\varepsilon_f\end{pmatrix}
\begin{pmatrix}n_\text{R}\\n_\text{F}\end{pmatrix}
\label{eqn:matrix_method}
\end{align}




\[
\varepsilon_f=\frac{n_\text{signal}^\text{data} - n_\text{signal}^\text{MC}}{n_\text{baseline}^\text{data} - n_\text{baseline}^\text{MC}}
\]


\[
\varepsilon_r=\frac{n_\text{signal}^\text{data}}{n_\text{baseline}^\text{data} - n_\text{baseline}^\text{BKG}}
\]




\begin{table}[htb!]
\def\arraystretch{1.15}
\def\arraystretch{1.15}
\centering
\resizebox{\textwidth}{!}{
\begin{tabular}{|c|c|c|c|} \hline\hline
             $10<\pt<12$ &              $12<\pt<14$ &              $14<\pt<17$ &             $17<\pt<20$ \\
$0.10 \pm 0.01 \pm 0.00$ & $0.10 \pm 0.01 \pm 0.01$ & $0.12 \pm 0.01 \pm 0.01$ & $0.08 \pm 0.02 \pm 0.00$ \\
\hline 
\end{tabular}}

\resizebox{\textwidth}{!}{
\begin{tabular}{|c|c|c|c|} \hline
             $20<\pt<25$ &              $25<\pt<30$ &              $30<\pt<40$ &             $40>\pt$ \\
$0.07 \pm 0.02 \pm 0.01$ & $0.11 \pm 0.03 \pm 0.01$ & $0.20 \pm 0.07 \pm 0.03$ & $0.25 \pm 0.10 \pm 0.05$ \\
\hline \hline
\end{tabular}}
\caption{Electron fake rate measured in data and the associated statistical uncertainty. 
The systematic uncertainty originating from the subtraction of ``backgrounds'' with only prompt leptons is also displayed. }
\label{table:fake_rates_electron}
\end{table}


\begin{table}[htb!]
\def\arraystretch{1.15}
\def\arraystretch{1.15}
\centering
\resizebox{\textwidth}{!}{
\begin{tabular}{|c|c|c|c|} \hline\hline
\multicolumn{2}{|c|}{$10<\pt<12~\GeV$}         & \multicolumn{2}{c|}{$12<\pt<14$}                  \\  
%\hline 
$|\eta|<2.3$             & $|\eta|>2.3$             & $|\eta|<2.3$             & $|\eta|>2.3$            \\
%\hline
$0.14 \pm 0.01 \pm 0.00$ & $0.22 \pm 0.05 \pm 0.00$ & $0.11 \pm 0.01 \pm 0.00$ & $0.24 \pm 0.06 \pm 0.00$ \\ 
\hline
\end{tabular}}

\resizebox{\textwidth}{!}{
\begin{tabular}{|c|c|c|c|} \hline
\multicolumn{2}{|c|}{$14<\pt<17$}                    & \multicolumn{2}{c|}{$17<\pt< 20~\GeV$}       \\       
%\hline
$|\eta|<2.3$             & $|\eta|>2.3$             & $|\eta|<2.3$             & $|\eta|>2.3$            \\    
%\hline
$0.12 \pm 0.01 \pm 0.00$ & $0.09 \pm 0.05 \pm 0.00$ & $0.09 \pm 0.01 \pm 0.00$ & $0.21 \pm 0.07 \pm 0.00$ \\
\hline 
\end{tabular}}

\resizebox{\textwidth}{!}{
\begin{tabular}{|c|c|c|c|c|} \hline
             $20<\pt<30$ &              $30<\pt<40$ &              $40<\pt<60$ &                $\pt>60$ \\
%\hline
$0.07 \pm 0.02 \pm 0.00$ & $0.12 \pm 0.05 \pm 0.01$ & $0.16 \pm 0.09 \pm 0.04$ & $0.49 \pm 0.10 \pm 0.07$ \\
\hline \hline
\end{tabular}}
\caption{Muon fake rate measured in data and the associated statistical uncertainty. 
The systematic uncertainty originating from the subtraction of ``backgrounds'' with only prompt leptons is also displayed. }
\label{table:fake_rates_muon}
\end{table}


\end{document}


